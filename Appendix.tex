\chapter{Higher Algebra Backgrounds}

In this appendix, we recall some notions and results from \cite{HTT} and \cite{HA} that are used repeatedly in this thesis.


\section{Monoids and Groups in $\infty$-Categories}
Let $\CC$ be a pointed $\infty$-category with finite limits. 
We recall that a \emph{monoid object} in an $\infty$-category $\CC$ \cite[
Definition 4.1.2.5.]{HA} is a simplicial object $X: \Delta^{op}\to \CC$ satisfying the Segal condition:
the collection of face maps $X([n])\to X(\{i-1,i\})$ for $1\leq i \leq n$ exhibits $X([n])$ as a product of $\{X(\{i-1,i\})\}_{1\leq i \leq n}$. 
We will denote the full subcategory of monoids in $\CC$ by $\Mon(\CC)$.

Under the identification $\Mon(\CC)\simeq \Mon_{\Ass}(\CC)$ (see \cite[Proposition 4.1.2.10.]{HA}), a monoid $X$ is equipped with a (homotopy coherently) associative multiplication:
$$
m: X\times X\to X;
$$
we say that $X$ is a \emph{group} in $\CC$ \cite[Definition 5.2.6.2.]{HA} if both shearing maps
\begin{align*}
			(p_1, m):X\times X\to X\times X\\
	(m,p_2):X\times X\to X\times X
\end{align*}
are equivalences. 
We will denote the $\infty$-category of groups in $\CC$ as $\Grp(\CC)$.
By \cite[Remark 5.2.6.5.]{HA}, this is equivalent to the definition of \emph{groupoid objects} in \cite[Definition 6.1.2.7.]{HTT}.

We now give another characterization of group objects in terms of presheaves on $\CC$.
Since the Yoneda embedding preserves limits, it induces a functor
$$
\Grp(\CC) \to  \Grp(\Fun(\CC^{op}, \Space)) \simeq \Fun(\CC^{op}, \Grp(\Space)).
$$
Hence we can identify group objects in $\CC$ as those representable presheaves that factor through $\Grp(\Space)$. 

The \emph{loop object} $\Omega Y$ of an object $Y\in \CC$ is defined as
$$
\Omega Y:=0 \underset { Y }{\times } 0.
$$
We want to show that $\Omega Y$ is a group object of $\CC$. 
\begin{lemma}
\label{Loop of an object is a group}
    Let $\CC$ be a pointed $\infty$-category with finite limits, then $\Omega Y$ is a group object for any $Y\in \CC$.
\end{lemma}
\begin{proof}
    The image of the Yoneda embedding of $\Omega Y$ is given by 
    $$
    X \mapsto \map_{\CC}(X, \Omega Y)\simeq \Omega \map_{\CC}(X, Y),
    $$
    where latter is in $\Grp(\Space)$ for any $X\in \CC$.
\end{proof}
\begin{remark}
In particular, $\Omega Y$ is a monoid in $\CC$. The multiplication map is given by the "concatenation" map
$$
0 \underset { Y }{\times } 0 \underset { Y }{\times } 0
\simeq\Omega Y\times \Omega Y\rightarrow \Omega Y\simeq 0\underset{Y}{\times}Y\underset{Y}{\times} 0 
$$
which is unique up to contractible ambiguity. 
\end{remark}


\begin{proposition}
\label{induced fully faithfulness on group objects}
Let $\CC$ and $\CD$ be $\infty$-categories with finite products. Suppose $F:\CC \hookrightarrow \CD$ is a fully faithful, product-preserving functor, then
	\begin{enumerate}[(1)]
		\item the functor $F$ induces a fully faithful embedding $\Mon(F):\Mon(\CC)\to \Mon(\CD)$ between the category of associative monoids.
		\item the functor $F$ induces a fully faithful embedding $\Grp(F):\Grp(\CC)\to \Grp(\CD)$ between the category of group objects.
	\end{enumerate}
\end{proposition}
\begin{proof}
	Observe first that the functor $F$ induces a fully faithful embedding 
	$$
	\Fun(\Delta^{op},\CC)\to \Fun(\Delta^{op},\CD)
	$$ between the $\infty$-categories of simplicial objects.
	Then $(1)$ follows from the fact that $F$ send monoids to monoids and $\Mon(\CC)$ is a full subcategory of $\Fun(\Delta^{op},\CC)$; and $(2)$ follows from the fact that $\Grp(\CC)$ is a full subcategory of $\Mon(\CC)$.
\end{proof}
%%%%


\begin{definition}
	\cite[Definition 6.1.2.7.]{HTT}
	A simplicial object $U_{\bullet}$ in $\CD$ is a \textit{groupoid object} if for every $n\geq 0$ and every partition $[n]=S\cup S'$ such that $S\cap S'=\{s\}$, the diagram
	\[
	\begin{tikzcd}
		U([n]) \arrow[r] \arrow[d] & U(S) \arrow[d]\\
		U(S') \arrow[r]    & U(\{s\}) 
	\end{tikzcd}
	\]
	is a pullback square in the $\infty$-category $\CC$.
\end{definition}


%%%%





\begin{proposition}
\label{Restrict adjoints to full subcategory}
	Let $\adj{F}{\CC}{\CD}{G}$ be a pair of adjoint functors between $\infty$-categories.
	Let $\CC^0$ be a full subcategory of $\CC$ so that the essential image of $G$ is contained in $\CC^0$. Then $F^0:=F|_{\CC^0}$ restricts to an adjoint pair 
	$$
	\adj{F^0}{\CC^0}{\CD}{G}.
	$$
\end{proposition}
\begin{proof}
    By Proposition 5.2.2.8 of \cite{HTT}, the pair of adjoint functors gives a unit natural transformation
    \[
    id_{\CC} \to G\circ F
    \]
    such that for every pair of objects $c\in \CC$ and $d\in \CD$ the composite
    \[
    \map_{\CD}(F(c),d) \to \map_\CC(G\circ F(c),G(d)) \to \map_{\CC}(c,G(d))
    \]
    is an equivalence .
	The assumptions then give a natural transformation $id_{\CC^{0}}\to G\circ F$.
	For every $c'\in \CC^0$ and $d\in \CD$, the composite 
	\[
	\map_{\CD}(F(c'),d) \to \map_{\CC^0}(G\circ F(c'),G(d)) \to \map_{\CC^{0}}(c',G(d))
	\] 
	is an equivalence, since one observes that
	$$
	\map_{\CC^0}(G\circ F(c'),G(d))\simeq \map_{\CC}(G\circ F(c'),G(d))
	$$
	and 
	$$
	\map_{\CC^0}(c',G(d))\simeq \map_{\CC}(G\circ F(c'),G(d)).
	$$
	\end{proof}

\begin{definition}
\cite[Definition 2.1]{Gepner-Moritz-Nikolaus}
\label{preadditive}
    An $\infty$-category is \emph{preadditive} if the canonical morphism 
    $$
    X\coprod Y \to X\times Y
    $$
    is an equivalence for any pair of objects $X,Y \in \CC$.
\end{definition}

Let $\CMon(\CC)$ denote the $\infty$-category of commutative monoids in $\CC$.
\begin{proposition}
\label{Prop 2.3 GMN}
\cite[Proposition 2.3]{Gepner-Moritz-Nikolaus}
Let $\CC$ be an $\infty$-category with finite products and finite coproducts, 
    then the following are equivalent:
    \begin{enumerate}
        \item The $\infty$-category $\CC$ is preadditive.
        \item The homotopy category $h\CC$ is preadditive.
        \item The forgetful functor $\CMon(\CC) \to \CC$ is an equivalence.
    \end{enumerate}
\end{proposition}

If $M\in \CC$ is an object in a preadditive $\infty$-category $\CC$, then it $M$ can be equipped with a commutativee monoid structure by Proposition \ref{Prop 2.3 GMN}.
We will call the monoid structure map $\nabla:M\oplus M \to M$ the \emph{fold map}. 
The \emph{shearing map} defined as
$$
s: M\oplus M \xrightarrow{(pr_1, \nabla)} M\oplus M,
$$
where $pr_1$ denotes the projection to the first factor.

\begin{definition}
\cite[Definition 2.6]{Gepner-Moritz-Nikolaus}
    \label{additive}
    A preadditive $\infty$-category $\CC$ is \emph{additive} if the shearing map $s$ is an equivalence.
\end{definition}

\begin{proposition}
\label{Prop 2.8 GMN}
\cite[Proposition 2.8]{Gepner-Moritz-Nikolaus}
Let $\CC$ be an $\infty$-category with finite products and finite coproducts, 
    then the following are equivalent:
    \begin{enumerate}
      \item The $\infty$-category $\CC$ is additive.
      \item The homotopy category $h\CC$ is additive.
      \item The forgetful functor $\Grp(\CC)\to \CC$ is an equivalence.
    \end{enumerate}
\end{proposition}

Proposition \ref{Prop 2.3 GMN} and Proposition \ref{Prop 2.8 GMN} then imply that 
there are equivalences
$$
\Grp(\CC)\simeq\CMon(\CC) \simeq \CC
$$
if $\CC$ is additive. Moreover, if $\CC$ is stable, we have the following corollary regarding the $\infty$-category $\Alg_{\CO}(\CC)$ of $\CO$-algebras in $\CC$.
\begin{corollary}
    Let $\CC$ be a presentably stable $\infty$-category.
    Let $B_{\CO}(M)$ denote the bar construction of a monoid $M$ in $\Alg_{\CO}(\CC)$.
    Then the pair of adjoint functors
$$
\adj{B_{\CO}}{\Grp(\Alg_{\CO}(\CC))}{\Alg_{\CO}(\CC)}{\Omega_{\CO}}
$$
are mutually inverses.
\end{corollary}
\label{Alg(C) equivalent to groups in Alg(C)}
\begin{proof}
    Since both of the functors $B_{\CO}$ and $\oblv_{\CO}$ preserves sifted colimits, the bar construction $B(M)$ of $M\in \Grp(\Alg_{\CO}(\CC))$ is computed in $\CC$.
    Since $\CC$ is stable, the bar complex $\Barconstruction(0, M, 0)_{\bullet}$ comes from the \v{C}ech nerve $0 \to \Sigma M$, which implies that
    $B_{\CO}M \simeq \Sigma M$.
    Therefore we have a commuting diagram
	\[
	\begin{tikzcd}
		\Grp(\Alg_{\CO}(\CC)) & \Alg_{\CO}(\CC) \\
		\CC &
		\CC ;
		\arrow[from=1-1, to=1-2, "B_{\CO}", shift left]
		\arrow[from=1-2, to=1-1, "\Omega_{\CO}", shift left]
		\arrow[from=1-1, to=2-1, "\oblv_{\Grp}\circ \oblv_{\CO}" left]
		\arrow[from=1-2, to=2-2, "\oblv_{\CO}"]
		\arrow[from=2-1, to=2-2, "\Sigma", shift left]
		\arrow[from=2-2, to=2-1, "\Omega", shift left]
	\end{tikzcd}
	\]
	Since the vertical forgetful functors are conservative, there are equivalences
$$	    
\oblv_{\CO}\circ B_{\CO} \circ  \Omega_{\CO} (X)
	    \simeq \Sigma \Omega X \simeq X
$$
and 
$$
\oblv_{\Grp}\circ \oblv_{\CO} \circ \Omega_{\CO} \circ B_{\CO} X 
\simeq 
\Omega \Sigma X
\simeq
X.
$$
\end{proof}




We consider now the $\infty$-category $\Sp^{\geq r}_{\tame}$ of $r$-tame spectra.
Let $\Sigma$ and $\Omega$ denote the suspension and loop functor in $\Sp$ respectively.
Note that the category $\Sp^{\geq r}_{\tame}$ is not closed under $\Sigma$, so it's not an inverse of $\Omega$.
However, it's easy to see we have the following lemma.
\begin{lemma}
\label{Relations between different tameness}
	The adjunction
	\[
	\adj{\Sigma}{\Sp^{\geq r-1}_{\tame}}{\Sp^{\geq r}_{\tame}}{\Omega}
	\]
	are mutual inverses.
\end{lemma} 
Moreover, if we let $L_{r-1, \tame}$ and $L_{r, \tame}$ be $(r-1)$-localization and $r$-localization functors, respectively.
It's easy to see there are equivalences
$$
L_{r-1, \tame} \Omega X \simeq \Omega L_{r, \tame}X
$$
for an $r$-connective spectrum, and 
$$
L_{r, \tame} \Sigma X \simeq \Sigma L_{r-1, \tame}X
$$
for an $(r-1)$-connective spectrum.


Since $\Alg_{\Lie}(\Sp^{\geq r}_{tame})$ is Cartesian symmetric monoidal, the loop functor 
$$
\Omega: \Sp^{\geq r}_{\tame}
\to 
\Sp^{\geq r-1}_{\tame}
$$
induces a functor
\[
\Omega_{\Lie}:\Alg_{\Lie}(\Sp^{\geq r}_{\tame}) \to \Grp( \Alg_{\Lie}(\Sp^{\geq r-1}_{\tame}) )
\]
which takes a $r$-tame Lie algebra $X$ to a $(r-1)$-tame Lie algebra $\Omega_{\Lie}X$ whose underlying spectrum is $\Omega X$.
\begin{proposition}
	\label{B and Omega are mutally inverses}
	The functor 
	\[
	\Omega_{\spLie}: 
	\Alg_{\spLie}(\Sp^{\geq r}_{\tame}) \to \Grp( \Alg_{\spLie}(\Sp^{\geq r-1}_{\tame}))
	\]
	is an equivalence.
\end{proposition}
\begin{proof}
	Let $B_{\spLie}$ denote the bar construction functor that is left adjoint to $\Omega_{\spLie}'$. Since the forgetful functor 	
	$\Alg_{\spLie}(\Sp^{\geq r}_{\tame}) \to \Sp^{\geq r}_{\tame}$ preserves sifted colimits, the bar construction is computed in the underlying category $\Sp^{\geq r}_{\tame}$ of $r$-tame spectra;
	Let $X_{\bullet}$ be a simplicial object in $\Sp^{\geq r-1}_{\tame}$, the geometric realization of $X_{\bullet}$ is computed by first taking the geometric realization of $X_{\bullet}$ in $\Sp$ then applying the $r$-tame localization functor $L_{\tame}$, but $|X_{\bullet}|\simeq \Sigma X$ which is already $r$-tame, hence we have $B_{\spLie}X\simeq \Sigma X$.
	Therefore we have a commuting diagram
	\[
	\begin{tikzcd}
		\Grp( \Alg_{\spLie}(\Sp^{\geq r-1}_{\tame}) ) & \Alg_{\spLie}(\Sp^{\geq r}_{\tame})\\
		\Sp^{\geq r-1}_{\tame} &
		\Sp^{\geq r}_{\tame};
		\arrow[from=1-1, to=1-2, "B_{\spLie}", shift left]
		\arrow[from=1-2, to=1-1, "\Omega_{\spLie}'", shift left]
		\arrow[from=1-1, to=2-1]
		\arrow[from=1-2, to=2-2]
		\arrow[from=2-1, to=2-2, "\Sigma", shift left]
		\arrow[from=2-2, to=2-1, "\Omega", shift left]
	\end{tikzcd}
	\]
	Hence $B_{\spLie}$ and $\Omega_{\spLie}'$ are mutually inverses because $\Sigma$ and $\Omega$ are mutually inverses.
\end{proof}

%$$
%\Omega_{\Lie}:\Alg_{\Lie}(\Sp^{\geq r}_{\text{$r$-tame}}) \to \Grp(\Alg_{\Lie}(\Sp^{\geq r}_{\text{$r$-tame}}))
%$$
%factors through $\Grp(\Alg_{\Lie}(\Sp^{\geq r-1}_{\text{$(r-1)$-tame}}))$.
%Let 
%$$
%B_{\Lie}:\Grp(\Alg_{\Lie}(\Sp^{\geq r-1}_{\text{$(r-1)$-tame}})) \to
%\Alg_{\Lie}(\Sp^{\geq r-1}_{\text{$(r-1)$-tame}})
%$$
%be the bar construction functor.
%We claim the following
%\begin{lemma}
%	The composition
%	\[
%	U:
%	\Grp(\Alg_{\Lie}(\Sp^{\geq r-1}_{\text{$(r-1)$-tame}})) 
%	\xrightarrow{B_{\Lie}}
%\Alg_{\Lie}(\Sp^{\geq r-1}_{\text{$(r-1)$-tame}})
%\to 
%\Sp^{\geq r-1}_{\text{$(r-1)$-tame}}
%\to 
%\Sp
%\]
%can be identified with the suspension $\Sigma X$ of the underlying spectrum of $X\in \Grp(\Alg_{\Lie}(\Sp^{\geq r-1}_{\text{$(r-1)$-tame}}))$. Hence, $U$ factors over $\Sp^{\geq r}_{\text{$r$-tame}}$.
%\end{lemma}
%\begin{proof}
%	
%\end{proof}
%

\section{Filtered and Graded Objects in Infinity-Categories}
In this section, we introduce the notion of filtered and graded objects in a symmetric monoidal $\infty$-category $\CC$.
The main reference for this section is \cite{Brantner-Mathew}.
\begin{definition}
	Let $\BZ$ denote the poset of the integers and let $\CC$ be an $\infty$-category.
	The $\infty$-category of \emph{filtered objects in $\CC$} is defined as the functor category 
	$$
	\Fil(\CC) :=\Fun(\BZ, \CC).
	$$
\end{definition}

Let $\BZ^{\operatorname{disc}}$ denote the groupoid with objects the integers and isomorphisms.
\begin{definition}
	The $\infty$-category of \emph{graded objects in $\CC$} is defined as the functor category 
	$$
	\Graded(\CC):=\Fun(\BZ^{\operatorname{disc}}, \CC).
	$$
\end{definition}

\begin{remark}
	We can also extend the definitions above to the category $\Fil^{+}(\CC)$ (resp. $\Gr^{+}(\CC)$) of \emph{non-negatively filtered} (resp. \emph{non-negatively graded}) objects.
\end{remark}

The natural inclusion $ \BZ^{\operatorname{disc}}\hookrightarrow \BZ$ induces a forgetful functor
$
\Fil(\CC) \to \Gr(\CC). 
$
We denote $$
(-)_*: \Graded(\CC) \to \Fil(\CC)
$$ the left Kan extension along the inclusion $ \BZ^{\operatorname{disc}}\hookrightarrow \BZ$; explicitly, the filtered object $X_*$ evaluated at $n$ is given by $\bigoplus_{k\leq n} X_k$.

We can also define the \emph{associated graded} of a filtered object 
$$
\Graded(-):
\Fil(\CC)
\to 
\Graded(\CC)
$$
to be $X_* \mapsto (n \mapsto \cofib(X_{n-1} \to X_{n}))$.
The following lemma follows immediately from an inductive argument.
\begin{lemma}
\label{Ass-gr is conservative}
	The associated graded functor $$\Graded(-):
\Fil^{+}(\CC)
\to 
\Gr^{+}(\CC)
$$
is conservative.
\end{lemma}

We also have the following obvious observation.
\begin{lemma}
	The composite 
	\[
	\Graded(\CC) \xrightarrow{(-)_*} 
	\Fil(\CC)
	\xrightarrow{\Graded(-)} 
	\Graded(\CC) 
	\]
	is equivalent to the identity functor.
\end{lemma}

The following remark from \cite{Brantner-Mathew} equips both the category of filtered objects and graded objects symmetric monoidal structure.
\begin{remark}
\cite[Definition 2.5]{Brantner-Mathew}
Suppose that $\CC$ is (nonunital) presentably symmetric monoidal $\infty$-category in which tensor product commuts with colimits.
Using Day convolution, one can equip both $\Fil(\CC)$ and $\Graded(\CC)$ with the structure of presentably (nonunital) symmetric monoidal $\infty$-categories. Furthermore, the associated graded functor $\Graded : \Fil(\CC) \to \Graded(\CC)$ is (nonunital) symmetric monoidal (cf. \cite[Sec. 2.23]{Glasman}).
\end{remark}

Consider now an $\infty$-operad $\CO$ in $\CC$, we can now state a theorem in \cite{Heuts_Koszul} which says that every $\CO$-algebra admits a canonical filtration so that its associated graded is trivial.
\begin{theorem}
\label{Canonical grading on an O-algebra}
\cite[Theorem 5.2 (2)]{Heuts_Koszul}
For an $\CO$-algebra $X$, there exists a canonical canonical filtered object $X_{*}$ so that 
\begin{enumerate}
    \item The filtration is \emph{exhaustive}, i.e. there is an equivalence
    $$
    \colim X_{*} \to X.
    $$
    \item The filtered $\CO$-algebra has associated graded
    $$
    \Graded(X_*)\simeq \Free_{\CO}(B\CO(n)\otimes X^{\otimes n })_{h\Sigma_n}.
    $$
\end{enumerate}


\end{theorem}



% We let $A = \BS \oplus \BS$ and view it as an augmented filtered commutative algebra with $A_0=\BS$ and 
% $A_n=\BS \oplus \BS$ for $n\geq 1$.
% For any $\CO$-algebra $X$, we obtain a filtered $\CO$-algebra $(\iota_0X)_*$ where $\iota_0$ is the functor that inserts an object to a graded object concentrated in degree $0$.
% Consider now the functor
% \begin{align*}
% 	\CAlg(\CC^{\Fil, \geq 0}) \times \Alg_{\CO}(\CC) 
% 	& \to 
% 	\Alg_{\CO}(\CC^{\Fil, \geq 0})\\
% 	(A, X) & \mapsto A\otimes (\iota_0X)_*.
% \end{align*}
% Note that $A$ is augmented, hence we have a natural morphism 
% \[
% \epsilon: A\otimes (\iota_0X)_* 
% \to 
% (\iota_0X)_* .
% \]

% \begin{definition}
% 	The \emph{canonical filtration} is a functor 
% 	$$
% 	(-)^{\Fil}: \Alg_{\CO}(\CC)\to \Alg_{\CO}(\CC^{\Fil,\geq 0})
% 	$$ 
% 	given by
% 	$X\mapsto   \fib(A\otimes (\iota_0X)_* \to (\iota_0X)_*)$.
% \end{definition}

% \begin{proposition}
% 	\label{canonical filtration}
% 	The associated graded $\gr(X^{\Fil})$ of the canonical filtration on $X$ is the trivial graded $\CO$-algebra, i.e. 
% 	$$
% 	\gr(X^{\Fil}) \simeq \trivial^{\gr}_{\CO}(\iota_{1} \circ \oblv_{\CO}X).
% 	$$
% % Is the following even true?
% %\begin{enumerate}
% %	\item \todo{do we really need this?} the direct colimit of $X^{\Fil}$ is equivalent to $X$, i.e. $$\colim X^{\Fil} \simeq X.$$
% %\end{enumerate}
% \end{proposition}
% \begin{proof}
% %	We abuse notation by writing $(\iota_0X)_*\in \CC^{\Fil, \geq 0}$ as $X$ in the following.
% 	Note that $\gr(X^{\Fil})$ is a graded $\CO$-algebra with $X$ concentrated in degree $1$, hence it is a trivial $\CO$-algebra by degree reason.
	
	
	
	
% \end{proof}
	


%\begin{lemma}
%\label{Algebras lifted to filtered algebras}
%	The composite
%	$$
%	\Alg_{\CO}(\CC) \xrightarrow{(-)^{\Fil}}
%	\Alg_{\CO}(\CC^{\Fil,\geq 0})
%	\xrightarrow{\oblv_{\Fil}}
%	\Alg_{\CO}(\CC)
%	$$
%	is equivalent to the identity functor on $\Alg_{\CO}(\CC)$.
%\end{lemma}
%\begin{proof}
%	Since the forgetful functor $\oblv_{\CO}: \Alg_{\CO}(\CC)\to \CC$ is conservative, it suffices to show
%	$$
%	\oblv_{\CO}\circ \oblv_{\Fil}(X^{\Fil}) \simeq X,
%	$$
%	which follows from the equivalence $\fib(X\times X \to  X)\simeq X$. 
%\end{proof}






%One can also extend the definitions above to the category $\gr(\CC)$ of \emph{non-negatively graded} objects


\clearpage







\section{The Barr-Beck-Lurie Theorem}
\begin{theorem}
\label{Barr-Beck-Lurie theorem}
	[Theorem 4.7.3.5. \cite{HA}]
	Let $\adj{F}{\CC}{\CD}{G}$ be an adjoint pair of $\infty$-categories. Then $G$ is monadic if and only if 
	\begin{enumerate}
		\item $G$ is conservative.
		\item If $X_\bullet$ is a $G$-split simplicial object in $\CD$, then its geometric realization exists in $\CD$ and $G$ preserves geometric realization of $X_{\bullet}$.
	\end{enumerate}
\end{theorem}

In practice, the category $\CD$ often admits all geometric realizations for simplicial objects. In this case, we have a technically convenient criteria for determining monadicity of a functor.
\begin{corollary}[\cite{HeutsSurvey}]
\label{Cor of Barr-Beck-Lurie theorem}
Suppose $\adj{F}{\CC}{\CD}{G}$ an adjoint pair and that $\CD$ admits colimits of $G$-split simplicial objects. Then this pair is monadic if and only if for every object $X$ of $\CD$, the map
$$
\abs{(FG)^{\bullet+1}X} \to X
$$
arising from the simplicial resolution described above is an equivalence.
\end{corollary}
\begin{proof}
	Suppose this pair is monadic, then it satisfies the conditions of Theorem \ref{Barr-Beck-Lurie theorem}. For an object $X$ in $\CD$, the simplicial object $(FG)^{\bullet+1}X$ is a $G$-split. Indeed, the simplicial object $G((FG)^{\bullet+1}X)$ admits a contracting homotopy via the unit natural transformation $X \to GF(X)$.
	Now consider the map 
	$$
	\abs{(FG)^{\bullet+1}X} \to X
	$$
	and apply $G$ to it, one has
	\begin{align*}
		G(\abs{(FG)^{\bullet+1}X}) & \simeq \abs{G(FG)^{\bullet+1}X}\\
		& \simeq G(X)
	\end{align*}
	where the first equivalence is due to the fact that $G$ preserves geometric realization of $G$-split objects and the second equivalence is due to the assumption that $(FG)^{\bullet+1}X$ is $G$-split.  
	Since $G$ is conservative, we conclude that $\abs{(FG)^{\bullet+1}X} \to X$ is an equivalence.
	
	Suppose now $\abs{(FG)^{\bullet+1}X} \to X$ is an equivalence for every $X$ in $\CD$ and $G(f): G(X) \to G(Y)$ is an equivalence in $\CC$ for some morphism $f:X \to Y$, then $(FG)^{\bullet+1} X \to (FG)^{\bullet+1}Y$ is an equivalence of simplicial objects in $\CD$. Therefore, 
	$$
	\abs{(FG)^{\bullet+1} X} \to \abs{(FG)^{\bullet+1} Y}
	$$
	is an equivalence and hence is $f: X\to Y$. So $G$ is conservative.
	We now claim that $G$ preserves geometric realization of $G$-split objects.
	Let $X_{\bullet}$ be a $G$-split simplicial object and consider the following commuting diagram
	\[
	\begin{tikzcd}
		\abs{(FG)^{\bullet+1} \abs{X_{\bullet}}} \ar[r] \ar[d]& \abs{(FG)^{\bullet+1} X_{-1} \ar[d]}     \\
		\abs{X_{\bullet}} \ar[r] & X_{-1}  .
	\end{tikzcd}
	\]
	We claim the bottom horizontal arrow is an equivalence.
	Note that two vertical morphisms are equivalences by our assumption. We claim that the top horizontal map is an equivalence as well. Indeed, if we view $(FG)^{p+1} X_{q}$ is an bisimplicial set then 
	$$
	\abs{(FG)^{\bullet+1} \abs{X_{\bullet}}} 
	\simeq 
	\colim_{p} \colim_q (FG)^{q+1}X_{p} .
	$$
	For fixed $q$, the simplicial object $(FG)^{q+1}X_{\bullet}$ is split since it's a composite of functors starting with $G$, hence one has 
	$$
	\colim_{p}(FG)^{q+1}X_{p}\simeq (FG)^{q+1}X_{-1}
	$$
	and $\colim_{q} (FG)^{q+1} X_{-1} \to X_{-1}$  is an equivalence by the assumption.
	Therefore, we conclude that 
	$$
	G(\abs{X_{\bullet}}) \simeq G(X_{-1}) \simeq \abs{G(X_{\bullet})}
	$$
	where the last equivalence follows from the fact that $X_{\bullet}$ is $G$-split.
\end{proof}

    As an application in homotopy theory, we prove the following folklore proposition that is well-known among seasoned homotopy theoriests.
\begin{proposition}
	\label{Coalgebra model for simply-connected spaces}
	For $r\geq 2$, the functor $\Sigma^{\infty}:\Space^{\geq r}_{*} \to \Sp^{\geq r}$ is comonadic. In other words, there is an equivalence of $\infty$-categories:
	\[
	\phi: \Space^{\geq r}_{*} \to \coAlg_{\Sigma^{\infty}\Omega^{\infty}}(\Sp^{\geq r}).
	\]
 between the $\infty$-category of $r$-connective spaces and the $\infty$-category of $r$-connective $\Sigma^{\infty}\Omega^{\infty}$-coalgebra.
\end{proposition}
\begin{proof}
By Corollary \ref{Cor of Barr-Beck-Lurie theorem}, it suffices to show there is an equivalence 
$$
X\to \Tot (\Omega^{\infty}\Sigma^{\infty})^{\bullet+1}X
$$
for every $X\in \Space_{*}^{\geq r}$. We prove this by induction on the Postnikov tower. For $X$ being an Eilenberg-Maclane space $K(A,n)\simeq \Omega^{\infty-n}HA$, its associated augmented cosimplicial object $(\Omega^{\infty}\Sigma^{\infty})^{\bullet+1}X$ splits, with the contracting homotopy induced by the counit $\Sigma^{\infty}\Omega^{\infty} \Omega^n HA\to \Omega^n HA$. For the inductive step, we have a principal fibration sequence
\[
K(\pi_n X,n) \to \tau_{\leq n}X \to \tau_{\leq n-1}X 
\to 
K(\pi_n X,n+1).
\]
By the principal fibration lemma \cite{Bousfield-KanYellow}, the functor $\Tot (\Omega^{\infty}\Sigma^{\infty})^{\bullet+1}$ preserves principal fibrations. Hence the vertical sequences in the following diagram are fiber sequences.
\[
\begin{tikzcd}
	\tau_{\leq n}X & \Tot (\Omega^{\infty}\Sigma^{\infty})^{\bullet+1} (\tau_{\leq n}X)\\
	\tau_{\leq n-1}X  & \Tot (\Omega^{\infty}\Sigma^{\infty})^{\bullet+1} (\tau_{\leq n-1}X)\\
	K(\pi_n X,n+1)   & \Tot (\Omega^{\infty}\Sigma^{\infty})^{\bullet+1} K(\pi_n X,n+1)
	\arrow[from=1-1, to=1-2]
	\arrow[from=1-1, to=2-1]
	\arrow[from=2-1, to=2-2]
	\arrow[from=1-2, to=2-2]
	\arrow[from=2-1, to=3-1]
	\arrow[from=3-1, to=3-2]
	\arrow[from=2-1, to=3-1]
	\arrow[from=2-2, to=3-2]
\end{tikzcd}
\]
Observe that the bottom two horizontal arrows are equivalences by the inductive hypothesis, hence the induced map on the fibers is an equivalence. This completes the inductive step of the proof.
\end{proof}







\section{Construction of the Comparison Functor}
\label{Construction of the Comparison Functor}
In this section, we construct the comparison functor from divided power conilpotent commutative coalgebras in tame spectra to commutative coalgebras in tame spectra.
$$
\zeta:\coCAlg^{\divpow,\nil}(\Sp^{\geq r}_{\tame}) \to
\coCAlg(\Sp^{\geq r}_{\tame}).
$$
We will explain this in a more general setting.
First, we need a result from Heine.
\begin{theorem}
\label{Monads-Alg correspondence}
\cite[Theorem 5.1]{Heine_Monads}
Let $\CC$ be a presentable $\infty$-category, then the $\infty$-category of comonads on $\CC$ is a localization of the over category $\Pr^{L}_{/\CC}$ that assigns a comonad $Q$ to $\LcoMod_{Q}(\CC)$.
\end{theorem}

Our strategy is to use Heine's theorem to construct the comparison functor $\zeta$ by producing a map of their comonads.
Let $\CC$ be a presentable symmetric monoidal $\infty$-category whose tensor product is compatible with colimits.
Suppose $L:\CC \to \CD$ is a symmetric monoidal localization.
and we let $j:\CD \to \CC$ denote the embedding that is right adjoint to $L$.
We then have an adjunction on the following functor categories.
\begin{proposition}
\label{Prop A.4.2}
There is an adjunction
$$
    \adj{L\circ (-)\circ j}{\Fun(\CC, \CC)}{\Fun(\CD, \CD)}{j\circ (-)\circ L}.
$$
\end{proposition}
\begin{proof}
    We need to show there are two pairs of adjunctions
    $$
    \adj{L_{*}}{\Fun(\CC, \CC)}{\Fun(\CC, \CD)}{j_*}
    $$
    and
    $$
    \adj{j^{*}}{\Fun(\CC, \CD)}{\Fun(\CD, \CD)}{L^*},
    $$
    which is an easy exercise to check the triangle identities of both.
\end{proof}
Let $Q_{\CC}$ and $Q_{\CD}$ denote the comonad arising from the forgetful-cofree adjunction 
$$\adj{\oblv_{\CC}}{\coCAlg(\CC)}{\CC}{\cofree_{\CC}}$$
and
$$\adj{\oblv_{\CD}}{\coCAlg(\CD)}{\CD}{\cofree_{\CD}}$$
\begin{lemma}
    There is an equivalence of comonads
    $$
    L_* j^*Q_{\CC}\simeq Q_{\CD}
    $$
    on $\CD$.
\end{lemma}



We observe the functor $j_*L^*: \Fun(\CD,\CD) \to \Fun(\CC,\CC)$ is monoidal; indeed, for two endofunctors $F,G$ on $\CC$ we have
\begin{align*}
    j_*L^*(F\circ G) & \simeq j\circ F \circ G \circ L\\
                     & \simeq j\circ F \circ L \circ j \circ  G \circ L \\
                     & \simeq j_*L^*(F)\circ j_*L^*(G)
\end{align*}
where the second last equivalence follows from $ L \circ j \simeq \id_{\CD}$.
Hence $j_*L^*$ preserves monads and comonads. Furthermore, the left adjoint $j^*L_*$ is oplax monoidal so it also preserves comnads. 
The counit map of the adjunction in Proposition \ref{Prop A.4.2} evaluating on any comonad $Q$ on $\CD$
$$
j^*L_*j_*L^*Q  \to  Q
$$
is an equivalence of comonads. Therefore, the functor $j_*L^*:\Comonad(\CD) \to \Comonad(\CC)$ is fully faithful and we record the following lemma.

\begin{lemma}
\label{Pullback diagram in Prop A4.2}
The following diagram of $\infty$-categories
\[
\begin{tikzcd}
	\LcoMod_{j_*L^*Q_\CD}(\CC) & \LcoMod_{Q_\CD}(\CD)\\
    \CC & \CD
	\arrow[from=1-1, to= 1-2]
	\arrow[from=1-1, to=2-1]
	\arrow[from=1-2, to=2-2, "\oblv"]
	\arrow[from=2-1, to= 2-2, "L"]
\end{tikzcd}
\]
is a pullback diagram.
\end{lemma}

Now suppose there is a functor
$$
\theta': \coCAlg^{\divpow, \nil}(\CC) \to \coCAlg(\CC),
$$
then postcomposes with the localization functor gives
$$
\theta: \coCAlg^{\divpow, \nil}(\CC) \to \coCAlg(\CD).
$$
Therefore, we obtain a functor 
$$
\LcoMod_{F_{\Com}}(\CC) \to
\LcoMod_{j_*L^*Q_\CD}(\CC)
$$
as the canonical functor into the pullback in the diagram of Proposition \ref{Pullback diagram in Prop A4.2}. 
Then by Theorem \ref{Monads-Alg correspondence}, there is a map of comonads
$F_{\Com}\to j_*L^*Q_\CD$ on $\CC$, which corresponds to a map of comonads 
$\Gamma:L\circ F_{\Com}\circ j \to Q_{\CD}$ by adjunction.
\begin{definition}
    We define the comparison functor
    \[
    \zeta: \coCAlg^{\divpow,\nil}(\CD) :=\LcoMod_{L_{\tame}\circ F_{\Com}}(\CD) \to \coCAlg(\CD):=\LcoMod_{Q_{\CD}}
    \]
    to be the functor corresponding to $\Gamma$.
\end{definition}

To construct the comparison functor in our case, take $\CC=\Sp^{\geq r}$ and $\CD=\Sp^{\geq r}_{\tame}$. There is a comparison functor on coalgebras in $r$-connective spectra
$$
\theta': \coCAlg^{\divpow, \nil}(\Sp^{\geq r}) \to \coCAlg(\Sp^{\geq r})
$$
by \cite{Francis-Gaitsgory}.
Combining with the above discussion, we obtain the comparison functor
\[
\zeta:\coCAlg^{\divpow,\nil}(\Sp^{\geq r}_{\tame}) \to
\coCAlg(\Sp^{\geq r}_{\tame}).
\]











\section{Stuff I don't need anymoree...}

\subsection{Smashing localization}
Let $\CC^{\otimes}$ be a symmetric monoidal $\infty$-category. We say a localization functor $L:\CC\to\CC$ is \emph{smashing} if it is given by $L(-)\simeq (-)\otimes X$ for some object $X\in\CC$.
The aim of this section is to prove the following result concerning localization in $\Sp$.

\begin{proposition}
\label{HR is smashing}
	Let $R$ be a localization of the integers $\BZ$. Then the loacalization $L_R:\Sp\to \Sp$ with respect to the homology theory with coefficient in $R$ is smashing.
\end{proposition}

\begin{remark}
	The statement above is \cite[Proposition 2.4]{BousfieldSpectra}.
\end{remark}

Let $A$ be an abelian group and let $SA$ denote the Moore spectrum associated to $A$. 
\begin{theorem}
	[Universal coefficient theorem]
%	\cite{Adams' blue book}
	Let $X$ be any spectrum. Then we have a short exact sequence:
	\[
	0\to 
	\pi_{*}X\otimes A \to
	\pi_{*}(X\otimes S A)
	\to
	\Tor_{\BZ}(\pi_{*-1}X, A)\to 
	0.
	\]
	More generally, let $E$ be another spectrum, we have a short exact sequence:
	\[
	0\to 
	E_n(X)\otimes A 
	\to
	(EA)_n(X)
	\to
	\Tor_\BZ(E_{n-1}(X),A)
	\to
	0
	\]
	where $EA:= E\otimes SA$.
\end{theorem}

\begin{corollary}
	For any abelian group $A$, we have 
	\[
	HA\simeq H\BZ\otimes SA.
	\]
\end{corollary}
\begin{proof}
	Let $X=H\BZ$ in the Universal coefficient theorem.
\end{proof}

\begin{corollary}
Let $R$ be a localization of $\BZ$.
	Then the map
	\[
	SR\to SR\otimes SR
	\]
	induced from the unit map $\BS \to SR$ is an equivalence.
\end{corollary}
\begin{proof}
	Let $E=H\BZ$ and $X=SR$ in the universal coefficient theorem. We obtain at $n=0$
	an isomorphism
	$$
	R\otimes R \cong H\BZ_0(SR \otimes SR)
	$$
	When $n= 1$, we have
	\[
	H\BZ_1(SR \otimes SR) \cong \Tor_\BZ(H\BZ_{0}(SR),R)=\Tor_\BZ(R,R).
	\]
	Note that $R$ has two important properties
	\begin{enumerate}
		\item $R\cong R\otimes R$;
		\item $\Tor_{\BZ}(R,R)=0$.
	\end{enumerate}
	Therefore, we conclude that $SR\otimes SR\simeq SR$. 
\end{proof}


\begin{proposition}
	Let $\CO$ be a connected operad, the following map 
	$$
	\tau_{\leq n}\CO \rightarrow 
	\Cobar(\Barconstruction(\tau_{\leq n}\CO,\CO,\mathds{1}),\Barconstruction(\CO),\mathds{1})
	$$
	is an isomorphism for $n\geq 1$.
\end{proposition}
\begin{proof}
	We prove this by induction on $n$.
	For $n=1$, the map $\triv\simeq \tau_{\leq 1}\CO \rightarrow \Cobar(\Barconstruction(\CO),\Barconstruction(\CO),\triv)$ is clearly an isomorphism.
	For $n>0$, we first consider the following diagram in $\Opd^{aug}(\CC)$.
	\[
	\begin{tikzcd}
	\tau_{\leq n}\CO \ar[d]\ar[r] & \Cobar \big(\Barconstruction(\tau_{\leq n}\CO,\CO,\triv),\Barconstruction (\CO), \triv	 \big) \ar[d]  \\
	\tau_{\leq n-1}\CO \ar[r]    & 
	\Cobar \big(\Barconstruction(\tau_{\leq n-1}\CO,\CO,\triv),\Barconstruction (\CO), \triv	 \big)
	\end{tikzcd}
	\]
	Taking the fibers of the vertical maps, we obtain another diagram
		\[
	\begin{tikzcd}
	\CO_n \arrow[d] \arrow[r] & 
	\Cobar \big(\Barconstruction(\CO_n,\CO,\triv),\Barconstruction (\CO), \triv	 \big) \ar[d] \\
	\tau_{\leq n}\CO \ar[d]\ar[r] & \Cobar \big(\Barconstruction(\tau_{\leq n}\CO,\CO,\triv),\Barconstruction (\CO), \triv	 \big) \ar[d]  \\
	\tau_{\leq n-1}\CO \ar[r]    & 
	\Cobar \big(\Barconstruction(\tau_{\leq n-1}\CO,\CO,\triv),\Barconstruction (\CO), \triv	 \big)
	\end{tikzcd}
	\]
	where $\CO_n$ is the operad whose underlying symmetric sequence is 
	$$
	 \CO_n:= \begin{cases}
    \CO(n), & \text{for } k = n \\
   0, & \text{otherwise. }         \\
  \end{cases}
	$$
	Observe that 
	\begin{align*}
			\Cobar \big(\Barconstruction(\CO_n,\CO,\triv),\Barconstruction (\CO), \triv \big) &
			\simeq \Cobar \big(\CO_n\circ \Barconstruction(\CO),\Barconstruction (\CO), \triv	 \big)\\
			& \simeq \CO_n,
	\end{align*}
	which implies that the top horizontal map is an equivalence. The bottom horizontal map is an equivalence by induction hypothesis. Therefore, the mid-horizontal map is an equivalence as well. This concludes the proposition.
	
\end{proof}
\begin{corollary}
		If an operad $\CO$ is connected, then $\CO$ is a Koszul operad.
\end{corollary}
\begin{proof}
	
\end{proof}

\subsection{The Universal Enveloping Algebra Functor}
Let $k$ be a field of characteristc $0$. Denote the category of $k$-vector spaces by $\vectk$.
If $A$ is a unital associative algebra over $k$, then the commutator operation endows $A$ with a Lie algebra structure. This construction is functorial so that it determines a functor from the category $\Alg_{\Ass}(\vect_k)$ of unital associative algebras to the category $\Alg_{\Lie}(\vect_k)$ of Lie algebras. 
Conversely, given a Lie algebra $L$ one can define its universal enveloping algebra $U(L)$; as a vector space, one can construct it as the tensor algebra $T(L)$ modulo the ideal $I$ generated by 
 elements of the form $x\otimes y - y\otimes x-[x,y]$. 
This construction is functorial and one check it defines a functor
$$
U:\Alg_{\Lie}(\vect_k) \rightarrow \Alg_{\Ass}(\vect_k),
$$
which is left adjoint to the forgetful functor.
We'll refer the functor $U$ as the \emph{enveloping algebra functor}. 

The enveloping algebra functor is well studied both in algebra and topology. We state two results concerning the enveloping algebra functor $U$ that will be relevant later.
\begin{itemize}
	\item The enveloping algebra functor $U$ is symmetric monoidal; that is, if we endow the category of Lie algebras $\Alg_{\Lie}(\vect_k)$ with the Cartesian symmetric monoidal structure and $\Alg_{\Ass}(\vect_k)$ with the usual symmetric monoidal structure with tensor product as monoidal product, then $U$ sends categorical products to tensor products. As a consequence, $U(L)$ has a canonical coalgebra structure for any Lie algebra $L\in \Alg_{\Lie}(\vect_k)$.
	\item Over a field of characteristic, the Poincare-Witt-Birkhoff theorem has a strong form that gives a simple description of the universal enveloping algebra $U(L)$. 
	\begin{theorem}
		[Poincare-Birkhoff-Witt]
		\cite{Quillen_RHT}[Appendix B Theorem 2.3]
		Let $L$ be a Lie algebra over $k$ and let $i:L\to U(L)$ denote the unit map of. Then the "averaging" map
		\begin{align*}
			S(L) & \to U(L) \\
			x_1\cdots x_n & \mapsto \frac{1}{n!}\sum_{\sigma\in \Sigma_n} i(x_{\sigma(1)})\cdots i(x_{\sigma(n)})
		\end{align*}	
		is an isomorphism of coalgebras.
	\end{theorem}
	
\end{itemize}

The goal of this paper is to discuss coalgebras and the enveloping algebra functor in a more general context. 


%\begin{lemma}
%\label{conservativity of forgetful functor}
%	The forgetful functor 
%	\[
%	\Grp(\CS^{\geq r-1}_{\text{($r-1$)-tame}}) \xrightarrow{\oblv_{\Grp}} 
%	\CS^{\geq r-1}_{\text{($r-1$)-tame}}
%	\]
%	is a conservative functor.
%\end{lemma}
%\begin{proof}
%	Since $\Grp(\CS^{\geq r-1}_{\text{($r-1$)-tame}})$ is a full subcategory of $\Mon(\CS^{\geq r-1}_{\text{($r-1$)-tame}})$, it suffices to show the forgetful functor $\oblv_{\Mon}: \Mon(\CS^{\geq r-1}_{\text{($r-1$)-tame}})
%	\to
%	\CS^{\geq r-1}_{\text{($r-1$)-tame}}$ is conservative.
%	Note that we have a commutative diagram
%	\[
%\begin{tikzcd}
%	\Mon(\CS^{\geq r-1}_{\text{($r-1$)-tame}}) & \CS^{\geq r-1}_{\text{($r-1$)-tame}}\\
%	\Mon(\CS^{\geq r-1}_{*}) & 
%	\CS^{\geq r-1}_{*}
%	\arrow[from=1-1, to= 1-2, "\oblv_{\Mon}"]
%%	\arrow[from=1-2, to= 1-1, shift left, "R"]
%	\arrow[from=1-1, to=2-1]
%	\arrow[from=1-2, to=2-2]
%	\arrow[from=2-1, to= 2-2, "\oblv_{\Mon}"]
%%	\arrow[from=2-2, to= 2-1, shift left, "R'"]
%\end{tikzcd}
%\]
%where both horizontal arrows are fully faithful and the bottom arrow is conservative by \cite{HA}[Corollary 5.2.6.18.]. Hence the top arrow is conservative as well.
%\end{proof}

\subsection{Computational proof of Proposition 4.1.1.}

\todo[inline]{Above is an alternative approach.}
Since $R_k$ is a PID, by the structure theorem of finitely generated modules over PIDs, we are further reduced to check the cases $V=R_k$ and $V=\BZ/p^m$ where $p$ is a prime not inverted in $R_k$. 

\begin{lemma}
	For $V=R_k$, we have
\begin{enumerate}
	\item If $n$ is odd,
	 $$
	 \pi_* \Sigma_{tame}^{\infty}K(V,n) =
	 \begin{cases}
		R_k & \text{if $*=n$,}\\
		0 & \text{otherwise}.
	 \end{cases}
	 $$
	\item If $n$ is even,
	$$
	 \pi_* \Sigma_{tame}^{\infty}K(V,n) =
	 \begin{cases}
		R_{in-r} & \text{if $*=in$,}\\
		0 & \text{otherwise}.
	 \end{cases}
	 $$
\end{enumerate}
\end{lemma}
\begin{proof}
Observe that $K(V,n)$ is the $R_k$-localization of $K(\BZ,n)$ and combine with Lemma \ref{htpy groups of tame spectra can be computed by homology}, we see that 
	\begin{align*}
		\pi_* \Sigma_{\tame}^{\infty}K(V, n) & = H_* K(V, n)\otimes R_{*-r}\\
		& = H_* K(\BZ, n)\otimes R_{*-r}.
	\end{align*}
	Recall that the first $p$-torsion in the homology groups of $K(\BZ, n)$ is at degree $2p-2+n$, but $p$ is already invertible in $R_{k+2p-2}$. 
	Hence, one has
	$$
	H_* K(\BZ,n)\otimes R_{*-r} = \big(H_* K(\BZ,n)/\text{torsions}\big) \otimes R_{*-r}
	$$ and the result follows from the classical calculation of $H_* K(\BZ,n)/\text{torsions}$ \cite[Proposition 1.20]{Hatcher}. 
\end{proof}


We now consider the case for $V=\BZ/p$,
where $p$ is a prime not inverted in $R_k$.
For $p=2$, the only tame Eilenberg-Maclane space is $K(\BZ/2, r)$. The tame symmetric coalgebra on $\Sigma^r H\BZ/2$ is equivalent to itself, since higher tenosr power of $\Sigma^r H\BZ/2$ is trivial for degree reason. On the other hand, $\Sigma_{\tame}^{\infty}K(\BZ/2,r)\to \Sigma^r H\BZ/2$ is a tame equivalence.
Hence the map
$
 \Sigma_{\tame}^{\infty}K(\BZ/2,r)\to \\Sym_{\tame} \Sigma^{r}H\BZ/2
$
is an equivalence.
 
For $p$ an odd prime.
First, we review the result of the classical computation of the cohomology of $K(\BZ/p,n)$ by Cartan in \cite{Cartan_I} and \cite{Cartan_II}; See also \cite[Chapter 6]{McCleary_SS}.
Let $\beta$ denote the Bockstein operation and $P^i$ denote the $i$-th mod-$p$ Steenrod operations.
The $\BF_p$-cohomology of $K(\BZ/p,n)$ is known classically, as the free graded commutative $\BF_p$-algebra generated by the set 
$$
\{
St^I(\iota_n)=\beta^{\epsilon_0}P^{s_1}\cdots \beta^{\epsilon_m}P^{\epsilon_m}(\iota_n) |
e(I)<n 
\}
$$
where $I$ is an admissible sequence (cf. \cite[Theorem 4.46]{McCleary_SS}) of nonnegative integers $I = (\epsilon_0, s_1, \epsilon_1, \dots, s_m, \epsilon_m)$,  $\iota_n$ is the fundamental class of $K(\BZ/p,n)$ and $e(I)$ is the \emph{excess} computed as
\[
e(I):=2s_1p+2\epsilon_0-\abs{I}.
\]

Since the Steenrod operation $P^i$ has degree $2i(p-1)$, and the prime $p$ is inverted in $R_{k}$ for $k\geq 2p-3$, hence the generators of the form
$$
P^i(\iota_n), \text{ for } i\geq 1
$$  
live in degrees where $p$ has been inverted. Therefore, we obtain the following proposition.
\begin{proposition}
\label{F_p tame cohomology of K(Z/p,n)}
	The cohomology $H^*(K(\BZ/p,n);\BF_p)\otimes R_{*-r}$ of the Eilenberg-Maclane space with coefficients in the tame ring system is given by $(r+2p-4)$-truncation of the free graded commutative $\BF_p$-algebra generated by 
	$\iota_n$ and $\beta(\iota_n)$ tensoring with the tame ring system.
\end{proposition}

Next we claim the cohomology $H^*(K(\BZ/p,n);\BZ)\otimes R_{*-r}$ with coefficients in the tame ring system admits no summand of $\BZ/p^i$ for $i>1$;
For this we need to analyze the Bockstein cohomology $BH^*(K(V,n))$ group of $K(V,n)$.


We claim the Bockstein cohomology groups $BH^*(K(\BZ/p,n);\BF_p)$ are trivial in degree $[n, r+2p-4]$.
By Proposition \ref{F_p tame cohomology of K(Z/p,n)}, we only need to consider the differential graded algebra generated by the classes $\iota_n$ and $\beta(\iota_n)$. 

When $n$ is odd, the cohomology $H^*(K(\BZ/p,n);\BF_p)$ in degree $[n, r+2p-4]$ is a truncated $\BF_{p}$-DGA (cf. \cite[Section 10.1]{McCleary_SS})
$$
\Lambda_{\BF_p}(\iota_n)\otimes \BF_p[\beta(\iota_n)]
$$
with differential $d(\iota_n)=\beta(\iota_n)$ and $d(\beta(\iota_n))=\beta^2(\iota_n)=0$. Hence $\Lambda_{\BF_p}(\iota_n)\otimes \BF_p[\beta(\iota_n)]$ is acyclic, and the Bockstein cohomology groups $BH^*(K(\BZ/p,n);\BF_p)$ are trivial.

When $n$ is even, the cohomology $H^*(K(\BZ/p,n);\BF_p)$ in degree $[n, r+2p-4]$ is a truncated $F_{p}$-algebra of the DGA
$$
 \BF_p[\iota_n]\otimes \Lambda_{\BF_p}(\beta(\iota_n))
$$
with the same differential as in the odd case. We claim that $BH^j(K(\BZ/p,n);\BF_p)$ is trivial for $j<pn$; indeed, we have $d\iota_n^m=m\cdot\iota_n^{m-1}$ so it is an isomorphism for $m<p$ and is zero for $m=p$. 
Since we assume $n=r+k$ and $kp\geq 0$, it suffices to compare $rp$ with $r+2p-4$;
\begin{equation*}
    rp - (r+2p-4) = r(p-1) -2(p-1) + 2 = (r-2)(p-1)+2>0
\end{equation*}
as $r\geq 4$ and $p$ is an odd prime.

\begin{proposition}
[\cite{Hatcher} Theorem 3E.3]
	Let $H_{n}(X;\BZ)$ be finitely generated for all $n$, then each $\BZ/p^{k}$ summand of $H^n(X;\BZ)$ with $k>1$ contributes $\BZ/p$ summands to both $BH^{n-1}(X;\BF_p)$ and $BH^{n}(X;\BF_p)$, where $BH^n(X;\BF_p)$ denotes the $n$-th Bockstein cohomology group.
\end{proposition}
The proposition above then imply there is no $\BZ/p^k$ summand in lower degree of $H^*K(\BZ/p,n)$.
\begin{proposition}
\label{integreal cohomology of Eilenberg-Maclane space}
	Let $p$ be an odd prime.
	The integral cohomology group 
	$H^*K(\BZ/p,n)$ has no summand $\BZ/p^k$ for $k>1$ in degree less than $np$.
\end{proposition}

As a consequence of Proposition \ref{integreal cohomology of Eilenberg-Maclane space}, we can slice the long exact sequence 
\[
\cdots \to 
H^{m}(X;\BZ)\xrightarrow{\cdot p} H^{m}(X;\BZ)
\xrightarrow{\rho}
H^{m}(X;\BZ/p)\xrightarrow{\tilde{\beta}}
H^{m+1}(X;\BZ)
\to
\cdots
\]
in \cite[Section 3.E]{Hatcher}, into short exact sequences 
\[
0 \to H^{m}(X;\BZ)
\xrightarrow{\rho}
H^{m}(X;\BZ/p)
\xrightarrow{\tilde{\beta}}
H^{m+1}(X;\BZ)
\to 
0
\]
for $m<np$.
Note that $\rho$ is injective, so we have 
\begin{align*}
    \operatorname{Im}\rho & \cong \ker (\tilde{\beta}:H^{m}(X;\BZ/p)\to H^{m+1}(X;\BZ))\\
                        & \cong \ker (\beta:H^{m}(X;\BZ/p)\to H^{m+1}(X;\BZ/p))
\end{align*}
since $\beta = \rho\tilde{\beta}$.
Hence $\operatorname{Im}\rho$ is generated by the class $\beta(\iota_n)$ and
we conclude our computation in the following proposition.

\begin{proposition}
\label{tame homology of K(Z/p,n)}
The cohomology of $K(\BZ/p, n)$ with coefficients in the tame ring system is given by
	\[
	H^{m}(K(\BZ/p,n);\BZ)\otimes R_{m-r}\cong
	\begin{cases}
		$R_{m-r}\otimes\BZ/p$,  & \text{ for $m = l n +1$ in $[r, r+2p-4]$;} \\
		$0$,        &\text{ otherwise.}
	\end{cases}
	\]
	for an integer $l\geq 1$.
	
	The homology of $K(\BZ/p, n)$ with coefficients in the tame ring system is then given by
	\[
	H_{m}(K(\BZ/p,n);\BZ)\otimes R_{m-r}\cong
	\begin{cases}
		$R_{m-r}\otimes\BZ/p$,  & \text{ for $m = l n$ in $[r, r+2p-4]$;} \\
		$0$,        &\text{ otherwise.}
	\end{cases}
	\]
	for an integer $l\geq 1$.
\end{proposition}

By Remark \ref{suspension tame sends Moore space to EM-space}, there are equivalences
    \begin{align*}
			\Sym_{\tame} \Sigma^n H\BZ_p & \simeq \Sym_{\tame} \Sigma^{\infty}_{\tame}M(\BZ/p,n)\\
			&  \simeq L_{\tame}\Sym \Sigma^{\infty}  M(\BZ/p,n).
	\end{align*}
With this identification, we are reduced to show the map
$$
\Sigma^{\infty}K(\BZ/p,n) \to 
	\Sym \Sigma^{\infty}  M(\BZ/p,n)
$$
induces an isomorphism on homology with coefficients in the tame ring system.

We now recall the mod-$p$ homology of $\Sym \Sigma^{\infty} X$ for a spectrum $X\in \Sp$ from \cite[IX.2.1]{H_inf}. 
For $I = (\epsilon_1, s_1, , \dots, \epsilon_m, s_m)$ a sequence of non-negative integers, it determines the homology operation 
$Q^I =\beta^{\epsilon_1}Q^{s_1}\dots \beta^{\epsilon m}Q^{s_m}$ where $Q^s$ denotes the Dyer-Lashof operations with degree $|Q|=2s(p-1)$, see \cite{Cohen-Lada-May_homology} for more details.



Let $A$ be a set of basis of $H_*(X;\BF_p)$ and let $\CM$ be the set of \emph{standard indecomposables} for $\Sym X$,
\[
\CM(X):=\{
Q^I x | x \in A,  \text{ admissible and } e(I)+b(I) > \abs{x};
\}
\]
where $b(I)=\epsilon_1$ and the \emph{excess} $e(I)$ (see \cite[I.2]{Cohen-Lada-May_homology} for more details) of $I$ is defined as 
\[
e(I):= 2s_1-\epsilon_1 - \sum_{j=2}^{m}[2s_j(p-1)-\epsilon_j].
\]
When $I$ is the empty sequence, we adopt the convention that $e(I)=\infty$, $b(I)=0$.
We denote $\Free_{\BF_p}(\CM(X))$ the free commutative $\BF_p$-algebra generated by $\CM$.
Since the homology of $\Sym X$ is also a commutative $\BF_p$-algebra, we have a canonical map
\[
\lambda: \Free_{\BF_p}(\CM(X)) \to
H_{*} \Sym X.
\]
\begin{theorem}
	\cite[IX.2.1]{H_inf}
	The map 
	\[
\lambda: \Free_{\BF_p}(\CM(X)) \to
H_{*} \Sym X.
\]
is an isomorphism of $\BF_p$-algebras.
\end{theorem}

Back to the case $X=\Sigma^{\infty} M(\BZ/p,n)$, whose homology is $\BZ/p$ at degree $n$ and $0$ otherwise.
Since the Dyer-Lashof operation $Q^s$ raises degree by $2s(p-1)$ \cite[I, Theorem 1.1]{Cohen-Lada-May_homology}, so $H_* (\Sigma^{\infty} M(\BZ/p,n); \BF_p)$ is generated by $\iota_n$ and $\beta \iota$ in the tame range. After running the same Bockstein argument as we did for the cohomology $H^* K(\BZ/p,n)$, we conclude the following analogue of Proposition \ref{tame homology of K(Z/p)}.
\begin{corollary}
The homology $H_* (\Sigma^{\infty} M(\BZ/p,n); \BF_p)$ with coefficients in the tame ring system is given by the $(r+2p-3)$-truncation of the free graded commutative $F_{p}$-algebra generated by $\iota_n$
\end{corollary}

By Corollary \ref{htpy groups of tame spectra can be computed by homology}, 



Since $\pi_* \Sym_{\tame} \Sigma^n H\BF_p$ is the free commutative $\BF_p$-algebra generated by an element in degree $n$, hence we have the following corollary.
\begin{corollary}
\label{base case}
	The map $\Sigma^{\infty}K(\BZ/p,n) \to 
	\Sym \Sigma^nH\BF_p$ is a tame equivalence.
\end{corollary}

For $p$ an odd prime and $I=(\epsilon_1, s_1, \dots, \epsilon_k, s_k)$ such that $\epsilon_j\in \{0,1\}$.
We say $I$ is \emph{admissible} if $s_i\geq ps_{i+1}\epsilon_i$ for $1\leq i <k$. We write 
$$
P^I:=\beta^{\epsilon_0}P^{s_1}\beta^{\epsilon_1}\cdots P^{s_k}\beta^{\epsilon_k}.
$$
\begin{theorem}
	[??]\todo{ref?}
	The admissible monomials $P^I\in \CA^{*}$ form basis for the mod-$p$ Steenrod algebra $\CA^*$.
\end{theorem}




\begin{corollary}
	There is an isomorphism of graded $\BF_p$-vector spaces
	\[
	\tau_{\leq r+2p-3}H^{*}( K(\BZ/p,n);\BF_p) 
	\to
	\tau_{\leq r+2p-3} (H\BF_p)^*\Sigma^nH\BF_p.
	\]
\end{corollary}





Now observe that $H^{*}(K(\BZ/p,n);\BF_p)$ and $H_{*}(\Sym \Sigma^nH\BF_p;\BF_p)$ are isomorphic as DGAs.
We can then run the Bockstein argument, and see that their integral cohomology doesn't contain $\BZ/p^k$ for $k>1$ below degree $np$.


For the inductive step, recall that there is a principal fiber sequence
\[
K(\BZ/p^k,n)\to 
K(\BZ/p^k,n+1)
\xrightarrow{\beta} K(\BZ/p^{k+1},n+1)
\]
and hence we can write $K(\BZ/p^{k+1},n+1)\simeq \abs{\Barconstruction(K(\BZ/p^k,n+1),K(\BZ/p^k,n),*)}$.
Note that 
\begin{equation}
\label{principal fiber seq for EM-spaces}
	\Sigma^{\infty}_{+}K(\BZ/p^k,n)\to 
\Sigma^{\infty}_{+}K(\BZ/p^k,n+1)
\to 
\Sigma^{\infty}_{+}
K(\BZ/p^{k+1},n+1)
\end{equation}
is a principal fiber sequence of $\E_{\infty}$-rings,
since $\Sigma^{\infty}_{+}$ is symmetric monoidal and preserves small colimits.

\begin{theorem}
	The map $\Sigma^{\infty}K(\BZ/p^k,n) \to 
	\Sym \Sigma^nH\BZ/{p^k}$ is a tame equivalence for any $k\geq 1$.
\end{theorem}
\begin{proof}
	The case for $k=1$ is theorem \ref{base case}.
	For the inductive step, observe that 
	\[
		\Sigma^{\infty}_{+}K(\BZ/p^k,n)\to 
\Sigma^{\infty}_{+}K(\BZ/p^k,n+1)
\to 
\Sigma^{\infty}_{+}
K(\BZ/p^{k+1},n+1)
\]
is a cofiber sequence in $\Sp$ by the discussion above.
Consider now the cofiber sequence in $\Sp$
\[
\Sigma^nH\BZ/{p^k}\to \Sigma^{n+1}H\BZ/{p^k}
\to 
\Sigma^{n+1}H\BZ/{p^{k+1}},
\]
apply the functor $\Sym$ and we obtain a cofiber sequence in $\CAlg(\Sp)$
\[
\Sym\Sigma^nH\BZ/{p^k}\to 
\Sym\Sigma^{n+1}H\BZ/{p^k}
\to 
\Sym
\Sigma^{n+1}H\BZ/{p^{k+1}}.
\]
The theorem now follows from the fact that the collection of tame equivalences is closed under colimits.

\end{proof}

To conclude, we record the following corollary, which completes the proof of Proposition \ref{Fully faithfulness of C_tame on EM spaces}.
\begin{corollary}
\label{Suspension of Eilenberg-Maclane spaces are symmetric algebra}
	For any $R_k$-module $V$, the map 
	$$\Sigma^{\infty}K(V,r+k) \to 
	\Sym \Sigma^{r+k}HV$$ is a tame equivalence for any $k\geq 1$.
\end{corollary}
\subsection{Ching-Harper}



% For our purpose, we are interested in the case when $F:\CC \to \CD$ is a localization functor.
% The identification of $\Alg_{\spLie}(\Sp^{\geq r}_{\operatorname{tame}})$ as a localization of $\Alg_{\spLie}(\Sp^{\geq r})$ will be a direct consequence of a more general result below.







The following proposition allows us to identify $\Alg_{\Lie}(\Sp^{\geq r}_{tame})$ as a full subcategory of Lie algebras in $Mod_{H\BZ}$.



% By a dual argument, we can also identify the category of tame coalgebras as a full subcategory of $\coAlg^{nil, dp}_{\Com}(\Mod_{H\BZ}^{\geq r})$.
% \begin{proposition}
% 	The $\infty$-category $\coAlg^{nil, dp}_{\Com}(\Sp^{\geq r}_{tame})$ can be identified as the full subcategory of $\coAlg^{nil, dp}_{\Com}(\Mod_{H\BZ}^{\geq r})$ spanned by coalgebras whose underlying $H\BZ$-modules are tame.	
% \end{proposition}

We want to show the functor 
$$
B_{\Lie}: \Alg_{\Lie}(\Mod_{H\BZ})^{\geq r} \to \coAlg^{nil, dp}_{\Com}(\Mod_{H\BZ}^{\geq r+1})
$$ 
restricts to a functor:
$$
B_{tame}: \Alg_{\Lie}(\Sp^{\geq r}_{tame})\to \coAlg^{nil, dp}_{\Com}(\Sp^{\geq r+1}_{tame})
$$
and it is an equivalence if we precompose $B_{tame}$ with the functor $\Omega_{\Lie}$. We claim that the composition $B_{tame}\circ \Omega_{\Lie}$ is an equivalence of $\infty$-categories.

\begin{remark}
\label{symmetric monoidal structure on Mod_Z tame}
	Note that $\Sp^{\geq r}_{tame}$ admits a symmetric monoidal structure with the tensor product $X\otimes Y$ given by the tame localization $L_{tame}(X\otimes Y)$ of the smash product of $X$ and $Y$. 
	Moreover, the functor $L_{tame}:\Mod_{H\BZ}^{\geq r}\to (\Mod_{H\BZ}^{\geq r})_{tame}$ is symmetric monoidal; this follows from the fact that the composite map $M \otimes_{\BZ} N \to L_{tame}M\otimes_{\BZ} N \to L_{tame}M\otimes_{\BZ} L_{tame}N$ is a tame equivalence for any $M,N\in \Mod_{H\BZ}^{\geq r}$.
\end{remark}

\subsubsection{Group objects}
Consider $\CC$ as a Cartesian symmetric monoidal $\infty$-category, then by the dual argument of \cite[Proposition 2.4.3.9.]{HA}, the forgetful functor $\coCAlg(\CC)\to \CC$ is an equivalence.
In particular, for every $Y\in \coCAlg(\CC)\simeq \CC$, the totalization of the cobar construction $\Cobar(0,X,0)^{\bullet}$ can be identified with $0\times_X 0 \simeq \Omega Y$.
Therefore, we can identify $\Omega Y \simeq \Tot \Cobar(0,Y,0)^{\bullet}$ as an associative monoid in $\CC$.


We can then define a simplicial object $(Y^{S^1})_{\bullet}$ in $\CD$ 
$$
(Y^{S^1})_{n}:=(\Omega Y)^{\times n},
$$ for which inner face maps are given by concatenation maps, outer face maps are given by projection to zero object, and degeneracy maps are given by insertion of zero object. Therefore, we obtain the following proposition.
\begin{proposition}
	Let $\CC$ be a pointed $\infty$-category with finite limits. For any object $Y\in \CC$, the simplicial object $(Y^{S^1})_{\bullet}$ is a group object of $\CC$.
\end{proposition}
\begin{proof}
    
\end{proof}

\begin{lemma}
\label{loop objects are grouplike monoid objects}
	Let $\CC$ be a pointed $\infty$-category with finite limits. For any object $X\in \CC$, its loop object $\Omega X$ is a group object.
\end{lemma}
\begin{proof}
	Since we have defined the underlying monoid object of the associative monoid $\Omega X$ as the simplicial object
	$$
	(X^{S^1})_n=(\Omega X)^{n}.
	$$
	We claim this is indeed a groupoid object $\CC$.
    Let $S\cup S'$ be a partition of $[n]$ with $S\cap S'=\{s\}$, then the following commutative diagram
    \[
\begin{tikzcd}
	(\Omega X)^{ n} & (\Omega X)^{|S|-1} \\
	(\Omega X)^{ |S'|-1}   &  0
	\arrow[from=1-1, to= 1-2]
%	\arrow[from=1-2, to= 1-1, shift left, "R"]
	\arrow[from=1-1, to=2-1]
	\arrow[from=1-2, to=2-2]
	\arrow[from=2-1, to= 2-2]
%	\arrow[from=2-2, to= 2-1, shift left, "R'"]
\end{tikzcd}
    \]
	is a pullback diagram, since all the maps involved are projections.
	The claim then follows immediately by \cite[Remark 5.2.6.5.]{HA}.
\end{proof}


We first show that if the underlying $\infty$-category is pointed with finite limits, then the category of groups is equivalent to the category of monoids.
\begin{lemma}
    Let $\CC$ be a pointed $\infty$-category with finite limits.
    Then a morphism $f:Y \to Z$ is an equivalence if and only if $Y\times_Z 0\to 0$ is an equivalence.
\end{lemma}
\begin{proof}
    The "only if " part is clear. 
    We now prove the "if" part. We claim that
    $$
    \map_{\CC}(X, Y)
    \xrightarrow{f_*}
    \map_{\CC}(X, Z)
    $$
    is an equivalence for any $X\in \CC$. Since $Y\times_Z 0\to 0$ is an equivalence, 
    $$
    \map_{\CC}(X,Y\times_Z 0) 
    \simeq
    \map_{\CC}(X,Y) \times_{\map_{\CC}(X,Z)} \{*\} 
    $$
     is contractible. The long exact sequence for the fiber sequence 
    $$
    \map_{\CC}(X,Y\times_Z 0)  \to \map_{\CC}(X, Y ) \to \map_{\CC}(X, Z)
    $$
    then implies $\map_{\CC}(X, Y ) \to \map_{\CC}(X, Z)$ is an equivalence.
    
\end{proof}


\subsection{Original proof of trivial algebra C_tame S^{r-1}}
The lemma below would be crucial to establish the essential surjectivity of $C_{\tame}$.

\begin{lemma}
% 	\label{Sigma S^n is a trivial coalgebra}
	$C_{\tame} L_{\tame}S^{r-1}$ is a trivial coalgebra in $\coCAlg(\Sp^{\geq r-1}_{\tame})$.
\end{lemma}
\begin{proof}
    
	Using Proposition \ref{inductive construction of coalgebras}, we can build a commutative coalgebra $X$ in $\Sp$ by assembling compatible coalgebra structures of $X$ in $\coAlg_{\phi^n \Com}(\Sp)$ for each $n$. 
	We prove by induction on $n$ that $\Sigma^{\infty}_{\tame}S^{r-1}$ is a trivial coalgebra in $\coAlg_{\phi^n \Com}(\Sp)$ after tame localization.
	Assume $\Sigma^{\infty}_{\tame}S^{r-1}$ is a trivial $(\phi^n \Com)$-coalgebra, by Proposition \ref{inductive construction of coalgebras} and the vanishing of Tate diagonal in tame spectra, specifying a $(\phi^{n+1} \Com)$-coalgebra structure  on $\Sigma^{\infty}_{\tame}S^{r-1}$ is equivalent to a lift in the following diagram
	\[
	\begin{tikzcd}
		& L_{\tame}(\Sigma^{\infty}_{\tame}S^{r-1} \otimes \cdots \otimes \Sigma^{\infty}_{\tame}S^{r-1})_{h\Sigma_n}\\
		\Sigma^{\infty}_{\tame}S^{r-1} & L_{\tame}(\phi^{n-1}\Com(n)\otimes \Sigma^{\infty}_{\tame}S^{r-1} \otimes \cdots \otimes\Sigma^{\infty}_{\tame}S^{r-1})_{h\Sigma_n}
		\arrow[from = 2-1, to = 1-2, dashed]
		\arrow[from = 2-1, to = 2-2, "0" below]
		\arrow[from = 1-2, to = 2-2].
	\end{tikzcd}
	\]
	Let $F$ denote the fiber of the vertical map. We claim that the connectivity of $F$ is at least $r$, hence any lift is null-homotopic.
	The connectivity of $L_{\tame}(\Sigma^{\infty}_{\tame}S^{r-1} \otimes \cdots \otimes \Sigma^{\infty}_{\tame}S^{r-1})_{h\Sigma_n}$ is $n(r-1)$, which is larger than $r+1$ (recall $r\geq 4$). Since the possible largest dimension of a non-degenerate simplex in $\phi^{n-1}\Com(n)$ (cf. \cite[Example 4.7]{Heuts_Goodwillie}) is $n-3$, the connectivity of 
	$$
	L_{\tame}(\phi^{n-1}\Com(n)\otimes \Sigma^{\infty}_{\tame}S^{r-1} \otimes \cdots \otimes\Sigma^{\infty}_{\tame}S^{r-1})_{h\Sigma_n}
	$$
	is at least $n(r-1)-(n-3) > r+1$. Hence the connectivity of $F$ is larger than $r$, and the lemma is proved.
	
\end{proof}

\begin{remark}
	The proof above is almost identical to the proof in \cite{Heuts_Goodwillie}[Lemma 6.17]. 
	There he shows that $\Sigma^{\infty}S^{r-1}$ is a trivial coalgebra in $\coAlg^{nil, dp}_{\Com}(\tau_{p-1}\Sp^{\geq r-1})$,
	where $\coAlg^{nil, dp}_{\Com}(\tau_{p-1}\Sp^{\geq r-1})$ denotes the $\infty$-category of conilpotent, divided power coalgebras $X$ in $\Sp$ with $(p-1)!$ inverted in $\Sp$ and with coherent structure maps $X\to (X^{\otimes k})_{h\Sigma_k}$ for $1\leq k \leq p-1$. 
	The important ingredients of both proof are the inductive construction of coalgebras and the vanishing of Tate construction.
\end{remark}



\subsection{Tame localization for spectra}
Let $m_p:\BS^n\to \BS^n$ denote the multiplication-by-$p$ map on $\BS^n$. A connective spectrum $X$ is $m_p$-local if 
\[
\map_{\Sp}(\BS^n,X) \xrightarrow{m_p^*} \map_{\Sp}(\BS^n,X)
\]
is a weak homotopy equivalence.
\begin{lemma}
\label{local w.r.t. multiplication-by-p map}
	A spectrum $X$ is $m_p$-local if and only if $\pi_{n+k}(X)$ is uniquely $p$-divisible for $k\geq 0$.
\end{lemma}
\begin{proof}
	We compute the induced map on the homotopy groups of $\map_{\Sp}(\BS^n,X)$. Note that for $k\geq 0$, one has
	\begin{align*}
		\pi_k\map_{\Sp}(\BS^n,X) & \cong \pi_0 \Omega^k\map_{\Sp}(\BS^n,X)\\
		     					 & \cong \pi_0 \map_{\Sp}(\BS^{n+k},X)\\
		     					 & \cong \pi_{n+k} (X),
	\end{align*}
	and $m_p^*$ induces a multiplication-by-$p$ map $ \pi_{n+k} (X) \xrightarrow{\cdot p} \pi_{n+k} (X)$. Therefore, $X$ is $m_p$-local if and only if $\pi_{n+k}X$ is uniquely $p$-divisible for all $k\geq 0$.
	
\end{proof}
Let $\Sp_{m_p}$ denote the full subcategory of spectra whose homotopy groups are uniquely $p$-divisible above degree $n$.
Note that the localization functor $L_{m_p}:\Sp \to \Sp_{m_p}$ exists by Proposition 5.5.4.15. of \cite{HTT}.
Let's now compare $L_{m_p}$ with the Bousfield localization with respect to the homology theory $H_{*}(-;\BZ[\frac{1}{p}])$. A spectrum is $\BZ[\frac{1}{p}]$-local if and only if its homotopy groups are uniquely $p$-divisible \cite[Proposition 2.4]{BousfieldSpectra}. Therefore, a $\BZ[\frac{1}{p}]$-local spectrum is also $L_{m_p}$-local and we record the following useful lemma.
\begin{lemma}
\label{L-equivalence implies Z[1/p]-iso}
	If $f:X\to Y$ is a $L_{m_{p}}$-equivalence of spectra, then it induces an isomorphism on 
	$$
	\pi_{*}(X)\otimes \BZ[\frac{1}{p}]
	\to
	\pi_{*}(Y)\otimes \BZ[\frac{1}{p}].
	$$
\end{lemma}

The inverse is true if we restricts to connective spectra.
\begin{lemma}
\label{L-equivalence is equivalent to pi_* iso on n-conn spectra}
	Let $f:X\to Y$ be a map between $n$-connective spectra, then $f$ is a $L_{m_{p}}$-equivalence if and only if it induces an isomorphism on 
	$
	\pi_{*}(X)\otimes \BZ[\frac{1}{p}]
	\to
	\pi_{*}(Y)\otimes \BZ[\frac{1}{p}].
	$
\end{lemma}
\begin{proof}
	Let $Z$ be a $L_{m_p}$-local spectrum and let $f$ be a $\BZ[\frac{1}{p}]$-equivalence. Consider the commutative diagram
	\[
	\begin{tikzcd}
		\map_{\Sp}(Y,Z) & \map_{\Sp}(X,Z)\\
		\map_{\Sp}(Y,\tau_{\geq n}Z) & \map_{\Sp}(X,\tau_{\geq n}Z)
		\arrow[from=1-1, to=1-2]
		\arrow[from=2-1, to=2-2]
		\arrow[from=1-1, to=2-1,"\simeq"]
		\arrow[from=1-2, to=2-2,"\simeq"]
	\end{tikzcd}
	\]
	where the vertical maps are equivalences. Since the homotopy groups of $\tau_{\geq n}Z$ are uniquely $p$-divisible, it is $\BZ[\frac{1}{p}]$-local and hence the bottom map is an equivalence as well.
\end{proof}



\begin{proposition}
\label{connectivity of local equivalences}
	Let $g: X\to Y$ be a $n$-connective map in $\Sp$. Then every $L_g$-equivalence is $n$-connective.
\end{proposition}
\begin{proof}
	Let $S=\{g\}$ and denote $\overline{S}$ the strongly saturated class of morphisms generated by $S$. Proposition 5.5.4.15. of \cite{HTT} then implies that  $\overline{S}= \{L_g\text{-equivalences}\}$, hence it suffices to show that $\overline{S}$ is contained in the class of $n$-connective maps in $\Sp$. Note that the class of $n$-connective maps is saturated in the sense of Definition 5.5.5.1 of \cite{HTT}. Since every strongly saturated class of morphisms in $\Sp$ is also saturated by Example 5.5.5.5. of \cite{HTT}, we have
	$$
	\{L_g\text{-equivalences}\}=\overline{S} \subseteq \tilde{S} \subseteq \{n\text{-connective maps}\}
	$$
	where $\tilde{S}$ is the collection of saturated morphisms generated by $S$.
\end{proof}
\begin{remark}
	So in particular, every $L_{m_p}$-equivalence is $n$-connective.
\end{remark}

As a corollary, we deduce that $L_{m_p}$-localization commutes with taking $n$-connective cover.
\begin{corollary}
\label{L-localization commutes with connective cover}
	The $L_{m_p}$-localization commutes with $n$-connective cover, that is, 
	$$
	L_{m_p}(\tau_{\geq n}X)\simeq \tau_{\geq n}(L_{m_p} X).
	$$
	for any $X\in \Sp$.
\end{corollary}
\begin{proof}
	It suffices to show $\tau_{\geq n}X\to \tau_{\geq n}L_{m_{p}}X$ is a $L_{m_p}$-equivalence, as $\tau_{\geq n}L_{m_{p}}X$ is $L_{m_p}$-local. Consider the following ladder of fiber sequences
	\[
	\begin{tikzcd}
		\tau_{\geq n}X & X & \tau_{\leq n-1}X    \\
		\tau_{\geq n}L_{m_p}X & L_{m_p}X & \tau_{\leq n-1} L_{m_p}X\simeq \tau_{\leq n-1}X
		\arrow[from=1-1, to=1-2]
		\arrow[from=2-1, to=2-2]
		\arrow[from=1-3, to=2-3]
		\arrow[from=1-2, to=1-3]
		\arrow[from=2-2, to=2-3]
		\arrow[from=1-1, to=2-1]
		\arrow[from=1-2, to=2-2]
	\end{tikzcd}
	\]
	which induces a ladder of long exact sequences of homotopy groups. For each $k\geq n$, we have a commutative diagram
	\[
	\begin{tikzcd}
		\pi_{k}(\tau_{\geq n}X)\otimes \BZ[\frac{1}{p}] & \pi_{k}(X) \otimes \BZ[\frac{1}{p}] \\
		\pi_{k}(\tau_{\geq n}L_{m_p}X)\otimes \BZ[\frac{1}{p}]  & \pi_{k}(L_{m_p}X)\otimes \BZ[\frac{1}{p}] 
		\arrow[from=1-1, to=1-2]
		\arrow[from=2-1, to=2-2]
		\arrow[from=1-1, to=2-1]
		\arrow[from=1-2, to=2-2],
	\end{tikzcd}
	\]
	where the horizontal maps are isomorphisms and the right vertical map is also an isomorphism by Lemma \ref{L-equivalence implies Z[1/p]-iso}, hence the left map is also an isomorphism.
	The corollary then follows from Lemma \ref{L-equivalence is equivalent to pi_* iso on n-conn spectra}.
\end{proof}


We can now compute the homotopy groups of the $L_{m_p}$-localization on a spectrum.

\begin{corollary}
	The homotopy groups of $L_{m_p}X$ is given by
	\[
	\pi_* L_{m_p}X = \begin{cases}
		\pi_*X & \text{if $*< n$}\\
		\pi_*X\otimes \BZ[\frac{1}{p}] & \text{if $*\geq n$}.
	\end{cases}
	\]
\end{corollary}
\begin{proof}
	Since $m_p$ is $n$-connective, the map $X \to L_{m_p}X$ is also $n$-connective by Proposition \ref{connectivity of local equivalences}. Since $L_{m_p}$ commutes with taking $n$-connective cover by Corollary \ref{L-localization commutes with connective cover}.
	 Let's now assume $X$ is $n$-connective. We claim that $L_{m_p}X\simeq L_{\BZ[\frac{1}{p}]}X$, that is, $L_{m_p}X$ is the localization of $X$ with respect to the homology theory $H_{*}(-,\BZ[\frac{1}{p}])$.
	
	Indeed, the homotopy groups of $L_{m_{p}}X$ is uniquely $p$-divisible, so it's $L_{\BZ[\frac{1}{p}]}$-local and we have a unique factorization
	\[
	\begin{tikzcd}
		X &   & L_{m_p}X\\
		  & L_{\BZ[\frac{1}{p}]}X &
	\arrow[from=1-1, to=1-3]
	\arrow[from=1-1, to=2-2]
	\arrow[from=2-2, to=1-3, dashed].
	\end{tikzcd}
	\]
By Lemma \ref{L-equivalence is equivalent to pi_* iso on n-conn spectra}, the horizontal map is a  $L_{\BZ[\frac{1}{p}]}X$-equivalence
	Hence the dashed diagonal map is also a $L_{\BZ[\frac{1}{p}]}$-equivalence and we have $L_{m_p}X\simeq L_{\BZ[\frac{1}{p}]}X$, which implies $\pi_*L_{m_p}X=\pi_*X\otimes \BZ[\frac{1}{p}]$ for $*\geq n$.
\end{proof}
 \begin{lemma}
	A map $f:X \to Y$ of spectra is a $L_{m_p}$-equivalence if and only if 
	\begin{enumerate}
	 	\item $f$ is $n$-connective.
		\item $f$ induces isomorphism on 
	$$
	\pi_{n+k}X\otimes \BZ[\frac{1}{p}]\to \pi_{n+k}Y\otimes \BZ[\frac{1}{p}]
	$$
	for $k\geq 0$.

	\end{enumerate}
	\end{lemma}
\begin{proof}
Note that $f$ is a $L_{m_p}$-equivalence if and only if  
$$
L_{m_p}X \to L_{m_p}Y
$$
is an equivalence. By our computation of the homotopy groups, this is equivalent to requiring that $f$ is $n$-connective and induces isomorphism on 
	$
	\pi_{n+k}X\otimes \BZ[\frac{1}{p}]\to \pi_{n+k}Y\otimes \BZ[\frac{1}{p}]
	$
	for $k\geq 0$.
\end{proof}

To obtain tame localization, we only need to assemble the multiplication maps for different primes.
We define a map between wedges of sphere spectra
\begin{equation}
\label{f}
	f: \bigvee_{p\in P}\BS^{s+2p-3} \to \bigvee_{p\in P}\BS^{s+2p-3}
\end{equation}
which can be seen as a diagonal matrix with primes on the diagonal. Combining with Lemma \ref{local w.r.t. multiplication-by-p map}, this allows us to give the following equivalent characterization of tame spectra.
\begin{corollary}
	A spectrum $X\in \Sp^{\geq s}$ is $s$-tame if and only if it is $f$-local.
\end{corollary}

\begin{definition}
	A map of spectra $g:X\to Y$ in $\Sp^{\geq s}$ is a \emph{$s$-tame equivalence} if for any $s$-tame spectrum $Z$, the induced map
	\[
	\map_*(Y,Z) \xrightarrow{g^*} \map_*(X,Z)
	\]
	is an equivalence.
\end{definition}



\begin{proposition}
	A map of spectra $g:X\to Y$ in $\Sp^{\geq s}$ is a $s$-tame equivalence if and only if it induces isomorphisms
	\[
	\pi_{s+j}(X)\otimes R_j \to \pi_{s+j}(Y) \otimes R_j
	\]
	for $j\geq 0$.
\end{proposition}




%
%
%The theory of associative monoid objects in a symmetric monoidal $\infty$-category is developed in great generality in \cite[chapter 4.1]{HA}. 
%An \textit{ad hoc} way to define the composition product in $\SSeq(\CC)$ (in the one-object case) is explained in  Brantner's thesis  \cite[Section 4.1.2.]{BrantnerPhD}. The general approach was elaborated by Haugseng \cite{Haugsengsymseq}. 
%
%

%
%
%
%
%
%
%
%
%
%
%\subsection{Other coalgebras}
%There are three more different notions of coalgebras over a cooperad $\CP$, see section 3.5 of \cite{Francis-Gaitsgory}.
%
%\todo[inline]{More background to be put in this section.}
%
%\subsection{Gaitsgory's conjectures}
%
%\begin{conjecture}
%	The restriction functor 
%	$$
%	\resstriction: \coAlg_{\Com^{\vee}}^{nil,dp}(\CC) \rightarrow  \coAlg_{\Com^{\vee}}^{dp}(\CC)
%	$$
%	is fully faithful.
%\end{conjecture}
%
%\todo{The proof will be in an upcoming paper by Heuts. Now I will also give a sketch here.}
%
%Let $\Opd_{\infty}$ denote the $\infty$-category of $\infty$-operads. We let $(\Opd_{\infty})_{\leq n}$ be the full subcategory spanned by those $\infty$-operads for which $\CO(k)\simeq 0$ for $k>n$.
% Since the inclusion $(\Opd_{\infty})_{\leq n}\hookrightarrow \Opd_{\infty}$ preserves limits and filtered colimits, there is a localization functor 
% $$
% \tau_{n}: \Opd_{\infty}\rightarrow (\Opd_{\infty})_{\leq n}.
% $$
% We will refer to $\tau_n \CO$ the \textit{$n$-truncation} of $\CO$. Informally, we can think of $\tau_n \CO$ as an $\infty$-operad with 
% \[
% \tau_n \CO (k)= \left.
%  \begin{cases}
%    \CO(k), & \text{for } k \leq n \\
%   0. & \text{for } k > n          \\
%  \end{cases}
%  \right.
%\] 
%
%The unit map $\CO\rightarrow \tau_n\CO$ then induces a pair of adjoint functors: 
%\[
%\begin{tikzcd}
%	\Alg_{\CO} \arrow[r,shift left=1,"\tau_n"] & \Alg_{\tau_n\CO(\CC)}(\CC) \arrow[l,shift left=1]
%\end{tikzcd}
%\]
%where the left adjoint is given by $X\mapsto \tau_n\CO\circ_{\CO}X$. We denote the unit of this adjunction by $t_n$.
%\begin{definition}
%	We say an $\infty$-operad $\CO$ in $\CC$ has \emph{good completion} if for every object $X\in \CC$, there is an equivalence
%	\[
%	\trivial_{\CO}X\simeq \lim t_n(\trivial_{\CO} X).
%	\]
%\end{definition}
%
%\begin{proposition}
%	If $\CO$ has good completion in $\CC$, then the restriction functor
%	$$
%	\resstriction: \coAlg_{\Com^{\vee}}^{nil,dp}(\CC) \rightarrow  \coAlg_{\Com^{\vee}}^{dp}(\CC)
%	$$
%	is fully faithful with essential image the full subcategory generated by colimits of trivial coalgebras.
%\end{proposition}
%
%
%
%Our goal in this section is then to endow $\coAlg_{\Com^{\vee}}^{nil,dp}(\CC)$ with a Cartesian symmetric monoidal structure. Recall that the functor induced from the norm map:
%$$
%\norm:\coAlg_{\Com^{\vee}}^{dp}(\CC) 
%\rightarrow
%\coAlg_{\Com^{\vee}}(\CC).
%$$
%
%
%
%Let $\coCAlg(\CC):=\CAlg(\CC^{op})^{op}$ denote the $\infty$-category of cocommutative coalgebra objects in $\CC$.
%\begin{lemma}[Heuts]
%	There is an equivalence of $\infty$-categories:
%	$$
%	\coAlg_{\Com^{\vee}}(\CC)\simeq \coCAlg(\CC)
%	$$
%\end{lemma}
%
%\begin{corollary}
%	Under the assumption that $\CC$ is $1$-semiadditive, there is a fully faithful embedding $\coAlg_{\Com^{\vee}}^{nil,dp}\hookrightarrow \coCAlg(\CC)$. Moreover, $\coAlg_{\Com^{\vee}}^{nil,dp}$ is a Cartesian symmetric monoidal $\infty$-category.
%\end{corollary}
%\section{The Higher Enveloping Algebra Functor}
% Knudsen \cite{KnudsenHEA} introduced a higher enveloping algebra functor in the $\infty$-categorical setting
%$$
%U_n:\Alg_{\Lie}(\CC)\rightarrow \Alg_{\E_n}(\CC) 
%$$
%where $\Alg_{\Lie}(\CC)$ is the $\infty$-category of algebras over the $\Lie$ operad in $\CC$ and $\Alg_{\E_n}(\CC)$ is the $\infty$-category of $\E_n$-algebras in $\CC$. 
%The main work lies in the construction of a forgetful functor $\Alg_{\E_n}(\CC)\rightarrow \Alg_{\Lie}(\CC)$; the higher enveloping algebra functor $U_n$ is then defined as the left adjoint of the forgetful functor. Moreover, there is a Poincare-Birkhoff-Witt theorem:
%$$
%U_n(L)\oplus 1_{\CC}\simeq B_{\Lie}(\Omega^nL).
%$$
%
%Motivated by the PBW theorem, we give a direct definition of the higher enveloping algebra functor $U_n$. We first show the Lie algebras inherits a Cartesian symmetric monoidal structure from its underlying category.
%
%\begin{proposition}
%	The $\infty$-category $\Alg_{\Lie}(\CC)$ of Lie algebras in $\CC$  admits a Cartesian symmetric monoidal structure. Moreover, the categorical product in $\Alg_{\Lie}(\CC)$ is computed in the underlying $\infty$-category $\CC$.
%\end{proposition}
%\begin{proof}
%	Since the $\infty$-category $\CC$ admits finite products, the $\infty$-category $\Alg_{\Lie}(\CC)$ of Lie algebras in $\CC$ also admits finite products and they are computed in $\CC$ by \cite{HA} proposition 3.2.2.1.
%Let $\Alg_{\Lie}(\CC)^{\times}$ be the simplicial set obtained by applying \cite{HA} construction 2.4.1.4 to $\Alg_{\Lie}(\CC)$. As $\Alg_{\Lie}(\CC)$ admits finite products, it admits a Cartesian symmetric monoidal structure by \cite{HA} proposition 2.4.1.5.
%\end{proof}
%
%\begin{remark}
%	We let $\Alg_{\Lie}(\CC)^{\times}$ denote the Cartesian symmetric monoidal $\infty$-category on $\Alg_{\Lie}(\CC)$.
%\end{remark}
%
%Since limits in $\Alg_{\Lie}(\CC)$ are computed in the underlying $\infty$-category $\CC$, it follows that $\Alg_{\Lie}(\CC)$ also admits totalizations of cosimplicial objects. 
%
%
%
%
%
%
%
%
%By Corollary \ref{n-fold loop objects are E_n-algebras}, the $n$-fold loop functor induces a functor
%$$
%\Alg_{\Lie}(\CC)\xrightarrow{\Omega^{n}} \Alg_{\E_n}(\Alg_{\Lie}(\CC)).
%$$
%Post-compose with the functor $B_{\Lie}$, we then obtain the following composite
%\begin{align*}
%	\Alg_{\Lie}(\CC) & \xrightarrow{\Omega^n}
%\Alg_{\E_n}\big(\Alg_{\Lie}(\CC)\big) \subseteq \Fun\big(\E_n^{\otimes},\Alg_{\Lie}(\CC)^{\times}\big)\\
%&\xrightarrow{(B_{\Lie})_{*}}
%\Fun\big(\E_n^{\otimes},\coAlg_{\Com^{\vee}}^{nil,dp}(\CC)^{\times}\big).
%\end{align*}
%where we let $\coAlg_{\Com^{\vee}}^{nil,dp}(\CC)^{\times}$ denote the Cartesian symmtric monoidal structure on $\coAlg_{\Com^{\vee}}^{nil,dp}(\CC)$.
%
%\begin{lemma}
%	Let $\CC^{\times}$ and $\CD^{\times}$ be Cartesian symmetric monoidal $\infty$-categories. Any functor $F:\CC\rightarrow \CD$ can be lifted to a oplax monoidal functor $F^{\times}:\CC^{\times}\rightarrow \CD^{\times}$.
%\end{lemma}
%\begin{proof}
%	\todo{to do}
%\end{proof}
%
%Our next goal is to show that the induced functor $B_{\Lie}^{\times}:\Alg_{\Lie}(\CC)^{\times}\rightarrow \coAlg_{\Com^{\vee}}^{nil,dp}(\CC)^{\times}$ is symmetric monoidal.
%
%\begin{proposition}
%	The functor $B_{\Lie}^{\times}:\Alg_{\Lie}(\CC)^{\times}\rightarrow \coAlg_{\Com^{\vee}}^{nil,dp}(\CC)^{\times}$ is symmetric monoidal.
%\end{proposition}
%\begin{proof}
%	\todo{todo}
%\end{proof}
%
%
%
%
%\begin{definition}
%	We define the higher enveloping algebra functor $U_n$ as 
%	\begin{align*}
%		U_n:\Alg_{\Lie}(\CC) & \rightarrow \Alg_{\E_n}(\CC)\\
%		L                    & \mapsto \oblv^{nil,dp}_{\Com^{\vee}}\circ B_{\Lie}(\Omega^n L).
%	\end{align*}
%
%\end{definition}
%
%\begin{remark}
%The higher enveloping algebra functor defined here is equivalent to the one in \cite{KnudsenHEA} by the PBW theorem.	
%\end{remark}
%
%
%\begin{definition}
%	Let $\CD$ be an $\infty$-category. An $\E_n$-Hopf algebra object in $\CD$ is a grouplike $\E_n$-algebra object in $\coCAlg(\CC)$. We let $\Hopfalgebra_{\E_n}(\CC)$ denote the $\infty$-category of $\E_n$-Hopf algebra objects in $\CD$.
%\end{definition}
%
%\begin{corollary}
%We have a commutative diagram:
%\[
%\begin{tikzcd}
%	\Alg_{\Lie}(\CC) \arrow[rr,"U_n"] \arrow[dr,"B_{\Lie}\circ \Omega^n"]& &\Alg_{\E_n}(\CC)\\
%	& \Hopfalgebra_{\E_n}(\CC) \arrow[ur] &
%\end{tikzcd}
%\]
%In other words,	the universal enveloping algebra functor $U_n$ factors through the $\infty$-category of $E_n$-Hopf algebras in $\CC$. 
%\end{corollary}
%
%\begin{remark}
%	More precisely, $U_n$ factors through the $\infty$-category of conilpotent cocommutative $\E_n$-Hopf algebras. 
%\end{remark}
%
%
%\begin{conjecture}
%The functor $B_{\Lie}\circ \Omega:\Alg_{\Lie}(\CC) \rightarrow \Hopfalgebra_{\E_1}(\CC)$ is an equivalence of $\infty$-categories.
%\end{conjecture}
%
%\section{Appendix}
%
%In this section, we will provide some background, and prove some results of higher algebra which we used in this paper.
%\subsection{Backgroud on higher algebra}
%Our first goal is to show under certain assumptions of the underlying $\infty$-category $\CD$, the loop functor $\Omega$ sends an arbitrary object $X$ to an monoid object in $\CD$. Then by Dunn's additivity theorem, this implies that the $k$-fold loop object $\Omega^k X$ is an $\E_k$-monoid object in $\CD$. Then we will further show that the loop functor lands in the full subcategor of grouplike associative monoids.
%\begin{definition}
%	A simplicial object in an $\CD$ is a functor of $\infty$-categories
%$$
%U_{\bullet}: \nerve (\Delta)^{op}\rightarrow \CD.
%$$
%An augmented simplicial object in an $\infty$-category $\CD$ is a functor of $\infty$-categories 
%$$
%U_{\bullet}^{+}: \nerve (\Delta_{+})^{op}\rightarrow \CD.
%$$
%We let $\CD_{\Delta}:=\Fun(\nerve (\Delta)^{op}, \CD)$, resp. $\CD_{\Delta_+}:=\Fun(\nerve (\Delta_+)^{op}, \CD)$, denote the $\infty$-category of (resp. augmented ) simplicial objects in $\CD$. 
%\end{definition}
%
%\begin{definition}
%	[Definition 4.1.2.5. \cite{HA}]
%	Let $\CD$ be an $\infty$-category. A \textit{monoid object} of $\CC$ is a simplicial object $X:\nerve (\Delta)^{op}\rightarrow \CD$ with the property that, for each $n\geq 0$, the collection of face maps induced from the inclusions $[1]\simeq\{i-1,i\} \hookrightarrow [n]$, exhibits $X_n$ as a $n$-fold product of $X_1$.
%	We let $\Mon(\CD)$ be the full subcategory of $\Fun(\nerve(\Delta)^{op},\CD)$ spanned by monoid objects of $\CD$.
%\end{definition}
%
%Let $\CD$ be a pointed $\infty$-category which admits pullbacks. 
%Recall the loop object $\Omega Y$ of an object $Y\in \CD$ is defined as:
%$$
%\Omega Y:=0 \underset { Y }{\times } 0.
%$$
%There is "concatenation" map
%$$
%0 \underset { Y }{\times } 0 \underset { Y }{\times } 0
%\simeq\Omega Y\times \Omega Y\rightarrow \Omega Y\simeq 0\underset{Y}{\times}Y\underset{Y}{\times} 0 
%$$
%which is unique up to a contractible ambiguity. We can define a simplicial object $(Y^{S^1})_{\bullet}$ in $\CD$ as $(Y^{S^1})_{n}:=(\Omega Y)^{\times n}$, for which inner face maps are given by concatenation map, outer face maps are given by projection to zero object, and degeneracy maps are given by insertion of zero object. Therefore, we obtain the following proposition:
%\begin{proposition}
%	Let $\CD$ be a pointed $\infty$-category which admits pullbacks. For any object $X\in \CD$, the simplicial object $(X^{S^1})_{\bullet}$ is a monoid object of $\CD$.
%\end{proposition}
%
%\begin{remark}
%	We will abuse notation by writing the simplicial object $(X^{S^1})_{\bullet}$ as $\Omega X$; note this is not harmful at all, as the value of $X^{S^1}$ on $\Delta^{op}$ entirely depends on objects of $(X^{S^1})_{1}\simeq \Omega X$.
%\end{remark}
%
%\begin{proposition}
%	 \cite[Proposition~4.1.2.10.]{HA}
%	For every $\infty$-category $\CD$ which admits finite products, there is an equivalence of $\infty$-categories $\theta:\Mon_{\Ass}(\CD)\rightarrow \Mon(\CD)$.
%\end{proposition}
%
%Moreover, since there is an equivalence of $\infty$-operads
%\todo{references to be added from higher algebra}
%$$
%\E_1^{\otimes} \simeq \Ass^{\otimes},
%$$
%we conclude that $\Mon_{\E_1}(\CD)\simeq \Mon_{\Ass}(\CD)\simeq \Mon(\CD)$. Therefore, we can view $\Omega Y$ as an $\E_1$-monoid in $\CD$.
%
%\begin{corollary}
%	Let $\CD$ be a pointed $\infty$-category which admits pullbacks. For any object $X\in \CD$, the loop space object $\Omega X$ is an $\E_1$-monoid object of $\CD$.
%\end{corollary}
%
%The following corollary uses Dunn's additivity theorem \cite[Proposition 5.1.2.2.]{HA}.
%\begin{corollary}
%\label{n-fold loop objects are E_n-algebras}
%	Let $\CD$ be a pointed Cartesian symmetric monoidal $\infty$-category which admits finite limits. For any object $X\in \CD$, the $n$-fold loop object $\Omega^n X$ is an $\E_n$-algebra object in $\CD$.
%\end{corollary}
%
%\begin{proof}
%	Since $\CD^{\otimes}$ is a Cartesian symmetric monoidal $\infty$-category, we have an equivalence 
%	$\Alg_{E_{1}}(\CD)\rightarrow \Mon_{\E_1}(\CD)$
%	by \cite[Proposition 2.4.2.5.]{HA}. Therefore, we can identify $\Omega X$ as an $\E_1$-algebra of $\CD$. The additivity theorem (Theorem 5.1.2.2. \cite{HA}) then allows us to iterate the process
%	$$
%	\Alg_{\E_{k+1}}(\CD)\simeq \Alg_{\E_1}(\Alg_{\E_k}(\CD)),
%	$$
%	which induces an equivalence of $\infty$-categories
%	$$
%	\Alg_{\E_n}(\CD)\simeq \Alg_{\E_1}(\Alg_{E_1}(\cdots (\Alg_{E_1}(\CD))\cdots)
%	$$, hence $\Omega^n X$ is an $\E_n$-algebra object in $\CD$.
%\end{proof}
%
%Next, we state the definition of grouplike monoid objects as stated in \cite{HA} and show $\Omega X$ is a grouplike $\E_1$-monoid.
%
%\begin{definition}
%\cite[Definition 5.2.6.2.]{HA}
%	Let $\CD$ be an $\infty$-category which admits finite products and let $G$ be an associative monoid object of $\CC$. Let $m:G\times G\rightarrow G$ be the multiplication maps, and let $p_1,p_2:G\times G\rightarrow G$ be the projection maps onto the first and second factors, respectively. We will say that $G$ is \textit{grouplike} if the maps
%	$$
%	(p_1,m):G\times G\rightarrow G
%	$$
%	$$
%	(m,p_2):G\times G\rightarrow G
%	$$
%	are isomorphisms. 
%\end{definition}
%
%\begin{definition}
%	\cite[Definition 6.1.2.7.]{HTT}
%	A simplicial object $U_{\bullet}$ in $\CD$ is a \textit{groupoid object} if for every $n\geq 0$ and every partition $[n]=S\cup S'$ such that $S\cap S'=\{s\}$, the diagram
%	\[
%	\begin{tikzcd}
%		U([n]) \arrow[r] \arrow[d] & U(S) \arrow[d]\\
%		U(S') \arrow[r]    & U(\{s\}) 
%	\end{tikzcd}
%	\]
%	is a pullback square in the $\infty$-category $\CC$.
%\end{definition}
%
%\begin{lemma}
%	\cite[Remark 5.2.6.5.]{HA}
%	\label{Char of grouplike objects}
%	Let $\CD$ be an $\infty$-category with finite products. Let $G$ be a monoid object in $\CD$, then $G$ is a grouplike if and only if its underlying simplicial object is a groupoid object in $\CD$.
%\end{lemma}
%
%\begin{corollary}
%\label{loop objects are grouplike monoid objects}
%	Let $\CD$ be an $\infty$-category with finite limits. For any object $X\in \CD$, its loop object $\Omega X$ is a grouplike monoid object.
%\end{corollary}
%\begin{proof}
%	Since we have defined the underlying monoid object of the associative monoid $\Omega X$ as the simplicial object
%	$$
%	(X^{S^1})_n=(\Omega X)^{\times n},
%	$$
%	which is automatically a groupoid object. The claim then follows immediately by Lemma \ref{Char of grouplike objects}.
%\end{proof}
%
%\subsection{Symmetric Sequences}
%We recall the foundation of the theory of $\infty$-operads in $\CC$ in this section. 
%
%\begin{definition}
%	 The $\infty$-category of \textit{symmetric sequences} $\SSeq(\CS)$ in spaces is defined as the functor category 
%	 $$
%	 \Fun(\Fin^{\simeq},\Space),
%	 $$
%	 where $\Fin^{\simeq}$ denotes the (nerve of) the category of finite sets with bijections.
%	 
%	 More generally, let $\CD$ be an arbitrary symmetric monoidal $\infty$-category. We define the $\infty$-category $\SSeq(\CD)$ of \textit{symmetric sequences in $\CD$} as the functor category $\Fun(\Fin^{\simeq},\CD)$.
%\end{definition}
%
%
%
%
%\begin{comment}
%% It turns out I don't need these notions...
%\begin{definition}
%	\cite[Definition 7.2.2.1]{HTT}
%	Let $\CD$ be an $\infty$-category which admits a final object.
%	A \textit{group object} of $\CD$ is a groupoid object $U_{\bullet}:\nerve\Delta^{op}\rightarrow \CD$ for which $U_0$ is a final object of $\CD$.
%\end{definition}
%
%
%
%\begin{definition}
%	An augmented simplicial object $U^{+}_{\bullet}$ in $\CD$ is a \textit{\v{C}ech nerve} if $U^{+}_{\bullet}$ is a right Kan extension of 
%		$
%		U^{+}_{\bullet}|_{\nerve (\Delta^{\leq 0}_{+})^{op}}.
%		$
%\end{definition}
%
%\begin{proposition}
%	\cite[Proposition 6.1.2.11.]{HTT}
%	Let $\CD$ be an $\infty$-category and let $U^{+}_{\bullet}:\nerve(\Delta_{+})^{op}\rightarrow \CD$ be an augmented simplicial object of $\CD$. Then $U^{+}_{\bullet}$ is a \v{C}ech nerve	 if and only if 
%the underlying simplicial object $U_{\bullet}$ is a groupoid object of $\CD$ and the diagram 
%		\[
%		\begin{tikzcd}
%		U_{2} \arrow[r] \arrow[d] & U_{1} \arrow[d]\\
%		U_{1} \arrow[r]           & U_{0}
%		\end{tikzcd}
%		\]
%		is a pullback diagram in $\CD$.
%\end{proposition}
%\end{comment}
%
%
%
%
%\section{Appendix}
%\todo[inline]{Things below will be in appendix...}
%A functor bewteen $\infty$-topos $F:\CX\to \CX'$ is said to \emph{preserve connectivity} if for any $n$-conncective object $X\in \CX$, $F(X)$ is also $n$-connective.
%\todo[inline]{The lemma below is definitely true for any $\infty$-topos, but I don't think we need to do things that general here.}
%\begin{lemma}
%\label{products commute with products of Eilenberg-Maclane spaces}
%Let $F: \CS_*^{\geq 1} \to \CS_*^{\geq 1}$ be an endofunctor on the $\infty$-category of pointed spaces. Suppose $F$ preserves finite products and connectivity, and let $X$ be a space that is equivalent to a product of Eilenberg-Maclane spaces, i.e.
%$$
%X\simeq \prod_{i} K(\pi_i X, i),
%$$ then 
%$$F(X)\simeq \prod_i F(K(\pi_i X, i)).$$
%\end{lemma}
%\begin{proof}
%	Since $F$ preserves finite products, for a fixed $n$, we have 
%	$$
%	F(X) \simeq F\big (\prod_{i}^n K(\pi_i X, i)\big)\times  F\big(\prod_{i\geq n+1}K(\pi_i X, i) \big).
%	$$
%	Since $F$ preserves finite products and connectivity, if we take $n$-truncation on both side, we get
%	\begin{align*}
%	\tau_{\leq n}F(X) & \simeq \tau_{\leq n} F \big( \prod_{i}^n K(\pi_i X, i) \big)		\\
%					  & \simeq \tau_{\leq n} \prod_{i}^n F\big( K(\pi_i X, i) \big)\\
%					  & \simeq \tau_{\leq n} \prod_{i} F\big( K(\pi_i X, i) \big),
%	\end{align*}
%	which means that the canonical map $F(X)\to \prod_{i} F\big( K(\pi_i X, i) \big)$ is an $n$-equivalence. The lemma then follows from the convergence of Postnikov tower.
%	
%	\end{proof}
%	
%	
%	
%\subsection{Results from higher algebra}
%	Let $\adj{F}{\CC}{\CD}{G}$ be an adjunction between $\infty$-categories. 
%	Let $T:=GF$ denote the endomorphism monad induced by the adjunction and we let $\Alg_{T}(\CC)$ denote the $\infty$-category of $T$-modules in $\CD$.
%	Then $F$ factors as a commutative diagram of $\infty$-categories:
%	\[
%\begin{tikzcd}
%C \arrow[rr, shift left, "F"] \arrow[rd, shift left] &                                                 & D \arrow[ll, shift left, "G"] \arrow[ld, shift left,"G'"] \\
%                                                & \Alg_{T}(\CC) \arrow[lu, shift left] \arrow[ru, shift left] &                                                
%\end{tikzcd}
%	\] 
%	
%	
%	
%The following result can be extracted from the proof of \cite{HA}[Lemma 4.7.3.13.].
%\begin{lemma}
%	Suppose every $G$-split object in $\CD$ admits a colimit in $\CD$ and is preserved by $G$. Then the functor $G'$ admits a left adjoint $F'$. Moreover, $F'G'X$ can be resolved by the geometric realization of the simplicial object $(FG)^{\bullet+1}X$, i.e.
%	$$
%	F'G'X \simeq  \abs{(FG)^{\bullet+1}X}
%	$$
%	for any object $X\in \CD$. 
%\end{lemma}
%
%The following proposition can be easily deduced by dualizing \cite{HA}[Proposition 3.2.4.7.].
%\begin{proposition}
%	 Let $\CC$ be a symmetric monoidal $\infty$-category, then the symmetric monoidal structure on the category $\coCAlg(\CC)$ of commutative coalgebras is Cartesian.
%\end{proposition}
%
%Let $\CC$ be the $\infty$-category $\Sp$ of spectra in the previous proposition, then we can conclude that the Cartesian symmetric monoidal structure on $\coCAlg(\Sp)$ is given by the smash product $\otimes$ in $\Sp$.
%It is a well-known fact that the functor 
%$$
%\Sigma^{\infty}_+: \CS \to \Sp
%$$
%sends products in $\CS$ to smash product in $\Sp$.
%Hence, the factorization $C: \CS \to \coCAlg(\Sp)$ preserves finite products.
%In the case of tame spaces and tame coalgebras, we have the following lemma.
%
%\begin{lemma}
%	[Finite products lemma]
%	The functor 
%	$$
%	C_{tame}: \CS^{\geq r-1}_{\text{($r-1$)-tame}}  
%	\to 
%	\coCAlg(\Sp^{\geq r-1}_{\text{($r-1$)-tame}})
%	$$
%	preserves finite products.
%\end{lemma}
%\begin{proof}
%	Since the forgetful functor $\oblv:\coCAlg(\Sp^{\geq r-1}_{\text{($r-1$)-tame}}) 
%	\to
%	\Sp^{\geq r-1}_{\text{($r-1$)-tame}}$ is conservative, it suffices to show the composite $\Sigma^{\infty}_{tame}$ preserves finite products. Recall that $\Sigma^{\infty}_{tame}$ is by definition the composite $L_{tame}\circ \Sigma^{\infty}_{+}$. The lemma then follows from the fact that the symmetric monoidal product of $X, Y$ in $\Sp^{\geq r-1}_{\text{($r-1$)-tame}}$ is given by $L_{tame}(X\otimes Y)$.
%\end{proof}
%
%
%\begin{proposition}
%The unit of $RC_{tame}$ preserves connectivity and finite products. 
%\end{proposition}
%\begin{proof}
%	For any tame space $X\in \CS^{\geq r-1}_{\text{($r-1$)-tame}}$, we have 
%	$$
%	RC_{tame}X \simeq  \Tot (\Omega^{\infty}\Sigma^{\infty}_{tame})^{\bullet+1}X.
%	$$ 
%	The first part of the proposition now follows from ??, because both $\Sigma^{\infty}_{tame}$ and $\Omega^{\infty}$ preserve connectivity. 
%	Since $C_{tame}$ preserves finite products and $R$ is a right adjoint, their composition also preserves finite products.
%	\end{proof}
%
%
%
%\begin{proposition}
%	\cite{HA}[Proposition 2.4.2.5.]
%	 Let $\mathcal{C}^{\otimes}$ be a symmetric monoidal $\infty$-category, $\pi: \mathcal{C}^{\otimes} \rightarrow \mathcal{D}$ a Cartesian structure, and $\mathcal{O}^{\otimes}$ an $\infty$-operad. Then composition with $\pi$ induces an equivalence of $\infty$-categories $\operatorname{Alg}_{\mathcal{O}}(\mathcal{C}) \rightarrow \operatorname{Mon}_{\mathcal{O}}(\mathcal{D})$.
%\end{proposition}
%
%\begin{lemma}
%\cite{HA}[Lemma 3.2.2.6.]
%Let $p: \mathcal{C}^{\otimes} \rightarrow \mathcal{O}^{\otimes}$ be a fibration of $\infty$-operads, and let $\gamma: A \rightarrow A^{\prime}$ be a morphism in $\operatorname{Alg}_{/ \mathcal{O}}(\mathcal{C})$. The following conditions are equivalent:
%\begin{enumerate}[(1)]
%	\item The morphism $\gamma$ is an equivalence in $\operatorname{Alg}_{/ 0}(\mathcal{C})$.
%	\item For every object $X \in \mathcal{O}$, the morphism $\gamma(X): A(X) \rightarrow A^{\prime}(X)$ is an equivalence in $\mathcal{C}$.
%\end{enumerate}
%\end{lemma}
%
%
%
%
%\subsection{Background on H-spaces}
%Let $H$ be an associative $H$-space, then precomposing with the cross-product gives a multiplication, the \emph{Pontryagin product}, on the homology $H_*(X;R)$:
%$$
%H_*(X;R)\otimes H_*(X;R) \xrightarrow{\times} H_*(X\times X;R)\xrightarrow{\mu_*} H_*(X;R).
%$$
%Since the diagonal of $X$ induces a cocommutative multiplication on $H_*(X;R)$, hence $H_*(X;R)$ is a Hopf algebra. 
%Dually, the cohomology $H^*(X;R)$ of the H-space $X$ is a commutative Hopf algebra with the multiplication induced by diagonal and comultiplication induced by the product of H-spaces.
%
%
%
%
%\begin{lemma}
%	If the cohomology class $[\alpha]\in H^n(X;R)$ is a $k$-invariant that represents a map
%	$$
%	\alpha: X \to K(R,n)
%	$$
%	of H-spaces. Then $[\alpha]$ is a primitive element in the Hopf algebra $H^n(X;R)$.
%\end{lemma}
%
%
%
%\begin{proof}
%	Note the comultiplication $\psi:H^*(X;R)\to H^*(X;R)\otimes H^*(X;R)$ is induced by the multiplication $\mu: X\times X\to X$, hence $\psi([\alpha])$ is represented by the following composition:
%	\[
%	
%	\]
%\end{proof}
%


\subsection{Comonadicity}
The main theorem of this note is an analogue of Proposition \ref{Coalgebra model for simply-connected spaces} in the setting of tame spaces.
Before we give the proof of the main theorem, we make an important observation.
We claim that the following composite:
\[
\coAlg_{\Sigma^{\infty}\Omega^{\infty}}(\Sp^{\geq r})
\xrightarrow{\oblv}
\Sp^{\geq r}
\xrightarrow{L_{tame}}
\Sp^{\geq r}_{tame}
\]
factors through the $\infty$-category $\coAlg_{\Sigma^{\infty}_{tame}\Omega^{\infty}}(\Sp^{\geq r}_{tame})$ of $\Sigma^{\infty}_{tame}\Omega^{\infty}$-coalgebras in $\Sp^{\geq r}_{tame}$;
that is to say, the tame localization of a $\Sigma^{\infty}\Omega^\infty$-coalgebra admits a $\Sigma^{\infty}_{tame}\Omega^{\infty}$-coalgebra structure.
Indeed, 
for $X\in \Space_*^{\geq r}$, we have an equivalence $\Sigma^{\infty}_{tame}X \simeq \Sigma^{\infty}_{tame}L_{tame}X$, since $\Sigma^{\infty}_{tame}$ sends tame equivalences to equivalences of tame spectra. Then we obtain a commuting diagram
\[
\begin{tikzcd}
\Space^{\geq r}_{*} & & \Sp^{\geq r}\\
& \coAlg_{\Sigma^{\infty}\Omega^{\infty}}(\Sp^{\geq r})&\\
\Space^{\geq r}_{tame}&  & \Sp^{\geq r}_{tame}
\arrow[from=1-1,to=1-3, "\Sigma^{\infty}"]
\arrow[from=1-1,to=2-2, "\simeq"]
\arrow[from=2-2,to=1-3, "\oblv" below]
\arrow[from=1-1, to=3-1, "L_{tame}"]
\arrow[from=1-3, to=3-3, "L_{tame}"]
\arrow[from=3-1, to=3-3, "\Sigma_{tame}^{\infty}"]
\end{tikzcd}
\]
which implies $\Sigma^{\infty}_{tame}(L_{tame}X)\simeq  L_{tame}(\oblv\circ\phi(X))$. 
Hence we obtain a factorization
\[
\begin{tikzcd}
	\coAlg_{\Sigma^{\infty}\Omega^{\infty}}(\Sp^{\geq r}) &  & \Sp^{\geq r}_{tame}\\
	 & \coAlg_{\Sigma^{\infty}_{tame}\Omega^{\infty}}(\Sp_{tame}^{\geq r}) &
	 \arrow[from=1-1, to=1-3, "L_{tame}\circ \oblv"]
	 \arrow[from=1-1, to=2-2, "L'_{tame}"]
	 \arrow[from=2-2, to=1-3, "\oblv_{tame}"]	 
\end{tikzcd}
\]
where $\oblv_{tame}$ denotes the forgetful functor sending a $\Sigma^{\infty}_{tame}\Omega^{\infty}$-coalgebra to its underlying tame spectrum. 

\begin{lemma}
	\label{Lemma for the main theorem}
	Let $X\in \coAlg_{\Sigma^{\infty}\Omega^{\infty}}(\Sp^{\geq r})$. Then the induced map 	\[
	\Tot (\Omega^{\infty}\Sigma^{\infty})^{\bullet} \Omega^{\infty}X
	\to 
	\Tot (\Omega^{\infty}\Sigma^{\infty}_{tame})^{\bullet} \Omega^{\infty}L_{tame}X
	\]
	on the totalization of cosimplicial spaces $(\Omega^{\infty}\Sigma^{\infty})^{\bullet} \Omega^{\infty}X$ and $(\Omega^{\infty}\Sigma^{\infty}_{tame})^{\bullet} \Omega^{\infty}L_{tame}X$
	is a tame equivalence of spaces.
\end{lemma}
\begin{proof}
	Since tame equivalences is closed under limits\todo{not true!!!}, it suffices to show 
	\[
	(\Omega^{\infty}\Sigma^{\infty})^{n} \Omega^{\infty}X
	\to 
    (\Omega^{\infty}\Sigma^{\infty}_{tame})^{n} \Omega^{\infty}L_{tame}X
    \]
    is a tame equivalence for each $n\geq 0$.
    For $n=0$, this is obvious since $\Omega^{\infty}$ preserves tame equivalence. Suppose now 
    $(\Omega^{\infty}\Sigma^{\infty})^{n-1} \Omega^{\infty}X
	\to 
    (\Omega^{\infty}\Sigma^{\infty}_{tame})^{n-1} \Omega^{\infty}L_{tame}X$ is a tame equivalence;
    observe that $(\Omega^{\infty}\Sigma^{\infty})^{n} \Omega^{\infty}X
	\to 
    (\Omega^{\infty}\Sigma^{\infty}_{tame})^{n} \Omega^{\infty}L_{tame}X$ is also a tame equivalence as it is the composition of two tame equivalences
    $$ \Omega^{\infty}\Sigma^{\infty}(\Omega^{\infty}\Sigma^{\infty})^{n-1} \Omega^{\infty}X
	\to 
    \Omega^{\infty}\Sigma^{\infty}(\Omega^{\infty}\Sigma^{\infty}_{tame})^{n-1} \Omega^{\infty}L_{tame}X
    $$  
    and 
    $$\Omega^{\infty}\Sigma^{\infty}(\Omega^{\infty}\Sigma^{\infty}_{tame})^{n-1} \Omega^{\infty}L_{tame}X
    \to \Omega^{\infty}\Sigma^{\infty}_{tame}(\Omega^{\infty}\Sigma^{\infty}_{tame})^{n-1} \Omega^{\infty}L_{tame}X.$$
	\end{proof}

\begin{theorem}
	The adjoint pair $(\Sigma^{\infty}_{tame},\Omega^{\infty})$ is comonadic. That is, we have an equivalence of $\infty$-categories
	\[
	\Sigma^{\infty}_{tame}: \Space^{\geq r}_{tame} \to \coAlg_{\Sigma^{\infty}_{tame}\Omega^{\infty}}(\Sp^{\geq r}_{tame}).
	\]
\end{theorem}
\begin{proof}



%The functor $L'_{tame}$ preserves all colimits 
%$$
%L'_{tame}: \coAlg_{\Sigma^{\infty}\Omega^{\infty}}(\Sp^{\geq r}) \to \coAlg_{\Sigma^{\infty}_{tame}\Omega^{\infty}}(\Sp_{tame}^{\geq r})
%$$ 
%by $L'_{tame}$.


We let $\CC'$ denote the full subcategory of $\coAlg_{\Sigma^{\infty}\Omega^{\infty}}(\Sp^{\geq r})$ spanned by coalgebras whose underlying spectra satisfying:
\[
\Tot (\Omega^\infty \Sigma^{\infty})^{\bullet} \Omega^{\infty}X 
\text{ is a tame space.}
\tag{$\ast$}
\]
Note that a $\Sigma^{\infty}\Omega^{\infty}$-coalgebra $Y$ is in $\CC'$ if and only if it is in the esential image of $\Space^{\geq r}_{tame}$ via $\phi$, since $\phi^{-1}(Y)\simeq \Tot (\Omega^\infty \Sigma^{\infty})^{\bullet} \Omega^{\infty}Y $.
Hence, we have $\CC'\simeq \Space^{\geq r}_{tame}$.

We claim the functor $L'_{tame}: \coAlg_{\Sigma^{\infty}\Omega^{\infty}}(\Sp^{\geq r}) \to \coAlg_{\Sigma^{\infty}_{tame}\Omega^{\infty}}(\Sp_{tame}^{\geq r})$ restricts to an equivalence on $\CC'$.
%We first observe that $\CC'$ is equivalent to the $\infty$-category $\Space^{\geq r}_{tame}$ of tame spaces; note that $\phi$ is an equivalence and a $\Sigma^{\infty}\Omega^{\infty}$-coalgebra $\Sigma^{\infty}X$ is tame if and only if its corresponding space $\phi^{-1}(X)$ is tame.

%$$
%\map_{\Space^{\geq r}}(S^n,X)\simeq \map_{\coAlg_{\Sigma^{\infty}\Omega^\infty}}(\Sigma'^{\infty}S^n,\Sigma'^{\infty}X) \neq
%\map_{\Sp}(\BS^n,\Sigma'^{\infty}X)
%$$


%We next claim that $\coAlg_{\Sigma^{\infty}_{tame}\Omega^{\infty}}(\Sp_{tame}^{\geq r})$ is a full subcategory of $\coAlg_{\Sigma^{\infty}\Omega^{\infty}}(\Sp^{\geq r})$ and $L'_{tame}: \coAlg_{\Sigma^{\infty}\Omega^{\infty}}(\Sp^{\geq r}) \to \coAlg_{\Sigma^{\infty}_{tame}\Omega^{\infty}}(\Sp_{tame}^{\geq r})$ is the reflective localization functor at the class of maps whose underlying maps in $\Sp^{\geq r}$ are tame equivalences.

%Let $X\in \coAlg_{\Sigma^{\infty}\Omega^{\infty}}(\Sp^{\geq r})$ and $A\in \Sp^{\geq r}_{tame}$, then we have
%\[
%\begin{align*}
%	\map_{\coAlg_{\Sigma^{\infty}\Omega^{\infty}}(\Sp^{\geq r})}(X,\cofree(A))  & \simeq  \map_{\Sp^{\geq r}}(X,A) \\
%	& \simeq \map_{\Sp^{\geq r}_{tame}}(L_{tame}X,A) &\\
%	& \simeq \map_{\coAlg_{\Sigma^{\infty}_{tame}\Omega^{\infty}}(\Sp_{tame}^{\geq r})}(L'_{tame}X,\cofree_{tame}(A))\\
%	& \simeq \map_{\coAlg_{\Sigma^{\infty}_{tame}\Omega^{\infty}}(\Sp_{tame}^{\geq r})}(L'_{tame}X,L'_{tame}\cofree(A))
%\end{align*}
%\]
%where $\cofree_{tame}:\Sp^{\geq r}_{tame}\to \coAlg_{\Sigma^{\infty}_{tame}\Omega^{\infty}}$ is the right adjoint to the forgetful functor. 

Let $Y,Z\in \CC'$, we compute:
\[
\begin{aligned}
	\map_{\CC'}(Y,Z) & \simeq \map_{\coAlg_{\Sigma^{\infty}\Omega^{\infty}}(\Sp^{\geq r})}\big(Y,Z\big) \\
	& \simeq \map_{\coAlg_{\Sigma^{\infty}\Omega^{\infty}}(\Sp^{\geq r})}\big(Y,\Tot \cofree^{\bullet+1} Z\big)\\
	& \simeq \Tot \map_{\coAlg_{\Sigma^{\infty}\Omega^{\infty}}(\Sp^{\geq r})}\big(Y, \cofree^{\bullet+1} Z\big)\\
	& \simeq \Tot \map_{\Sp^{\geq r}}\big(Y, (\Sigma^{\infty}\Omega^{\infty})^{\bullet} Z\big)\\
	& \simeq  \map_{\Sp^{\geq r}}\big(Y, \Tot (\Sigma^{\infty}\Omega^{\infty})^{\bullet} Z\big)\\
	& \simeqexpl{$(\ast)$ and Lemma \ref{Lemma for the main theorem}}  \map_{\Sp^{\geq r}}\big(Y, \Tot (\Sigma^{\infty}_{tame}\Omega^{\infty})^{\bullet} L_{tame} Z\big)\\
	& \simeq  \map_{\Sp^{\geq r}_{tame}}\big(L_{tame}Y, \Tot (\Sigma^{\infty}_{tame}\Omega^{\infty})^{\bullet} L_{tame} Z\big)\\
	& \simeq \map_{\coAlg_{\Sigma^{\infty}_{tame}\Omega^{\infty}}(\Sp^{\geq r}_{tame})}\big(L'_{tame}Y,\Tot \cofree^{\bullet+1}_{tame} (L'_{tame}Z)\big)\\
	& \simeq \map_{\coAlg_{\Sigma^{\infty}_{tame}\Omega^{\infty}}(\Sp^{\geq r}_{tame})}\big(L'_{tame}Y,L'_{tame}Z \big)
\end{aligned}
\]
which implies that $L'_{tame}|_{\CC'}$ is fully faithful. It remains to show $L'_{tame}$ is essentially surjective. Let $X\in \coAlg_{\Sigma^{\infty}_{tame}\Omega^{\infty}} (\Sp_{tame}^{\geq r})$, we show that $X$ also admits a $\Sigma^{\infty}\Omega^{\infty}$-coalgebra structure. 
This follows from the fully faithfulness of $L'_{tame}|_{\CC'}$; observe that the structure map $X\to \Sigma^{\infty}_{tame}\Omega^{\infty}X$ can be viewed as a map $\gamma_X:X \to \cofree_{tame}(X)$ of $\Sigma^{\infty}_{tame}\Omega^{\infty}$-coalgebras, but this corresponds to a $\Sigma^{\infty}\Omega^{\infty}$-coalgebra map $X \to \cofree (X)$ up to a contractible choice of ambiguity. 
This concludes our proof $\Space^{\geq r}_{tame}\simeq \CC'\simeq \coAlg_{\Sigma^{\infty}_{tame}\Omega^{\infty}}(\Sp_{tame}^{\geq r})$.
\end{proof}

We now turn to the algebraic description of tame spaces. 
Let $D(\BZ)$ denote the derived $\infty$-category of the abelian category $\Ab$ of abelian groups, there is an equivalence of $\infty$-categories $D(\BZ)\simeq \Mod_{H\BZ}$ by Remark 7.1.1.6 of \cite{HA}. Note that the objects of $D(\BZ)$ are projective, bounded below chain complexes of abelian groups.
The $\infty$-category $(\Mod_{H\BZ})_{tame}$ of tame $H\BZ$-modules then can be identified with $D(\BZ)_{tame}^{\geq r}$, the full subcategory spanned by chain complexes $M_{*}$ satisfying the tame condition,
$$
M_{k}\simeq 0 \text{ for $k<r$ and } H_{r+j}(M_{*})\otimes R_{j}\cong H_{r+j}(M_{*}) \text{ for $j\geq 0$ }.
$$

A non-unital cocommutative coalgebra $C$ in $D(\BZ)$ is a chain complex $C$ equipped with maps $\gamma_n: C \to (C^{\otimes n})^{h\Sigma_n}$ for $n\geq 1$, such that it satisfies the usual coassociativity and cocommutativity conditions. 

\begin{remark}
	Note that $\Sp^{\geq r}_{tame}$ admits a symmetric monoidal structure with the tensor product $X\otimes_{tame} Y$ given by the tame localization $L_{tame}(X\otimes Y)$ of the smash product of $X$ and $Y$. Observe that the functor $L_{tame}:\Mod_{H\BZ}^{\geq r}\to (\Mod_{H\BZ}^{\geq r})_{tame}$ is symmetric monoidal; this follows from the fact that the map $M \otimes_{\BZ} N \to L_{tame}M\otimes_{\BZ} N \to L_{tame}M\otimes_{\BZ} L_{tame}N$ is a tame equivalence for any $M,N\in \Mod_{H\BZ}^{\geq r}$.
\end{remark}


%\begin{question}
%    I understand that the goal is to show $\Sigma^{\infty}_{tame}\Omega^{\infty}$ agrees with the cofree commutative comonad on $\Sp^{\geq r}_{tame}$ and we indeed have 
%    $$
%    \Sigma^{\infty}_{tame}\Omega^{\infty}X\simeq L_{tame}\big(\Sym^{\geq 1}(X) \big),
%    $$ since we invert primes fast enough to kill Tate construction; but I don't know how to show the cofree commutative coalgebra comonad is given by $L_{tame}\big(\Sym^{\geq 1}(X)\big)$.
%\end{question}
%
%\begin{lemma}
%	Let $X\in \Sp^{B\Sigma_n}$ be a $\Sigma_n$-spectrum whose underlying spectrum is uniquely $n!$-divisible, then 
%	$$
%	X^{t\Sigma_n}\simeq *.
%	$$
%\end{lemma}






\begin{proposition}
	Let $X\in \Sp^{\geq r}$, then the norm map	
	
	$$
	(X^{\otimes n})_{h\Sigma_n} \xrightarrow{Nm} (X^{\otimes n})^{h\Sigma_n} 
	$$
	admits a retract after tame localization. 
% 	THE FOLLOWING IS NOT KNOWN
%	Moreover, the tame localization of $(X^{\otimes n})^{t\Sigma_n}$ vanishes, i.e. 
%	$$
%	L_{tame}(X^{\otimes n})^{t\Sigma_n}\simeq *.
%	$$
\end{proposition}
\begin{proof}
	Note we have a commutative diagram
	\[
	\begin{tikzcd}
		(X^{\otimes n})_{h\Sigma_n} & (X^{\otimes n})^{h\Sigma_n}\\
		(L_{tame}X^{\otimes n})_{h\Sigma_n} & (L_{tame}X^{\otimes n})^{h\Sigma_n}
		\arrow[from=1-1, to=1-2, "Nm"]
		\arrow[from=1-1, to=2-1]
		\arrow[from=2-1, to=2-2, "\simeq"]
		\arrow[from=1-2, to=2-2]
	\end{tikzcd}
	\]
	where the left map is a tame equivalence and the bottom map is an equivalence by Lemma \ref{Tate vanishing for tame spectra}.
After tame localization we then have 	
\[
	\begin{tikzcd}
		L_{tame}(X^{\otimes n})_{h\Sigma_n} & L_{tame}(X^{\otimes n})^{h\Sigma_n}\\
		L_{tame}(L_{tame}X^{\otimes n})_{h\Sigma_n} & L_{tame}(L_{tame}X^{\otimes n})^{h\Sigma_n}
		\arrow[from=1-1, to=1-2, "Nm"]
		\arrow[from=1-1, to=2-1, "\simeq"]
		\arrow[from=2-1, to=2-2, "\simeq"]
		\arrow[from=1-2, to=2-2]
	\end{tikzcd}
\]
the retract is given by composite of the right map followed by the bottom and left maps.
%	Observe that 
%	$$
%	L_{tame}(X^{\otimes n})_{h\Sigma_n} \xrightarrow{L_{tame}Nm} L_{tame}(X^{\otimes n})^{h\Sigma_n} 
%	\to
%	L_{tame}(X^{\otimes n})^{t\Sigma_n}
%	$$
%	is a cofiber sequence in $\Sp^{\geq r}_{tame}$, hence 
%	$L_{tame}(X^{\otimes n})^{t\Sigma_n}$ is contractible.
\end{proof}

\clearpage
\begin{question}
	Let $G$ be a finite group and let $X\in \Fun(BG,\Sp)$. Do we have 
	$$
	L(X)^{hG}\simeq (LX)^{hG} \text{   ?}
	$$
	Note that we have a fiber sequence in $\Sp$
	$$
	(LX)_{hG} \to (LX)^{hG} \to (LX)^{tG}
	$$
	and a cofiber sequence in $\Sp_{tame}$
	$$
	L(X)_{hG} \to L(X)^{hG} \to L(X)^{tG}.
	$$
	Note that in the case I'm interested in, $(LX)^{tG}\simeq *$ and $L(X)_{hG}\simeq (LX)_{hG}$.
	So we have a cofiber sequence in $\Sp_{tame}$
	$$
	(LX)^{hG} \to L(X)^{hG} \to L(X)^{tG}
	$$
	So it suffices to show $(-)^{tG}$ preserves tame equivalences, but this cannot be ture:
	consider the 0-tame equivalence $\BS\to H\BZ$,
	then $(\BS^{\otimes 2})^{tC_2}\simeq \BS^{\wedge}_{2}\to  (L_{tame}(\BS^{\otimes 2}))^{tC_2}\simeq *$, but $\pi_0 \BS^{\wedge}_2\neq 0$.
\end{question}


%\begin{proposition}
%	The Goodwillie tower of $\Sigma^{\infty}\Omega^{\infty}$ splits after tame localization.
%\end{proposition}
%\begin{proof}
%
%	Note that we have a pullback diagram in $\Sp$
%	\[	
%	\begin{tikzcd}
%		P_n\Sigma^{\infty}\Omega^{\infty} (X) 
%		& (X^{\otimes n})^{h\Sigma_n}\\
%		P_{n-1}\Sigma^{\infty}\Omega^{\infty} (X)  
%		& (X^{\otimes n})^{t\Sigma_n}
%		\arrow[from=1-1, to=1-2]
%		\arrow[from=2-1, to=2-2]
%		\arrow[from=1-1, to=2-1]
%		\arrow[from=1-2, to=2-2].
%	\end{tikzcd}
%	\]
%	For $n=1$, we have $P_1\Sigma^{\infty}\Omega^{\infty} (X)\simeq D_{1}\Sigma^{\infty}\Omega^{\infty} (X)\simeq X$.
%	Suppose we have 
%	$$
%	L_{tame}P_{n-1}\Sigma^{\infty}\Omega^{\infty}(X)\simeq
%	\bigoplus_{i=1}^{n-1} L_{tame} (X^{\otimes n})_{h\Sigma_n}.
%	$$
%	
%\end{proof}


