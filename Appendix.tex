\chapter{Higher Algebra Preliminaries}

In this appendix, we recall some notions and results from \cite{HTT} and \cite{HA} that are used repeatedly in this thesis.


\section{Monoids and Groups in $\infty$-Categories}
Let $\CC$ be a pointed $\infty$-category with finite limits. 
We recall that a \emph{monoid object} in an $\infty$-category $\CC$ \cite[
Definition 4.1.2.5.]{HA} is a simplicial object $X: \Delta^{op}\to \CC$ satisfying the Segal condition:
the collection of face maps $X([n])\to X(\{i-1,i\})$ for $1\leq i \leq n$ exhibits $X([n])$ as a product of $\{X(\{i-1,i\})\}_{1\leq i \leq n}$. 
We will denote the full subcategory of monoids in $\CC$ by $\Mon(\CC)$.

Under the identification $\Mon(\CC)\simeq \Mon_{\Ass}(\CC)$ (see \cite[Proposition 4.1.2.10.]{HA}), a monoid $X$ is equipped with a (homotopy coherently) associative multiplication:
$$
m: X\times X\to X;
$$
we say that $X$ is a \emph{group} in $\CC$ \cite[Definition 5.2.6.2.]{HA} if both shearing maps
\begin{align*}
			(p_1, m):X\times X\to X\times X\\
	(m,p_2):X\times X\to X\times X
\end{align*}
are equivalences. 
We will denote the $\infty$-category of groups in $\CC$ as $\Grp(\CC)$.
\begin{definition}
	\cite[Definition 6.1.2.7.]{HTT}
	A simplicial object $U_{\bullet}$ in $\CD$ is a \textit{groupoid object} if for every $n\geq 0$ and every partition $[n]=S\cup S'$ such that $S\cap S'=\{s\}$, the diagram
	\[
	\begin{tikzcd}
		U([n]) \arrow[r] \arrow[d] & U(S) \arrow[d]\\
		U(S') \arrow[r]    & U(\{s\}) 
	\end{tikzcd}
	\]
	is a pullback square in the $\infty$-category $\CC$.
\end{definition}
\begin{remark}
By \cite[Remark 5.2.6.5.]{HA}, the notion of group in an $\infty$-category is equivalent to the definition of \emph{groupoid objects} in \cite[Definition 6.1.2.7.]{HTT}. For a groupoid object $X_{\bullet}$, we write $X_1$ for the corresponding group object.
\end{remark}


We now give another characterization of group objects in terms of presheaves on $\CC$.
Since the Yoneda embedding preserves limits, it induces a functor
$$
\Grp(\CC) \to  \Grp(\Fun(\CC^{op}, \Space)) \simeq \Fun(\CC^{op}, \Grp(\Space)).
$$
Hence we can identify group objects in $\CC$ as those representable presheaves that factor through $\Grp(\Space)$. 

The \emph{loop object} $\Omega Y$ of an object $Y\in \CC$ is defined as
$$
\Omega Y:=0 \underset { Y }{\times } 0.
$$
We want to show that $\Omega Y$ is a group object of $\CC$. 
\begin{lemma}
\label{Loop of an object is a group}
    Let $\CC$ be a pointed $\infty$-category with finite limits, then $\Omega Y$ is a group object for any $Y\in \CC$.
\end{lemma}
\begin{proof}
    The image of the Yoneda embedding of $\Omega Y$ is given by 
    $$
    X \mapsto \map_{\CC}(X, \Omega Y)\simeq \Omega \map_{\CC}(X, Y),
    $$
    where latter is in $\Grp(\Space)$ for any $X\in \CC$.
\end{proof}
\begin{remark}
In particular, $\Omega Y$ is a monoid in $\CC$. The multiplication map is given by the "concatenation" map
$$
0 \underset { Y }{\times } 0 \underset { Y }{\times } 0
\simeq\Omega Y\times \Omega Y\rightarrow \Omega Y\simeq 0\underset{Y}{\times}Y\underset{Y}{\times} 0 
$$
which is unique up to contractible ambiguity. 
\end{remark}


\begin{proposition}
\label{induced fully faithfulness on group objects}
Let $\CC$ and $\CD$ be $\infty$-categories with finite products. Suppose $F:\CC \hookrightarrow \CD$ is a fully faithful, product-preserving functor, then
	\begin{enumerate}[(1)]
		\item the functor $F$ induces a fully faithful embedding $\Mon(F):\Mon(\CC)\to \Mon(\CD)$ between the category of associative monoids.
		\item the functor $F$ induces a fully faithful embedding $\Grp(F):\Grp(\CC)\to \Grp(\CD)$ between the category of group objects.
	\end{enumerate}
\end{proposition}
\begin{proof}
	Observe first that the functor $F$ induces a fully faithful embedding 
	$$
	\Fun(\Delta^{op},\CC)\to \Fun(\Delta^{op},\CD)
	$$ between the $\infty$-categories of simplicial objects.
	Then $(1)$ follows from the fact that $F$ send monoids to monoids and $\Mon(\CC)$ is a full subcategory of $\Fun(\Delta^{op},\CC)$; and $(2)$ follows from the fact that $\Grp(\CC)$ is a full subcategory of $\Mon(\CC)$.
\end{proof}
%%%%







%%%%





\begin{proposition}
\label{Restrict adjoints to full subcategory}
	Let $\adj{F}{\CC}{\CD}{G}$ be a pair of adjoint functors between $\infty$-categories.
	Let $\CC^0$ be a full subcategory of $\CC$ so that the essential image of $G$ is contained in $\CC^0$. Then $F^0:=F|_{\CC^0}$ restricts to an adjoint pair 
	$$
	\adj{F^0}{\CC^0}{\CD}{G}.
	$$
\end{proposition}
\begin{proof}
    By Proposition 5.2.2.8 of \cite{HTT}, the pair of adjoint functors gives a unit natural transformation
    \[
    id_{\CC} \to G\circ F
    \]
    such that for every pair of objects $c\in \CC$ and $d\in \CD$ the composite
    \[
    \map_{\CD}(F(c),d) \to \map_\CC(G\circ F(c),G(d)) \to \map_{\CC}(c,G(d))
    \]
    is an equivalence .
	The assumptions then give a natural transformation $id_{\CC^{0}}\to G\circ F$.
	For every $c'\in \CC^0$ and $d\in \CD$, the composite 
	\[
	\map_{\CD}(F(c'),d) \to \map_{\CC^0}(G\circ F(c'),G(d)) \to \map_{\CC^{0}}(c',G(d))
	\] 
	is an equivalence, since one observes that
	$$
	\map_{\CC^0}(G\circ F(c'),G(d))\simeq \map_{\CC}(G\circ F(c'),G(d))
	$$
	and 
	$$
	\map_{\CC^0}(c',G(d))\simeq \map_{\CC}(G\circ F(c'),G(d)).
	$$
	\end{proof}

\begin{definition}
\cite[Definition 2.1]{Gepner-Moritz-Nikolaus}
\label{preadditive}
    An $\infty$-category is \emph{preadditive} if the canonical morphism 
    $$
    X\coprod Y \to X\times Y
    $$
    is an equivalence for any pair of objects $X,Y \in \CC$.
\end{definition}

\begin{example}
\begin{enumerate}
    \item Any stable $\infty$-category is preadditive.
    \item The $\infty$-category $\Sp^{\geq r}$ of $r$-connective spectra is preadditive;
    indeed, the coproduct of $X,Y\in \Sp^{r}$ is computed in $\Sp$ and the product of $X,Y\in \Sp^{r}$ is computed by $\tau_{\geq r} (X\oplus Y)\simeq X\oplus Y$, where $\oplus$ denotes the direct sum in $\Sp$.
    \item The $\infty$-category $\Sp^{\geq r}_{\tame}$ of $r$-tame spectra is preadditive; the product of $X,Y\in \Sp^{\geq r}_{\tame}$ is computed in $\Sp^{\geq r}$ and the coproduct is computed by $L_{\tame}(X\oplus Y) \simeq X\oplus Y$.
\end{enumerate}

\end{example}


Let $\CMon(\CC)$ \cite[Definition 2.4.2.1.]{HA} denote the $\infty$-category of commutative monoids in $\CC$.
\begin{proposition}
\label{Prop 2.3 GMN}
\cite[Proposition 2.3]{Gepner-Moritz-Nikolaus}
Let $\CC$ be an $\infty$-category with finite products and finite coproducts, 
    then the following are equivalent:
    \begin{enumerate}
        \item The $\infty$-category $\CC$ is preadditive.
        \item The homotopy category $h\CC$ is preadditive.
        \item The forgetful functor $\CMon(\CC) \to \CC$ is an equivalence.
    \end{enumerate}
\end{proposition}




If $M\in \CC$ is an object in a preadditive $\infty$-category $\CC$, then $M$ can be equipped with a commutative monoid structure by Proposition \ref{Prop 2.3 GMN}.
We will call the monoid structure map $\nabla:M\oplus M \to M$ the \emph{fold map}. 
The \emph{shearing map} defined as
$$
s: M\oplus M \xrightarrow{(pr_1, \nabla)} M\oplus M,
$$
where $pr_1$ denotes the projection to the first factor.

\begin{definition}
\cite[Definition 2.6]{Gepner-Moritz-Nikolaus}
    \label{additive}
    A preadditive $\infty$-category $\CC$ is \emph{additive} if the shearing map $s$ is an equivalence.
\end{definition}

\begin{proposition}
\label{Prop 2.8 GMN}
\cite[Proposition 2.8]{Gepner-Moritz-Nikolaus}
Let $\CC$ be an $\infty$-category with finite products and finite coproducts, 
    then the following are equivalent:
    \begin{enumerate}
      \item The $\infty$-category $\CC$ is additive.
      \item The homotopy category $h\CC$ is additive.
      \item The forgetful functor $\Grp(\CC)\to \CC$ is an equivalence.
    \end{enumerate}
\end{proposition}

Proposition \ref{Prop 2.3 GMN} and Proposition \ref{Prop 2.8 GMN} then imply that 
there are equivalences
$$
\Grp(\CC)\simeq\CMon(\CC) \simeq \CC
$$
if $\CC$ is additive. Moreover, if $\CC$ is stable, we have the following corollary regarding the $\infty$-category $\Alg_{\CO}(\CC)$ of $\CO$-algebras in $\CC$.
\begin{corollary}
    Let $\CC$ be a presentably stable $\infty$-category.
    Let $B_{\CO}(M)$ denote the bar construction of a monoid $M$ (with respect to the monoid structure) in $\Alg_{\CO}(\CC)$.
    Then the pair of adjoint functors
$$
\adj{B_{\CO}}{\Grp(\Alg_{\CO}(\CC))}{\Alg_{\CO}(\CC)}{\Omega_{\CO}}
$$
is an equivalence.
\end{corollary}
\label{Alg(C) equivalent to groups in Alg(C)}
\begin{proof}
    Since both of the functors $B_{\CO}$ and $\oblv_{\CO}$ preserves sifted colimits, the bar construction $B_{\CO}(M)$ of $M\in \Grp(\Alg_{\CO}(\CC))$ is computed in $\CC$.
    Since $\CC$ is stable, the bar complex $\Barconstruction(0, M, 0)_{\bullet}$ comes from the \v{C}ech nerve of $0 \to \Sigma M$, which implies that
    $B_{\CO}M \simeq \Sigma M$.
    Therefore we obtain two commuting diagrams
	\[
	\begin{tikzcd}
		\Grp(\Alg_{\CO}(\CC)) & \Alg_{\CO}(\CC) \\
		\CC &
		\CC ;
		\arrow[from=1-1, to=1-2, "B_{\CO}"]
		\arrow[from=1-1, to=2-1, "\oblv_{\CO}\circ \oblv_{\Grp}" left]
		\arrow[from=1-2, to=2-2, "\oblv_{\CO}"]
		\arrow[from=2-1, to=2-2, "\Sigma"]
	\end{tikzcd}
	\]
	and 
	\[
	\begin{tikzcd}
		\Grp(\Alg_{\CO}(\CC)) & \Alg_{\CO}(\CC) \\
		\CC &
		\CC.
		\arrow[from=1-2, to=1-1, "\Omega_{\CO}"]
		\arrow[from=1-1, to=2-1, "\oblv_{\CO}\circ \oblv_{\Grp}" left]
		\arrow[from=1-2, to=2-2, "\oblv_{\CO}"]
		\arrow[from=2-2, to=2-1, "\Omega"]
	\end{tikzcd}
	\]
	Since the vertical forgetful functors are conservative, there are equivalences
$$	    
\oblv_{\CO}\circ B_{\CO} \circ  \Omega_{\CO} (X)
	    \simeq \Sigma \Omega (\oblv_{\CO}X) \simeq \oblv_{\CO} X
$$
for any $X\in  \Alg_{\CO}(\CC)$,
and 
$$
\oblv_{\CO}\circ \oblv_{\Grp} \circ \Omega_{\CO} \circ B_{\CO} Y
\simeq 
\Omega \Sigma(\oblv_{\CO}\circ \oblv_{\Grp} Y)
\simeq
(\oblv_{\CO}\circ \oblv_{\Grp}Y)
$$
for any $Y\in\Grp(\Alg_{\CO}(\CC)) $.
\end{proof}




We consider now the $\infty$-category $\Sp^{\geq r}_{\tame}$ of $r$-tame spectra.
Let $\Sigma$ and $\Omega$ denote the suspension and loop functor in $\Sp$ respectively.
Note that the category $\Sp^{\geq r}_{\tame}$ is not closed under $\Sigma$, so it's not an inverse of $\Omega$.
However, it's easy to see we have the following lemma.
\begin{lemma}
\label{Relations between different tameness}
	The adjunction
	\[
	\adj{\Sigma}{\Sp^{\geq r-1}_{\tame}}{\Sp^{\geq r}_{\tame}}{\Omega}
	\]
	is an equivalence.
\end{lemma} 
Moreover, if we let $L_{r-1, \tame}$ and $L_{r, \tame}$ be $(r-1)$-localization and $r$-localization functors, respectively.
It's easy to see there are equivalences
$$
L_{r-1, \tame} \Omega X \simeq \Omega L_{r, \tame}X
$$
for an $r$-connective spectrum $X$, and 
$$
L_{r, \tame} \Sigma Y \simeq \Sigma L_{r-1, \tame} Y
$$
for an $(r-1)$-connective spectrum $Y$.


Since $\Alg_{\spLie}(\Sp^{\geq r}_{tame})$ is Cartesian symmetric monoidal, the loop functor 
$$
\Omega: \Sp^{\geq r}_{\tame}
\to 
\Sp^{\geq r-1}_{\tame}
$$
induces a functor
\[
\Omega_{\spLie}:\Alg_{\spLie}(\Sp^{\geq r}_{\tame}) \to \Grp( \Alg_{\spLie}(\Sp^{\geq r-1}_{\tame}) )
\]
which takes a $r$-tame Lie algebra $X$ to a $(r-1)$-tame Lie algebra $\Omega_{\spLie}X$ whose underlying spectrum is $\Omega X$.
\begin{proposition}
	\label{B and Omega are mutally inverses}
	The functor 
	\[
	\Omega_{\spLie}: 
	\Alg_{\spLie}(\Sp^{\geq r}_{\tame}) \to \Grp( \Alg_{\spLie}(\Sp^{\geq r-1}_{\tame}))
	\]
	is an equivalence.
\end{proposition}
\begin{proof}
	Let $B_{\spLie}$ denote the bar construction functor that is left adjoint to $\Omega_{\spLie}$. Since the forgetful functor 	
	$\Alg_{\spLie}(\Sp^{\geq r}_{\tame}) \to \Sp^{\geq r}_{\tame}$ preserves sifted colimits, the bar construction is computed in the underlying category $\Sp^{\geq r}_{\tame}$ of $r$-tame spectra.
	
	Let $X_{\bullet}$ be a groupoid object in $\Sp^{\geq r-1}_{\tame}$, the geometric realization of $X_{\bullet}$ is computed by first taking the geometric realization of $X_{\bullet}$ in $\Sp$, then applying the $r$-tame localization functor $L_{\tame}$, but $|X_{\bullet}|\simeq \Sigma X$ which is already $r$-tame, hence we have $B_{\spLie}X\simeq \Sigma X$.
	Therefore we have a commuting diagram
	\[
	\begin{tikzcd}
		\Grp( \Alg_{\spLie}(\Sp^{\geq r-1}_{\tame}) ) & \Alg_{\spLie}(\Sp^{\geq r}_{\tame})\\
		\Sp^{\geq r-1}_{\tame} &
		\Sp^{\geq r}_{\tame};
		\arrow[from=1-1, to=1-2, "B_{\spLie}", shift left]
		\arrow[from=1-2, to=1-1, "\Omega_{\spLie}", shift left]
		\arrow[from=1-1, to=2-1]
		\arrow[from=1-2, to=2-2]
		\arrow[from=2-1, to=2-2, "\Sigma", shift left]
		\arrow[from=2-2, to=2-1, "\Omega", shift left]
	\end{tikzcd}
	\]
	Hence $B_{\spLie}$ and $\Omega_{\spLie}$ are mutually inverses because $\Sigma$ and $\Omega$ are mutually inverses.
\end{proof}

%$$
%\Omega_{\Lie}:\Alg_{\Lie}(\Sp^{\geq r}_{\text{$r$-tame}}) \to \Grp(\Alg_{\Lie}(\Sp^{\geq r}_{\text{$r$-tame}}))
%$$
%factors through $\Grp(\Alg_{\Lie}(\Sp^{\geq r-1}_{\text{$(r-1)$-tame}}))$.
%Let 
%$$
%B_{\Lie}:\Grp(\Alg_{\Lie}(\Sp^{\geq r-1}_{\text{$(r-1)$-tame}})) \to
%\Alg_{\Lie}(\Sp^{\geq r-1}_{\text{$(r-1)$-tame}})
%$$
%be the bar construction functor.
%We claim the following
%\begin{lemma}
%	The composition
%	\[
%	U:
%	\Grp(\Alg_{\Lie}(\Sp^{\geq r-1}_{\text{$(r-1)$-tame}})) 
%	\xrightarrow{B_{\Lie}}
%\Alg_{\Lie}(\Sp^{\geq r-1}_{\text{$(r-1)$-tame}})
%\to 
%\Sp^{\geq r-1}_{\text{$(r-1)$-tame}}
%\to 
%\Sp
%\]
%can be identified with the suspension $\Sigma X$ of the underlying spectrum of $X\in \Grp(\Alg_{\Lie}(\Sp^{\geq r-1}_{\text{$(r-1)$-tame}}))$. Hence, $U$ factors over $\Sp^{\geq r}_{\text{$r$-tame}}$.
%\end{lemma}
%\begin{proof}
%	
%\end{proof}
%

\section{Filtered and Graded Objects in Infinity-Categories}
In this section, we introduce the notion of filtered and graded objects in a symmetric monoidal $\infty$-category $\CC$.
The main reference for this section is \cite{Brantner-Mathew}.
\begin{definition}
	Let $\BZ$ denote the poset of the integers and let $\CC$ be an $\infty$-category.
	The $\infty$-category of \emph{filtered objects in $\CC$} is defined as the functor category 
	$$
	\Fil(\CC) :=\Fun(\BZ, \CC).
	$$
\end{definition}

Let $\BZ^{\operatorname{disc}}$ denote the groupoid with objects the integers and isomorphisms.
\begin{definition}
	The $\infty$-category of \emph{graded objects in $\CC$} is defined as the functor category 
	$$
	\Graded(\CC):=\Fun(\BZ^{\operatorname{disc}}, \CC).
	$$
\end{definition}

\begin{remark}
	We can also extend the definitions above to the category $\Fil^{+}(\CC)$ (resp. $\Gr^{+}(\CC)$) of \emph{non-negatively filtered} (resp. \emph{non-negatively graded}) objects by restricting to the category $\BZ_{\geq 0}$ (resp. $\BZ^{\operatorname{disc}}_{\geq 0}$) of non-negative integers.
\end{remark}

The natural inclusion $ \BZ^{\operatorname{disc}}\hookrightarrow \BZ$ induces a forgetful functor
$
\Fil(\CC) \to \Gr(\CC). 
$
We denote by
$$
U: \Graded(\CC) \to \Fil(\CC)
$$ the left Kan extension along the inclusion $ \BZ^{\operatorname{disc}}\hookrightarrow \BZ$; explicitly, the filtered object $X_*$ evaluated at $n$ is given by $\bigoplus_{k\leq n} X_k$.

We can also define the \emph{associated graded} of a filtered object 
$$
\Graded(-):
\Fil(\CC)
\to 
\Graded(\CC)
$$
to be $X_* \mapsto (n \mapsto \cofib(X_{n-1} \to X_{n}))$.
The following lemma follows immediately from an inductive argument.
\begin{lemma}
\label{Ass-gr is conservative}
	The associated graded functor $$\Graded(-):
\Fil^{+}(\CC)
\to 
\Gr^{+}(\CC)
$$
is conservative.
\end{lemma}

We also have the following obvious observation.
\begin{lemma}
	The composite 
	\[
	\Graded(\CC) \xrightarrow{U} 
	\Fil(\CC)
	\xrightarrow{\Graded(-)} 
	\Graded(\CC) 
	\]
	is equivalent to the identity functor.
\end{lemma}

The following remark from \cite{Brantner-Mathew} equips the categories of both filtered objects and graded objects with symmetric monoidal structure.
\begin{remark}
\cite[Definition 2.5]{Brantner-Mathew}
\label{BM def 2.5}
Suppose that $\CC$ is (nonunital) presentably symmetric monoidal $\infty$-category in which tensor product commutes with colimits.
Using Day convolution, one can equip both $\Fil(\CC)$ and $\Graded(\CC)$ with the structure of presentably (nonunital) symmetric monoidal $\infty$-categories. Furthermore, the associated graded functor $\Graded : \Fil(\CC) \to \Graded(\CC)$ is (nonunital) symmetric monoidal (cf. \cite[Sec. 2.23]{Glasman}).
\end{remark}

Consider now an $\infty$-operad $\CO$ in $\CC$. We can now state a theorem in \cite{Heuts_Koszul} which says that every $\CO$-algebra admits a \emph{canonical filtration} so that its associated graded is free.
\begin{theorem}
\label{Canonical grading on an O-algebra}
\cite[Theorem 5.2 (2)]{Heuts_Koszul}
For an $\CO$-algebra $X$, there exists a canonical filtered object $X_{*}$ so that 
\begin{enumerate}
    \item The filtration is \emph{exhaustive}, i.e., there is an equivalence
    $$
    \colim X_{*} \to X.
    $$
    \item The filtered $\CO$-algebra has associated graded
    $$
    \Graded(X_*)\simeq \Free_{\CO}(B\CO(n)\otimes X^{\otimes n })_{h\Sigma_n}.
    $$
\end{enumerate}


\end{theorem}



% We let $A = \BS \oplus \BS$ and view it as an augmented filtered commutative algebra with $A_0=\BS$ and 
% $A_n=\BS \oplus \BS$ for $n\geq 1$.
% For any $\CO$-algebra $X$, we obtain a filtered $\CO$-algebra $(\iota_0X)_*$ where $\iota_0$ is the functor that inserts an object to a graded object concentrated in degree $0$.
% Consider now the functor
% \begin{align*}
% 	\CAlg(\CC^{\Fil, \geq 0}) \times \Alg_{\CO}(\CC) 
% 	& \to 
% 	\Alg_{\CO}(\CC^{\Fil, \geq 0})\\
% 	(A, X) & \mapsto A\otimes (\iota_0X)_*.
% \end{align*}
% Note that $A$ is augmented, hence we have a natural morphism 
% \[
% \epsilon: A\otimes (\iota_0X)_* 
% \to 
% (\iota_0X)_* .
% \]

% \begin{definition}
% 	The \emph{canonical filtration} is a functor 
% 	$$
% 	(-)^{\Fil}: \Alg_{\CO}(\CC)\to \Alg_{\CO}(\CC^{\Fil,\geq 0})
% 	$$ 
% 	given by
% 	$X\mapsto   \fib(A\otimes (\iota_0X)_* \to (\iota_0X)_*)$.
% \end{definition}

% \begin{proposition}
% 	\label{canonical filtration}
% 	The associated graded $\gr(X^{\Fil})$ of the canonical filtration on $X$ is the trivial graded $\CO$-algebra, i.e. 
% 	$$
% 	\gr(X^{\Fil}) \simeq \trivial^{\gr}_{\CO}(\iota_{1} \circ \oblv_{\CO}X).
% 	$$
% % Is the following even true?
% %\begin{enumerate}
% %	\item \todo{do we really need this?} the direct colimit of $X^{\Fil}$ is equivalent to $X$, i.e. $$\colim X^{\Fil} \simeq X.$$
% %\end{enumerate}
% \end{proposition}
% \begin{proof}
% %	We abuse notation by writing $(\iota_0X)_*\in \CC^{\Fil, \geq 0}$ as $X$ in the following.
% 	Note that $\gr(X^{\Fil})$ is a graded $\CO$-algebra with $X$ concentrated in degree $1$, hence it is a trivial $\CO$-algebra by degree reason.
	
	
	
	
% \end{proof}
	


%\begin{lemma}
%\label{Algebras lifted to filtered algebras}
%	The composite
%	$$
%	\Alg_{\CO}(\CC) \xrightarrow{(-)^{\Fil}}
%	\Alg_{\CO}(\CC^{\Fil,\geq 0})
%	\xrightarrow{\oblv_{\Fil}}
%	\Alg_{\CO}(\CC)
%	$$
%	is equivalent to the identity functor on $\Alg_{\CO}(\CC)$.
%\end{lemma}
%\begin{proof}
%	Since the forgetful functor $\oblv_{\CO}: \Alg_{\CO}(\CC)\to \CC$ is conservative, it suffices to show
%	$$
%	\oblv_{\CO}\circ \oblv_{\Fil}(X^{\Fil}) \simeq X,
%	$$
%	which follows from the equivalence $\fib(X\times X \to  X)\simeq X$. 
%\end{proof}






%One can also extend the definitions above to the category $\gr(\CC)$ of \emph{non-negatively graded} objects


\clearpage







\section{The Barr-Beck-Lurie Theorem}
\begin{theorem}
\label{Barr-Beck-Lurie theorem}
	 \cite[Theorem 4.7.3.5.]{HA}
	Let $\adj{F}{\CC}{\CD}{G}$ be an adjoint pair of $\infty$-categories. Then $G$ is monadic if and only if 
	\begin{enumerate}
		\item $G$ is conservative, and
		\item If $X_\bullet$ is a $G$-split simplicial object in $\CD$, then its geometric realization exists in $\CD$ and $G$ preserves geometric realization of $X_{\bullet}$.
	\end{enumerate}
\end{theorem}

In practice, the category $\CD$ often admits all geometric realizations for simplicial objects. In this case, we have a technically convenient criteria for determining monadicity of a functor. We learn the proof of the following corollary from Heuts.
\begin{corollary}[\cite{HeutsSurvey}]
\label{Cor of Barr-Beck-Lurie theorem}
If $\adj{F}{\CC}{\CD}{G}$ an adjoint pair and that $\CD$ admits colimits of $G$-split simplicial objects. Then this pair is monadic if and only if for every object $X$ of $\CD$, the map
$$
\abs{(FG)^{\bullet+1}X} \to X
$$
arising from the simplicial resolution described above is an equivalence.
\end{corollary}
\begin{proof}
	Suppose this pair is monadic, then it satisfies the conditions of Theorem \ref{Barr-Beck-Lurie theorem}. 
	For an object $X$ in $\CD$, the simplicial object $(FG)^{\bullet+1}X$ is $G$-split. Indeed, the simplicial object $G((FG)^{\bullet+1}X)$ admits a contracting homotopy via the unit natural transformation $X \to GF(X)$.
	Applying $G$ to the map 
	$$
	\abs{(FG)^{\bullet+1}X} \to X
	$$
	one obtains
	\begin{align*}
		G(\abs{(FG)^{\bullet+1}X}) & \simeq \abs{G(FG)^{\bullet+1}X}\\
		& \simeq G(X)
	\end{align*}
	where the first equivalence is due to the fact that $G$ preserves geometric realization of $G$-split objects and the second equivalence is due to the assumption that $(FG)^{\bullet+1}X$ is $G$-split.  
	Since $G$ is conservative, we conclude that $\abs{(FG)^{\bullet+1}X} \to X$ is an equivalence.
	
	Suppose now $\abs{(FG)^{\bullet+1}X} \to X$ is an equivalence for every $X$ in $\CD$. If $G(f): G(X) \to G(Y)$ is an equivalence in $\CC$ for some morphism $f:X \to Y$, then $(FG)^{\bullet+1} X \to (FG)^{\bullet+1}Y$ is an equivalence of simplicial objects in $\CD$. Therefore, 
	$$
	\abs{(FG)^{\bullet+1} X} \to \abs{(FG)^{\bullet+1} Y}
	$$
	is an equivalence and hence so is $f: X\to Y$. So $G$ is conservative.
	We now claim that $G$ preserves geometric realization of $G$-split objects.
	Let $X_{\bullet}$ be a $G$-split simplicial object and consider the following commuting diagram
	\[
	\begin{tikzcd}
		\abs{(FG)^{\bullet+1} \abs{X_{\bullet}}} \ar[r] \ar[d]& \abs{(FG)^{\bullet+1} X_{-1} \ar[d]}     \\
		\abs{X_{\bullet}} \ar[r] & X_{-1}  .
	\end{tikzcd}
	\]
	
	We claim the bottom horizontal arrow is an equivalence.
	Note that two vertical morphisms are equivalences by our assumption. We claim that the top horizontal map is an equivalence as well. Indeed, if we view $(FG)^{p+1} X_{q}$ as a bisimplicial object, then 
	$$
	\abs{(FG)^{\bullet+1} \abs{X_{\bullet}}} 
	\simeq 
	\colim_{p} \colim_q (FG)^{q+1}X_{p} .
	$$
	For fixed $q$, the simplicial object $(FG)^{q+1}X_{\bullet}$ is split since it's a composite of functors starting with $G$, hence one has 
	$$
	\colim_{p}(FG)^{q+1}X_{p}\simeq (FG)^{q+1}X_{-1}
	$$
	and $\colim_{q} (FG)^{q+1} X_{-1} \to X_{-1}$  is an equivalence by the assumption.
	Therefore, we conclude that 
	$$
	G(\abs{X_{\bullet}}) \simeq G(X_{-1}) \simeq \abs{G(X_{\bullet})}
	$$
	where the last equivalence follows from the fact that $X_{\bullet}$ is $G$-split.
\end{proof}

    As an application of the Corollary \ref{Cor of Barr-Beck-Lurie theorem} in homotopy theory, we prove the following folklore proposition that is well-known among seasoned homotopy theoriests.
\begin{proposition}
	\label{Coalgebra model for simply-connected spaces}
	For $r\geq 2$, the functor $\Sigma^{\infty}:\Space^{\geq r}_{*} \to \Sp^{\geq r}$ is comonadic. In other words, there is an equivalence of $\infty$-categories:
	\[
	\phi: \Space^{\geq r}_{*} \to \coAlg_{\Sigma^{\infty}\Omega^{\infty}}(\Sp^{\geq r}).
	\]
 between the $\infty$-category of $r$-connective spaces and the $\infty$-category of $r$-connective $\Sigma^{\infty}\Omega^{\infty}$-coalgebras.
\end{proposition}
\begin{proof}
By the dual of Corollary \ref{Cor of Barr-Beck-Lurie theorem}, it suffices to show there is an equivalence 
$$
X\to \Tot (\Omega^{\infty}\Sigma^{\infty})^{\bullet+1}X
$$
for every $X\in \Space_{*}^{\geq r}$. We prove this by induction on the Postnikov tower. For $X$ an Eilenberg-Maclane space $K(A,n)\simeq \Omega^{\infty-n}HA$, its associated augmented cosimplicial object $(\Omega^{\infty}\Sigma^{\infty})^{\bullet+1}X$ splits, with the contracting homotopy induced by the counit $\Sigma^{\infty}\Omega^{\infty} \Omega^n HA\to \Omega^n HA$. 

For the inductive step, we have a principal fibration sequence
\[
K(\pi_n X,n) \to \tau_{\leq n}X \to \tau_{\leq n-1}X 
\to 
K(\pi_n X,n+1).
\]
By the principal fibration lemma \cite{Bousfield-KanYellow}, the functor $\Tot (\Omega^{\infty}\Sigma^{\infty})^{\bullet+1}$ preserves principal fibrations. Hence the vertical sequences in the following diagram are fiber sequences.
\[
\begin{tikzcd}
	\tau_{\leq n}X & \Tot (\Omega^{\infty}\Sigma^{\infty})^{\bullet+1} (\tau_{\leq n}X)\\
	\tau_{\leq n-1}X  & \Tot (\Omega^{\infty}\Sigma^{\infty})^{\bullet+1} (\tau_{\leq n-1}X)\\
	K(\pi_n X,n+1)   & \Tot (\Omega^{\infty}\Sigma^{\infty})^{\bullet+1} K(\pi_n X,n+1)
	\arrow[from=1-1, to=1-2]
	\arrow[from=1-1, to=2-1]
	\arrow[from=2-1, to=2-2]
	\arrow[from=1-2, to=2-2]
	\arrow[from=2-1, to=3-1]
	\arrow[from=3-1, to=3-2]
	\arrow[from=2-1, to=3-1]
	\arrow[from=2-2, to=3-2]
\end{tikzcd}
\]
Observe that the bottom two horizontal arrows are equivalences by the inductive hypothesis, hence the induced map on the fibers is an equivalence. This completes the inductive step of the proof.
\end{proof}







\section{Construction of the Comparison Functor}
\label{Construction of the Comparison Functor}
In this section, we construct the comparison functor from divided power conilpotent commutative coalgebras in tame spectra to commutative coalgebras in tame spectra
$$
\zeta:\coCAlg^{\divpow,\nil}(\Sp^{\geq r}_{\tame}) \to
\coCAlg(\Sp^{\geq r}_{\tame}).
$$

We fix a pre-additive, presentable, symmetric monoidal $\infty$-category $\CC$ in which the tensor product is compatible with colimits. 
Consider the comonad $K$ on $\CC$ given by 
$$
K(X) := \prod_n (X^{\otimes n})_{h\Sigma n}.
$$
\begin{definition}
    We define the \emph{$\infty$-category of divided power commutative coalgebras} in $\CC$ as 
$$
\coCAlg^{\divpow}(\CC):= \LcoMod_{K}(\CC).
$$
\end{definition}
\begin{remark}
\label{Remark A4.2}
If $\CC=\Sp^{\geq r}_{\tame}$, then there is an equivalence of $\infty$-categories 
$$
\coCAlg^{\divpow}(\Sp^{\geq r}_{\tame})
\to 
\coCAlg(\Sp^{\geq r}_{\tame})
$$
by Lemma \ref{Tate vanishing for tame spectra}.
\end{remark}

\begin{remark}
\label{Remark A4.3}
When $\CC=\Sp$, there is a comparison functor
$$
\coCAlg^{\divpow, \nil}(\Sp) \to  \coCAlg^{\divpow}(\Sp)
$$
(see \cite[Section 3.5]{Francis-Gaitsgory} or \cite{Heuts_Koszul}), which induces a comparison functor
\begin{equation}
\label{comparison functor for r-conn Sp}
    \coCAlg^{\divpow, \nil}(\Sp^{\geq r}) \to  \coCAlg^{\divpow}(\Sp^{\geq r})
\end{equation}
since colimits in $\Sp^{\geq r}$ are computed in $\Sp$.
\end{remark}

By Remark \ref{Remark A4.2}, it suffices to construct a comparison functor
$$
\zeta':\coCAlg^{\divpow, \nil}(\Sp^{\geq r}_{\tame})
\to 
\coCAlg^{\divpow}(\Sp^{\geq r}_{\tame}).
$$
For the rest of this section, we explain how to obtain $\zeta'$ given (\ref{comparison functor for r-conn Sp}).
The crux for this construction is the following theorem by Heine, which allows us to identify maps between comonads with functors between comodules
\begin{theorem}
\label{Monads-Alg correspondence}
\cite[Theorem 5.1]{Heine_Monads}
Let $\CC$ be a presentable $\infty$-category, then the $\infty$-category of comonads on $\CC$ is a localization of the over category $\Pr^{L}_{/\CC}$ that assigns a comonad $Q$ to $\LcoMod_{Q}(\CC)$.
\end{theorem}


Suppose $L:\CC \to \CD$ is a symmetric monoidal localization
and we let $j:\CD \to \CC$ denote the embedding that is right adjoint to $L$.
We then have an adjunction on the $\infty$-category of endofunctors.
\begin{proposition}
\label{Prop A.4.2}
There is an adjunction
\begin{equation}
\label{A4.2 adjunction}
    \adj{L\circ (-)\circ j}{\Fun(\CC, \CC)}{\Fun(\CD, \CD)}{j\circ (-)\circ L}.
\end{equation}
\end{proposition}
\begin{proof}
    We need to show there are two pairs of adjunctions
    $$
    \adj{L_{*}}{\Fun(\CC, \CC)}{\Fun(\CC, \CD)}{j_*}
    $$
    and
    $$
    \adj{j^{*}}{\Fun(\CC, \CD)}{\Fun(\CD, \CD)}{L^*}.
    $$
    It is an easy exercise to check the triangle identities of both.
\end{proof}


We observe the functor $j_*L^*: \Fun(\CD,\CD) \to \Fun(\CC,\CC)$ is monoidal; indeed, for two endofunctors $F,G$ on $\CC$ we have
\begin{align*}
    j_*L^*(F\circ G) & \simeq j\circ F \circ G \circ L\\
                     & \simeq j\circ F \circ L \circ j \circ  G \circ L \\
                     & \simeq j_*L^*(F)\circ j_*L^*(G)
\end{align*}
where the second-to-last equivalence follows from $ L \circ j \simeq \id_{\CD}$.
Hence $j_*L^*$ preserves monads and comonads. Furthermore, the left adjoint $j^*L_*$ is oplax monoidal, so it also preserves comnads. 
The counit map of adjunction (\ref{map above A.2})  evaluated on any comonad $Q$ on $\CD$
\begin{equation}
\label{map above A.2}
    j^*L_*j_*L^*Q  \to  Q
\end{equation}
is an equivalence of comonads. Therefore, the functor $j_*L^*:\Comonad(\CD) \to \Comonad(\CC)$ is fully faithful.
% , and (\ref{map above A.2}) corresponds to a map
% $j_*L^* Q_{\CD} \to Q_{\CD}$.

Let $Q_{\CC}$ and $Q_{\CD}$ denote the comonads arising from the forgetful-cofree adjunction 
$$\adj{\oblv_{\CC}}{\coCAlg^{\divpow}(\CC)}{\CC}{\cofree_{\CC}}$$
and
$$
\adj{\oblv_{\CD}}{\coCAlg^{\divpow}(\CD)}{\CD}{\cofree_{\CD}}.
$$
For every $X\in\CD$,
there is a natural map
$$
Q_{\CC}(X)= \prod_{n} (X^{\otimes n})_{h\Sigma_n,\CC} \to 
Q_{\CD}(X)= \prod_{n} (X^{\otimes n})_{h\Sigma_n,\CD}
$$
where we use $(X^{\otimes n})_{h\Sigma_n,\CC}$ (resp. $(X^{\otimes n})_{h\Sigma_n,\CD}$) to indicate the homotopy orbits is computed in $\CC$ (resp. $\CD$).
Therefore, we obtain a map of comonads 
\begin{equation}
\label{LQ_Cj to Q_D}
    LQ_{\CC}j \to Q_{\CD}.
\end{equation}

Consider now the composite of comonads
\begin{equation}
\label{A.3}
    Q_{\CC} \to 
   j\circ L \circ Q_{\CC} \circ j\circ L
    \to 
    j\circ Q_{\CD} \circ L
\end{equation}
where the first map is the unit of adjunction (\ref{A4.2 adjunction}), and the second map (\ref{LQ_Cj to Q_D}).

Since $\CC$ is preadditive, the comonad $F_{\Com}$ is equivalent to the symmetric algebra functor $\Sym_{\CC}$, hence there is a natural transformation
$$
\Sym_{\CC} \to \id_{\CC}
$$
induced from the coagumentation.

\begin{remark}
\label{F_Com to Q_C}
Suppose there is a map of comonads
$$
\theta: F_{\Com} \to 
Q_{\CC},
$$
then $\theta$ must be the canonical map.
    More concretely, if $X\in \CC$, then the map $F_{\Com}(X) \to 
Q_{\CC}(X)$ is the unique map (up to contractible ambiguity) that makes the following diagram commute
\[
\begin{tikzcd}
	 F_{\Com}(X) &   & X \\
	& Q_{\CC}(X) .  &
	\arrow[from=1-1, to = 1-3]
	\arrow[from=1-1 , to =2-2]
	\arrow[from=2-2, to=1-3]
\end{tikzcd}
\]
\end{remark}


Postcomposing with (\ref{A.3}) then induces a map 
$$
F_{\Com}  \to 
Q_{\CC}
\to 
j\circ Q_{\CD} \circ L,
$$
which corresponds to 
$$
\Gamma: L\circ F_{\Com} \circ j \to L\circ Q_{\CC} \circ j \to Q_{\CD}
$$
by the adjunction (\ref{A4.2 adjunction}).

Applying Theorem \ref{Monads-Alg correspondence}, we define the comparison functor as follows.
\begin{definition}
    The comparison functor
    \[
    \zeta': \coCAlg^{\divpow,\nil}(\CD) :=\LcoMod_{L\circ F_{\Com}}(\CD) \to \coCAlg^{\divpow}(\CD):=\LcoMod_{Q_{\CD}}
    \]
    is the functor corresponding to $\Gamma$ under the correspondence of Theorem \ref{Monads-Alg correspondence}.
\end{definition}

To construct the comparison functor in the case of tame spectra, we let $\CC=\Sp^{\geq r}$ and $\CD=\Sp^{\geq r}_{\tame}$. There is a comparison functor on coalgebras in $r$-connective spectra
$$
\coCAlg^{\divpow, \nil}(\Sp^{\geq r}) \to  \coCAlg^{\divpow}(\Sp^{\geq r})
$$
by Remark \ref{Remark A4.3}.
Combining with the discussion above, we obtain the comparison functor
\[
\zeta:\coCAlg^{\divpow,\nil}(\Sp^{\geq r}_{\tame}) \to
\coCAlg^{\divpow}(\Sp^{\geq r}_{\tame})\simeq
\coCAlg(\Sp^{\geq r}_{\tame}).
\]



% \begin{lemma}
% \label{Pullback diagram in Prop A4.2}
% The following diagram of $\infty$-categories
% \[
% \begin{tikzcd}
% 	\LcoMod_{j_*L^*Q_\CD}(\CC) & \LcoMod_{Q_\CD}(\CD)\\
%     \CC & \CD
% 	\arrow[from=1-1, to= 1-2]
% 	\arrow[from=1-1, to=2-1]
% 	\arrow[from=1-2, to=2-2, "\oblv"]
% 	\arrow[from=2-1, to= 2-2, "L"]
% \end{tikzcd}
% \]
% is a pullback diagram.
% \end{lemma}

% Now suppose there is a functor
% $$
% \theta': \coCAlg^{\divpow, \nil}(\CC) \to \coCAlg(\CC),
% $$
% then postcomposes with the localization functor gives
% $$
% \theta: \coCAlg^{\divpow, \nil}(\CC) \to \coCAlg(\CD).
% $$
% Therefore, we obtain a functor 
% $$
% \LcoMod_{F_{\Com}}(\CC) \to
% \LcoMod_{j_*L^*Q_\CD}(\CC)
% $$
% as the canonical functor into the pullback in the diagram of Proposition \ref{Pullback diagram in Prop A4.2}. 
% Then by Theorem \ref{Monads-Alg correspondence}, there is a map of comonads
% $F_{\Com}\to j_*L^*Q_\CD$ on $\CC$, which corresponds to a map of comonads 
% $\Gamma:L\circ F_{\Com}\circ j \to Q_{\CD}$ by adjunction.










