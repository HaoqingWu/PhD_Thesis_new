\chapter{Introduction}

\section{A Brief History of Algebraic Models for Spaces}

Algebraic topology is the study of classification of topological spaces up to certain equivalence relations via algebraic invariants. 
The homotopy groups $\pi_* X$ of a space $X$ is one of these invariants.
For $n=1$, $\pi_1(X)$ is called the \emph{fundamental group} of $X$. For $n\geq 2$, $\pi_n(X)$ are abelian groups. 

Spheres are the building blocks in topology, since the category of spaces is generated under (homotopy) colimits by $S^n$ for $n\geq 0$. Perhaps surprisingly, we only know very little about the homotopy groups of spheres. 

To motivate the study of algebraic models of spaces, we now survey some important results about the homotopy groups of the spheres $\pi_m(S^n)$ for $m,n \geq 1$.
From our first course of topology, we know that the fundamental group of the circle $S^1$ is isomorphic to the group of integers $\BZ$. Moreover, $\pi_1( S^1)$ is the only non-vanishing homotopy group of $S^1$, i.e. $S^1$ is an Eilenberg-Maclane space of type $K(\BZ,1)$.

In 1930's, Hurewicz showed that, if $X$ simply-connected,  the lowest non-vanishing homology group of a space $X$ is isomorphic to the lowest non-vanishing homotopy group. Hurewicz's theorem then implies the following theorem, which states that any map from a low dimensional sphere to a higher dimensional sphere is null-homotopic.
\begin{theorem}
    [Hurewicz]
    The homotopry group $\pi_m(S^n)$ is zero for $m<n$ and is isomorphic to $\BZ$ if $m=n$.
\end{theorem}
\clearpage

Shortly after Hurewicz, Freudenthal discovered the statbility of the groups $\pi_m(S^n)$.
\begin{theorem}
[Freudenthal's Suspension Theorem]
    The suspension morphism 
    $$
    \sigma: S^n \to \Omega S^{n+1}
    $$
    induces isomorphisms:
    $$
    \pi_{n+k}(S^n) \xrightarrow{\sigma_*} 
    \pi_{n+k+1}(S^{n+1})
    $$
    for $k<n-1$.
\end{theorem}
Freudenthal's theorem motivates the definition of the stable homotopy groups of spheres
$$
\pi^{s}_{k}:= \colim_n \pi_{n+k}(S^n), \forall k \in \BZ
$$
which led to the study of a whole new subject, the \emph{stable homotopy theory}.
The stable homotopy groups of spheres are slightly easier to compute than the unstable ones, but still largely unknown.

In the 1950's, Serre used his spectral sequence to compute $\pi_m(S^n)$ modulo torsion.
\begin{theorem}
[Serre]
    For each $n\geq 1$, the homotopy groups of the sphere $S^n$ are finitely generated abelian groups.
    Moreover, after tensoring with $\BQ$, 
    \[
    \pi_m (S^n)\otimes \BQ \cong
    \begin{cases}
    \BQ & \text{ if $m=n$,}\\
    \BQ & \text{ if $m=2n-1$ and $n$ is even,}\\
    0   & \text{ otherwise.}
    \end{cases}
    \]
\end{theorem}

Serre's theorem initiated the study of the homotopy theory of spaces modulo torsion, which is the subject called the \emph{rational homotopy theory}. A simply-connected space $X$ is \emph{rational} if its homotopy groups $\pi_* X$ are vector spaces over $\BQ$.
Quillen \cite{Quillen_RHT} showed that rational homotopy theory has a complete algebraic description. 
We shall now use the modern language of $\infty$-categories from \cite{HTT} to explain Quillen's theorem in more detail.

Let $\CS^{\geq 2}_*$ denote the $\infty$-category of pointed, simply-connected spaces and let $\Space^{\geq 2}_{\BQ}$ denote the full subcategory of $\CS^{\geq 2}_*$ spanned by rational spaces.
A map $f:X\to Y$ in $\CS^{\geq 2}_*$ is a \emph{rational equivalence} if it induces isomorhpisms on rational homotopy groups
\[
\pi_*(f): \pi_*X \otimes \BQ \to \pi_*Y \otimes \BQ;
\]
or equivalently, if it induces isomorphisms on rational homology groups
\[
H_*(f): H_*(X;\BQ)  \to H_*(Y;\BQ).
\]
For any space $X\in \CS^{\geq 2}*$, there exists a map $\eta:X\to X_{\BQ}$ such that
\begin{enumerate}
    \item $X_{\BQ}$ is a rational space;
    \item $\eta$ is a rational equivalence.
\end{enumerate}
The $\infty$-category $\Space^{\geq 2}_{\BQ}$ of simply-connected rational spaces can be constructed by formally inverting rational equivalences in $\CS^{\geq 2}_*$.

Let $\Ch_{\BQ}$ denote the $\infty$-category of rational chain complexes. The $\infty$-categorical version of Quillen's theorem can be stated as follows.
\begin{theorem}
    \cite{Quillen_RHT}
    There are equivalences of $\infty$-categories
    $$
	\coCAlg(Ch_{\BQ})^{\geq 2}\simeq \CS^{\geq 2}_{\BQ} \simeq \Lie(Ch_{\BQ})^{\geq 1},
	$$
	where $\Lie(Ch_{\BQ})^{\geq 1}$ denotes the $\infty$-category of connected rational dg Lie algebras, and $\coCAlg(Ch_{\BQ})^{\geq 2}$ denotes the $\infty$-category of simply-connected rational commutative dg coalgebras.
\end{theorem}
Quillen's model for rational homotopy theory is not only conceptual, but also allows computation using algebraic gadgets. 
Later Sullivan \cite{Sullivan} defined the \emph{polynomial de Rham complex functor} $\CA_{PL}:(\Space^{\geq 2}_{\BQ})^{op} \to \CAlg(\Ch_{\BQ})$, where the target is the category of commutative differential graded algebras (CDGAs) over $\BQ$.
The functor $\CA_{PL}$ is a complete homotopy invariant, in the sense that two rational spaces $X$ and $Y$ of finite type (i.e. their homology groups are finitely generated) are equivalent if and only if $\CA_{PL}(X)$ and $\CA_{PL}(Y)$ are equivalent (i.e. connected by a zig-zag of quasi-isomorphisms) as CDGAs over $\BQ$.
Moreover, every $\CA_{PL}(X)$ is equivalent to a \emph{minimal model} $\Lambda_X$, which is generally quite computable. The rational homotopy groups of $X$ can be directly computed from the minimal model $\Lambda_X$. 

From a modern prospective, Sullivan's functor $\CA_{PL}$ is essentially equivalent to the rational cochain functor $C^*(-;\BQ)$, and $C^*(X;\BQ)$ has a $\E_\infty$-algebra structure coming from the chain-level cup products. Sullivan's theorem can then be rephrased in the language of $\infty$-categories as follows.
\begin{theorem}
    \cite{Sullivan}
    Let $\CS^{\geq 2, \operatorname{fin}}_{\BQ}$ be the $\infty$-category of simply-connected spaces of finite type, and let  $\CAlg(\Ch_{\BQ})$ be the $\infty$-category of $\E_{\infty}$-algebras over $\BQ$.
    The cochain functor 
$$
C^*(-;\BQ): \CS^{\geq 2, \operatorname{fin}}_{\BQ}
\to 
\CAlg(\Ch_{\BQ})
$$ 
is fully faithful.
\end{theorem}

One natural question to ask is what other subcategories of spaces admit concrete algebraic descriptions as in the case of rational spaces.
We will now survey some results that answer this question.
Recall that we obtain the $\infty$-category $\CS^{\geq 2}_{\BQ}$ by formally inverting rational equivalences in $\CS^{\geq 2}_{*}$. This technique is called \emph{localization}, and it can be streamlined in a general setting as follows.
\begin{definition}
\label{localization}
Let $\CC$ be an $\infty$-category and let $S$ be a (small) collection of morphisms in $\CC$. 
An object $X$ in $\CC$ is \emph{$S$-local} if the induced map on mapping spaces
$$
\map_{\CC}(B, X) \to \map_{\CC}(A, X)
$$
is a weak equivalence for all $f:A \to B$ in $S$. A map $g:Y \to Z$ is an \emph{$S$-equivalence} if the induced map on mapping spaces
$$
\map_{\CC}(Z, X) \to \map_{\CC}(Y, X)
$$
is an equivalence for every $S$-local object $X$.
\end{definition}
Let $\CC'$ be the full subcategory of $\CC$ spanned by $S$-local objects. $\CC'$ is called a \emph{localization} of $\CC$ if the embedding functor $\CC' \hookrightarrow \CC$ admits a left adjoint.
\begin{proposition}
\cite[Proposition 5.5.4.15.]{HTT}
\label{Prop 5.5.4.15. HTT2}
If $\CC$ is a presentable $\infty$-category and $S$ is a collection of morphisms in $\CC$, then there exists a localization functor
$$
L:\CC \to \CC'
$$
so that a map $f:A\to B$ in $\CC$ is an $S$-equivalence if and only if $Lf$ is an equivalence in $\CC'$.
\end{proposition}

Using Proposition \ref{Prop 5.5.4.15. HTT2}, one can show that the $\infty$-category $\CS^{\geq 2}_{\BQ}$ of simply-connected rational spaces is obtained by inverting rational homology equivalences. 
A next step to consider is to take 
$S$ to be the collection of morphisms in $\CS^{\geq 2}_*$ that induce isomorphisms on mod-$p$ homology groups, i.e., $f: X \to Y$ is in $S$ if 
$$
H_*(f):H_{*}(X;\BF_p) \xrightarrow{\cong} H_*(Y; \BF_p).
$$
An $S$-local space is called a \emph{$p$-complete space}. By Proposition \ref{Prop 5.5.4.15. HTT2}, any simply-connected space $X$ admits a
\emph{$p$-completion} $X \to X^{\wedge}_p$, i.e., $X^{\wedge}_p$ is $p$-complete and the map $X \to X^{\wedge}_p$ induces isomorphisms on mod-$p$ homology groups.

Let $\overline{\BF}_p$ denote the algebraic closure of $\BF_p$. Mandell constructed an algebraic model for the $\infty$-category $\CS^{\geq 2, \operatorname{fin}}_p$ of simply-connected $p$-complete spaces of finite type.
\begin{theorem}
    \cite{Mandell_p-adic}
    The cochain functor 
    $$
    C^*(-;\overline{\BF}_p): 
    (\CS^{\geq 2, \operatorname{fin}}_p)^{op} 
    \to 
    \CAlg(\Ch_{\overline{\BF}_p})
    $$
    is fully faithful.
\end{theorem}
One may then wonder whether the integral homotopy type might be characterized by the integral cochain functor
$$
C^*(-;\BZ): 
    (\CS^{\geq 2, \operatorname{fin}})^{op} 
    \to 
    \CAlg(\Ch_{\BZ}).
$$
Unfortunately, this is not the case due to the following theorem proved again by Mandell.
\begin{theorem}
    \cite{Mandell_Cochain}
    The integral cochain functor $C^*(-;\BZ)$ is faithful but not full.
\end{theorem}
The next best hope is to find a way to assemble information from rationalization and $p$-completions.
In number theory, one can recover the ring of integers $\BZ$ by the following pullback square:
\[
\begin{tikzcd}
\BZ & \prod_{p} \BZ^{\wedge}_{p}  \\
\BQ & (\prod_{p} \BZ^{\wedge}_{p}) \otimes \BQ
        \arrow[from=1-1, to=1-2]
		\arrow[from=2-1, to=2-2]
		\arrow[from=1-1, to=2-1]
		\arrow[from=1-2, to=2-2]
\end{tikzcd}
\]
where $\BZ^{\wedge}_p$ denotes the ring of $p$-adic integers. In homotopy theory, Bousfield--Kan \cite{Bousfield-KanYellow} and Sullivan \cite{Sullivan05} proved that there is pullback square for any nilpotent space $X$:
\[
\begin{tikzcd}
X & \prod_{p} X^{\wedge}_{p}  \\
X_\BQ & (\prod_{p} X^{\wedge}_{p}) \otimes \BQ.
        \arrow[from=1-1, to=1-2]
		\arrow[from=2-1, to=2-2]
		\arrow[from=1-1, to=2-1]
		\arrow[from=1-2, to=2-2]
\end{tikzcd}
\]

One might then hope to construct the integral model for nilpotent spaces by assembling Sullivan's cochain model for rational spaces and Mandell's cochain model for $p$-complete spaces. However, the difficulty lies in how to assemble information from $\E_{\infty}$-rings over fields of different characteristics.

Mandell's theorem suggests that we need structures more than just commutativity to capture the whole information of integral homotopy.
We now introduce the recent results of Yuan \cite{Yuan} concerning the integral homotopy type.

For each prime $p$, Nikolaus-Scholze \cite{Nikolaus-Scholze} showed that any $\E_{\infty}$-ring $A$ admits a \emph{Frobenius action} $\varphi_{A}:A \to A^{tC_{p}}$, where $(-)^{tC_p}$ denotes the Tate construction of a $C_p$-spectrum.
Yuan \cite{Yuan} then defines a $p$-complete $\E_\infty$-ring to be \emph{$p$-perfect} by imposing conditions on the Frobenius action and showed that the $\infty$-category $\CAlg_p^{\operatorname{perf}}$ of $p$-perfect $\E_\infty$-rings
admits an $S^1$-action.
A canonical example of a $p$-perfect $\E_{\infty}$-ring is the $p$-complete sphere spectrum $\BS^{\wedge}_p$.

Yuan defined the $\infty$-category $\CAlg_{p}^{\varphi=1}$ of \emph{$p$-Frobenius fixed $\E_{\infty}$-rings} as the $S^1$-fixed points of $\CAlg_p^{\operatorname{perf}}$, and refer to a lift $A_{\varphi=1}\in \CAlg_{p}^{\varphi=1}$ of a
$p$-perfect $\E_{\infty}$-ring $A$ as the \emph{$F_p$-trivialization} of $A$.
% $\CAlg_{p}^{\varphi=1}$ and the mapping
% spectrum $(\BS^{\wedge}_{p})_{\varphi=1}^X$ is a $p$-Frobenius fixed $\infty$-ring.
The $p$-complete sphere spectrum $\BS^{\wedge}_p$ admits an $F_p$-trivialization and hence the cochain $(\BS^{\wedge}_{p})_{\varphi=1}^{X}$ lies in the $\infty$-category $\CAlg_{p}^{\varphi=1} $.
Yuan then constructed a new algebraic model for simply-connected $p$-complete spaces of finite type.
\begin{theorem}
    \cite[Theorem B]{Yuan}
    The functor $
    (\CS^{\geq 2, \operatorname{fin}}_p)^{op} 
    \to 
    \CAlg_{p}^{\varphi=1} 
    $ that sends $X$ to $(\BS^{\wedge}_{p})_{\varphi=1}^{X}$ is fully faithful.
\end{theorem}

It turns out that the cochain of $X$ with coefficient in the $p$-Frobenius fixed $\E_\infty$-ring $(\BS^{\wedge}_{p})_{\varphi=1}$ can be assembled for different primes $p$ and give rise to an algebraic model for the integral homotopy type.

An $\E_{\infty}$-ring $A$ is called \emph{Frobenius fixed} if its $p$-completion $A^{\wedge}_{p}$ admits $F_p$-trivialization for each prime $p$. Let $\CAlg^{\varphi=1}$ denote the $\infty$-category of Frobenius fixed $\E_\infty$-rings.
\begin{theorem}
    \cite[Theorem C]{Yuan}
    The functor $
    (\CS_{*}^{\geq 2, \operatorname{fin}})^{op} 
    \to 
    \CAlg^{\varphi=1} 
    $ that sends $X$ to $\BS_{\varphi=1}^{X}$ is fully faithful.
\end{theorem}

So far we have only seen the (co)algebra analogues of Quillen's rational homotopy theory.
We now state a result of Heuts \cite{heuts2018lie}, which concerns a Lie algebra model for $v_n$-periodic spaces.
Since the prerequisites to understand the exact statement are substantial, we give only a rough explaination here and refer the readers to \cite{heuts2018lie} and the survey paper \cite{HeutsSurvey} for more details.

For each prime $p$, there is a sequence of homology thoeries $$K(0), K(1), K(2), \dots, K(\infty)$$ 
on $p$-local spaces, called the \emph{Morava K-theories}. 
For certain $n$, these homology theories are well-known. For instance, $K(0)$ is the rational homology theory, $K(1)$ is the mod-$p$ complex K-theory and $K(\infty)$ is the mod-$p$ homology theory.
These homology theories are the fundamental objects in \emph{chromatic homotopy theory}.

A $p$-local finite complex $V$ is of \emph{type $n$} if $K_*(m)V$ vanishes for $m<n$ and is non-zero for $m=n$.
By the famous periodicity theorem of Hopkins-Smith \cite{Hopkins-Smith}, any $p$-local finite complex $V$ of type $n$ admits a \emph{$v_n$-self map} $v:\Sigma^d V\to V$ so that $K_*(m)v$ is an isomoprhism for $m=n$ and $K_*(m)v$ is the zero map for $m\neq n$.
Hence, for any $p$-local space $X$, the $v_n$-self map $v$ acts invertibly on the homotopy groups of the mapping space $\map_*(V,X)$ and one can define the \emph{$v_n$-periodic homotopy groups} of $X$ as
$$
v^{-1}\pi_*(X;V):= \BZ[v^{\pm 1}]\otimes_{\BZ[v] }\pi_* \map_*(V,X).
$$

A map $f:X \to Y$ between two $p$-local spaces $X,Y$ is called a \emph{$v_n$-periodic equivalence} if it induces isomorphisms on the $v_n$-periodic homotopy groups.
We remark that the $v_n$-periodic homotopy groups depend on the choice of the type $n$ complex $V$, but the class of $v_n$-periodic equivalences does not.
The $\infty$-category $\CS_{v_n}$ of \emph{$v_n$-periodic spaces} is then obtained by formally inverting the $v_n$-periodic equivalences.
In a similar manner, we can also define the $\infty$-category $\Sp_{v_n}$ of $v_n$-periodic spectra.

Heuts proved that the $\infty$-category of $v_n$-periodic spaces admits a Lie algebra model.
\begin{theorem}
\cite{heuts2018lie}
The $\infty$-category of $v_n$-periodic spaces is equivalent to the $\infty$-category $\Alg_{\spLie}(\Sp_{v_n})$ of spectral Lie algebras in $\Sp_{v_n}$.
\end{theorem}
We will give the definition of the $\infty$-category of spectral Lie algebras in Chapter 3.

We now introduce the algebraic model for Dwyer's tame homotopy theory \cite{Dwyer}, which is also the main topic of this thesis.
All the localizations we have seen so far are obtained by first inverting a set of primes for all spaces. For rational spaces, we invert all the primes at all degrees of the homotopy groups.
What if we only invert more primes as the degree of the homotopy groups increases?
The motivation of doing so was inspired by the following theorem of Serre.
\begin{theorem}[Serre]
	For $n\geq 3$, the first $p$-torsion in the homotopy group $\pi_{k}(S^{n})$ appears at degree $ n+2p-3$.
\end{theorem}

A \emph{ring system} $R_*$ is a sequence of subrings $R_j$ of $\BQ$ for $j\geq 0$ such that  $R_m$ is a subring of $R_n$ if $m<n$. In this thesis, we will be interested in only one ring system, defined as follows.
\begin{definition}
\label{tame ring system}
The \emph{tame ring system} $\{R_j\}_{j\geq 0}$ is defined as 
$$
R_{j}:=\BZ[\frac{1}{k}|k\leq \frac{j+3}{2}].
$$
In other words, $R_j$ is the smallest subring of $\BQ$ in which $p$ is inverted for all primes $p \leq \frac{j+3}{2}$.
\end{definition}

Let $r\geq 3$ be an integer. 
Dwyer \cite{Dwyer} defined the tame model structure on the category $\CS^{\geq r}_*$ of pointed $r$-connective spaces, in which a map $f$ is a weak equivalence if $\pi_{r+j}(f)\otimes R_j$ is an isomorphism for all $j\geq 0$.
On the algebraic side, he defined the tame model structure on the category $\Alg_{\Lie}(\Ch_\BZ)^{\geq r-1}$ of $(r-1)$-connective dg Lie algebras over the integer $\BZ$, in which a map $f$ is a weak equivalence if the induced maps on homology groups with coefficient in the tame ring system are isomorphisms, that is, $H_{r+j-1}(f)\otimes R_j$ is an isomorphism for all $j\geq 0$.
There seems to exist no direct homotopy functor that connects these two categories on the model category level.
Dwyer then defined the notion of \emph{Lazard algebras}, which are Lie algebras with just enough extra structure so that the Baker-Campbell-Hausdorff formula makes sense. He equipped the category of simplicial Lazard algebras with a model structure and proved that it is the intermediate category of a zig-zag of Quillen equivalences between $\CS^{\geq r}_*$ and $\Alg_{\Lie}(\Ch_\BZ)^{\geq r-1}$.
\begin{theorem}
\cite{Dwyer}
With the model structures described above,
there is a zig-zag of Quillen equivalences between $\CS^{\geq r}_*$ and $\Alg_{\Lie}(\Ch(\BZ))^{\geq r-1}$.
\end{theorem}

We end this section by stating a result of Anick \cite{AnickHopf} which motivates us to consider the Hopf algebra model for tame spaces.
Let $p$ be a prime number and let $R=\BZ[{\frac{1}{(p-1)!}}]$. 
A free differential graded (dg) Lie algebra is \emph{$r$-mild}  if it is generated in the range of dimension from $r$ to $rp-1$. Denote the category of free $r$-mild dg Lie algebras over $R$  by $\Alg_{\Lie}(\Ch_{r}(R))$. Anick introduced a notion of \emph{Hopf algebra up to homotopy} (Hah) \cite[Definition 4.1]{AnickHopf}, which is a generalization of dg cocommutative Hopf algebras with the usual structure diagrams commuting up to homotopy. Let $\Hah_{r}(R)$ denote the category of $r$-mild Hah over $R$.

\begin{theorem}[\cite{AnickHopf} Theorem 4.8]
The universal enveloping algebra functor $U:\Alg_{\Lie}(\Ch_{r}(R))\rightarrow \Hah_{r}(R)$ induces an equivalence on their homotopy categroies:
$$
\Ho(\Alg_{\Lie}(\Ch_{r}(R)) )\simeq \Ho(\Hah_{r}(R)).
$$
\end{theorem}
% Fix some integer $n$. 
% Let $R$ be the subring of $\BQ$ which contains $\frac{1}{p}$ for all $p<n$. He defined the notion of \emph{Hopf algebras up to homotopy} .


\section{Outline of the Thesis}
The main objectives of the thesis are four-fold.
In chapter 2, we will streamline the discussion of tame homotopy theory using the modern language of $\infty$-categories.
A space $X$ is \emph{$r$-tame} if its homotopy groups are modules over the tame ring system, i.e.
$\pi_{r+j}X$ is an $R_j$-module for all $j\geq 0$.
We show that the $\infty$-category of $r$-tame spaces can be obtained by inverting maps that induce isomorphisms on homotopy groups with coefficients in the tame ring system. We will refer to such maps as \emph{tame equivalences}.
More concretely, we prove that the $\infty$-category $\CS^{\geq r}_{\tame}$ of $r$-tame spaces is a localization of the $\infty$-category $\CS^{\geq r}_*$ of pointed $r$-connective spaces at the class of tame equivalences.

Secondly, we define the $\infty$-category $\Sp^{\geq r}_{\tame}$ of \emph{$r$-tame spectra} and tame equivalences between them in a similar manner. We show that $\Sp^{\geq r}_{\tame}$ is a localization of the $\infty$-category $\Sp^{\geq r}$ of $r$-connective spectra.
The $\infty$-category $\Sp^{\geq r}_{\tame}$ of tame spectra appears to have a nice algebraic description.
Let $(\Mod_{H\BZ}^{\geq r})_{\tame}$ be the $\infty$-category of $r$-connective $H\BZ$-modules whose underlying spectra are tame.
\begin{thmx}
\label{Theorem B}
There is an equivalence of $\infty$-categories
$$
\Sp^{\geq r}_{\tame} \simeq (\Mod_{H\BZ}^{\geq r})_{\tame}.
$$
\end{thmx}

In Chapter 3, we define the $\infty$-category $\Alg_{\spLie}(\Sp^{\geq r}_{\tame})$ of Lie algebras in tame spectra , which we will refer as the \emph{tame spectral Lie algebras}.
We apply Koszul duality to produce a universal enveloping algebra functor
$$
U:\Alg_{\spLie}(\Sp^{\geq r}_{\tame}) \to \Hopfalgebra(\Sp^{\geq r}_{\tame}),
$$ 
where the target is the $\infty$-category of Hopf algebras in $\Sp^{\geq r}_{\tame}$. 

In chapter 4, we prove a new Hopf algebra model for tame spaces.
\begin{thmx}
\label{Theorem C}
There is an equivalence of $\infty$-categories 
$$
\Space^{\geq r}_{\tame} 
\simeq
\Hopfalgebra(\Sp^{\geq r-1}_{\tame}).
$$
\end{thmx}
Using the Hopf algebra model, we prove the following theorem which can be seen as a Milnor-Moore theorem for tame spectra.
\begin{thmx}
The universal enveloping algebra functor $U$ is an equivalence of $\infty$-categories.
\end{thmx}

Assembling these three theorems, we then recover Dwyer's Lie algebra model for tame spaces.




\section{Conventions and Notation}
Throughout this paper, we will freely use the language of $\infty$-categories (i.e. quasi-categories) developed in \cite{HTT} and higher alegebras from \cite{HA}.
We will try our best to provide explicit references to the relevant results in these books.

\subsection{Conventions}
\begin{itemize}
%   No need to fix a stable \infty-category anymore
%	\item We fix a presentably stable symmetric monoidal $\infty$-category $\CC$ in which tensor product preserves colimits in each variable.
	\item If $\CD$ is an ordinary category, then we won't distinguish $\CD$ and its nerve $N\CD$ (when viewed as an $\infty$-category).
	\item We say a morphism $f:X\rightarrow Y$ in an $\infty$-category $\CD$ is an \emph{equivalence} if it is an isomorphism after passing to the homotopy category $\hcat\CD$.
    \item If $\CD$ is an $\infty$-category, we denote by $\CD^{\simeq}$ the \emph{core} of $\CD$, i.e. the largest Kan subcomplex contained in $\CD$.  
    \item Let $n$ be a non-negative integer. 
    We will call a space (or spectrum) $X$ \emph{$n$-connective} if $\pi_i(X)=0$ for $i <n$. Dually, we will call a space $X$ (or spectrum) is $n$-truncated if $\pi_i(X)=0$ for $i >n$.
    \item A map $f: X \to Y$ between spaces or spectra is $n$-connective (resp. $n$-truncated) if the (homotopy) fibers of $f$ are $n$-connective (resp. $n$-truncated).
\end{itemize}

\subsection{Notation}
\begin{itemize}
	\item $\Delta$ denotes the category of non-empty finite linearly ordered sets.
	\item $\Delta_{+}$ denotes the category of (possibly empty) finite linearly ordered sets. We will abuse notation by denoting the empty set by $[-1]$.
	\item $\Delta^{\leq n}_{+}$ is the full subcategory of $\Delta_{+}$ spanned by the objects $\{[k]\}_{-1\leq k\leq n}$.
	\item $\Fin^{nu}$ is the category of non-empty finite sets.
	\item $\Fin$ is the category of (possibly empty) finite sets. 
	\item Let $\CC$ and $\CD$ be $\infty$-categories. We let $\Fun(\CC,\CD)$ denote the $\infty$-category of functors from $\CC$ to $\CD$.
	\item $\CS_{*}$ is the $\infty$-category of spaces and $\Sp$ is the $\infty$-category of spectra.
	\item $\Cat_{\infty}$ is the $\infty$-category of (small) $\infty$-categories.
	\item $\Pr^{L}$ is the $\infty$-category of presentable $\infty$-categories with colimit-preserving functors as morphisms \cite[Definition 5.5.3.1.]{HTT}.
	Let $\CC,\CD \in \Pr^L$. We let $\Fun^L(\CC,\CD)$ (resp. $\Fun^{R}(\CC, \CD)$) denote  the $\infty$-category of functors from $\CC$ to $\CD$ that are left adjoints (resp. right adjoints).
	\item $\mathds{1}$ denotes the trivial $\infty$-operad, which is the unit object in the monoidal category of $\infty$-operads with respect to the composition product.
	\item If $\CC$ is a monoidal $\infty$-category. We let $\Alg(\CC)$ denote the $\infty$-category of associative algebras in $\CC$.
	If $\CC$ is a symmetric monoidal $\infty$-category, we let $\CAlg(\CC)$ denote the 
	$\infty$-category of commutative algebras in $\CC$.
	\item 
	Informally, a monad (resp. comonad) (See \cite[Definition 4.7.3.2.]{HA}) $T$ on $\CC$ is an associative (resp. coassociative) algebra object in the $\infty$-category of endofunctors on $
	CC$. 
	$\LMod_{T}(\CC)$ denotes the $\infty$-category of left modules over the monad $T$ as defined in \cite[Section 4.2]{HA}. Dually, $\LcoMod_{\CQ}(\CC)$ denotes the $\infty$-category of left comodules over the comonad $\CQ$.
\end{itemize}