\chapter{Introduction}

\section{A Brief History of Algebraic Models for Spaces}
\begin{enumerate}
	\item Adams-Hilton
	\item Quillen (Lie algebras, coalgebras -- rational spaces)
	\item Dwyer (Lie algebras -- tame spaces)
	\item Anick (Hopf algebras -- truncated spaces with some primes inverted)
	\item Mandell ($\E_{\infty}$-algebras -- p-complete spaces of finite type)
	\item Heuts (Lie algebras -- $v_n$-periodic spaces)
	\item Yuan ($\E_{\infty}$-algebras with higher Frobenius -- simply-connected finite complexes)
\end{enumerate}


\section{Conventions and Notations}
Throughout this paper, we will freely use the language of $\infty$-categories (i.e. quasi-categories) developed in \cite{HTT} and $\infty$-operads from \cite{HA}.
We will try our best to give the numberings of the relavant results in these two books.

\subsection{Conventions}
\begin{itemize}
%   No need to fix a stable \infty-category anymore
%	\item We fix a presentably stable symmetric monoidal $\infty$-category $\CC$ in which tensor product preserves colimits in each variable.
	\item If $\CD$ is an ordinary category, then we won't distinguish $\CD$ and its nerve $N\CD$ (when viewed as an $\infty$-category).
	\item We say a morphism $f:X\rightarrow Y$ in an $\infty$-category $\CD$ is an \emph{equivalence} if it is an isomorphism after passing to the homotopy category $\hcat\CD$.
    \item If $\CD$ is an $\infty$-category, we denote by $\CD^{\simeq}$ the \emph{core} of $\CD$, i.e. the largest Kan subcomplex contained in $\CD$.  
\end{itemize}

\subsection{Notations}
\begin{itemize}
	\item $\Delta$ denotes the category of non-empty finite linearly ordered sets.
	\item $\Delta_{+}$ denotes the category of (possibly empty) finite linearly ordered sets. We will abuse notation by denoting the empty set by $[-1]$.
	\item $\Delta^{\leq n}_{+}$ is the full subcategory of $\Delta_{+}$ spanned by the objects $\{[k]\}_{-1\leq k\leq n}$.
	\item $\Fin^{nu}$ is the category of non-empty finite sets.
	\item $\Fin$ is the category of (possibly empty) finite sets. 
	\item $\Fun(\CC,\CD)$ is the $\infty$-category of functors from $\CC$ to $\CD$.
	\item $\Cat_{\infty}$ is the $\infty$-category of (small) $\infty$-categories.
	\item $\Pr^{L}$ is the $\infty$-category of presentable $\infty$-categories with colimit-preserving morphisms as morphisms.
	\item $\mathds{1}$ denotes the trivial $\infty$-operad, which is the unit object in the monoidal category of $\infty$-operads with respect to composition products.
\end{itemize}