\chapter{Koszul Duality}

In this chapter, we provide the necessary backgrounds on $\infty$-operads and $\infty$-cooperads.
Our main reference will be \cite{BrantnerPhD},  Heuts' upcoming paper, Gaitsgory's notes, \cite{Hadrianphdthesis} and \cite{HA}.
 
In section 3.1, we define an $\infty$-operad in a symmetric monoidal $\infty$-category $\CC$, as an associative algebra object in the category of symmetric sequences in $\CC$.

In section 3.2, we discuss algebras over an operad and some relavant constructions surrounding this area. 

In section 3.3, we introduce the Koszul duality functor. Then we show that the loop of a $\CO$-algebra in tame spectra is automatically a trivial algebra.

In section 3.4, we define the universal enveloping algebra functor from the category of Lie algebras to the category of commutative coalgebras. 
In Section \ref{Commutative coalgebras in tame spectra}, we prove that there is essentially no difference between the category of conilpotent, divided commutative coalgebras and the category of commutative coalgebras.
Moreover, we will show there is an equivalence from tame Lie algebras to tame Hopf algebras. This completes the proof of our first main theorem.

\section{(co)Operads in $\infty$-Categoriees}
In this section, we collect and extend some results from \cite{Francis-Gaitsgory}, \cite{HA} and \cite{HeutsSurvey}.
For convenience of the illustration, let's assume $\CC$ is a presentably symmetric monoidal $\infty$-category throughout this chapter.

Let $\Fin^{\simeq}$ denote the $\infty$-category of finite sets with bijections between them, i.e. it's the core of $\Fin$. 
Note that $\Fin^{\simeq}$ carries a natural symmetric monoidal structure with tensor products given by disjoint unions. 
Moreover, it's the free symmtric monoidal $\infty$-category generated by the one-object category $\{*\}$.
The $\infty$-category $\CP(\Fin^{\simeq})$ of presheaves admits a symmtric monoidal structure given by Day convolution \cite{HA}[Example 2.2.6.9.], hence we can also consider it as a presentably symmetric monoidal $\infty$-category. This suggests an alternative description of $\CC$
\[
\Fun_{\CAlg(\Pr^L)}(\CP(\Fin^{\simeq}), \CC) 
\simeq 
\Fun_{\CAlg(\Cat_{\infty})} (\Fin^{\simeq}, \CC)
\simeq
\Fun(\{*\}, \CC)
\simeq 
\CC.
\]


\begin{definition}
	A \emph{symmetric sequence} in $\CC$ is a functor 
	$$
	A:\Fin^{\simeq} \to \CC.
	$$ We denote the $\infty$-category of symmetric sequences in $\CC$ by $\SSeq(\CC)$
\end{definition}
\begin{remark}
	Informally, one can think of a symmetric sequence $A$ in $\CC$ as a sequence of objects $\{A(n)\}_{n\geq 0}$ in $\CC$ where $A(n):= A((n))$ carries an action of $\Sigma_n$ for each $n$.
	We will refer to $n$ as the \emph{arity} of the symmetric sequence $A$. We will sometimes describe an $\infty$-operad informally by giving a sequence $\{A(0), A(1), \dots \}$.
	
\end{remark}
Let $\otimes$ denote the symmetric monoidal product in $\Pr^L$ defined in \cite{HA}[Chapter 4]. The following lemma points out that the $\infty$-category $\SSeq(\CC)$ of symmetric sequences in $\CC$ is tensored over the $\infty$-category $\SSeq(\Space)$ of symmetric sequences in spaces.
\begin{lemma}
	 Let $\CC$ be a presentably symmetric moniodal $\infty$-category.
	Then we have 
	$$
	\SSeq(\CC) \simeq \SSeq(\Space)\otimes \CC.
	$$
\end{lemma}
\begin{proof}F
First note that 
$
\SSeq(\Space)=\Fun(\Fin^{\simeq}, \Space)\simeq \CP(\Fin^{\simeq}),
$
since $\Fin^{\simeq}$ is isomorphic to its opposite category.
Using \cite{HA}[Proposition 4.8.1.17.], we deduce that there are equivalences
\begin{align*}
	\SSeq(\Space)\otimes \CC 
& \simeq 
\Fun^R(\CP(\Fin^{\simeq})^{op},\CC)\\
&\simeq
(\Fun^{L}(\CP(\Fin^{\simeq}), \CC^{op}))^{op}\\
& \simeq
\Fun(\Fin^{\simeq},\CC^{op}) ^{op}\\
& \simeq \SSeq(\CC).
\end{align*}



\end{proof}

% The category of symmetric sequences $\SSeq(\CD)$ in $\CD$ has a monoidal product $\circ$ called the \emph{composition product}, defined so that the functor $\theta$ is a monoidal functor with respect to the composition monoidal structure in the codomain.   An \emph{ordinary operad} in $\CD$ is then defined as an associative monoid object with respect to this composition product.
%
%For our usage, we need a version of $\infty$-operads enriched in $\CC$. For this we need three ingredients, symmetric sequences, composition products and associative monoid objects. 
%The $\infty$-category $\SSeq(\CC)$ of \emph{symmetric sequences} in $\CC$ is defined as $\Fun(\Fin^{\simeq},\CC)$. 
%Given a symmetric sequence $\CO$ in $\SSeq(\CC)$, it has an associated monad on $\CC$ which is informally on an object $X\in \CC$ given by 
%$$
%\coprod_{n\geq 0} \CO(n)\otimes_{\Sigma_n} X^{\otimes n}.
%$$

\begin{remark}
Note that there is a canonical functor 
\[
\CC \to \SSeq(\CC)
\]
which sends an object $X$ to the symmetric sequence taking value $X$ in arity $1$ and the final object $*$ elsewise. 
The $\infty$-category $\SSeq(\CC)$ of symmetric sequences is equipped with a monoidal structure which corresponds to the composition of functors in the category $\Fun_{\CC/}^{\otimes,\Pr^{L}}(\SSeq(\CC),\SSeq(\CC))$; more precisely, there are equivalences of $\infty$-categories
\begin{align*}
	\Fun_{\CAlg(\Pr^L)_{\CC/}}(\SSeq(\CC),\SSeq(\CC))
	& \simeq 
\Fun_{\CAlg(\Pr^L)}(\CP(\Fin^{\simeq}),\SSeq(\CC))\\
& \simeq
\Fun_{\CAlg(\Cat_{\infty})}(\Fin^{\simeq},\SSeq(\CC))\\
& \simeq \Fun(\{*\}, \SSeq(\CC))\\
& \simeq \SSeq(\CC).
\end{align*}
\end{remark}


It is shown in \cite{Haugsengsymseq} and \cite{BrantnerPhD} that the associated endofunctor of a symmetric sequence $A$ has the explicit form given by
\begin{align*}
		F_{A}: \SSeq(\CC) & \rightarrow \Fun(\CC,\CC)\\
	A        & \mapsto     (\coprod_{n\geq 0} A(n)\otimes_{\Sigma_n} X^{\otimes n})_{h\Sigma_n},
\end{align*}
where $\Sigma_n$ acts on the $n$-th tensor power of $X$ via permutation.
We refer $F_{A}$ as the \emph{Schur functor} associated to $A$.
Given two symmetric sequences $X$ and $Y$ in $\CC$, we will write $X\circ Y$ for their monoidal product on $\SSeq(\CC)$ and refer it as the \emph{composition product}.


\begin{definition}
	\label{inf operads}
	The \emph{$\infty$-category $\Opd(\CC)$ of $\infty$-operads in $\CC$} (resp. $\infty$-category $\Coop(\CC)$ of $\infty$-cooperads in $\CC$) as the $\infty$-category $\Alg(\SSeq(\CC))$ of asscociative algebras (resp. coalgebras) in $\SSeq(\CC)$ with respect to the composition product.
\end{definition}

\begin{remark}
	We denote by $\triv_{\CC}$ the symmetric sequence that takes value the tensor unit $1_\CC$ of $\CC$ at $\{*\}$ and the zero object otherwise. One then checks that $\triv_{\CC}$ corresponds to the identity endofunctor on $\SSeq(\CC)$. Hence $\triv_{\CC}$ serves as a monoidal unit for the composition product in $\SSeq(\CC)$.
	We will refer to $\triv_{\CC}$ as the \emph{trivial $\infty$-operad} in $\CC$.
\end{remark}


\begin{remark}
Although Lurie's approach to $\infty$-operads \cite{HA}[Chapter 2] is seemingly different from the one we give here. 
Haugseng \cite{Haugsengsymseq} and Heine \cite{Hadrianphdthesis} show that Lurie's model is equivalent to the definition of $\infty$-operads in $\Space$. Hence all the results in $\cite{HA}$ transferred smoothly to the setting of $\infty$-operads in spaces.
\end{remark}

In this paper, we will work exclusively with \emph{non-unital $\infty$-operads}. Roughly speaking, a non-unital $\infty$-operad $\CO$ is an operad without nullary operations.
\begin{definition}
	\label{non-unital operad}
	An $\infty$-operad $\CO$ in $\CC$ is \emph{non-unital} if $\CO(0)$ is equivalent to the initial object of $\CC$. 
\end{definition}



\begin{remark}
Consider the canonical embedding $(\Fin^{nu})^{\simeq}\hookrightarrow \Fin^{\simeq}$. One can also define a \emph{non-unital symmetric sequence in $\CC$}  as a functor 
\[
\CO:(\Fin^{nu})^{\simeq}\to \CC.
\]
Hence for any symmetric sequence $\CO$ in $\CC$, 
the underlying non-unital symmetric sequence $\CO_{nu}$ is given by the restriction along $(\Fin^{nu})^{\simeq}\hookrightarrow \Fin^{\simeq}$.
\end{remark}

\begin{remark}
	Under Lurie's model of $\infty$-operads in $\Space$, a non-unital $\infty$-operad $\CO^{\otimes}$ is defined as an $\infty$-operad whose structure map $\CO^{\otimes} \to \Fin_*$ factors through $\Surj$, where $\Surj$ is the subcategory consisting of surjective maps. This is equivalent to our definition in which case $\CO(0)\simeq \emptyset$, the initial object in $\Space$.
\end{remark}

% No need to make this definition
% \begin{definition}
% \label{non-unitalization of operads}
% 		Let $\CO^{\otimes}$ be an $\infty$-operad in $\Space$, we define (as in \cite{HA}[Definition 5.4.4.1.]) the \emph{non-unitalization of $\CO^{\otimes}$} by $\CO^{\otimes}_{nu}:= \Surj\times_{\Fin_*}\CO^{\otimes}$.
% \end{definition}

% Dually, we define $\infty$-cooperads as coassociative coalgebras in $\SSeq(\CC)$ with respect to the composition product.
% \begin{definition}
% 	\label{infty cooperads}
% 	We define the $\infty$-category $\Coop(\CC):=\Alg(\SSeq(\CC)^{op})^{op}$ of \emph{$\infty$-cooperads in $\CC$} as the opposite $\infty$-category of the $\infty$-category of associtive algebras in of $\SSeq(\CC^{op})$.
% \end{definition}








\begin{definition}
	An $\infty$-operad $\CO$ in $\CC$ is \emph{augmented} if it admits a map of $\infty$-operads $\epsilon:\CO\rightarrow \triv_{\CC}$ such that $\epsilon\circ \eta\simeq id_{\mathds{1}}$, where $\eta$ is the unit of $\CO$. We will denote the $\infty$-category of augmented operads and coaugmented cooperads in $\CC$ by $\operatorname{Op}^{aug}(\CC)$ and $\operatorname{coOp}^{aug}(\CC)$, respectively.

\end{definition}

\begin{definition}
	An (non-unital) $\infty$-operad $\CO$ is \emph{connected} \footnote{In \cite{ChingBar}, operads with this property are called reduced. However, reduced operads might have other meaning in other literature.}  if $\CO(1)\simeq 1_{\CC}$. 
\end{definition}
\begin{remark}
A connected $\infty$-operad is canonically augmented with the augmentation $\epsilon:\CO\to \triv_{\CC}$.
\end{remark}  

\begin{remark}
From now on, whenever we say operads (resp. cooperads) we mean non-unital connected $\infty$-operads (resp. $\infty$-cooperads).
\end{remark}


\subsection{Truncations of $\infty$-operads}
% \todo{I'm still expecting Gijs' paper for more details to cite in this subsection}
In this section, we will discuss various notions of "truncations" of operads, which would allow us to produce natural filtrations on algebras (resp. coalgebras) over operads (resp. cooperads).
We will follow \cite{Heuts_Koszul}.

We fix a pointed presentably symmtric monoidal $\infty$-category $\CC$ with tensor product compatible with colimits.
Let $\Fin^{nu}_{\leq n}$ denote the full subcategory of $\Fin^{nu}$ spanned by (non-empty) finite sets with cardinality less or equal to $n$.
We will refer to
$$
\SSeq_{\leq n}(\CC):=\Fun(\Fin^{nu}_{\leq n}, \CC)
$$ 
as the $\infty$-category $\SSeq_{\leq n}(\CC)$ of \emph{$n$-truncated symmetric sequences in $\CC$}.
Similar to $\SSeq(\CC)$, the $\infty$-category of $n$-truncated symmetric sequences in $\CC$ also admits a monoidal product.
We can now define $n$-truncated operads and cooperads in $\CC$.
\begin{definition}
    We define the $\infty$-category of \emph{$n$-truncated operads in $\CC$} as 
    $$
    \operatorname{Op}_{\leq n}(\CC) :=\Alg(\SSeq_{\leq n}(\CC)).
    $$
    Dually, we define 
    $$
     \operatorname{coOp}_{\leq n}(\CC) := \coAlg(\SSeq_{\leq n}(\CC)) 
    $$
    to be the $\infty$-category of \emph{$n$-truncated cooperads in $\CC$}.
\end{definition}

We now explain the relations between operads and their truncations following \cite{Heuts_Koszul}.
Consider the functor 
$$
\zeta_n: \SSeq_{\leq n}(\CC) \to \SSeq(\CC)
$$
given by inserting zero objects in the arity above $n$, one can check that it's both left and right adjoint to the restriction functor
$$
(-)_{\leq n}: \SSeq(\CC) \to 
\SSeq_{\leq n}(\CC).
$$

\begin{lemma}
	\cite{Hadrianphdthesis}[Lemma 2.16]
	If $\CC$ is a pointed symmetric monoidal $\infty$-category with colimits.
	Then the restriction functor
$$
(-)_{\leq n}: \SSeq(\CC) \to 
\SSeq_{\leq n}(\CC)
$$
	is monoidal. Hence it lifts to functors between algebras and coalgebras,
	such that the following squares commute
\[
\begin{tikzcd}
%%%% The nodes %%%%%%%%%
	\Opd(\CC)  & 
	\operatorname{Op}_{\leq n}(\CC)\\
	\SSeq(\CC) & \SSeq_{\leq n}(\CC)
%%%% Now the arrows %%%%
	\arrow[from=1-1, to= 1-2, "\rho_n"]
	\arrow[from=1-1, to=2-1, "\oblv" left]
	\arrow[from=1-2, to=2-2, "\oblv'" ]
	\arrow[from=2-1, to= 2-2, "(-)_{\leq n}"]
\end{tikzcd}
\]
and
\[
\begin{tikzcd}
%%%% The nodes %%%%%%%%%
	 \Coop(\CC)  & 
	\operatorname{coOp}_{\leq n}(\CC)\\
	\SSeq(\CC) & \SSeq_{\leq n}(\CC)
%%%% Now the arrows %%%%
	\arrow[from=1-1, to= 1-2, "\rho^n"]
	\arrow[from=1-1, to=2-1, "\oblv" left]
	\arrow[from=1-2, to=2-2, "\oblv'" ]
	\arrow[from=2-1, to= 2-2, "(-)_{\leq n}"].
\end{tikzcd}
\]
where the vertical arrows are forgetful functors.
\end{lemma}

The Lemma above has an immediate corollary.
\begin{corollary}
\label{functors between operads adn their truncations}
    The functor 
    $$
    \zeta_n: \SSeq_{\leq n}(\CC) \to \SSeq(\CC)
    $$
    admits both a lax monoidal and an oplax monoidal strucutre. 
\end{corollary}
    The lax monoidal structure on $\zeta_n$ induces a functor
    $$
    \tau_{n}:\operatorname{Op}_{\leq n}(\CC)
    \to 
    \Opd(\CC)
    $$
    which is right adjoint to $\rho_{n}$. On the other hand, the functor $\rho_n:\Opd(\CC) \to \operatorname{Op}_{\leq n}(\CC)$ preserves limits and filtered colimits, hence by the adjoint functor theorem \cite[Corollary 5.5.2.9.]{HTT} it admits a left adjoint 
    $$
    \varphi_n: \operatorname{Op}_{\leq n}(\CC)
    \to 
    \Opd(\CC).
    $$
    To sum up, we have a diagram consisting of functors described above
\begin{equation}
\label{Op-Op_n adjunction}
	\begin{tikzcd}
	%%%% The nodes %%%%%%%%%
 \Opd(\CC) & \operatorname{Op}_{\leq n}(\CC) 
%%%% Now the arrows %%%%
	\arrow[from=1-2, to= 1-1, bend right, shift right = 3, "\varphi_n" above]
	\arrow[from=1-2, to=1-1, bend left, shift left = 3, "\tau_n" below]
	\arrow[from=1-1, to=1-2, "\rho_n"]
\end{tikzcd}
\end{equation}
where the left adjoints are above the right adjoints.
    

    
    Dually, the oplax structure on $\zeta_n$ induces a functor
    $$
    \tau^{n}:\operatorname{coOp}_{\leq n}(\CC)
    \to 
    \Coop(\CC)
    $$
    which is left adjoint to the restriction $\rho^n$. The restriction $\rho^n$ admits a right adjoint
    $$
    \varphi^{n}: \operatorname{coOp}_{\leq n}(\CC)
    \to 
    \Coop(\CC)
    $$
    and we have the following diagram which summarizes the situation for cooperads
    
    \begin{equation}
\label{coOp-coOp_n adjunction}
	\begin{tikzcd}
	%%%% The nodes %%%%%%%%%
\Coop(\CC) & \operatorname{coOp}_{\leq n}(\CC)
%%%% Now the arrows %%%%
	\arrow[from=1-2, to= 1-1, bend right, shift right = 3, "\tau^n" above]
	\arrow[from=1-2, to=1-1, bend left, shift left = 3, "\varphi^n" below]
	\arrow[from=1-1, to=1-2, "\rho^n"].
\end{tikzcd}
\end{equation}


\begin{notation}
	We will abuse notation by writing 
	\begin{itemize}
	    \item 
	$\CO \to \tau_{n} \CO$
	for the unit of the bottom adjunction in \eqref{Op-Op_n adjunction};
	\item $
	\varphi_n \CO \to \CO
	$ for the counit of the top adjunction in \eqref{Op-Op_n adjunction};
	\item 	$\tau^n \CQ \to \CQ$ for the counit of the top adjunction in \eqref{coOp-coOp_n adjunction};
	\item 	$
	\CQ \to \varphi^n \CQ
	$
	for the unit of the bottom adjunction in \eqref{coOp-coOp_n adjunction}.
	\end{itemize}
\end{notation}

\begin{remark}
\begin{itemize}
	\item For any $\infty$-operad $\CO$ in $\CC$, there is a sequence of operad maps
	\[
	\varphi_n\CO \to \CO \to \tau_n \CO.
	\]
	The operad $\tau_n\CO$ is the $n$-truncation of $\CO$, i.e., operations of arity higher than $n$ are set to zero.
	The map $\CO \to \tau_n \CO$ is terminal among those operad maps from $\CO$ that are equivalences in arities up to $n$.
	The operad $\varphi_n\CO$ is equivalent to $\CO$ in arities up to $n$, while higher arity operations are "freely generated" by operations of arity less or equal to $n$.
	Hence, the map $\varphi_n\CO \to \CO$ is initial among those operad maps to $\CO$ that are equivalences in arities up to $n$.

	\item Similarly, for any $\infty$-cooperad $\CQ$ in $\CC$, there is a sequence cooperad maps
	\[
	\tau^n \CQ \to \CQ \to \varphi^n \CQ.
	\]
	The cooperad $\tau^n\CQ$ is the $n$-truncation of $\CQ$, i.e., cooperations of arity higher than $n$ are set to zero. The map $\CQ \to \varphi^n\CQ$ is terminal among those cooperad maps from $\CO$ that are equivalences in arities up to $n$.
    The cooperad $\varphi^n \CQ$ is "cofreely generated" by cooperations of arities up to $n$.
    The map $\tau^n\CQ \to \CQ$ is terminal among those cooperad maps from $\CQ$ that are equivalences in arities up to $n$.
\end{itemize}
\end{remark}

\begin{remark}
	Using the remark above, we see that there is a direct system for an $\infty$-operad $\CO$, 
\begin{equation}
\varphi_1\CO \to \varphi_2\CO \to \cdots \to \CO.
\end{equation}
Similarly, there is an inverse system for the $n$-truncation of $\infty$-operads.
\begin{equation}
\cdots \to 
\tau_2 \CO
\to 
\tau_1 \CO
\to 
\CO
\end{equation}
\end{remark}
%\label{telescope for O_n}
%	\begin{tikzcd}
%	%%%% The nodes %%%%%%%%%
%	\vdots & \\
%	\varphi_3 \CO & \\
%	\varphi_2 \CO & \\
%	\varphi_1 \CO & \CO
%	%%%% Now the arrows %%%%
%	\arrow[from=2-1, to= 1-1]
%	\arrow[from=2-1, to= 4-2]
%	\arrow[from=3-1, to=2-1]
%	\arrow[from=3-1, to=4-2]
%	\arrow[from=4-1, to=3-1]
%	\arrow[from=4-1, to=4-2]
%	\end{tikzcd}



\begin{proposition}
\label{Operad as a colimits}
	Let $\CO$ be an $\infty$-operad in $\CC$, 
	then there is an equivalence
	\[
	\CO \simeq
	\colim_n \varphi_n \CO
	\]
	in $\Opd(\CC)$.
\end{proposition}
\begin{proof}
    It suffices to check the equivalence aritywise.
    Fix an arity $k$, then the direct system of objects 
    $$
    \varphi_1\CO(k) \to \varphi_2\CO(k) \to \cdots    
    $$
    becomes constant for $n \geq k$.
\end{proof}





\section{Algebras over Operads}
In this section, we will review the definition of algebras over an operad.
Recall that a symmetric sequence $\CO\in\SSeq(\CC)$ acts on $\CC$ via its Schur functor $F_{\CO}$.
% \[
% \SSeq(\CC)
% \simeq
% \Fun_{\CAlg(\Pr^L)_{\CC/}}(\SSeq(\CC)\SSeq(\CC)).
% \]
If $\CO$ is an $\infty$-operad, then its Schur functor $F_\CO$ is a monad, hence we can consider the $\infty$-category $\LMod_{F_\CO}$ of left $\CO$-modules in $\CC$.
%\begin{align*}
%SSeq(\CC) \times \CC & \to \CC\\
%(\CO, X)   & \mapsto  \coprod_{n\geq 0} (\CO(n)\otimes X^{\otimes n})_{h\Sigma_n}.
%\end{align*}
\begin{definition}
	\label{algebras over an operad}
	Let $\CO$ be an $\infty$-operad in $\CC$. The $\infty$-category of $\CO$-algebras is $\Alg_{\CO}(\CC):= \LMod_{F_\CO}(\CC)$.
\end{definition}

For a symmetric sequence $A$ in $\CC$, we consider the following \emph{extended power functors}
$$
D^{A}_n(X): = (A(n)\otimes X^{\otimes n })_{h\Sigma}, 
\quad
D_{A}^n(X): = (A(n)\otimes X^{\otimes n })^{h\Sigma}
$$
Informally, an $\CO$-algebra $X$ in $\CC$ is equipped with maps
$$
D^{\CO}_n(X) \to X
$$
for each $n\geq 1$ and homotopy coherent data that keeps track of the associativity.

\begin{definition}
Let $F:\CC\to \CC$ be an endofunctor.
	We define the $\infty$-category $\Alg_{F}(X)$ of \emph{$F$-algebras in $\CC$} to be the pullback of the following diagram in $\Cat_{\infty}$
\[
\begin{tikzcd}
%%%% The nodes %%%%%%%%%
	\Alg_{F}(X) & 
	\Fun(\Delta^1, \CC)\\
	\CC & 
	\CC \times \CC
%%%% Now the arrows %%%%
	\arrow[from=1-1, to= 1-2]
	\arrow[from=1-1, to=2-1]
	\arrow[from=1-2, to=2-2, "(ev_0\text{,} ev_1)"]
	\arrow[from=2-1, to= 2-2, "(F\text{,} id)"].
\end{tikzcd}
\]
That is, $\Alg_{F}(X)$ is the subcategory of $\CC$ consisting of objects $X$ equipped with a morphism $F(X)\to X$.
\end{definition}

Proposition \ref{Operad as a colimits} motivates us to ask the following question: Can we write an $\CO$-algebra as the limit of $\varphi_k\CO$-algebras as in the case of Postnikov decomposition of a simply-connected space? 

Heuts answers the question in the following theorem.
\todo{I checked the proof Gijs' paper and I think he didn't use any property of stable $\infty$-category how do I say this then?}
\begin{theorem}
\cite[Theorem 4.1]{Heuts_Koszul}
\label{Thm 4.1 of Heuts Koszul Duality paper}
For each $n \geq 2$, the commutative square of $\infty$-categories
\[
\begin{tikzcd}
%%%% The nodes %%%%%%%%%
	\Alg_{\varphi_n\CO}(\CC) & 
	\Alg_{D^{\CO}_{n}}(\CC)\\
	\Alg_{\varphi_{n-1}\CO} (\CC)  & 
	\Alg_{D_{n}^{\varphi_{n-1}\CO}}(\CC)
%%%% Now the arrows %%%%
	\arrow[from=1-1, to= 1-2]
	\arrow[from=1-1, to=2-1]
	\arrow[from=1-2, to=2-2]
	\arrow[from=2-1, to= 2-2].
\end{tikzcd}
\]
is a pullback square. Furthermore, the natural map
$$
\operatorname{Alg}_{\mathcal{O}}(\mathrm{C}) \rightarrow \lim _{n} \operatorname{Alg}_{\varphi_{n} \mathcal{O}}(\mathcal{C})
$$
is an equivalence of $\infty$-categories.
\end{theorem}

\begin{remark}
Theorem \ref{Thm 4.1 of Heuts Koszul Duality paper} has the following informal interpretation: suppose $X$ is a $\phi_{n-1}\CO$-algebra, then to specify a $\varphi_n\CO$-algebra on $X$, it suffices to equip $X$ with a multiplication map $\mu_n: D^{\CO}_n(X)\to X$ so that it is compatible with the $\phi_{n-1}\CO$-algebra structure maps. We refer the readers to [Heuts] for further details.
\end{remark}

We end this section with the example of commutative operad.
\begin{example}
	Consider the \emph{symmetric algebra functor} on $\CC$,
	\begin{align*}
		\Sym:\CC & \to  \CC\\
		X & \mapsto \amalg_{n\geq 1} (X^{\otimes n})_{h\Sigma_n}.
	\end{align*}
	One can easily check $\Sym$ is the Schur functor of the symmetric sequence $\Com:=(1_\CC,1_\CC,\dots)$, where $1_\CC$ is the tensor unit of $\CC$.
	Since $\Sym$ is a monad on $\CC$, $\Com$ is an operad, which is called the \emph{commutative operad} in $\CC$.
% Not necessairly anymore
% 	Similarly, the \emph{non-unital symmetric algebra functor} $\Sym^{\geq 1}:\CC \to \CC$ given by $X\mapsto \amalg_{n\geq 1} (X^{\otimes n})_{h\Sigma_n}$ defines a non-unital $\infty$-operad in $\CC$ which we will denote by $\Com_{nu}$.
\end{example}



\section{Coalgebras over cooperads and Koszul Duality}
In this section, we introduce coalgebras over an cooperad and discuss Koszul duality for operads in the sense of \cite{Ginzburg-Kapranov}. 

We first discuss the bar-cobar duality for $\infty$-operads. The general form of bar-cobar duality is exhibited in the form of associative algebras and coassociative coalgebras in a "nice" $\infty$-category in \cite{HA}.
\begin{proposition}[\cite{HA}[Remark 5.2.2.19.] 
Let $\CD$ be a pointed monoidal $\infty$-category with geometric realizations of simplicial objects and totalizations of cosimplicial objects. Then we have an adjunction
\[
\adj{\Barconstruction}{\Alg^{\operatorname{aug}} (\CD)}{\coAlg^{\operatorname{aug}}(\CD)}{\Cobar}.
\]
\end{proposition}
Let $\CD= \SSeq(\CC)$ then we obtain an adjunction between (augmented) operads and (coaugmented) cooperads.
	$$
	\adj{\Barconstruction}{\Opd^{aug}(\CC)}{\Coop^{aug}(\CC)}{\Cobar}.
	$$
Explicityly, the bar construction $\Barconstruction(\CO)$ of an operad $\CO$ is computed as the geometric realization of the simplicial object
	\[
		\xymatrix{ \mathds{1}\ar@<0ex>[r]  &  
		\CO \ar@<1ex>[r] \ar@<-1ex>[r] \ar@<1ex>[l] \ar@<-1ex>[l]  &  \CO\circ \CO \ar@<2ex>[l] \ar@<0ex>[l] \ar@<-2ex>[l] \cdots}
\]
and we will write 
$$
B\CO := |\Barconstruction(\triv, \CO, \triv)_{\bullet}|.
$$
	Similarly, the cobar construction $\Cobar(\CQ)$ of a cooperad $\CQ$ is computed as the totalization of the cosimplicial object
	\[
		\xymatrix{ \mathds{1} \ar@<1ex>[r] \ar@<-1ex>[r] &  
		\CQ \ar@<0ex>[l] \ar@<2ex>[r] \ar@<0ex>[r] \ar@<-2ex>[r] & \CQ\circ \CQ \ar@<1ex>[l] \ar@<-1ex>[l] \cdots}.
\]
and we will write 
$$
C\CQ := |\Cobar(\triv, \CQ, \triv)_{\bullet}|.
$$



\begin{definition}
	An $\infty$-operad $\CO\in \Opd^{aug}(\CC)$ is \emph{Koszul} if the unit map
	$$
	\CO\rightarrow \Cobar\circ \Barconstruction(\CO)
	$$
	is an equivalence.
\end{definition}

If we restrict to $\infty$-operads in a stable $\infty$-category, then every connected $\infty$-operad is Koszul.

\begin{proposition}
\cite[Proposition 3.4]{Heuts_Koszul}
\label{Connected operads are Koszul}
Let $\CC$ be a presentbly stable symmtric monoidal $\infty$-category. Then the bar-cobar adjunction
	$$
	\adj{\Barconstruction}{\Opd^{aug}(\CC)}{\Coop^{aug}(\CC)}{\Cobar}.
	$$
	restricts to an equivalence on the  $\infty$-categories of connected operads and cooperads.
\end{proposition}


We now introduce functors that will play important roles throughout the rest of the thesis. 
Let $\CO$ be a connected operad with its canonical augmentation $\epsilon:\CO\rightarrow \mathds{1}$.
Then the restriction along $\epsilon$ gives a functor 
$$
\trivial_{\CO}: \CC\simeq\Alg_{\mathds{1}}(\CC)\rightarrow \Alg_{\CO}(\CC)
$$
which we will call the \emph{trivial $\CO$-algebra functor}.
The trivial $\CO$-algebra functor admits a left adjoint $\cot_{\CO}:\CC\rightarrow \Alg_{\CO}(\CC)$. The functor $\cot_{\CO}$ will be called the \emph{cotangent fiber functor}.

Similarly, the restriction along the unit map $\eta:\mathds{1}\rightarrow \CO$ gives a functor
$$
\oblv_{\CO}: \Alg_{\CO}(\CC)
\rightarrow 
\CC\simeq\Alg_{\mathds{1}}(\CC)$$
which we will refer as the \emph{$\CO$-forgetful functor}.
The $\CO$-forgetful functor admits a left adjoint $\Free_{\CO}:\CC\rightarrow \Alg_{\CO}(\CC)$ which we will call the \emph{free $\CO$-algebra functor}.
    
The composite
$$
\mathds{1}\xrightarrow{\eta} \CO \xrightarrow{\epsilon} \mathds{1}
$$
is equivalent to the identity, hence we obtain
$$
\cot_{\CO} \circ \Free_{\CO} \simeq \id
$$
and 
$$
\oblv_{\CO}\circ \trivial_{\CO} \simeq \id.
$$
The situation can be summarized in the following diagram
\[
\begin{tikzcd}
	%%%% The nodes %%%%%%%%%
\CC & \Alg_{\CO}(\CC) & \CC
%%%% Now the arrows %%%%
	\arrow[from=1-1, to= 1-2, shift left = 1, "\Free_{\CO}" above]
	\arrow[from=1-2, to=1-1, shift left = 1, "\oblv{\CO}" below]
	\arrow[from=1-2, to= 1-3, shift left = 1, "\cot_{\CO}" above]
	\arrow[from=1-3, to=1-2, shift left = 1, "\trivial_{\CO}" below]
\end{tikzcd}
\]
in which the left adjoints are above their right adjoints.

The following corollary of the Barr-Beck-Lurie theorem says that any $\CO$-algebra $X$ is canonically equivalent to the geomectric realization of a simplicial object of $\CO$-algebras.
\begin{proposition}
\label{free resolutino of O-algebras}
Let $X$ be an $\CO$-algebra, then it can be resolved as the geometric realization of the simplicial objects $(\Free_{\CO}\circ \oblv_{\CO})_{\bullet + 1}X$, i.e. 
\begin{equation}
\label{(3.5)}
    X \simeq
|(\Free_{\CO}\circ \oblv_{\CO})_{\bullet + 1}X|
\simeq |\Barconstruction(\CO, \CO, X)_{\bullet}|.
\end{equation}
\end{proposition}
\begin{proof}
    This follows from the fact that the adjuntion $(\Free_{\CO}, \oblv_{\CO})$ is monadic and Corollary \ref{Cor of Barr-Beck-Lurie theorem}.
    
\end{proof}
Applying $\cot_{\CO}$ to both sides of \eqref{(3.5)} yields a formula for computing the cotangent fiber of $X$. 
\begin{corollary}
\cite[Proposition 4.4]{Heuts_Koszul}
The cotangent fiber of an $\CO$-algebra $X$ can be computed as 
$$
\cot_{\CO} \simeq 
|(\oblv_{\CO}\circ \Free_{\CO})_{\bullet}\oblv_{\CO}X|
$$
where the geometric realization is taken in the underlying $\infty$-category $\CC$.
\end{corollary}

The comonad $\cot_{\CO}\circ \trivial_{\CO}$ then admits an explicit description.
\begin{proposition}
\label{cot triv is F_BO}
\cite[Proposition 4.5]{Heuts_Koszul}
The comonad $\cot_{\CO}\circ \trivial_{\CO}$ is naturally equivalent to the comonad $F_{B\CO}$ associated with the cooperad $B\CO$.
\end{proposition}

We are now ready to define the $\infty$-category of divided power, conilpotent coalgebras over a cooperad.
\begin{definition}
\label{coalgebras over a cooperad}
	Let $\CQ$ be an $\infty$-cooperad in $\CC$, we define the $\infty$-category of divided power, conilpotent coalgebras over a cooperad $\CQ$ as 
	$$
	\coAlg^{\divpow, \nil}_{\CQ}(\CC):= \operatorname{LcoMod}_{F_{\CQ}}(\CC).
	$$
\end{definition}

Analogous to the case of algebras over an operad, we have two pair of adjunctions 
\[
\begin{tikzcd}
	%%%% The nodes %%%%%%%%%
\CC & \coAlg^{\divpow, \nil}_{\CQ}(\CC) & \CC
%%%% Now the arrows %%%%
	\arrow[from=1-1, to= 1-2, shift left = 1, "\cofree^{\nil}_{\CQ}" above]
	\arrow[from=1-2, to=1-1, shift left = 1, "\oblv^{\nil}_{\CQ}" below]
	\arrow[from=1-2, to= 1-3, shift left = 1, "\operatorname{Prim}^{\nil}_{\CQ}" above]
	\arrow[from=1-3, to=1-2, shift left = 1, "\trivial^{\nil}_{\CQ}" below]
\end{tikzcd}
\]
where the right adjoints are above the left adjoints,
and compositions of the horizontal functors are equivalent to the identity functor on $\CC$.
Moreover,we can resolve any divided power, conilpotent coalgebra by a totalization of cofree coalgebras.
\begin{proposition}
Let $X$ be a divided power, conilpoent $\CQ$-coalgebra, then it can be resolved as the totalization of the cosimplicial object $(\cofree_{\CQ}\circ \oblv_{\CQ})^{\bullet + 1}X$, i.e. 
\begin{equation}
\label{(3.5)}
    X \simeq
|(\cofree_{\CQ}\circ \oblv_{\CQ})^{\bullet + 1}X|
\simeq \Tot \Cobar(\CQ, \CQ, X)^{\bullet}.
\end{equation}
\end{proposition}




As a consequence of Proposition \ref{cot triv is F_BO}, the functor $\cot_{\CO}$ can be factored as 
$$
\Alg_{\CO}(\CC) \xrightarrow{\indec^{\nil}_{\CO}}
\coAlg^{\divpow, \nil}_{B\CO}(\CC)
\xrightarrow{\oblv_{B\CO}^{\nil}}
\CC.
$$
The functor $\indec^{\nil}_{\CO}$ admits a right adjoint $\Prim_{B\CO}$;
for $Y \in \coAlg^{\divpow, \nil}_{B\CO} $, the \emph{primitives} $\Prim_{B\CO}^{\nil}(Y)$ of $Y$ can be computed explicitly as \cite[Lemma 4.7]  {Heuts_Koszul}
$$
\Prim_{B\CO}^{\nil}(Y) \simeq \Tot ( (\trivial_{B\CO}\circ \cofree^{\nil}_{B\CO})^{\bullet} \oblv^{nil}_{B\CO} Y).
$$


We now introduce other types of coalgebras over a cooperad $\CQ$. 
As the dual notion of algebras over operad, one would expect a colagebra $X$ over a cooperad $\CQ$ to be an object $X$ in $\CC$ equipped with maps
$$
\epsilon_n: X \to  (\CQ(n) \otimes X^{\otimes n})^{h\Sigma_n}
$$
that are homotopy-coherently coassociative. Recall we have defined a cooperad $\CQ$ to be a comonoid in $\SSeq(\CC)$ with respect to the composition product, hence it's a monoid in $\Fun(\CC^{op}, \CC^{op})$. 
\begin{definition}
    \label{coalgebras over cooperads}
    We define the $\infty$-category of \emph{coalgebras over the cooperad $\CQ$} to be 
    $$
    \coAlg_{\CQ}(\CC) := \Alg_{\CQ}(\CC^{op})^{op}.
    $$
\end{definition}

We also have the following decomposition result for $\CQ$-coalgebras, which will be used in the proof of the essential surjectivity of the functor $C_{\operatorname{tame}}$.
\begin{proposition}
\cite[Theorem 4.12]{Heuts_Koszul}
\label{inductive construction of coalgebras}
    For $n\geq 2$, the following commutative square of $\infty$-categories 
\[
\begin{tikzcd}
%%%% The nodes %%%%%%%%%
	\coAlg_{\varphi^n \CQ}(\CC) & 
	\coAlg_{D_n^{\varphi^{n}\CQ}}(\CC)\\
	\coAlg_{\varphi^{n-1} \CQ}(\CC) & 
	\coAlg_{D_n^{\varphi^{n-1}\CQ}}(\CC)
%%%% Now the arrows %%%%
	\arrow[from=1-1, to= 1-2]
	\arrow[from=1-1, to=2-1]
	\arrow[from=1-2, to=2-2]
	\arrow[from=2-1, to= 2-2]
\end{tikzcd}
\]
is a pullback square. Moreover, the natural map
$$
\coAlg_{\CQ}(\CC) \to \lim_n \coAlg_{\varphi^n\CQ}(\CC)
$$
is an equivalence.
\end{proposition}

\section{Commutative Coalgebras in $\Sp^{\geq r}_{\tame}$}
\label{Commutative coalgebras in tame spectra}

In this section, we define and study the $\infty$-category of commutative coalgebras in the category of $r$-tame spectra. 
The main aim is to show the $\infty$-category of divided power, conilpotent commutative coalgebras is equivalent to the $\infty$-category of commutative coalgebras in $\Sp^{\geq r}_{\tame}$.
We start by collecting some results from \cite{LurieEllipticI}.

\begin{definition}
    Let $\mathcal{C}$ be a symmetric monoidal $\infty$-category. The $\infty$-category $\coCAlg(\CC)$ of commutative coalgebras in $\CC$ is defined to be
    $
    \coCAlg(\CC):= \CAlg(\CC^{op})^{op}.
    $
\end{definition}


\begin{proposition}
\label{Cor 3.1.5. Ellip}
\cite[Corollary 3.1.5]{LurieEllipticI}
	Let $\mathcal{C}$ be a symmetric monoidal $\infty$-category. Suppose that the $\infty$-category $\mathcal{C}$ is presentable and that the tensor product functor $\otimes: \mathcal{C} \times \mathcal{C} \rightarrow \mathcal{C}$ preserves colimits separately in each variable. Then the forgetful functor $\coCAlg(\mathcal{C}) \rightarrow \mathcal{C}$ admits a right adjoint $\cofree: \mathcal{C} \rightarrow \coCAlg(\mathcal{C})$.
\end{proposition}


\begin{corollary}
The forgetful functor $\coCAlg(\Sp_{\tame}^{\geq r})\to \Sp^{\geq r}_{\tame}$ admits a right adjoint functor which we will denote by $\cofree_{\tame}$.
\end{corollary}
\begin{proof}
	 Note that $\Sp^{\geq r}_{\tame}$ is presentable as it's a localization of the presentable $\infty$-category $\Sp^{\geq r}$. The tensor product $\hat{\otimes}$ in $\Sp_{\tame}^{\geq r}$ preserves colimits separately in each variable. The existence of the cofree commutative coalgebra functor is ensured by Proposition \ref{Cor 3.1.5. Ellip}.
\end{proof}

We now define the $\infty$-category of divided power, conilpotent coalgebras in $\Sp_{\tame}^{\geq r}$. Note that $\Sp_{\tame}^{\geq r}$ is a non-unital symmetric monoidal $\infty$-category, so there is no commutative cooperad in $\Sp_{\tame}^{\geq r}$. However, we can define a comonad which comes from the commutative cooperad in $\Sp$ via tame localization.

The comonad $F_{\Com}$ associated to the commutative cooperad in $\Sp$ is given by
$$
F_{\Com}(X) := \coprod_{n\geq 1} (X^{\otimes n})_{h\Sigma_n}.
$$
$F_{\Com}$ restricts to a comonad on the $\infty$-category $\Sp^{\geq r}$ of $r$-connective spectra.
Apply the localization functor $L_{\tame}$, we obtain a comonad $L_{\tame}F_{\Com}$ on $\Sp_{\tame}^{\geq r}$, given by 
$$
F_{\Com}(X) := \coprod_{n\geq 1} L_{\tame}(X^{\otimes n})_{h\Sigma_n}.
$$
as $L_{\tame}$ is colimit-preserving and symmetric monoidal.
\begin{definition}
    The $\infty$-category of \emph{divided power, conilpotent coalgebras in tame spectra} is defined to be the $\infty$-category of left comodules over the comonad $L_{\tame}F_{\Com}$
    \[
    \coCAlg^{\divpow, \nil}(\Sp_{\tame}^{\geq r}):= \operatorname{LcoMod}_{L_{\tame}F_{\Com}}(\Sp_{\tame}^{\geq r}).
    \]
\end{definition}



\todo{How to define this comparison functor $\zeta$?}

We now show that $\coCAlg(\Sp_{\tame}^{\geq r})$ is actually equivalent to $\coAlg^{\divpow, \nil}_{\Com}(\Sp_{\tame}^{\geq r})$.

\begin{proposition}
\label{all coalgebras are equivalent}
	The comparison functor
\[
\zeta: 
\coCAlg^{\divpow,\nil}(\Sp^{\geq r}_{\tame}) \to
\coCAlg(\Sp^{\geq r}_{\tame})
\]
is an equivalence of $\infty$-categories.
\end{proposition}
\begin{proof}
% 	We prove that the forgetful functor
% 	\[
% 	\operatorname{res'}: \coCAlg^{\dp,\nil}(\Sp^{\geq r}_{\tame})
% 	\to 
% 	\coCAlg^{\nil}(\Sp^{\geq r}_{\tame})
% 	\]
% 	is an equivalence of $\infty$-categories.
\todo{This proof needs to be updated}
	Since $\coCAlg^{\divpow,\nil}(\Sp^{\geq r}_{\tame})$ is comonadic over $\Sp^{\geq r}_{\tame}$, it suffices to check the comonad $U\circ \cofree^{\nil}$ induced from the forgetful-cofree adjunction
	$$
	\adj{U}{\coCAlg^{\nil}(\Sp^{\geq r}_{\tame})}{\Sp^{\geq r}_{\tame}}{\cofree^{\nil}}
	$$
	agrees with the comonad $F_{\Com}$. Indeed, we have a natural map
	$$
	\gamma: \bigoplus_{n\geq 1} (X^{\hat{\otimes}n})_{h\Sigma_n}
	\to 
	\prod_{n\geq 1} (X^{\hat{\otimes}n})^{h\Sigma_n};
	$$
	Fix an integer $k$, the $k$-truncation of $\gamma$ and there exists a maximum integer $l$ such that $X^{\hat{\otimes}(l+1)}$ is $(k+1)$-connective.
	Hence the $k$-truncation
	$$
	\tau_k (\gamma): \bigoplus_{n\geq 1}^l (X^{\hat{\otimes}n})_{h\Sigma_n}
	\to 
	\prod_{n\geq 1}^l (X^{\hat{\otimes}n})^{h\Sigma_n}
	$$
	 is an equivalence for every $k$, which implies $\gamma$ is an equivalence.
	\end{proof}
	
For a general symmetric monoidal $\infty$-category $\CC$, we don't know any explicit identification of the cofree coalgebra comonad. However, divided power, conilpotent commutative coalgebras we know the cofree comonad is given by the symmetric algebra functor.
\begin{corollary}
	The underlying spectrum of the cofree commutative coalgebra generated by $X\in \Sp^{\geq s}_{tame}$ is given by
	 $$
	 \cofree(X) \simeq \bigoplus_{n\geq 1} (X^{\hat{\otimes}n})_{h\Sigma_n}= \Sym(X) .
	 $$
\end{corollary}

%Combining the two adjunctions, we obtain a diagram:
%\[
%\begin{tikzcd}
%\CC \arrow[r,shift left=1,"\free_{\CO}"]  & \Alg_{\CO}(\CC) \arrow[r,shift left=1,"\TAQ_{\CO}"] \arrow[l,shift left=1,"\oblv_{\CO}"]  & \CC \arrow[l,shift left=1,"\trivial_{\CO}"] 
%\end{tikzcd}
%\]
%where left adjoints are on top and both composites are equivalent to $id_{\CC}$.
%% Denote $F_{\CO}:\CC\rightarrow \CC$ its associated monad.
%\subsection{Conilpotent coalgebras over a cooperad}
%Let $\CP$ be a connected cooperad.
%Let $F_{\CP}$ be the comonad associated to $\CP$. We denote the category of coalgebras over $F_{\CP}$ by $\coAlg^{nil,dp}_{\CP}(\CC)$. An object $X\in \coAlg^{nil,dp}_{\CP}(\CC)$ is equipped with a structure map:
%$$
%X\rightarrow \bigoplus_{n\geq 1}\big( \CP(n)\otimes X^{\otimes n}\big)_{h\Sigma_n}.
%$$
%We will refer to $\coAlg^{nil,dp}_{\CP}(\CC)$ \textit{the category of conilpotent divided power $\CP$-coalgebras}. 
%Dual to the algebra case, we have a diagram of two pair of adjoint functors:
%\[
%\begin{tikzcd}
%\CC \arrow[r,shift left=1,"\cofree^{nil,dp}_{\CP}"]  & \coAlg^{nil,dp}_{\CP}(\CC) \arrow[r,shift left=1,"\Prim^{nil,dp}_{\CP}"] \arrow[l,shift left=1,"\oblv^{nil,dp}_{\CP}"]  & \CC \arrow[l,shift left=1,"\trivial^{nil,dp}_{\CP}"] 
%\end{tikzcd}
%\]
%where functors at the bottom are left adjoints and both composites are equivalent to $id_\CC$.
%The adjective conilpotent stands for the direct sum and the adjective divided power corresponds to coinvariants in $F_{\CP}$.
%
%
%
%\begin{theorem}[\cite{Francis-Gaitsgory}, \cite{HeutsSurvey}]
%The functor $\TAQ_{\CO}$ factors as 
%$$
%\Alg_{\CO}(\CC) \xrightarrow{B_\CO} \coAlg^{nil,dp}_{\Barconstruction(\CO)}(\CC)\xrightarrow{\oblv_{\Barconstruction (\CO)}} \CC.
%$$	
%
%\end{theorem}
\section{The spectral Lie operad}
In this section, we introduce the operad that will be central to this thesis, the spectral Lie operad. 
We first need to provide some background on partition posets. 
Consider the set $\mathbf{P}(n)$ of partitions (i.e. equivalence relations) on the finite set $\{1, \cdots, n \}$. Note that $\mathbf{P}(n)$ is a poset with the minimal element the trivial partition and the maximal element the discrete partition.
Write $\mathbf{P}^{+}(n)$ (resp. $\mathbf{P}^{-}(n)$ ) for the subposet of $\mathbf{P}(n)$ obtained by deleting the minimal (resp. maximal) partition.

	The Goodwillie derivative $\partial_* \id$ of the identity functor $\id: \pSpace \to \pSpace$ forms a symmetric sequence in $\Sp$; each $\partial_n \id$ is a $\Sigma_n$-spectrum. In \cite{JohnsonDerivative} and \cite{Arone-Mahowald}, it was shown that
	$$
	\partial_n \id \simeq \BD(\Sigma^{\infty} K_n),
	$$
	where 
	$
	K_{n}:=|\mathbf{P}(n)| /\left(\left|\mathbf{P}^{+}(n)\right| \cup\left|\mathbf{P}^{-}(n)\right|\right)
	$
	and $\BD$ stands for the Spanier-Whitehead dual. 
	Let $\Com$ denote the commutative cooperad in $\Sp$.
	Ching \cite{ChingBar} identified the derivatives of the identity functor as the cobar construction of the commutative cooperad.
	\begin{proposition}
	\cite[Remark 8.9]{ChingBar}
	The cobar construction of the commutative cooperad is equivalent to the derivatives of the identity functor
	$$
	\Cobar(\Com)\simeq \partial_* \id.
	$$
	\end{proposition}
	
	\begin{definition}
	\label{Shifted Spectral Lie Operad}
	    	We define the \emph{shifted spectral Lie operad} to be
	    	$$
	    	\BL:= \partial_* \id.
	    	$$
	\end{definition}

	The name comes from the following computation. Let $\Lie$ denote the Lie operad in $\Ab$, then by the computation in \cite{ChingBar}[Example 9.50]
	$$
	H_{*}\BL(n) \cong 	
	\begin{cases}
	(\Lie(n)[1])\otimes (\BZ[-1])^{\otimes n}& \text{ if $*=1-n$},\\
	0 & \text{otherwise}
	\end{cases}
	$$	
	where $\Sigma_n$ acts on $(\BZ[-1])^{\otimes n}$ by permuting the factors. 


The mismatch of degrees between the spectral Lie operad $\BL$ and the Lie operad $\Lie$ in abelian groups is rather inconvenient for applications. To remedy this, we introduce a shift operation for $\infty$-operads in a stable $\infty$-category $\CC$. 
\begin{definition}
\cite[Section 3]{Camarena_mod2_free_spectral_Lie_algebra}
    Let $\CO$ be an $\infty$-operad, the \emph{desuspended $\infty$-operad} $\bf{\Sigma}^{-1} \CO$ of $\CO$ in $\CC$ is an $\infty$-operad  defined so that its associated monad is
	$$
	\Omega F_{\CO}\Sigma.
	$$
	Explicitly, $\bf{\Sigma}^{-1} \CO$ has underlying symmetric sequence 
	$$
	(\mathbf{\Sigma^{-1}} \CO ) (n)= \BS^{n}\otimes\Sigma^{-1} \CO(n)
	$$
	where $\Sigma_n$ acts on $\BS^{n}$ by permuting the $n$ factors of $\BS^{1}$ and $\Sigma_n$ acts on $\mathbf{\Sigma^{-1}} \CO$ via the its action on $\CO(n)$.
\end{definition}

For further details regarding the shift of an $\infty$-operad, we refer the readers to \cite[Section 2.2.4.]{Hadrianphdthesis} and \cite{Camarena_mod2_free_spectral_Lie_algebra}.
The following lemma indicates that shifting an $\infty$-operad is indeed harmless to the category of algebras we want to study.
\begin{lemma}
\label{Shift has no harm}
    The assignment 
    \begin{align*}
    \Omega': \Alg_{\CO } (\CC) &\to \Alg_{\mathbf{\Sigma} \CO }(\CC) \\
    X &\mapsto  \Omega X
    \end{align*}
    is an equivalence of $\infty$-categories.
\end{lemma}
\begin{proof}
	Consider the commuting diagram
\[
\begin{tikzcd}
	\Alg_{\CO }(\CC)  &   & \Alg_{ \Omega \CO \Sigma}(\CC) \\
	& \CC &
	\arrow[from=1-1, to = 1-3, "\Omega'"]
	\arrow[from=1-1 , to =2-2, "\Omega\circ \oblv_{\CO}" left]
	\arrow[from=1-3, to=2-2, "\oblv_{\mathbf{\Sigma}\CO}"]
\end{tikzcd}
\]
and we want to check the conditions of \cite{HA}[Corollary 4.7.3.16.].
Conditions (1), (2), (3) are immediate and note that both diagonal arrows in the diagram above are conservative, hence we are left to check (5).
Consider the map
\[
\varphi: \Free_{\mathbf{\Sigma}\CO}(X) \to \Omega' 
\circ \Free_{\CO} (\Sigma X)
\]
left adjoint to
\[
X \to \Omega \circ \oblv_{\CO}\circ
\Free_{\CO}
\circ
\Sigma(X)
\simeq 
\oblv_{\mathbf{\Sigma}\CO}
\circ
\Omega'\circ \Free_{\CO}\circ \Sigma (X).
\]
It suffices to check 
\[
\oblv_{\mathbf{\Sigma}\CO}(\varphi):
\mathbf{\Sigma}\CO(X) \to 
\oblv_{\mathbf{\Sigma}\CO} \circ \Omega'\circ \Free_{\CO} (\Sigma X)
\]
is an equivalence, where the latter is equivalent to 
$$
\Omega \circ \oblv_{\CO}\circ \Free_{\CO}(\Sigma X)\simeq \mathbf{\Sigma}\CO(X),
$$
and the proof is now complete.
\end{proof}

\begin{definition}
	\label{spectral Lie operad}
	We define the \emph{spectral Lie operad} as $$
	\spLie := \mathbf{\Sigma}^{-1} \BL.
	$$
\end{definition}

\begin{remark}
\label{Remark on shifted Lie operads}
The underlying symmetric sequence of the spectral Lie operad $\spLie:= \mathbf{\Sigma}^{-1}\BL$
	has the form 	
	$$
	\spLie(n)= \BS^{n} \otimes \Sigma^{-1} \BL(n),
	$$
	whose homology is exactly $\Lie(n)$ concentrated in degree $0$.
	On the other hand, $\spLie$ is the Goodwillie derivative of the functor $\Omega\Sigma$ on $\Space_*$, see \cite[Section 8]{GoodwillieIII}.
	Moreover, one can check the monad associated to $F_{\spLie}$ is given by $\Omega\BL \Sigma$; indeed, this can be seen by inspecting their associated monads
	\begin{align*}
		F_{\spLie}(X) &\simeq \coprod_{n\geq 1}
		\big(
		\BS^{n} \otimes \Sigma^{-1}\BL(n) 
		\otimes X^{\otimes n}
		\big)_{h\Sigma_n}\\
		& \simeq 
		\Sigma^{-1}
		\coprod_{n\geq 1}
		\big(
		\BL(n) 
		\otimes (\Sigma X)^{\otimes n}
		\big)_{h\Sigma_n}\\
		& \simeq 
		\Omega F_{\BL}\Sigma(X).
	\end{align*}	
\end{remark}






\section{Lie Algebras in Tame Spectra}

In this section, we discuss spectral Lie algebras in $\Sp$ and $\Sp^{\geq r}_{\operatorname{tame}}$. We will prove that the $\infty$-category of tame spectral Lie algebras can be identified as the full subcategory of spectral Lie algebras whose underlying spectra are tame.
We then define the Chevalley-Eilenberg functor that goes from the category of spectral Lie algebras to the category of divided power, conilpotent commutative coalgebras.


% Using the theory of Koszul duality introduced in the previous section, we will state the main result of \cite{Ching-Harper} and apply to the tame spectra case. The main result is that the $\infty$-category $\Alg_{\Lie}(\Sp_{tame})$ of tame spectral Lie algebras is equivalent to the $\infty$-category $\coAlg^{nil, dp}_{\Com}(\Sp_{tame})$ of conilpotent divided power coalgebras.
\begin{definition}
\label{Spectral Lie operad}
We define the $\infty$-category $\Alg_{\spLie}(\Sp)$ of \emph{spectral Lie algebras} to be the $\infty$-category of algebras over the spectral Lie operad $\spLie$.
\end{definition}
Note that our definition of spectral Lie algebras might be different from the one given in the literature, since we define $\spLie$ to be the spectral Lie operad. 
The advantage of this definition is that the homology $H_{*}\spLie$ is precisely the Lie operad in abelian groups; hence the homology of a spectral Lie algebra is a graded Lie algebra. 
Furthermore, it is demonstrated in \cite{Camarena_mod2_free_spectral_Lie_algebra} that using this definition of spectral Lie algebras makes computation easier.

To define the $\infty$-category of tame spectral Lie algebras, we need to introduce an action of a monad on $\Sp^{\geq r}_{\operatorname{tame}}$.
Note first that $F_{\spLie}$ is a monad on the $\infty$-category $\Sp^{\geq r}$ of $r$-connective spectra, since the colimits in $\Sp^{\geq r}$ are comuputed in $\Sp$.

\begin{lemma}
\label{induced maps on SSeq}
\cite{Hadrianphdthesis}[Remark 5.81]
Let $\CC$ and $\CD$ be symmetric monoidal $\infty$-categories and let 
$$
f: \CC \to \CD
$$
be a colimit-preserving, symmetric monoidal functor.
Then $F$ induces a functor 
$$
\SSeq(f): \SSeq(\CC) \to \SSeq(\CD)
$$
which is monoidal with respect to the composition product.
\end{lemma}	


Since the Schur functor $F_{\CO}$ associated to the $\infty$-operad $\CO$ in $\CC$ is a monad on $\CC$, under the assumption of Lemma \ref{induced maps on SSeq}, we obtain a monad $f_{!}(F_\CO)$ on $\CD$. Moreover, $f$ induces a functor:
\[
f' : \Alg_{\CO}(\CC) \to \Alg_{f_{!}(F_\CO)}(\CD).
\]

Take $\CO=\spLie$ and $f=L_{\tame}:\Sp^{\geq r} \to \Sp^{\geq r}_{\tame}$ the localization functor from $r$-connective spectra to $r$-tame spectra. The monad $f_{!}(F_\CO)$ on $\Sp^{\geq r}_{\tame}$ is simply given by the tame localization of the free spectral Lie algebra monad on $\Sp^{\geq r}$
$$
X \mapsto  
L_{\tame}
\coprod_{n\geq 1}
		\big(
		\spLie(n) \otimes X^{\otimes n}
		\big)_{h\Sigma_n}
		\simeq 
		L_{\tame}F_{\spLie}(X).
$$

We are now ready to define the $\infty$-category of tame spectral Lie algebras.
\begin{definition}
\label{Def of tame spectral Lie algebras}
    We define the $\infty$-category $\Alg_{\spLie}(\Sp^{\geq r}_{\tame})$ of \emph{tame spectral Lie algebras} to be 
    $$
    \Alg_{\spLie}(\Sp^{\geq r}_{\tame}) := \LMod_{	(L_{\tame})_{!}F_{\spLie}}(\Sp^{\geq r}_{\tame}).
    $$
\end{definition}

We claim the $\infty$-category of tame spectral Lie algebras is a full subcategory of the $\infty$-category of $r$-connective spectral Lie algebras whose underlying spectra are tame, it turns out that this is a consequence of the following more general result.
\begin{proposition}
\label{Lifting localization to algebra category}
Let $\CC$ and $\CD$ be symmetric monoidal $\infty$-categories. 
Suppose $f:\CC \to \CD$ is a symmetric monoidal localization functor.
Let $\CO$ be an $\infty$-operad in $\CC$, then the induced functor
$$
f' : \Alg_{\CO}(\CC) \to \Alg_{f_{!}(F_\CO)}(\CD).
$$
is a localization.
\end{proposition}
\begin{proof}
	The existence of the functor $f'$ follows from Lemma \ref{induced maps on SSeq}. It remains to show $f'$ is a localization functor. Let $g$ be the fully faithful right adjoint of $f$. We first claim that $g$ lifts to a functor 
	$$
	g': \Alg_{f_{!}(\CO)}(\CD)
	\to 
	\Alg_{\CO}(\CC)
	$$
	which is right adjoint to $f'$. 
	Since $f$ is symmetric monoidal, it follows that $g$ is lax symmetric monoidal; hence $g$ induces a functor $g': \Alg_{f_{!}(\CO)}(\CD)
	\to 
	\Alg_{\CO}(\CC)$.
	
	Since $g$ is fully faithful, the induced functor $g':\Alg_{f_{!}(\CO)}(\CD)
	\to
	\Alg_{\CO}(\CD)$ is also fully faithful by Lemma .  
	\todo{This Lemma will be in appendix?}
\end{proof}

\begin{corollary}
\label{Identification of tame Lie algebras}
	The $\infty$-category $\Alg_{\spLie}(\Sp^{\geq r}_{\tame})$ can be identified with the full subcategory of $\Alg_{\spLie}(\Sp)^{\geq r}$ spanned by spectral Lie algebras whose underlying spectrum are tame.
\end{corollary}
\begin{proof}
    Proposition \ref{Lifting localization to algebra category} provides a localization functor
	$f: \Alg_{\spLie}(\Sp)^{\geq r} \to \Alg_{\spLie}(\Sp^{\geq r}_{\tame})$. The corollary then follows from the commuting diagram
\[
\begin{tikzcd}
	\Alg_{\spLie}(\Sp)^{\geq r} & \Alg_{\spLie}(\Sp^{\geq r}_{\tame})\\
	\Sp^{\geq r}  & 
	\Sp^{\geq r}_{\tame}
	\arrow[from=1-2, to= 1-1]
	\arrow[from=1-1, to=2-1, "\oblv" left]
	\arrow[from=1-2, to=2-2, "\oblv'" ]
	\arrow[from=2-2, to= 2-1]
\end{tikzcd}
\]
where the horizontal maps are fully faithful and the vertical maps are conservative. 

\end{proof}

Recall that we have the identification
$$
\Sp^{\geq r}_{\tame} \simeq
\Mod_{\BZ,\tame}^{\geq r},
$$
and the localization $L_{\tame}:\Mod_{\BZ}^{\geq r} \to \Mod_{\BZ,\tame}^{\geq r}$. With the same proof as above, we obtain the following corollary.
\begin{corollary}
    The $\infty$-category $\Alg_{\spLie}(\Sp^{\geq r}_{\tame})$ of tame spectral Lie algebras can be identified with the full subcategory of $\Alg_{\spLie}(\Mod_{H\BZ})^{\geq r}$ spanned by Lie algebras whose underlying $H\BZ$-module spectra are tame.
\end{corollary}

\begin{remark}
\label{Identify tame Lie algebra with Dwyer's Lie algebra}
Since $\Mod_{H\BZ}^{\geq r}$ models the derived $\infty$-category $D(\BZ)^{\geq r}$, the Corollary implies that the $\infty$-category $\Alg_{\spLie}(\Sp^{\geq r}_{\tame})$ of tame Lie algebras model Dwyer's tame Lie algebras \cite{Dwyer}.
\end{remark}


We now discuss Koszul duality between Lie algebras and divided power, conilpotent commutative coalgebras.
We will call the functor
$$\indec_{\CO}: \Alg_{\CO}(\CC)\to \coAlg_{\Barconstruction(\CO)}(\CC)$$
the \emph{Chevalley-Eilenberg functor} when $\CO$ is a Lie operad. 
To justify the name, take $\CC = \Sp_{\BQ}$ the $\infty$-category of rational spectra, or equivalently, the $\infty$-category of rational chain complexes.
The functor 
$$
\cot_{\BL}: \Alg_{\BL}(\Sp_{\BQ}) 
\to 
\Sp_{\BQ}
$$
computes the Chevalley-Eilenberg cohomology of a dg Lie algebra over $\BQ$. 

Note that the operad $H_{*}\BL$ is a degree shift of the Lie operad in abelian groups; to avoid this inconvenience, it makes more sense to consider
$$
\operatorname{CE}: \Alg_{\spLie}(\Sp)\simeq \Alg_{\BL}(\Sp)
\xrightarrow{\cot_{\BL}}
\Sp
$$
and 
$$
\widetilde{\operatorname{CE}}: \Alg_{\spLie}(\Sp)\simeq \Alg_{\BL}(\Sp)
\xrightarrow{\indec_{\BL}}
\coAlg^{\divpow, \nil}_{\Com}(\Sp).
$$
We record the following definition for the Chevalley-Eilenberg functor.
\begin{definition}
\label{CChevalley-Eilenberg functor}
    We will call $$
    \widetilde{\operatorname{CE}}: \Alg_{\spLie}(\Sp)\simeq \Alg_{\BL}(\Sp)
\to 
\coAlg^{\divpow, \nil}_{\Com}(\Sp)
$$ the \emph{Chevalley-Eilenberg} functor.
\end{definition}

The $\infty$-category $\Sp^{\geq r}_{\tame}$ of tame spectra is not a localization of 
\begin{lemma}
    Let $\CO$ be a connected $\infty$-operad in $ \Sp$, then 
    $$
    \Barconstruction(L_{\tame}\CO ) \simeq L_{\tame} \Barconstruction(\CO).
    $$
\end{lemma}
\begin{proof}
    Unravelling the definition of bar construction, we need to show 
    $$
     |\Barconstruction(\triv, L_{\tame}\CO , \triv)_{\bullet}|
     \simeq
     L_{\tame}|\Barconstruction(\triv, \CO , \triv)_{\bullet}|.
    $$
    Since $L_{\tame}$ preserves colimits, we are reduced to check
    $$
    L_{\tame} (\underbrace{\CO \circ \cdots \circ \CO}_{\text{$n$-fold}} )
    \simeq 
    \underbrace{ L_{\tame}\CO \circ \cdots \circ  L_{\tame}\CO}_{\text{$n$-fold}} 
    $$
    for each $n\geq 1$.
    It suffices to check for two-fold composition, which can be done directly on the composition of their associated monads; but that follows from the fact that $L_{\tame}$ preserves colimits and is symmetric monoidal.
\end{proof}

As a consequence, taking $\CO=\spLie$ allows us to identify the bar construction of the monad $L_{\tame}\spLie$ as the comonad $L_{\tame}\Com$, i.e. the tame localization commutative cooperad in $\Sp$. Explicitly the associated comonad is given by
$$
X \mapsto \coprod_{n\geq 1} L_{\tame} (X^{\hat{\otimes} n})_{h\Sigma_n}.
$$


% We end this section with a result concerning the Chevalley-Eilenberg functor on tame spectra of different degrees.
% \begin{proposition}
% $$

% $$
% \end{proposition}







\clearpage






\begin{remark}
Proposition \ref{all coalgebras are equivalent} allows us to identify the codomain of the Chevalley-Eilenberg functor as $\coAlg(\Sp^{\geq}_{\tame})$, whose categorical product is given by the tensor product $\hat{\otimes}$ in $\Sp^{\geq}_{\tame}$.
\end{remark}

We now show that the Chevalley-Eilenberg $\widetilde{\operatorname{CE}}$ induces a functor on the categories of group objects, that is, the functor $\widetilde{\operatorname{CE}}$ is symmetric monoidal when we consider the Cartesian symmtric monoidal structures on both sides.
\begin{lemma}
\label{CE preserves products}
	The functor 
	$\widetilde{\operatorname{CE}}$
	preserves finite products.
\end{lemma}
\begin{proof}
	We want to show the morphism 
	\[
	\widetilde{\operatorname{CE}}(L \times L') 
	\to 
	\widetilde{\operatorname{CE}}(L)\otimes \widetilde{\operatorname{CE}}(L') 
	\]
	is an equivalence for $L,L' \in \Alg_{\Lie}(\Sp^{\geq r}_{\tame}$. 
	Note that $\widetilde{\operatorname{CE}}$ lifts to a functor on 
	$$
	\widetilde{\operatorname{CE}}^{\Fil}: \Alg_{\Lie}((\Sp^{\geq r}_{\tame})^{\Fil, \geq 0})
	\to 
	\coCAlg((\Sp^{\geq r}_{\tame}^{\Fil, \geq 0}),
	$$
	so it suffices to show the induced map
	$$
	\gr\circ \widetilde{\operatorname{CE}}^{\Fil}\big( (L \times L')^{\Fil} \big) 
	\simeq 
	\gr \widetilde{\operatorname{CE}}^{\Fil}(L) \otimes \gr\widetilde{\operatorname{CE}}^{\Fil}(L')
	$$
	is an equivalence since the functor $\gr$ is conservative. 
	Since $\widetilde{\operatorname{CE}}^{\Fil}$ preserves colimits, we have 
	\begin{align*}
		\gr\circ \widetilde{\operatorname{CE}}^{\Fil}\big( (L \times L')^{\Fil} \big) 
	& \simeq 
	\widetilde{\operatorname{CE}}^{\gr} \big( \gr (L \times L')^{\Fil} \big) \\
	& \simeq 
	\widetilde{\operatorname{CE}}^{\gr} \circ \trivial_{\Lie}^{\gr} (L\times L')\\
	& \simeq \Sym^{\gr} (L\times L').
	\end{align*}
	Similarly, we conclude 
	$$
	\gr \widetilde{\operatorname{CE}}^{\Fil}(L) \otimes \gr\widetilde{\operatorname{CE}}^{\Fil}(L') \simeq
	\Sym^{\gr} (L) \otimes  \Sym^{\gr} (L'), 
	$$
	and we are reduced to show the map 
	$$
	\Sym^{\gr} (L\times L') \to \Sym^{\gr} (L) \otimes  \Sym^{\gr} (L')
	$$
	which follows from the fact that $\Sym$ sends products to tensor products.
\end{proof}

\section{The Universal Enveloping Algebra Functor}
Let $k$ be a field of characteristc $0$. Denote the category of $k$-vector spaces by $\vectk$.
If $A$ is a unital associative algebra over $k$, then the commutator operation endows $A$ with a Lie algebra structure. This construction is functorial so that it determines a functor from the category $\Alg_{\Ass}(\vect_k)$ of unital associative algebras to the category $\Alg_{\Lie}(\vect_k)$ of Lie algebras. 
Conversely, given a Lie algebra $L$ one can define its universal enveloping algebra $U(L)$; as a vector space, one can construct it as the tensor algebra $T(L)$ modulo the ideal $I$ generated by 
 elements of the form $x\otimes y - y\otimes x-[x,y]$. 
This construction is functorial and one check it defines a functor
$$
U:\Alg_{\Lie}(\vect_k) \rightarrow \Alg_{\Ass}(\vect_k),
$$
which is left adjoint to the forgetful functor.
We'll refer the functor $U$ as the \emph{enveloping algebra functor}. 

The enveloping algebra functor is well studied both in algebra and topology. We state two results concerning the enveloping algebra functor $U$ that will be relevant later.
\begin{itemize}
	\item The enveloping algebra functor $U$ is symmetric monoidal; that is, if we endow the category of Lie algebras $\Alg_{\Lie}(\vect_k)$ with the Cartesian symmetric monoidal structure and $\Alg_{\Ass}(\vect_k)$ with the usual symmetric monoidal structure with tensor product as monoidal product, then $U$ sends categorical products to tensor products. As a consequence, $U(L)$ has a canonical coalgebra structure for any Lie algebra $L\in \Alg_{\Lie}(\vect_k)$.
	\item Over a field of characteristic, the Poincare-Witt-Birkhoff theorem has a strong form that gives a simple description of the universal enveloping algebra $U(L)$. 
	\begin{theorem}
		[Poincare-Birkhoff-Witt]
		\cite{Quillen_RHT}[Appendix B Theorem 2.3]
		Let $L$ be a Lie algebra over $k$ and let $i:L\to U(L)$ denote the unit map of. Then the "averaging" map
		\begin{align*}
			S(L) & \to U(L) \\
			x_1\cdots x_n & \mapsto \frac{1}{n!}\sum_{\sigma\in \Sigma_n} i(x_{\sigma(1)})\cdots i(x_{\sigma(n)})
		\end{align*}	
		is an isomorphism of coalgebras.
	\end{theorem}
	
\end{itemize}

The goal of this paper is to discuss coalgebras and the enveloping algebra functor in a more general context. 


\section{Proof of Theorem \ref{first main theorem}}
In the last section of this chapter, we prove the first main result of this paper. Since Lemma \ref{CE preserves products} ensures the functor $\widetilde{\operatorname{CE}}$ preserves group objects, hence we obtain a functor on the categories of groups
\[
	\Grp(\widetilde{\operatorname{CE}}):
	\Grp(\Alg_{\spLie}(\Sp^{\geq r-1}_{\text{($r-1$)-tame}}))
	\to 
	\Hopfalgebra(\Sp^{\geq r-1}_{\text{($r-1$)-tame}}).
\]
	
\begin{definition}
	The \emph{unviersal enveloping algebra functor}
	$$
	U: \Alg_{\spLie}(\Sp^{\geq r}_{\text{$r$-tame}}) 
	\to 
	\Hopfalgebra(\Sp^{\geq r-1}_{\text{($r-1$)-tame}})
	$$
	is defined as the composite $\Grp(\widetilde{\operatorname{CE}}) \circ \Omega_{\Lie}$.
\end{definition}

	
\begin{theorem}
\label{first main theorem}
	The universal enveloping algebra functor
	$$
	U: \Alg_{\spLie}(\Sp^{\geq r}_{\text{$r$-tame}})  
	\to
	\Hopfalgebra(\Sp^{\geq r-1}_{\text{($r-1$)-tame}})
	$$
	is an equivalence of $\infty$-categories.
\end{theorem}

Over the rational, if $L$ is a Lie algebra in $\Ch_{\BQ}$, then $\Omega_{\Lie}L$ is a trivial Lie algebra.
We claim the same phenomenon happens in the case of tame Lie algebras.

\begin{proposition}
\label{Triviality of loop of an O-algebra}
	Let $\Alg_{\spLie}(\Sp^{\geq r}_{\tame}) $ be a tame Lie algebra.
	Then we have an equivalence
	$$
	\Omega_{\spLie} L\simeq \trivial_{\spLie}(\Omega L).
	$$ 
\end{proposition}

Assuming Proposition \ref{Triviality of loop of an O-algebra}, we can now prove Theorem $\ref{first main theorem}$.
\begin{proof}
[Proof of  \ref{first main theorem}:]
	We first show the universal enveloping algebra functor
	\[
	U:
	\Alg_{\spLie}(\Sp^{\geq r}_{\text{$r$-tame}}) 
	\xrightarrow{\Omega_{\spLie}}
	\Grp(\Alg_{\spLie}(\Sp^{\geq r-1}_{\text{($r-1$)-tame}}))
	\xrightarrow{\Grp(\widetilde{\operatorname{CE}})}
	\Hopfalgebra(\Sp^{\geq r-1}_{\text{($r-1$)-tame}})
	\]
	is fully faithful. 
% 	Since $\Omega_{\spLie}X$ is also an $(r-2)$-tame spectrum, we can factor $U$ has follows,
% 	\[
% 	\begin{tikzcd}
% 	\Alg_{\Lie}(\Sp^{\geq r-1}_{\text{($r-1$)-tame}})	&    \Grp(\Alg_{\Lie}(\Sp^{\geq r-2}_{\text{($r-2$)-tame}}))
%  &
%   \Hopfalgebra(\Sp^{\geq r-1}_{\text{($r-2$)-tame}})
%     \\
% 		& \Grp(\Alg_{\Lie}(\Sp^{\geq r-2}_{\text{($r-1$)-tame}}))
%   & \Hopfalgebra(\Sp^{\geq r-1}_{\text{($r-1$)-tame}})
%   	\arrow[from=1-1, to=1-2, "\Omega_{\Lie}'"]
%   	\arrow[from=1-2, to=1-3, "\widetilde{\operatorname{CE}}'"]
%   	\arrow[from=1-1, to=2-2, "\Omega_{\Lie}" left]
%   	\arrow[from=1-2, to=2-2]
%   	\arrow[from=1-3, to=2-3]
%   	\arrow[from=2-2, to=2-3, "\widetilde{\operatorname{CE}}"]
% 	\end{tikzcd}
% 	\]
% 	where the two vertical arrows are fully faithful.
% 	We claim the upper horizontal arrow $\widetilde{\operatorname{CE}}'\circ \Omega_{\Lie}'$ is fully faithful.
    Let $\widetilde{\Prim}$ denote the right adjoint of $\Grp(\widetilde{\operatorname{CE}})$ and consider the unit map
	$$
	X \to B_{\spLie}\circ \widetilde{\Prim} \circ \Grp(\widetilde{\operatorname{CE}})\circ \Omega_{\spLie}(X)
	\simeq
	B_{\spLie}\circ
	\widetilde{\Prim} \circ\Grp(\widetilde{\operatorname{CE}} ) \circ \trivial_{\spLie}(\Omega X)
	$$
where the latter equivalence follows from Proposition \ref{Triviality of loop of an O-algebra}. Since $B_{\spLie}$ is an inverse to $\Omega_{\spLie}$ by Proposition \ref{B and Omega are mutally inverses}, it suffices to check the map
\[
\eta:  \Omega X \to 
\oblv_{\spLie}\circ 
\widetilde{\Prim} \circ \widetilde{\operatorname{CE}}\circ \trivial_{\spLie}(\Omega X)
\]
is an equivalence.
%, this is equivalent to show
%	the unit map	
%	
%	is an equivalence for any $(r-1)$-tame Lie lagebra $X$.
	By \cite{Francis-Gaitsgory}[Lemma 3.3.4.], we have a natural equivalence 
	$$\widetilde{\operatorname{CE}}\circ \trivial_{\spLie}\circ \Omega (X) 
	\simeq 
	\cofree(\Omega X);
	$$ 
	apply the functor $\oblv_{\spLie} \circ \widetilde{\Prim}$ on both sides, we have
	$$
	\oblv_{\spLie} \circ \widetilde{\Prim}\circ \widetilde{\operatorname{CE}} \circ 
	\trivial_{\spLie}(\Omega X) \simeq   	\oblv_{\spLie} \circ \widetilde{\Prim}\circ \cofree(\Omega X)
	\simeq  \Omega X
	$$
	which completes the proof of fully faithfulness of $U$.

    For essential surjectivity, note first that  $U$ preserves colimits, as it is a composition of $\Omega_{\spLie}$ and $\widetilde{\operatorname{CE}}$. 
    From the identification $\Hopfalgebra(\Sp^{\geq r-1}_{\text{($r-1$)-tame}})$ $\CS^{\geq r}_{tame}$ by Theorem \ref{2nd Main Theorem},
    it suffices to show that the Hopf algebra $H$, corresponding to the generator $S^r$ of $\CS^{\geq r}_{tame}$, lies in the image of $U$. 
    Note that $S^{r}$ is $r$-tame equivalent to $K(\BZ, r)$ and $\Omega K(\BZ, r)$ is  equivalent to $K(\BZ, r-1)$, hence the underlying $(r-1)$-tame spectrum of $H$ is 
    $$
    \Sigma^{\infty}_{\operatorname{tame}}\Omega S^{r}
    \simeq 
    \Sigma^{\infty}_{\operatorname{tame}} K(\BZ, r-1)
    \simeq \Sym(\Sigma^{r-1}H\BZ)
    $$ 
    where the second equivalence follows from Corollary \ref{Suspension of Eilenberg-Maclane spaces are symmetric algebra}.
    Take the trivial $r$-tame Lie algebra $\trivial_{\spLie}(\Sigma^{r}H\BZ)$, then
    \begin{align*}
        \operatorname{CE}\circ  \Omega_{\spLie}(\trivial_{\spLie}(\Sigma^{r}H\BZ))
        & \simeq 
        \operatorname{CE}\circ \trivial_{\spLie} (\Sigma^{r-1}H\BZ)\\
        & \simeq \Sym(\Sigma^{r-1}H\BZ),
    \end{align*}
    hence we see that $H$ is the image of $\trivial_{\spLie}(\Sigma^{r}H\BZ)$ under $U$.
    
\end{proof}

We now present the proof of Proposition \ref{Triviality of loop of an O-algebra}.
We learn this proof from Heuts. 
Let $\BL$ denote the free Lie algebra monad on the $\infty$-category $\Sp$ of spectra and consider the 
\emph{suspension morphism}
\[
\BL \xrightarrow{\sigma} \Omega \BL \Sigma.
\]
between these two monads on $\Sp$.

Observe that the desuspension $\Omega:\Sp \to \Sp$ lifts to a functor 
$$
\Omega':  \Alg_{\BL}(\Sp) \to \Alg_{\Omega \BL \Sigma}(\Sp)
$$
which is an equivalence by Lemma \ref{Shift has no harm}.
We can then identify the restriction functor $\Alg_{ \Omega \BL \Sigma}(\Sp)\xrightarrow{\sigma^{*}} \Alg_{\BL}(\Sp)$ as
\begin{align*}
	\Alg_{\BL}(\Sp)\simeq \Alg_{\Omega\BL\Sigma}(\Sp) & \xrightarrow{\sigma^*} \Alg_\BL(\Sp)\\
	X & \mapsto \Omega X.
\end{align*}

To complete the proof of Proposition \ref{Triviality of loop of an O-algebra},
we claim the suspension morphism $\sigma$ can be factored as 
$$
\BL \to \triv 
\to 
\Omega \BL \Sigma
$$
after tame localization;
since the restriction along $\BL \to \triv$ is indeed the trivial Lie algebra functor, this would complete the proof of Proposition \ref{Triviality of loop of an O-algebra}.

Let $\Lie_n$ denote the ordered set of Lie words with $n$ generators, i.e. every $w \in \Lie_k$ is a basis element of the free Lie algebra on $n$ generators.
For $X_1, \dots, X_n \in \Sp$, we define $w(X_1, \cdots, X_2):= X_1\otimes \cdots \otimes X_n$; that is, we let the Lie brackets in $w$ act as the smash products in $\Sp$. 
The following is a variant of the Hilton-Milnor theorem.
\begin{theorem}
	[Hilton-Milnor,  \cite{Arone-Kankaanrinta98}, \cite{Brantner-Heuts}]
	For any collection of spheres $\BS^{k_1}, \cdots, \BS^{k_n}$, there is an equivalence
	$$
	\Omega \BL \Sigma(\BS^{k_1}\oplus \cdots \oplus \BS^{k_n})
	\simeq 
	\bigoplus_{w\in \Lie_n} \Omega \BL( \Sigma w(\BS^{k_1}, \cdots, \BS^{k_n})).
	$$
\end{theorem}

Now we can rewrite the suspension morphism $\sigma$ evaluating at the wedge of spheres $\BS^{k_1}\oplus \cdots \oplus \BS^{k_n}$ as 
$$
\sigma: 
\BL(\BS^{k_1}\oplus \cdots \oplus \BS^{k_n})
\to 
\bigoplus_{w\in \Lie_n} \Omega \BL (\Sigma w(\BS^{k_1}, \cdots, \BS^{k_n})).
$$
Note that any map 
$$
\BS^{l} \to 
\BS^{k}
$$
is null-homotopic if $l < k$,
hence $\sigma$ has to factor through the wedge sum of Lie words of weight one, i.e. we have factorization
\[
\begin{tikzcd}
	\bigoplus_{i=1}^{n} \BL(\BS^{k_i}) &   &  \bigoplus_{w\in \Lie_n} \Omega \BL (\Sigma w(\BS^{k_1}, \cdots, \BS^{k_n}))\\
	& \bigoplus_{i=1}^{n}\Omega \BL (\Sigma \BS^{k_i})  &
	\arrow[from=1-1, to = 1-3, "\sigma"]
	\arrow[from=1-1 , to =2-2]
	\arrow[from=2-2, to=1-3]
\end{tikzcd}
\]
 factors over $\bigoplus_{i=1}^{n}\Omega \BL (\Sigma \BS^{k_i})$.
 It now suffices to check the following.
 
 \begin{lemma}
 	The suspension morphism evaluated on a sphere has a factorization
\[
\begin{tikzcd}
	 \BL(\BS^{k}) &   &  \Omega \BL (\Sigma \BS^{k})\\
	& \BS^{k}  &
	\arrow[from=1-1, to = 1-3, "\sigma"]
	\arrow[from=1-1 , to =2-2]
	\arrow[from=2-2, to=1-3]
\end{tikzcd}
\]
after tame localization.
\end{lemma}
\begin{proof}
	The lemma follows from Theorem \ref{odd spheres}; indeed, for $n=p^l$ and $l>0$, the connectivity of
	\[
	\big(
	\BL(n)\otimes (\BS^{k})^{\otimes n}
	\big)_{h\Sigma_n}
	\]
	is at least $kp^l-n-1$, which is larger than $r+2p-3$, hence 
	$\big(
	\BL(n)\otimes (\BS^{k})^{\otimes n}
	\big)_{h\Sigma_n}$ is contractible.
	\end{proof}
 \begin{theorem}
 	\cite{Arone-Mahowald}[Theorem 3.13, Theorem 4.4]
 	\label{odd spheres}
 	Let $X$ denote the $p$-localization of the sphere $\BS^k$ at a prime $p$, then
 	\begin{enumerate}
 		\item if $k$ is odd, then 
 			$$
 			\big(\BL(n)\otimes X^{\otimes n}
 			\big)_{h\Sigma_n}
 			\simeq *.
 			$$
 			if $n\neq p^l$ for some $l$.
 		\item if $k$ is even, then 
 			$$
 			\big(\BL(n)\otimes X^{\otimes n}
 			\big)_{h\Sigma_n}
 			\simeq *.
 			$$
 			if $n$ is not equal to $p^l$ or $2p^j$ for $l, j>0$.
 	\end{enumerate} 	
 \end{theorem}
 
%\begin{corollary}
%The suspension morphism $\sigma$ factors as 
%	$$
%\BL \to \triv 
%\to 
%\Omega \BL \Sigma
%$$
%after tame localiztion. 
%\end{corollary}
%
%\begin{proof}
%	Since all these three functors are sifted-colimit preserving functors, so it will suffice to check this on objects of the form of wedge of spheres.
%\end{proof}





%\begin{proof}
%	Still working on this proof, and here are some ideas on how to attack this:
%	\begin{itemize}
%		\item Need to define the notion of tensoring an $\CO$-algebra $X$ with a commutative algebra $A$ (this should work in any symmtric monoidal $\infty$-category). That is, a functor
%		\[
%		A \otimes -: \Alg_\CO\to \Alg_\CO
%		\]
%		\item In the rational case, this is indeed true; because $C^*(S^1;\BQ)$ is quasi-isomorphic to the cdga $H^*(S^1;\BQ)$.
%		\item Show $X\otimes A$ is trivial if $A$ is a trivial commutative algebra.
%		\item Show $\Omega L$ is equivalent to $L_{tame}H\BZ^{S^1}\otimes L$ and hence trivial. 
%	\end{itemize}
%\end{proof}
%The spectrum $H\BQ^{S^1}$ models the rational cochain of $S^1$. Moreover, by Sullivan's formality result, we have an equivalence of $\E_\infty$-algebras 
%\[
%H\BQ^{S^1} \simeq H^{*}(S^1; \BQ)
%\]
%in $\Mod_{H\BQ}$. Hence, the rational cochain $H\BQ^{S^1}$ is a trivial $\E_{\infty}$-algebra.
%
%Now to show $L_{tame}H\BZ^{S^1}$ is a trivial $\E_{\infty}$-algebra, we need a formality result for $L_{tame}H\BZ^{S^1}$.
%\begin{question}
%	How is the formality argument proved in the rational case?
%\end{question}
%
%\begin{proposition}
%	There is an equivalence 
%	\[
%	L_{tame}C^{*}(S^1;\BZ) \simeq \tilde{H}^*(S^1; \BZ)
%	\]
%of $\E_\infty$-algebras if we invert primes quick enough to kill all the Steenrod operations.
%\end{proposition}
%
%
%
%
%Maybe one can proceeds as follows:
%One first shows $L_{tame}H\BZ^{S^1}$ is a trivial $\E_{\infty}$-algebra in $\Sp$. For this, we need to show some kind of formality argument:
%the cochain $L_{tame}H\BZ^{S^1}$ is tame equivalent to the cdga $H^*(S^1; \BZ)$ in $\Mod_{\BZ}$.










