\chapter{Koszul Duality}

In this chapter, we provide the necessary backgrounds on $\infty$-operads and $\infty$-cooperads.
Our main reference will be \cite{BrantnerPhD},  Heuts' upcoming paper, Gaitsgory's notees, \cite{Hadrianphdthesis} and \cite{HA}.
 
In section 3.1, we define an $\infty$-operad (in a symmetric monoidal $\infty$-category $\CC$) as an associative algebra object in the category of symmetric sequences.

In section 3.2, we discuss algebras over an operad and some relavant constructions surrounding this area. 

In section 3.3, we introduce the Koszul duality functor. Then we show that the loop of a $\CO$-algebra in tame spectra is automatically a trivial algebra.

In section 3.4, we define the universal enveloping algebra functor from the category of Lie algebras to the category of commutative coalgebras. 
In Section \ref{Commutative coalgebras in tame spectra}, we prove that there is essentially no difference between the category of conilpotent, divided commutative coalgebras and the category of commutative coalgebras.
Moreover, we will show it is an equivalence from tame Lie algebras to tame Hopf algebras. This completes the proof of our first main theorem.

\section{(co)Operads in $\infty$-Categoriees}
In this section, we collect and extend some results from \cite{Francis-Gaitsgory}, \cite{HA} and \cite{HeutsSurvey}.
Let us fix a presentably symmetric monoidal $\infty$-category $\CC$ whose tensor product preserves colimits in each variable in this section.

Let $\Fin^{\simeq}$ denote the $\infty$-category of finite sets with bijections, i.e. it's the core of $\Fin$. 
Note that $\Fin^{\simeq}$ has a symmetric monoidal structure given by disjoint unions. Moreover, it's the free symmtric monoidal $\infty$-category generated by the one-object category $\{*\}$.
The $\infty$-category $\CP(\Fin^{\simeq})$ of presheaves admits a symmtric monoidal structure given by the Day convolution \cite{HA}[Example 2.2.6.9.], hence we can also consider it is an presentably symmetric monoidal $\infty$-category. These suggests an alternative description of $\CC$
\[
\Fun_{\CAlg(\Pr^L)}(\CP(\Fin^{\simeq}), \CC) 
\simeq 
\Fun_{\CAlg(\Cat_{\infty})} (\Fin^{\simeq}, \CC)
\simeq
\Fun(\{*\}, \CC)
\simeq 
\CC.
\]


\begin{definition}
	A \emph{symmetric sequence} in $\CC$ is a functor $\CO:\Fin^{\simeq} \to \CC$. We denote the $\infty$-category of symmetric sequences in $\CC$ by $\SSeq(\CC)$.
\end{definition}
\begin{remark}
	Informally, one can think of a symmetric sequence $\CO$ in $\CC$ as a sequence of objects $\{\CO(n)\}_{n\geq 0}$ in $\CC$ where $\CO(n):= \CO((n))$ carries an action of $\Sigma_n$ for each $n$.
	We will refer to $n$ as the \emph{arity} of the symmetric sequence $\CO$. We will sometimes describe an $\infty$-operad by informally giving a sequence $\{\CO(0), \CO(1), \dots \}$.
	
\end{remark}
Let $\otimes$ denote the symmetric monoidal product in $\Pr^L$ defined in \cite{HA}[Chapter 4]. The following lemma points out that 
	$\SSeq(\CC)$ is tensored over $\SSeq(\Space)$.
\begin{lemma}
	 Let $\CC$ be a presentably symmetric moniodal $\infty$-category.
	Then we have 
	$$
	\SSeq(\CC) \simeq \SSeq(\Space)\otimes \CC.
	$$
\end{lemma}
\begin{proof}
We first note that 
$
\SSeq(\Space)=\Fun(\Fin^{\simeq}, \Space)\simeq \CP(\Fin^{\simeq}),
$
since $\Fin^{\simeq}$ is isomorphic to its opposite category.
Using \cite{HA}[Proposition 4.8.1.17.], there are equivalences
\begin{align*}
	\SSeq(\Space)\otimes \CC 
& \simeq 
\Fun^R(\CP(\Fin^{\simeq})^{op},\CC)\\
&\simeq
(\Fun^{L}(\CP(\Fin^{\simeq}), \CC^{op}))^{op}\\
& \simeq
\Fun(\Fin^{\simeq},\CC^{op}) ^{op}\\
& \simeq \SSeq(\CC).
\end{align*}



\end{proof}

% The category of symmetric sequences $\SSeq(\CD)$ in $\CD$ has a monoidal product $\circ$ called the \emph{composition product}, defined so that the functor $\theta$ is a monoidal functor with respect to the composition monoidal structure in the codomain.   An \emph{ordinary operad} in $\CD$ is then defined as an associative monoid object with respect to this composition product.
%
%For our usage, we need a version of $\infty$-operads enriched in $\CC$. For this we need three ingredients, symmetric sequences, composition products and associative monoid objects. 
%The $\infty$-category $\SSeq(\CC)$ of \emph{symmetric sequences} in $\CC$ is defined as $\Fun(\Fin^{\simeq},\CC)$. 
%Given a symmetric sequence $\CO$ in $\SSeq(\CC)$, it has an associated monad on $\CC$ which is informally on an object $X\in \CC$ given by 
%$$
%\coprod_{n\geq 0} \CO(n)\otimes_{\Sigma_n} X^{\otimes n}.
%$$

Note that we have a canonical functor 
\[
\CC \to \SSeq(\CC)
\]
which sends an object $X$ to a symmetric sequence taking value at $X$ at arity 1 and the final object $*$ otherwise. 
The $\infty$-category $\SSeq(\CC)$ of symmetric sequences is equipped with a monoidal structure which corresponds to the composition of functors in the category $\Fun_{\CC/}^{\otimes,\Pr^{L}}(\SSeq(\CC),\SSeq(\CC))$; more precisely, there are equivalences of $\infty$-categories
\begin{align*}
	\Fun_{\CAlg(\Pr^L)_{\CC/}}(\SSeq(\CC),\SSeq(\CC))
	& \simeq 
\Fun_{\CAlg(\Pr^L)}(\CP(\Fin^{\simeq}),\SSeq(\CC))\\
& \simeq
\Fun_{\CAlg(\Cat_{\infty})}(\Fin^{\simeq},\SSeq(\CC))\\
& \simeq \SSeq(\CC).
\end{align*}

One can work out \todo{I think Rune/Hadrian/ Lukas worked that out} that the associated endofunctor of a symmetric sequence $\CO$ has the explicit form given by
\begin{align*}
		F_{\CO}: \SSeq(\CC) & \rightarrow \Fun(\CC,\CC)\\
	\CO        & \mapsto     (\coprod_{n\geq 0} \CO(n)\otimes_{\Sigma_n} X^{\otimes n})_{h\Sigma_n},
\end{align*}
where $\Sigma_n$ acts on the $n$-th tensor power of $X$ via permutation.
We refer $F_{\CO}$ as the \emph{Schur functor} associated to $\CO$.
Given two symmetric sequences $X$ and $Y$ in $\CC$, we will let $X\circ Y$ denote the monoidal product on $\SSeq(\CC)$ and refer it as the \emph{composition product}.

\begin{definition}
	\label{inf operads}
	We define the \emph{$\infty$-category $\Opd(\CC)$ of $\infty$-operads in $\CC$} as the $\infty$-category $\Alg(\SSeq(\CC))$ of asscociative algebras in $\SSeq(\CC)$.
\end{definition}

\begin{remark}
	We denote $\triv_{\CC}$ the symmetric sequence that takes value the tensor unit $1_\CC$ of $\CC$ at $\{*\}$ and zero object otherwise. One then checks that $\triv_{\CC}$ corresponds to the identity endofunctor on $\SSeq(\CC)$. Hence $\triv_{\CC}$ serves as a monoidal unit for the monoidal structure on $\SSeq(\CC)$.
	We will refer to $\triv_{\CC}$ as the \emph{trivial $\infty$-operad} in $\CC$.
\end{remark}


\begin{remark}
Although Lurie's approach to $\infty$-operads \cite{HA}[Chapter 2] is seemingly different from the one we give here. 
Haugseng \todo{which paper???} and Heine \cite{Hadrianphdthesis} show that Lurie's model is equivalent to the definition of $\infty$-operads in $\Space$.
\end{remark}

In this paper, we will exclusively work with \emph{non-unital $\infty$-operads}. Roughly speaking, a non-unital $\infty$-operad $\CO$ is an operad with no nullary operations.
\begin{definition}
	\label{non-unital operad}
	An $\infty$-operad $\CO$ in $\CC$ is \emph{non-unital} if $\CO(0)$ is equivalent to the initial object of $\CC$. We let $\operatorname{Op}_{\infty}^{nu}(\CC)$ denote the $\infty$-category of non-unital $\infty$-opeerads.
\end{definition}

\begin{remark}
Consider the canonical embedding $(\Fin^{nu})^{\simeq}\hookrightarrow \Fin^{\simeq}$, one can also define a \emph{non-unital symmetric sequence in $\CC$}  as a functor 
\[
\CO:(\Fin^{nu})^{\simeq}\to \CC.
\]
Hence for any symmetric sequence $\CO$ in $\CC$, 
the underlying non-unital symmetric sequence $\CO_{nu}$ is given by the restriction along $(\Fin^{nu})^{\simeq}\hookrightarrow \Fin^{\simeq}$.
\end{remark}

\begin{remark}
	Under Lurie's model of $\infty$-operads in $\Space$, a non-unital $\infty$-operad $\CO^{\otimes}$ is defined as an $\infty$-operad whose structure map $\CO^{\otimes} \to \Fin_*$ factors through $\Surj$, where $\Surj$ is the subcategory consisting of surjective maps. This is equivalent to our definition in which case $\CO(0)\simeq \emptyset$.
\end{remark}


\begin{definition}
\label{non-unitalization of operads}
		Let $\CO^{\otimes}$ be an $\infty$-operad in $\Space$, we define (as in \cite{HA}[Definition 5.4.4.1.]) the \emph{non-unitalization of $\CO^{\otimes}$} by $\CO^{\otimes}_{nu}:= \Surj\times_{\Fin_*}\CO^{\otimes}$.
\end{definition}




Dually, we define $\infty$-cooperads as coassociative coalgebras in $\SSeq(\CC)$ with respect to the composition product.
\begin{definition}
	\label{infty cooperads}
	We define the $\infty$-category $\Coop(\CC):=\Alg(\SSeq(\CC)^{op})^{op}$ of \emph{$\infty$-cooperads in $\CC$} as the opposite $\infty$-category of the $\infty$-category of associtive algebras in of $\SSeq(\CC^{op})$.
\end{definition}








\begin{definition}
	An $\infty$-operad $\CO$ in $\CC$ is \emph{augmented} if it admits map of $\infty$-operads $\epsilon:\CO\rightarrow \triv_{\CC}$ so that $\epsilon\circ \eta\simeq id_{\mathds{1}}$, where $\eta$ is the unit of $\CO$. We will denote the $\infty$-category of augmented operads and coaugmented cooperads in $\CC$ by $\operatorname{Op}^{aug}(\CC)$ and $\operatorname{coOp}^{aug}(\CC)$, respectively.

\end{definition}

\begin{definition}
	An non-unital $\infty$-operad $\CO$ is \emph{connected} \footnote{In \cite{ChingBar}, operads with this property are called reduced. However, reduced operads might have other meaning in other literature.}  if $\CO(1)\simeq 1_{\CC}$. 
\end{definition}


\subsection{Truncations of $\infty$-operads}
\todo{I'm still expecting Gijs' paper for more details to cite in this subsection}
In this section, we let $\CC$ be a pointed presentably symmtric monoidal $\infty$-category with tensor product compatible with colimits.
Let $\Fin^{nu}_{\leq n}$ denote the full subcategory of $\Fin^{nu}$ spanned by (non-empty) finite sets with cardinality less or equal to $n$.
The category $\Fun(\Fin^{nu}_{\leq n}, \CC)$ is a localization of $\Fun(\Fin^{nu}, \CC)$ via the restriction functor and the embedding $\Fun(\Fin^{nu}_{\leq n}, \CC) \hookrightarrow \Fun(\Fin^{nu}, \CC)$ is given by inserting zero objects in the arity above $n$.
\begin{notation}
	We denote $\operatorname{Op}_{\leq n}(\CC)$ the $\infty$-category of $\infty$-opeerads in $\CC$ whose underlying symmetric sequences belong to $\Fun(\Fin^{nu}_{\leq n}, \CC)$.	
\end{notation}

\begin{lemma}
	\cite{Hadrianphdthesis}[Lemma 2.16]
	Let $\CC$ be a pointed symmetric monoidal $\infty$-category with colimits.
	Then the localization functor
	$$
	(-)_{\leq n}: \Fun(\Fin^{nu}, \CC)
	\to
	\Fun(\Fin^{nu}_{\leq n}, \CC)
	$$
	is compatible with the composition product.
	Therefore we can lift $(-)_{\leq n}$ to a localization functor:
	$$
	\tau_{n}: \Opd(\CC)\to \operatorname{Op}_{\leq n}(\CC),
	$$
	so that the following square commutes
\[
\begin{tikzcd}
%%%% The nodes %%%%%%%%%
	\Opd(\CC)  & 
	\operatorname{Op}_{\leq n}(\CC)\\
	\Fun(\Fin^{nu}, \CC) & \Fun(\Fin^{nu}_{\leq n}, \CC)
%%%% Now the arrows %%%%
	\arrow[from=1-1, to= 1-2, "\tau_n"]
	\arrow[from=1-1, to=2-1, "\oblv" left]
	\arrow[from=1-2, to=2-2, "\oblv'" ]
	\arrow[from=2-1, to= 2-2, "(-)_{\leq n}"]
\end{tikzcd}
\]
	which we call the \emph{$n$-truncation functor}.
\end{lemma}


%The embedding $\operatorname{Op}_{\leq n}(\CC) \to \Opd(\CC)$ also admits a right adjoint $\phi_n$. That is, we have a colocalization functor
\clearpage
The $n$-truncation functor $\tau_n$  
\todo{how to define this functor? Is presentability enough for defining it?}  
also admits a left adjoint 
$$f_n: \operatorname{Op}_{\leq n}(\CC) \to \Opd(\CC).$$ 
Intuitively, $f_n \CO$ is freely generated by $\CO(k)$, for $1\leq k \leq n$, subject to the relations in that portion of $\CO$.
To summarize, we have a diagram with the functors described above
\begin{equation}
\label{Op-Op_n adjunction}
	\begin{tikzcd}
	%%%% The nodes %%%%%%%%%
\operatorname{Op}_{\leq n}(\CC) & \Opd(\CC)
%%%% Now the arrows %%%%
	\arrow[from=1-1, to= 1-2,bend left, shift left = 3, "f_n"]
	\arrow[from=1-1, to=1-2, bend right, shift right = 3, "inc" below]
	\arrow[from=1-2, to=1-1, "\tau_n"]
\end{tikzcd}
\end{equation}
where the left adjoints are above the right adjoints.

Similarly, for any non-unital cooperad $\CQ$ in $\CC$, we also have a pair of adjunctions
\begin{equation}
\label{coOp-coOp_n adjunction}
	\begin{tikzcd}
	%%%% The nodes %%%%%%%%%
\operatorname{coOp}_{\leq n}(\CC) & \Coop(\CC)
%%%% Now the arrows %%%%
	\arrow[from=1-1, to= 1-2,bend left, shift left = 3, "inc"]
	\arrow[from=1-1, to=1-2, bend right, shift right = 3, "f^n" below]
	\arrow[from=1-2, to=1-1, "\tau^n"]
\end{tikzcd}
\end{equation}
where the left adjoints are above right adjoints. 

\begin{notation}
	We will abuse notation by writting $\tau_{n}$ for the unit of the bottom adjunctions in \eqref{Op-Op_n adjunction}
	$$
	\Opd(\CC)  
	\xrightarrow{\tau_n}
	\operatorname{Op}_{\leq n}(\CC)
	\hookrightarrow
	\Opd(\CC)
	$$
	and $\tau^n$ for the counit of the top adjunction in \eqref{coOp-coOp_n adjunction}
	$$
	\Coop(\CC) 
	\xrightarrow{\tau^n}
	\operatorname{coOp}_{\leq n}(\CC)
	\hookrightarrow
	\Coop(\CC).
	$$
	We denote  $\phi_n$ for the counit of the top adjunctions in \eqref{Op-Op_n adjunction}
	$$
	\Opd(\CC)  
	\xrightarrow{\tau_n}
	\operatorname{Op}_{\leq n}(\CC)
	\xrightarrow{f_n}
	\Opd(\CC)
	$$
	and $\phi^n$ for the unit of the bottom adjunction in \eqref{coOp-coOp_n adjunction}
	$$
	\Coop(\CC) 
	\xrightarrow{\tau^n}
	\Coop(\CC)
	\xrightarrow{f^n}
	\Coop(\CC).
	$$
\end{notation}

\begin{remark}
\begin{itemize}
	\item For any (non-unital) $\infty$-operad $\CO$ in $\CC$, we have a sequence
	\[
	\phi_n\CO \to \CO \to \tau_n \CO.
	\]
	The map $\CO \to \tau_n \CO$ is terminal among those maps that are equivalences in arities less or equal to $n$ (localization).
	The map $\phi_n\CO \to \CO$ is initial among those maps that are equivalences in arities less or equal to $n$ (colocalization).

	\item For any (non-unital) $\infty$-cooperad $\CQ$ in $\CC$, we have a sequence
	\[
	\tau^n \CQ \to \CQ \to \phi^n \CQ.
	\]
	The map $\CO \to \phi^n \CO$ is terminal among those maps that are equivalences in arities less or equal to $n$ (localization).
	The map $\tau^n\CO \to \CO$ is initial among those maps that are equivalences in arities less or equal to $n$ (colocalization).
\end{itemize}
	
\end{remark}

\begin{remark}
	Using the remark above, we see that there is a direct system for any non-unital $\infty$-operad $\CO$, 
\begin{equation}
\phi_1\CO \to \phi_2\CO \to \cdots \to \CO.
\end{equation}
Similarly, there is an inverse system for the $n$-truncation of $\infty$-operads.
\begin{equation}
\cdots \to 
\tau_2 \CO
\to 
\tau_1 \CO
\to 
\CO
\end{equation}
\end{remark}
%\label{telescope for O_n}
%	\begin{tikzcd}
%	%%%% The nodes %%%%%%%%%
%	\vdots & \\
%	\phi_3 \CO & \\
%	\phi_2 \CO & \\
%	\phi_1 \CO & \CO
%	%%%% Now the arrows %%%%
%	\arrow[from=2-1, to= 1-1]
%	\arrow[from=2-1, to= 4-2]
%	\arrow[from=3-1, to=2-1]
%	\arrow[from=3-1, to=4-2]
%	\arrow[from=4-1, to=3-1]
%	\arrow[from=4-1, to=4-2]
%	\end{tikzcd}



\begin{proposition}
\todo{This was hinted in \cite{Heuts_Goodwillie}[Proposition 4.10], but it will be in Heuts' new paper.}
	Let $\CO$ be an $\infty$-operad in $\CC$, 
	then there is an equivalence
	\[
	\CO \simeq
	\colim_n \phi_n \CO
	\]
	in $\Opd(\CC)$.
\end{proposition}





Let $\Surj_{\leq n}$ denote the full subcategory of $\Surj$ spanned by finite sets with cardinality less or equal to $n$.
\section{(co)Operadic (co)Algebras}


A symmetric sequence $\CO\in\SSeq(\CC)$ acts on $\CC$ via its Schur functor under the equivalence
\[
\SSeq(\CC)
\simeq
\Fun_{\CAlg(\Pr^L)_{\CC/}}(\SSeq(\CC)\SSeq(\CC)).
\]
If $\CO$ is an $\infty$-operad, then we can consider the category of "$\CO$-modules in $\CC$".
%\begin{align*}
%SSeq(\CC) \times \CC & \to \CC\\
%(\CO, X)   & \mapsto  \coprod_{n\geq 0} (\CO(n)\otimes X^{\otimes n})_{h\Sigma_n}.
%\end{align*}
\begin{definition}
	\label{algebras over an operad}
	Let $\CO$ be an $\infty$-operad in $\CC$, we define the $\infty$-category of $\CO$-algebras as $\Alg_{\CO}(\CC):= \LMod_{\CO}(\CC)$.
\end{definition}

Dually, if $\CQ$ is a cooperad in $\CC$ we obtain an action of $\CQ$ on the opposite category of $\CC$. 
\begin{definition}
\label{coalgebras over a cooperad}
	Let $\CQ$ be an $\infty$-cooperad in $\CC$, we define the $\infty$-category of $\CQ$-coalgebras over $\CO$ as $\coAlg_{\CQ}(\CC):= \operatorname{LcoMod}_{\CQ}(\CC)$.
\end{definition}


Let $\CO$ be a connected operad. 
Then this assumption gives a unique augmentation $\epsilon:\CO\rightarrow \mathds{1}$ which is also a map of operads. 
Then the restriction along $\epsilon$ gives a functor 
$$
\trivial{\CO}: \CC\simeq\Alg_{\mathds{1}}(\CC)\rightarrow \Alg_{\CO}(\CC)
$$
which we will refer to as the \textit{trivial $\CO$-algebra functor}.
The trivial $\CO$-algebra functor admits a left adjoint $\TAQ_{\CO}:\CC\rightarrow \Alg_{\CO}(\CC)$. This notation stands for topological Andr\'e-Quillen homology.

Similarly, the restriction along the unit map $\eta:\mathds{1}\rightarrow \CO$ gives a functor
$$
\oblv_{\CO}: \Alg_{\CO}(\CC)
\rightarrow 
\CC\simeq\Alg_{\mathds{1}}(\CC)$$
which we will refer to as the \textit{$\CO$-forgetful functor}.
The $\CO$-forgetful functor admits a left adjoint $\Free_{\CO}:\CC\rightarrow \Alg_{\CO}(\CC)$ which we will refer to as the \textit{free $\CO$-algebra functor}. If $\CC$ is the category of chain complexes over $k$ and $\CO$ is the associative operad, then $\Free_{\CO}$ is the tensor algebra functor. 


\begin{definition}
Let $F:\CC\to \CC$ be an endofunctor.
	We define $\{F(X) \to X\}_{X \in \CC}$ to be the pullback of the following diagram in $\Cat_{\infty}$
\[
\begin{tikzcd}
%%%% The nodes %%%%%%%%%
	\{F(X) \to X\}_{X \in \CC} & 
	\Fun(\Delta^1, \CC)\\
	\CC & 
	\CC \times \CC
%%%% Now the arrows %%%%
	\arrow[from=1-1, to= 1-2]
	\arrow[from=1-1, to=2-1]
	\arrow[from=1-2, to=2-2, "(ev_0\text{,} ev_1)"]
	\arrow[from=2-1, to= 2-2, "(F\text{,} id)"].
\end{tikzcd}
\]
That is, $\{F(X) \to X\}_{X \in \CC}$ is the subcategory of $\CC$ consisting of objects $X$ equipped with a morphism $F(X)\to X$.
\end{definition}


The following proposition allows us to build an $\CO$-algebra inductively.
\todo[inline]{Below is analogous to Lemma 4.23 in \cite{Heuts_Goodwillie} but with out the assumption that $\CC$ is stable }
\begin{proposition}

	Let $\CC$ be a presentably symmetric monoidal $\infty$-category whose tensor product is compatible with colimits and let $\CO$ be a non-unital $\infty$-operads in $\CC$.
	Then we have a pull back diagram of $\infty$-categories:
\[
\begin{tikzcd}
%%%% The nodes %%%%%%%%%
	\Alg_{\phi_n \CO}(\CC) & 
	\{(\CO(n)\otimes X^{\otimes n})_{h\Sigma_n} \to X\}_{X \in \CC}\\
	\Alg_{\phi_{n-1} \CO}(\CC) & 
	\{(\phi_{n-1}\CO(n)\otimes X^{\otimes n})_{h\Sigma_n}\to X   \}_{X\in \CC}.
%%%% Now the arrows %%%%
	\arrow[from=1-1, to= 1-2]
	\arrow[from=1-1, to=2-1]
	\arrow[from=1-2, to=2-2]
	\arrow[from=2-1, to= 2-2]
\end{tikzcd}
\]
\end{proposition}

Dually, for coalgebras over non-unital cooperads $\CQ$, we have 
\begin{proposition}
\label{inductive construction of coalgebras}
	Let $\CC$ be a presentably symmetric monoidal $\infty$-category whose tensor product is compatible with colimits and let $\CQ$ be a non-unital $\infty$-operads in $\CC$.
	Then we have a pull back diagram of $\infty$-categories:
\[
\begin{tikzcd}
%%%% The nodes %%%%%%%%%
	\coAlg_{\phi^n \CQ}(\CC) & 
	\{X \to (\CQ(n)\otimes X^{\otimes n})^{h\Sigma_n} \}_{X \in \CC}\\
	\coAlg_{\phi^{n-1} \CQ}(\CC) & 
	\{ X \to (\phi^{n-1}\CQ(n)\otimes X^{\otimes n})^{h\Sigma_n}   \}_{X\in \CC}.
%%%% Now the arrows %%%%
	\arrow[from=1-1, to= 1-2]
	\arrow[from=1-1, to=2-1]
	\arrow[from=1-2, to=2-2]
	\arrow[from=2-1, to= 2-2]
\end{tikzcd}
\]
\end{proposition}


%Combining the two adjunctions, we obtain a diagram:
%\[
%\begin{tikzcd}
%\CC \arrow[r,shift left=1,"\free_{\CO}"]  & \Alg_{\CO}(\CC) \arrow[r,shift left=1,"\TAQ_{\CO}"] \arrow[l,shift left=1,"\oblv_{\CO}"]  & \CC \arrow[l,shift left=1,"\trivial_{\CO}"] 
%\end{tikzcd}
%\]
%where left adjoints are on top and both composites are equivalent to $id_{\CC}$.
%% Denote $F_{\CO}:\CC\rightarrow \CC$ its associated monad.
%\subsection{Conilpotent coalgebras over a cooperad}
%Let $\CP$ be a connected cooperad.
%Let $F_{\CP}$ be the comonad associated to $\CP$. We denote the category of coalgebras over $F_{\CP}$ by $\coAlg^{nil,dp}_{\CP}(\CC)$. An object $X\in \coAlg^{nil,dp}_{\CP}(\CC)$ is equipped with a structure map:
%$$
%X\rightarrow \bigoplus_{n\geq 1}\big( \CP(n)\otimes X^{\otimes n}\big)_{h\Sigma_n}.
%$$
%We will refer to $\coAlg^{nil,dp}_{\CP}(\CC)$ \textit{the category of conilpotent divided power $\CP$-coalgebras}. 
%Dual to the algebra case, we have a diagram of two pair of adjoint functors:
%\[
%\begin{tikzcd}
%\CC \arrow[r,shift left=1,"\cofree^{nil,dp}_{\CP}"]  & \coAlg^{nil,dp}_{\CP}(\CC) \arrow[r,shift left=1,"\Prim^{nil,dp}_{\CP}"] \arrow[l,shift left=1,"\oblv^{nil,dp}_{\CP}"]  & \CC \arrow[l,shift left=1,"\trivial^{nil,dp}_{\CP}"] 
%\end{tikzcd}
%\]
%where functors at the bottom are left adjoints and both composites are equivalent to $id_\CC$.
%The adjective conilpotent stands for the direct sum and the adjective divided power corresponds to coinvariants in $F_{\CP}$.
%
%
%
%\begin{theorem}[\cite{Francis-Gaitsgory}, \cite{HeutsSurvey}]
%The functor $\TAQ_{\CO}$ factors as 
%$$
%\Alg_{\CO}(\CC) \xrightarrow{B_\CO} \coAlg^{nil,dp}_{\Barconstruction(\CO)}(\CC)\xrightarrow{\oblv_{\Barconstruction (\CO)}} \CC.
%$$	
%
%\end{theorem}

\section{Examples}
\begin{example}
	Consider the \emph{symmetric algebra functor} on $\CC$,
	\begin{align*}
		\Sym:\CC & \to  \CC\\
		X & \mapsto \amalg_{n\geq 1} (X^{\otimes n})_{h\Sigma_n}.
	\end{align*}
	One can check $\Sym$ corresponds to the symmetric sequence $\Com:=(1_\CC,1_\CC,\dots)$, where $1_\CC$ is the tensor unit of $\CC$.
	Since $\Sym$ is a monad on $\CC$, 
	we will refer to its corresponding operad $\Com$ the \emph{commutative operad} in $\CC$.
	Similarly, the \emph{non-unital symmetric algebra functor} $\Sym^{\geq 1}:\CC \to \CC$ given by $X\mapsto \amalg_{n\geq 1} (X^{\otimes n})_{h\Sigma_n}$ defines a non-unital $\infty$-operad in $\CC$ which we will denote by $\Com_{nu}$.
\end{example}

We now give an example of an operad in $\Sp$ that will be central in this thesis. 
We need to provide some backgrounds on partion posets first. 
Consider the set $\mathbf{P}(n)$ of partitions (equivalence relations) on the set $\{1, \cdots, n \}$. Note that $\mathbf{P}(n)$ form a poset with the minimal element the trivial partition and the maximal element the discrete partition.
Write $\mathbf{P}^{+}(n)$ (resp. $\mathbf{P}^{-}(n)$ ) for the subposet of $\mathbf{P}(n)$ obtained by deleting the minimal (resp. maximal) partitions.
\begin{example}
	The Goodwillie derivative $\partial_* \id$ of the identity functor $\id: \pSpace \to \pSpace$ forms a symmetric sequence in $\Sp$; each $\partial_n \id$ is a $\Sigma_n$-spectrum. Their explicit form was shown in \cite{JohnsonDerivative} and \cite{Arone-Mahowald} as
	$$
	\partial_n \id \simeq D(\Sigma^{\infty} K_n)
	$$
	where 
	$
	K_{n}:=|\mathbf{P}(n)| /\left(\left|\mathbf{P}^{+}(n)\right| \cup\left|\mathbf{P}^{-}(n)\right|\right)
	$
	and $D$ stands for the Spanier-Whitehead dual. 
	Ching \cite{ChingBar} observes that the cobar construction of the commutative cooperad in $\Sp$ is precisely given by the Goodwillie derivatives  i.e. 
	$$
	\Cobar(\Com^{\vee})(n)\simeq \partial_* \id.
	$$
	Furthermore, he shows that the symmetric sequence $\partial_* \id$ forms an operad in $\Sp$ which we refer as the \emph{spectral Lie operad} $\BL$. 
	The name comes from the following computation; let $\Lie$ denote the Lie operad in $\Ab$, then we have \cite{ChingBar}[Example 9.50]
	$$
	H_{*}\BL(n) \cong 	
	\begin{cases}
	(\Lie(n)[1])\otimes (\BZ[-1])^{\otimes n}& \text{ if $*=1-n$},\\
	0 & \text{otherwise}
	\end{cases}
	$$	
	where $\Sigma_n$ acts on $(\BZ[-1])^{\otimes n}$ by permuting the factors. 
\end{example}

For reasons that will become clear later, the following shifted version of spectral Lie operad is the one  we will use throughout the rest of the thesis.
\begin{definition}
	\label{desuspended spectral Lie operad}
	We define the (desuspended) spectral Lie operad as $\spLie := \mathbf{\Sigma}^{-1} \BL$.
\end{definition}

\begin{remark}
\label{Remark on shifted Lie operads}
	Here is our convention for shifted operads.
	The shifted $\infty$-operad $\bf{\Sigma} \CO$ of $\CO$ in a stable $\infty$-category is an $\infty$-operad  defined so that its associated monad is
	$
	\Omega F_{\CO}\Sigma,
	$
	hence its underlying symmetric sequence has the form
	$$
	(\mathbf{\Sigma} \CO ) (n)= \BS^{n}\otimes\Sigma^{-1} \CO(n)
	$$
	where $\Sigma_n$ acts on $\BS$ by permuting the $n$ factors of $\BS$.
	Take $\CO=\BL$, then
	the underlying symmetric sequence of the (desuspended) Lie operad $\spLie:= \mathbf{\Sigma}^{-1}\BL$
	has the form 	
	$$
	\spLie(n)= \BS^{-n} \otimes \Sigma \BL(n),
	$$
	whose homology is exactly $\Lie(n)$ concentrated in degree $0$.
	
	Moreover, equipping a $\spLie$-algebra structure on a spectrum $X$ is equivalent to equipping $\Sigma^{-1} X$ a $\BL$-algebra structure; indeed, this can be seen by inspecting their associated monads
	\begin{align*}
		F_{\spLie}(X) &\simeq \coprod_{n\geq 1}
		\big(
		\BS^{-n} \otimes \Sigma\BL(n) 
		\otimes X^{\otimes n}
		\big)_{h\Sigma_n}\\
		& \simeq 
		\Sigma
		\coprod_{n\geq 1}
		\big(
		\BL(n) 
		\otimes (\Sigma^{-1} X)^{\otimes n}
		\big)_{h\Sigma_n}\\
		& \simeq 
		\Sigma F_{\BL}\Sigma^{-1}(X).
	\end{align*}	
\end{remark}


\section{Koszul Duality}
\begin{proposition}
	[Bar-cobar adjunction for $\infty$-operads, \cite{Haugsengsymseq} Corollary 4.3.2.]
	There is an adjunction
	$$
	\adj{\Barconstruction}{\Opd^{aug}(\CC)}{\Coop^{aug}(\CC)}{\Cobar},
	$$
	where $\Barconstruction(\CO)$ is computed as the geometric realization of the simplicial object
	\[
		\xymatrix{ \mathds{1}\ar@<0ex>[r]  &  
		\CO \ar@<1ex>[r] \ar@<-1ex>[r] \ar@<1ex>[l] \ar@<-1ex>[l]  &  \CO\circ \CO \ar@<2ex>[l] \ar@<0ex>[l] \ar@<-2ex>[l] \cdots}
\]
	and $\Cobar(\CP)$ of a cooperad $\CP$ is computed as the totalization of the cosimplicial object
	\[
		\xymatrix{ \mathds{1} \ar@<1ex>[r] \ar@<-1ex>[r] &  
		\CP \ar@<0ex>[l] \ar@<2ex>[r] \ar@<0ex>[r] \ar@<-2ex>[r] & \CP\circ \CP \ar@<1ex>[l] \ar@<-1ex>[l] \cdots}.
\]
\end{proposition}

We will be mostly concerned about the commutative cooperad $\Com^{\vee}$ and its cobar construction $\Cobar(\Com^{\vee})$. 
\begin{definition}
	The operad $\Lie$ in $\CC$ is defined to be the cobar construction of the commutative cooperad $\Com^{\vee}$.
\end{definition}
The notation $\Lie$ comes from the fact that if we let $\CC$ be the category of chain complexes over a field of characteristic $0$, then algebras over $\Lie$ are indeed differential graded Lie algebras. See \cite{ChingBar}.


\todo[inline]{Need a section for discussing Koszul duality for $\infty$-operads, which I'll follow an upcoming paper of Heuts.}

%\begin{comment}
	\begin{definition}
	An $\infty$-operad $\CO\in \Opd^{aug}(\CC)$ is \emph{Koszul} if the unit map
	$$
	\CO\rightarrow \Cobar\circ \Barconstruction(\CO)
	$$
	is an isomorphism.
\end{definition}
%\end{comment}

\begin{proposition}
	Let $\CO$ be a connected operad, the following map 
	$$
	\tau_{\leq n}\CO \rightarrow 
	\Cobar(\Barconstruction(\tau_{\leq n}\CO,\CO,\mathds{1}),\Barconstruction(\CO),\mathds{1})
	$$
	is an isomorphism for $n\geq 1$.
\end{proposition}
\begin{proof}
	We prove this by induction on $n$.
	For $n=1$, the map $\triv\simeq \tau_{\leq 1}\CO \rightarrow \Cobar(\Barconstruction(\CO),\Barconstruction(\CO),\triv)$ is clearly an isomorphism.
	For $n>0$, we first consider the following diagram in $\Opd^{aug}(\CC)$.
	\[
	\begin{tikzcd}
	\tau_{\leq n}\CO \ar[d]\ar[r] & \Cobar \big(\Barconstruction(\tau_{\leq n}\CO,\CO,\triv),\Barconstruction (\CO), \triv	 \big) \ar[d]  \\
	\tau_{\leq n-1}\CO \ar[r]    & 
	\Cobar \big(\Barconstruction(\tau_{\leq n-1}\CO,\CO,\triv),\Barconstruction (\CO), \triv	 \big)
	\end{tikzcd}
	\]
	Taking the fibers of the vertical maps, we obtain another diagram
		\[
	\begin{tikzcd}
	\CO_n \arrow[d] \arrow[r] & 
	\Cobar \big(\Barconstruction(\CO_n,\CO,\triv),\Barconstruction (\CO), \triv	 \big) \ar[d] \\
	\tau_{\leq n}\CO \ar[d]\ar[r] & \Cobar \big(\Barconstruction(\tau_{\leq n}\CO,\CO,\triv),\Barconstruction (\CO), \triv	 \big) \ar[d]  \\
	\tau_{\leq n-1}\CO \ar[r]    & 
	\Cobar \big(\Barconstruction(\tau_{\leq n-1}\CO,\CO,\triv),\Barconstruction (\CO), \triv	 \big)
	\end{tikzcd}
	\]
	where $\CO_n$ is the operad whose underlying symmetric sequence is 
	$$
	 \CO_n:= \begin{cases}
    \CO(n), & \text{for } k = n \\
   0, & \text{otherwise. }         \\
  \end{cases}
	$$
	Observe that 
	\begin{align*}
			\Cobar \big(\Barconstruction(\CO_n,\CO,\triv),\Barconstruction (\CO), \triv \big) &
			\simeq \Cobar \big(\CO_n\circ \Barconstruction(\CO),\Barconstruction (\CO), \triv	 \big)\\
			& \simeq \CO_n,
	\end{align*}
	which implies that the top horizontal map is an equivalence. The bottom horizontal map is an equivalence by induction hypothesis. Therefore, the mid-horizontal map is an equivalence as well. This concludes the proposition.
	
\end{proof}
\begin{corollary}
		If an operad $\CO$ is connected, then $\CO$ is a Koszul operad.
\end{corollary}
\begin{proof}
	
\end{proof}

\begin{theorem}
\label{Koszul dual of commutative cooperad}
\todo{Ref?}
	The Koszul dual of the commutative cooperad $\Com^{\vee}$ is the (desuspended) spectral Lie operad $\spLie$.
\end{theorem}


\section{The Universal Enveloping Algebra Functor}
Let $k$ be a field of characteristc $0$. Denote the category of $k$-vector spaces by $\vectk$.
If $A$ is a unital associative algebra over $k$, then the commutator operation endows $A$ with a Lie algebra structure. This construction is functorial so that it determines a functor from the category $\Alg_{\Ass}(\vect_k)$ of unital associative algebras to the category $\Alg_{\Lie}(\vect_k)$ of Lie algebras. 
Conversely, given a Lie algebra $L$ one can define its universal enveloping algebra $U(L)$; as a vector space, one can construct it as the tensor algebra $T(L)$ modulo the ideal $I$ generated by 
 elements of the form $x\otimes y - y\otimes x-[x,y]$. 
This construction is functorial and one check it defines a functor
$$
U:\Alg_{\Lie}(\vect_k) \rightarrow \Alg_{\Ass}(\vect_k),
$$
which is left adjoint to the forgetful functor.
We'll refer the functor $U$ as the \emph{enveloping algebra functor}. 

The enveloping algebra functor is well studied both in algebra and topology. We state two results concerning the enveloping algebra functor $U$ that will be relevant later.
\begin{itemize}
	\item The enveloping algebra functor $U$ is symmetric monoidal; that is, if we endow the category of Lie algebras $\Alg_{\Lie}(\vect_k)$ with the Cartesian symmetric monoidal structure and $\Alg_{\Ass}(\vect_k)$ with the usual symmetric monoidal structure with tensor product as monoidal product, then $U$ sends categorical products to tensor products. As a consequence, $U(L)$ has a canonical coalgebra structure for any Lie algebra $L\in \Alg_{\Lie}(\vect_k)$.
	\item Over a field of characteristic, the Poincare-Witt-Birkhoff theorem has a strong form that gives a simple description of the universal enveloping algebra $U(L)$. 
	\begin{theorem}
		[Poincare-Birkhoff-Witt]
		\cite{Quillen_RHT}[Appendix B Theorem 2.3]
		Let $L$ be a Lie algebra over $k$ and let $i:L\to U(L)$ denote the unit map of. Then the "averaging" map
		\begin{align*}
			S(L) & \to U(L) \\
			x_1\cdots x_n & \mapsto \frac{1}{n!}\sum_{\sigma\in \Sigma_n} i(x_{\sigma(1)})\cdots i(x_{\sigma(n)})
		\end{align*}	
		is an isomorphism of coalgebras.
	\end{theorem}
	
\end{itemize}

The goal of this paper is to discuss coalgebras and the enveloping algebra functor in a more general context. 

\section{Highly Structured Algebras in $\Sp$}
\todo[inline]{Added Feb. 1st:
The following paragraph is a summary of what needs to be done.}
We will discuss Lie algebras in the category of tame spectra. Essentially, this is simply the category of spectral Lie algebras whose underlying spectra are tame.
Using the theory of Koszul duality introduced earlier this section, we will state the main result of \cite{Ching-Harper} and apply to the tame spectra case. The main result is that the $\infty$-category $\Alg_{\Lie}(\Sp_{tame})$ of tame spectral Lie algebras is equivalent to the $\infty$-category $\coAlg^{nil, dp}_{\Com}(\Sp_{tame})$ of conilpotent divided power coalgebras.

\begin{definition}
	[The Lie operad in $\Mod_{H\BZ}$]
	\todo{to do}
\end{definition}

Since the spectral Lie operad is defined as the Koszul dual operad of the commutative cooperad in $\Sp$, we obtain the Koszul duality functor:
\[
B_{\Lie}: \Alg_{\Lie}(\Sp) \to \coAlg^{nil, dp}_{\Com}(\Sp).
\]

\begin{theorem}
\label{Ching-Harper's Theorem}
\cite{Ching-Harper}
\label{Ching-Harper's Koszul duality}
	Let $R$ be an $\E_{\infty}$-ring spectrum and $\CO$ an operad in the $\infty$-category $\Mod_R$ of $R$-module spectra with $\CO(1)=R$. Suppose $R$ and all $\CO(n)$ are connective. Then the restriction 
	\[
	B_{\CO}: \Alg_{\CO}(\Mod_R)^{\geq 1} \to \coAlg^{nil, dp}_{\Barconstruction(\CO)}(\Mod_R)^{\geq 1}
	\]
	 is an equivalence on the full subcategory of connected objects.
\end{theorem}

Let's now consider the case when $R=H\BZ$. The classical results of \cite{Ginzburg-Kapranov} show that the Koszul dual cooperad of the commutative operad in the $\infty$-category $\Mod_{H\BZ}$ is the commutative cooperad $\Com^{\vee}$ with a degree shift.
Combining this fact with Theorem \ref{Ching-Harper's Theorem}, we conclude the following:
\begin{corollary}
	The Koszul duality functor 
	\[
	B_{\Lie}: \Alg_{\Lie}(\Mod_{H\BZ})^{\geq 1} \to \coAlg^{nil, dp}_{\Com}(\Mod_{H\BZ})^{\geq 1}
	\]
	restricts to an equivalence of $\infty$-categories on the full subcategory of connected objects.
\end{corollary}


Recall we have an equivalence 
$$
\Sp^{\geq r}_{tame}
\simeq
\Mod_{H\BZ, tame}^{\geq r}
$$
of $\infty$-categories, and the localization functor $L_{tame}:\Mod_{H\BZ}^{\geq r}\to (\Mod_{H\BZ}^{\geq r})_{tame}$ is symmetric monoidal. Hence the Lie opeard acts on $\Sp_{tame}^{\geq r}$ via pulling back along the localization functor (See appendix \todo{See appendix}).
\begin{definition}
	We let $\Alg_{\Lie}(\Sp^{\geq r}_{tame}):= \LMod_{\Lie}(\Sp^{\geq r}_{tame})$ denote the $\infty$-category of left modules over the Lie monad in $Sp^{\geq r}_{tame}$ and we will refer its objects as \textit{tame Lie algebras}.
	
\end{definition}

The following proposition allows us to identify $\Alg_{\Lie}(\Sp^{\geq r}_{tame})$ as a full subcategory of Lie algebras in $Mod_{H\BZ}$.

\begin{proposition}
\label{Identification of tame Lie algebras}
	The $\infty$-category $\Alg_{\Lie}(\Sp^{\geq r}_{tame})$ can be identified as the full subcategory of $\Alg_{\Lie}(\Mod_{H\BZ}^{\geq r})$ spanned by Lie algebras whose underlying $H\BZ$-modules are tame.
\end{proposition}
\begin{proof}
	By Proposition \ref{Lifting localization to algebra category}, we obtain a localization functor
	$F: \Alg_{\Lie}(\Mod_{H\BZ}^{\geq r}) \to \Alg_{\Lie}(\Sp^{\geq r}_{tame})$. Since we have a commuting diagram
\[
\begin{tikzcd}
	\Alg_{\Lie}(\Mod_{H\BZ}^{\geq r}) & \Alg_{\Lie}(\Sp^{\geq r}_{tame})\\
	\Mod_{H\BZ}^{\geq r}  & 
	\Sp^{\geq r}_{tame}
	\arrow[from=1-2, to= 1-1]
	\arrow[from=1-1, to=2-1, "\oblv" left]
	\arrow[from=1-2, to=2-2, "\oblv'" ]
	\arrow[from=2-2, to= 2-1]
\end{tikzcd}
\]
where the horizontal maps are fully faithful and the vertical maps are conservative, the proposition now follows.
\end{proof}

By a dual argument, we can also identify the category of tame coalgebras as a full subcategory of $\coAlg^{nil, dp}_{\Com}(\Mod_{H\BZ}^{\geq r})$.
\begin{proposition}
	The $\infty$-category $\coAlg^{nil, dp}_{\Com}(\Sp^{\geq r}_{tame})$ can be identified as the full subcategory of $\coAlg^{nil, dp}_{\Com}(\Mod_{H\BZ}^{\geq r})$ spanned by coalgebras whose underlying $H\BZ$-modules are tame.	
\end{proposition}

We want to show the functor 
$$
B_{\Lie}: \Alg_{\Lie}(\Mod_{H\BZ})^{\geq r} \to \coAlg^{nil, dp}_{\Com}(\Mod_{H\BZ}^{\geq r+1})
$$ 
restricts to a functor:
$$
B_{tame}: \Alg_{\Lie}(\Sp^{\geq r}_{tame})\to \coAlg^{nil, dp}_{\Com}(\Sp^{\geq r+1}_{tame})
$$
and it is an equivalence if we precompose $B_{tame}$ with the functor $\Omega_{\Lie}$. We claim that the composition $B_{tame}\circ \Omega_{\Lie}$ is an equivalence of $\infty$-categories.

\begin{remark}
\label{symmetric monoidal structure on Mod_Z tame}
	Note that $\Sp^{\geq r}_{tame}$ admits a symmetric monoidal structure with the tensor product $X\hat{\otimes} Y$ given by the tame localization $L_{tame}(X\otimes Y)$ of the smash product of $X$ and $Y$. 
	Moreover, the functor $L_{tame}:\Mod_{H\BZ}^{\geq r}\to (\Mod_{H\BZ}^{\geq r})_{tame}$ is symmetric monoidal; this follows from the fact that the composite map $M \otimes_{\BZ} N \to L_{tame}M\otimes_{\BZ} N \to L_{tame}M\otimes_{\BZ} L_{tame}N$ is a tame equivalence for any $M,N\in \Mod_{H\BZ}^{\geq r}$.
\end{remark}



We will view $\Alg_{\CO}(\CC)\simeq \LMod_\CO(\CC)$ where we abuse notation by identifying $\CO$ as the associated monad of $\CO^{\otimes}$. 

\begin{lemma}
\label{induced maps on SSeq}
\cite{Hadrianphdthesis}[Remark 5.81]
Let $\CC$ and $\CD$ be symmetric monoidal $\infty$-categories and let 
$$
F: \CC \to \CD
$$
be a colimit-preserving, symetric monoidal functor.
Then $F$ induces a functor 
$$
\SSeq(F): \SSeq(\CC) \to \SSeq(\CD)
$$
which is monoidal with respect to the composition product.
\end{lemma}	

Since an operad $\CO$ in $\CC$ is simply an associative algebra object in $\SSeq(\CC)$, under the assumption of Lemma \ref{induced maps on SSeq}, we obtain an operad $F_{!}(\CO)$ in $\CD$. Moreover, it induces a functor:
\[
F' : \Alg_{\CO}(\CC) \to \Alg_{F_{!}(\CO)}(\CD).
\]
For our purpose, we are interested in the case when $F:\CC \to \CD$ is a localization functor.


\begin{proposition}
\label{Lifting localization to algebra category}
Let $\CC$ and $\CD$ be symmetric monoidal $\infty$-categories. 
Suppose $F:\CC \to \CD$ is a symmetric monoidal localization functor.
Let $\CO$ be an $\infty$-operad in $\CC$, then $F$ induces a localization 
$$
F': \Alg_{\CO}(\CC) \to  \Alg_{F_{!}(\CO)}(\CD)
$$
on the $\infty$-category of $\CO$-algebras.
\end{proposition}
\begin{proof}
	The existence of the functor $F'$ follows from Lemma \ref{induced maps on SSeq}. It remains to show $F'$ is a localization functor. Let $G$ be the fully faithful right adjoint of $F$. By Lemma [?], $G$ lifts to a functor 
	$$
	G': \Alg_{F_{!}(\CO)}(\CD)
	\to 
	\Alg_{\CO}(\CC)
	$$
	that is right adjoint to $F'$. 
	Since $G$ is fully faithful, the induced functor $G':\Alg_{F_{!}(\CO)}(\CD)	
	\to 
	\Alg_{\CO}(\CD)$ is also fully faithful by Lemma [?].  
	
	\todo{This Lemma will be in appendix?}
\end{proof}




\section{Commutative Coalgebras in $\Sp^{\geq r}_{tame}$}
\label{Commutative coalgebras in tame spectra}

In this section, we study the category of commutative coalgebras in the category of $r$-tame spectra. 
Recall that there are four types of coalgebras, but it turns out that they are all equivalent in tame spectra. This allows us to identify the cofree coalgebra functor explicitly.
We start by introducing some results from \cite{LurieEllipticI}.

\begin{proposition}
\label{Cor 3.1.5. Ellip}
\cite[Corollary 3.1.5]{LurieEllipticI}
	Let $\mathcal{C}$ be a symmetric monoidal $\infty$-category. Suppose that the $\infty$-category $\mathcal{C}$ is presentable and that the tensor product functor $\otimes: \mathcal{C} \times \mathcal{C} \rightarrow \mathcal{C}$ preseerves colimits separately in each variable. Then the forgetful functor $\coCAlg(\mathcal{C}) \rightarrow \mathcal{C}$ admits a right adjoint $\cofree: \mathcal{C} \rightarrow \operatorname{cCAlg}(\mathcal{C})$.
\end{proposition}


\begin{corollary}
The forgetful functor $\coCAlg(\Sp_{tame}^{\geq s})\to \Sp^{\geq s}_{tame}$ admits a right adjoint functor which we will denote by $\cofree_{tame}$.
\end{corollary}
\begin{proof}
	Since $\Sp^{\geq s}_{tame}$ is presentable; it's a localization of the presentable $\infty$-category $\Sp^{\geq s}$. The tensor product $\hat{\otimes}$ in $\Sp_{tame}^{\geq s}$ preserves colimits separately in each variable. The existence of the cofree commutative coalgebra functor is ensured by Proposition \ref{Cor 3.1.5. Ellip}.
\end{proof}

There are four different coalgebra categories in the $\infty$-category of tame spectra, we now explain their relations and claim they are equivalent to each other in $\Sp^{\geq r}_{tame}$.

Since $(X^{\hat{\otimes}n})^{h\Sigma_n}\simeq (X^{\hat{\otimes}n})_{h\Sigma_n}$ for any $X\in \Sp^{\geq r}_{tame}$, we have equivalences
$$
\coCAlg^{nil}(\Sp^{\geq r}_{tame})
\simeq
\coCAlg(\Sp^{\geq r}_{tame})
$$
and 
$$
\coCAlg^{nil,dp}(\Sp^{\geq r}_{tame})
\simeq
\coCAlg^{nil}(\Sp^{\geq r}_{tame}).
$$


\begin{proposition}
	The forgetful functor
	\[
\operatorname{res}: 
\coCAlg^{nil,dp}(\Sp^{\geq r}_{tame}) \to
\coCAlg(\Sp^{\geq r}_{tame})
\]
 is an equivalence of $\infty$-categories.
\end{proposition}
\begin{proof}
	We prove that the forgetful functor
	\[
	res': \coCAlg^{nil,dp}(\Sp^{\geq r}_{tame})
	\to 
	\coCAlg^{nil}(\Sp^{\geq r}_{tame})
	\]
	is an equivalence of $\infty$-categories.
	Since $\coCAlg^{nil,dp}(\Sp^{\geq r}_{tame})$ is comonadic over $\Sp^{\geq r}_{tame}$, it suffices to check the comonad $U\circ \cofree^{nil}$ induced from the adjunction
	$$
	\adj{U}{\coCAlg^{nil}(\Sp^{\geq r}_{tame})}{\Sp^{\geq r}_{tame}}{\cofree^{nil}}
	$$
	agrees with the comonad $F_{\Com}$. Indeed, we have a natural map
	$$
	\gamma: \bigoplus_{n\geq 1} (X^{\hat{\otimes}n})_{h\Sigma_n}
	\to 
	\prod_{n\geq 1} (X^{\hat{\otimes}n})_{h\Sigma_n};
	$$
	The $k$-truncation of $\gamma$
	$$
	\tau_k \gamma: \bigoplus_{n\geq 1}^l (X^{\hat{\otimes}n})_{h\Sigma_n}
	\to 
	\prod_{n\geq 1}^l (X^{\hat{\otimes}n})_{h\Sigma_n}
	$$
	 is an equivalence for every $k$, hence $\gamma$ is an equivalence.
	\end{proof}
	
\begin{corollary}
	The underlying spectrum of the cofree commutative coalgebra generated by $X\in \Sp^{\geq s}_{tame}$ is given by
	 $$
	 \cofree(X) \simeq \bigoplus_{n\geq 1} (X^{\hat{\otimes}n})_{h\Sigma_n}= \Sym(X) .
	 $$
\end{corollary}

We now show that the Chavalley-Eilenbereg induces a functor on the categories of group objects, that is, the functor $\widetilde{\operatorname{CE}}$ is symmetric monoidal when we consider the Cartesian symmtric monoidal structures on both sides.
\begin{lemma}
\label{CE preserves products}
	The functor 
	$\widetilde{\operatorname{CE}}$
	preserves finite products.
\end{lemma}
\begin{proof}
	We want to show the morphism 
	\[
	\widetilde{\operatorname{CE}}(L \times L') 
	\to 
	\widetilde{\operatorname{CE}}(L)\otimes \widetilde{\operatorname{CE}}(L') 
	\]
	is an equivalence for $L,L' \in \Alg_{\Lie}(\Sp^{\geq r-2}_{\text{($r-1$)-tame}})$. 
	Note that $\widetilde{\operatorname{CE}}$ lifts to a functor on 
	$$
	\widetilde{\operatorname{CE}}^{\Fil}: \Alg_{\Lie}((\Sp^{\geq r-2}_{\text{($r-1$)-tame}})^{\Fil, \geq 0})
	\to 
	\coCAlg((\Sp^{\geq r-2}_{\text{($r-1$)-tame}})^{\Fil, \geq 0}),
	$$
	so it suffices to show the induced map
	$$
	\gr\circ \widetilde{\operatorname{CE}}^{\Fil}\big( (L \times L')^{\Fil} \big) 
	\simeq 
	\gr \widetilde{\operatorname{CE}}^{\Fil}(L) \otimes \gr\widetilde{\operatorname{CE}}^{\Fil}(L')
	$$
	is an equivalence since the functor $\gr$ is conservative. 
	Since $\widetilde{\operatorname{CE}}^{\Fil}$ preserves colimits, we have 
	\begin{align*}
		\gr\circ \widetilde{\operatorname{CE}}^{\Fil}\big( (L \times L')^{\Fil} \big) 
	& \simeq 
	\widetilde{\operatorname{CE}}^{\gr} \big( \gr (L \times L')^{\Fil} \big) \\
	& \simeq 
	\widetilde{\operatorname{CE}}^{\gr} \circ \trivial_{\Lie}^{\gr} (L\times L')\\
	& \simeq \Sym^{\gr} (L\times L').
	\end{align*}
	Similarly, we conclude 
	$$
	\gr \widetilde{\operatorname{CE}}^{\Fil}(L) \otimes \gr\widetilde{\operatorname{CE}}^{\Fil}(L') \simeq
	\Sym^{\gr} (L) \otimes  \Sym^{\gr} (L'), 
	$$
	and we are reduced to show the map 
	$$
	\Sym^{\gr} (L\times L') \to \Sym^{\gr} (L) \otimes  \Sym^{\gr} (L')
	$$
	which follows from the fact that $\Sym$ sends products to tensor products.
\end{proof}



\section{Proof of Theorem \ref{first main theorem}}
In the last section of this chapter, we prove the first main result of this paper. Since Lemma \ref{CE preserves products} ensures the functor $\widetilde{\operatorname{CE}}$ preserves group objects, hence we obtain a functor on the categories of groups
\[
	\Grp(\widetilde{\operatorname{CE}}):
	\Grp(\Alg_{\spLie}(\Sp^{\geq r-1}_{\text{($r-1$)-tame}}))
	\to 
	\Hopfalgebra(\Sp^{\geq r-1}_{\text{($r-1$)-tame}}).
\]
	
\begin{definition}
	The \emph{unviersal enveloping algebra functor}
	$$
	U: \Alg_{\spLie}(\Sp^{\geq r}_{\text{$r$-tame}}) 
	\to 
	\Hopfalgebra(\Sp^{\geq r-1}_{\text{($r-1$)-tame}})
	$$
	is defined as the composite $\Grp(\widetilde{\operatorname{CE}}) \circ \Omega_{\Lie}$.
\end{definition}

	
\begin{theorem}
\label{first main theorem}
	The universal enveloping algebra functor
	$$
	U: \Alg_{\spLie}(\Sp^{\geq r}_{\text{$r$-tame}})  
	\to
	\Hopfalgebra(\Sp^{\geq r-1}_{\text{($r-1$)-tame}})
	$$
	is an equivalence of $\infty$-categories.
\end{theorem}

Over the rational, if $L$ is a Lie algebra in $\Ch_{\BQ}$, then $\Omega_{\Lie}L$ is a trivial Lie algebra.
We claim the same phenomenon happens in the case of tame Lie algebras.

\begin{proposition}
\label{Triviality of loop of an O-algebra}
	Let $\Alg_{\spLie}(\Sp^{\geq r}_{\text{$r$-tame}}) $ be a tame Lie algebra.
	Then we have an equivalence
	$$
	\Omega_{\spLie} L\simeq \trivial_{\spLie}(\Omega L).
	$$ 
\end{proposition}

Assuming Proposition \ref{Triviality of loop of an O-algebra}, we can now prove Theorem $\ref{first main theorem}$.
\begin{proof}
[Proof of  \ref{first main theorem}:]
	We first show the universal enveloping algebra functor
	\[
	U:
	\Alg_{\spLie}(\Sp^{\geq r}_{\text{$r$-tame}}) 
	\xrightarrow{\Omega_{\spLie}}
	\Grp(\Alg_{\spLie}(\Sp^{\geq r-1}_{\text{($r-1$)-tame}}))
	\xrightarrow{\Grp(\widetilde{\operatorname{CE}})}
	\Hopfalgebra(\Sp^{\geq r-1}_{\text{($r-1$)-tame}})
	\]
	is fully faithful. 
% 	Since $\Omega_{\spLie}X$ is also an $(r-2)$-tame spectrum, we can factor $U$ has follows,
% 	\[
% 	\begin{tikzcd}
% 	\Alg_{\Lie}(\Sp^{\geq r-1}_{\text{($r-1$)-tame}})	&    \Grp(\Alg_{\Lie}(\Sp^{\geq r-2}_{\text{($r-2$)-tame}}))
%  &
%   \Hopfalgebra(\Sp^{\geq r-1}_{\text{($r-2$)-tame}})
%     \\
% 		& \Grp(\Alg_{\Lie}(\Sp^{\geq r-2}_{\text{($r-1$)-tame}}))
%   & \Hopfalgebra(\Sp^{\geq r-1}_{\text{($r-1$)-tame}})
%   	\arrow[from=1-1, to=1-2, "\Omega_{\Lie}'"]
%   	\arrow[from=1-2, to=1-3, "\widetilde{\operatorname{CE}}'"]
%   	\arrow[from=1-1, to=2-2, "\Omega_{\Lie}" left]
%   	\arrow[from=1-2, to=2-2]
%   	\arrow[from=1-3, to=2-3]
%   	\arrow[from=2-2, to=2-3, "\widetilde{\operatorname{CE}}"]
% 	\end{tikzcd}
% 	\]
% 	where the two vertical arrows are fully faithful.
% 	We claim the upper horizontal arrow $\widetilde{\operatorname{CE}}'\circ \Omega_{\Lie}'$ is fully faithful.
    Let $\widetilde{\Prim}$ denote the right adjoint of $\widetilde{\operatorname{CE}}$ and consider the unit map
	$$
	X \to B_{\spLie}\circ \widetilde{\Prim} \circ \widetilde{\operatorname{CE}}\circ \Omega_{\spLie}(X)
	\simeq
	B_{\spLie}\circ
	\widetilde{\Prim} \circ \widetilde{\operatorname{CE}}\circ \trivial_{\spLie}(\Omega X)
	$$
where the latter equivalence follows from Proposition \ref{Triviality of loop of an O-algebra}. Since $B_{\spLie}$ is an inverse to $\Omega_{\spLie}$ by Proposition \ref{B and Omega are mutally inverses}, it suffices to check the map
\[
\eta: \Omega_{\spLie}X \to \widetilde{\Prim} \circ \widetilde{\operatorname{CE}}\circ \trivial_{\spLie}(\Omega X)
\]
is an equivalence.
%, this is equivalent to show
%	the unit map	
%	
%	is an equivalence for any $(r-1)$-tame Lie lagebra $X$.
	By \cite{Francis-Gaitsgory}[Lemma 3.3.4.], we have a natural equivalence 
	$$\widetilde{\operatorname{CE}}\circ \trivial_{\spLie}\circ \Omega (X) 
	\simeq 
	\cofree(X);
	$$ 
	apply the functor $\widetilde{\Prim}$ on both sides, we have
	
	$$
	\widetilde{\Prim}\circ \widetilde{\operatorname{CE}} \circ 
	\trivial_{\spLie}\circ \Omega(X)  \simeq   \widetilde{\Prim}\circ \cofree^{nil}_{dp}(X)
	\simeq  \Omega_{\spLie}X
	$$
	which completes the proof of fully faithfulness of $U$.

    For essential surjectivity, note first that  $U$ preserves colimits, as it is a composition of $\Omega_{\spLie}$ and $\widetilde{\operatorname{CE}}$. 
    We identify $\Hopfalgebra(\Sp^{\geq r-1}_{\text{($r-1$)-tame}})$ as $\CS^{\geq r}_{tame}$ by Theorem \ref{2nd Main Theorem}. 
    It suffices to show that the Hopf algebra $X$ corresponding to the generator $S^r$ lies in the image of $U$. Indeed, consider the free tame Lie algebra $\Free_{\spLie}(x)$ generated by an element in degree $r$, then 
\end{proof}

We now present the proof of Proposition \ref{Triviality of loop of an O-algebra}.
We learn this proof from Heuts. 
Let $\BL$ denote the free Lie algebra monad on the $\infty$-category $\Sp$ of spectra and consider the 
\emph{suspension morphism}
\[
\BL \xrightarrow{\sigma} \Omega \BL \Sigma.
\]
between these two monads on $\Sp$.

Observe that the suspension $\Omega:\Sp \to \Sp$ lifts to a functor 
$$
\Omega':  \Alg_{\BL}(\Sp) \to \Alg_{\Omega \BL \Sigma}(\Sp).
$$
\begin{lemma}
	The functor $\Omega'$ is an equivalence of $\infty$-categoreis.
\end{lemma}
\begin{proof}
	Consider the commuting diagram
\[
\begin{tikzcd}
	\Alg_{\BL }(\Sp)  &   & \Alg_{ \Omega \BL \Sigma}(\Sp) \\
	& \Sp &
	\arrow[from=1-1, to = 1-3, "\Omega'"]
	\arrow[from=1-1 , to =2-2, "\Omega\circ \oblv_{\BL}" left]
	\arrow[from=1-3, to=2-2, "\oblv_{\Omega\BL\Sigma}"]
\end{tikzcd}
\]
and we want to check the conditions of \cite{HA}[Corollary 4.7.3.16.].
Conditions (1), (2), (3) are immediate and note that both diagonal arrows in the diagram above are conservative, hence we are left to check (5).
Consider the map
\[
\phi: \Free_{\Omega\BL\Sigma}(X) \to \Omega' 
\circ \Free_{\BL} (\Sigma X)
\]
left adjoint to
\[
X \to \Omega \circ \oblv_{\BL}\circ
\Free_{\BL}
\circ
\Sigma(X)
\simeq 
\oblv_{\Omega \BL \Sigma}
\circ
\Omega'\circ \Free_{\BL}\circ \Sigma (X).
\]
It suffices to check 
\[
\oblv_{\Omega\BL\Sigma}(\phi):
\Omega \BL(\Sigma X) \to 
\oblv_{\Omega\BL\Sigma} \circ \Omega'\circ \Free_{\BL} (\Sigma X)
\]
is an equivalence, where the latter is equivalent to 
$$
\Omega \circ \oblv_{\BL}\circ \Free_{\BL}(\Sigma X)\simeq \Omega \BL \Sigma(X),
$$
hence the proof is complete.
\end{proof}

We can then identify the restriction functor $\Alg_{ \Omega \BL \Sigma}(\Sp)\xrightarrow{\sigma^{*}} \Alg_{\BL}(\Sp)$ as
\begin{align*}
	\Alg_{\BL}(\Sp)\simeq \Alg_{\Omega\BL\Sigma}(\Sp) & \xrightarrow{\sigma^*} \Alg_\BL(\Sp)\\
	X & \mapsto \Omega X.
\end{align*}

To complete the proof of Proposition \ref{Triviality of loop of an O-algebra},
we claim the suspension morphism $\sigma$ can be factored as 
$$
\BL \to \triv 
\to 
\Omega \BL \Sigma
$$
after tame localization;
since the restriction along $\BL \to \triv$ is indeed the trivial Lie algebra functor, this would complete the proof of Proposition \ref{Triviality of loop of an O-algebra}.

Let $\Lie_n$ denote the ordered set of Lie words with $n$ generators, i.e. every $w \in \Lie_k$ is a basis element of the free Lie algebra on $n$ generators.
For $X_1, \dots, X_n \in \Sp$, we define $w(X_1, \cdots, X_2):= X_1\otimes \cdots \otimes X_n$; that is, we let the Lie brackets in $w$ act as the smash products in $\Sp$. 
The following is a variant of the Hilton-Milnor theorem.
\begin{theorem}
	[Hilton-Milnor,  \cite{Arone-Kankaanrinta98}, \cite{Brantner-Heuts}]
	For any collection of spheres $\BS^{k_1}, \cdots, \BS^{k_n}$, there is an equivalence
	$$
	\Omega \BL \Sigma(\BS^{k_1}\oplus \cdots \oplus \BS^{k_n})
	\simeq 
	\bigoplus_{w\in \Lie_n} \Omega \BL( \Sigma w(\BS^{k_1}, \cdots, \BS^{k_n})).
	$$
\end{theorem}

Now we can rewrite the suspension morphism $\sigma$ evaluating at the wedge of spheres $\BS^{k_1}\oplus \cdots \oplus \BS^{k_n}$ as 
$$
\sigma: 
\BL(\BS^{k_1}\oplus \cdots \oplus \BS^{k_n})
\to 
\bigoplus_{w\in \Lie_n} \Omega \BL (\Sigma w(\BS^{k_1}, \cdots, \BS^{k_n})).
$$
Note that any map 
$$
\BS^{l} \to 
\BS^{k}
$$
is null-homotopic if $l < k$,
hence $\sigma$ has to factor through the wedge sum of Lie words of weight one, i.e. we have factorization
\[
\begin{tikzcd}
	\bigoplus_{i=1}^{n} \BL(\BS^{k_i}) &   &  \bigoplus_{w\in \Lie_n} \Omega \BL (\Sigma w(\BS^{k_1}, \cdots, \BS^{k_n}))\\
	& \bigoplus_{i=1}^{n}\Omega \BL (\Sigma \BS^{k_i})  &
	\arrow[from=1-1, to = 1-3, "\sigma"]
	\arrow[from=1-1 , to =2-2]
	\arrow[from=2-2, to=1-3]
\end{tikzcd}
\]
 factors over $\bigoplus_{i=1}^{n}\Omega \BL (\Sigma \BS^{k_i})$.
 It now suffices to check the following.
 
 \begin{lemma}
 	The suspension morphism evaluated on a sphere has a factorization
\[
\begin{tikzcd}
	 \BL(\BS^{k}) &   &  \Omega \BL (\Sigma \BS^{k})\\
	& \BS^{k}  &
	\arrow[from=1-1, to = 1-3, "\sigma"]
	\arrow[from=1-1 , to =2-2]
	\arrow[from=2-2, to=1-3]
\end{tikzcd}
\]
after tame localization.
\end{lemma}
\begin{proof}
	The lemma follows from Theorem \ref{odd spheres}; indeed, for $n=p^l$ and $l>0$, the connectivity of
	\[
	\big(
	\BL(n)\otimes (\BS^{k})^{\otimes n}
	\big)_{h\Sigma_n}
	\]
	is at least $kp^l-n-1$, which is larger than $r+2p-3$, hence 
	$\big(
	\BL(n)\otimes (\BS^{k})^{\otimes n}
	\big)_{h\Sigma_n}$ is contractible.
	\end{proof}
 \begin{theorem}
 	\cite{Arone-Mahowald}[Theorem 3.13, Theorem 4.4]
 	\label{odd spheres}
 	Let $X$ denote the $p$-localization of the sphere $\BS^k$ at a prime $p$, then
 	\begin{enumerate}
 		\item if $k$ is odd, then 
 			$$
 			\big(\BL(n)\otimes X^{\otimes n}
 			\big)_{h\Sigma_n}
 			\simeq *.
 			$$
 			if $n\neq p^l$ for some $l$.
 		\item if $k$ is even, then 
 			$$
 			\big(\BL(n)\otimes X^{\otimes n}
 			\big)_{h\Sigma_n}
 			\simeq *.
 			$$
 			if $n$ is not equal to $p^l$ or $2p^j$ for $l, j>0$.
 	\end{enumerate} 	
 \end{theorem}
 
%\begin{corollary}
%The suspension morphism $\sigma$ factors as 
%	$$
%\BL \to \triv 
%\to 
%\Omega \BL \Sigma
%$$
%after tame localiztion. 
%\end{corollary}
%
%\begin{proof}
%	Since all these three functors are sifted-colimit preserving functors, so it will suffice to check this on objects of the form of wedge of spheres.
%\end{proof}





%\begin{proof}
%	Still working on this proof, and here are some ideas on how to attack this:
%	\begin{itemize}
%		\item Need to define the notion of tensoring an $\CO$-algebra $X$ with a commutative algebra $A$ (this should work in any symmtric monoidal $\infty$-category). That is, a functor
%		\[
%		A \otimes -: \Alg_\CO\to \Alg_\CO
%		\]
%		\item In the rational case, this is indeed true; because $C^*(S^1;\BQ)$ is quasi-isomorphic to the cdga $H^*(S^1;\BQ)$.
%		\item Show $X\otimes A$ is trivial if $A$ is a trivial commutative algebra.
%		\item Show $\Omega L$ is equivalent to $L_{tame}H\BZ^{S^1}\otimes L$ and hence trivial. 
%	\end{itemize}
%\end{proof}
%The spectrum $H\BQ^{S^1}$ models the rational cochain of $S^1$. Moreover, by Sullivan's formality result, we have an equivalence of $\E_\infty$-algebras 
%\[
%H\BQ^{S^1} \simeq H^{*}(S^1; \BQ)
%\]
%in $\Mod_{H\BQ}$. Hence, the rational cochain $H\BQ^{S^1}$ is a trivial $\E_{\infty}$-algebra.
%
%Now to show $L_{tame}H\BZ^{S^1}$ is a trivial $\E_{\infty}$-algebra, we need a formality result for $L_{tame}H\BZ^{S^1}$.
%\begin{question}
%	How is the formality argument proved in the rational case?
%\end{question}
%
%\begin{proposition}
%	There is an equivalence 
%	\[
%	L_{tame}C^{*}(S^1;\BZ) \simeq \tilde{H}^*(S^1; \BZ)
%	\]
%of $\E_\infty$-algebras if we invert primes quick enough to kill all the Steenrod operations.
%\end{proposition}
%
%
%
%
%Maybe one can proceeds as follows:
%One first shows $L_{tame}H\BZ^{S^1}$ is a trivial $\E_{\infty}$-algebra in $\Sp$. For this, we need to show some kind of formality argument:
%the cochain $L_{tame}H\BZ^{S^1}$ is tame equivalent to the cdga $H^*(S^1; \BZ)$ in $\Mod_{\BZ}$.










