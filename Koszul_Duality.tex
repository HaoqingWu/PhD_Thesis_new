\chapter{Koszul Duality}

In this chapter, we provide the necessary background on $\infty$-operads and $\infty$-cooperads. For convenience, we will assume $\CC$ is a pointed, presentably symmetric monoidal $\infty$-category throughout this chapter.

Our main references will be \cite{BrantnerPhD},  \cite{Heuts_Koszul}, \cite{Francis-Gaitsgory}, \cite{Hadrianphdthesis} and \cite{HA}.
 
In \S \ref{(co)Operads in Infinity-Categories}, we define $\infty$-operads and $\infty$-cooperads in a symmetric monoidal $\infty$-category $\CC$, as associative algebras and associative coalgebras in the $\infty$-category of symmetric sequences in $\CC$, respectively.

In \S \ref{Algebras over Operads}, we discuss algebras over an $\infty$-operad and some relevant constructions which will be useful in the proof of the main theorems. 

In \S \ref{Coalgebras over cooperads and Koszul Duality}, we explain the bar-cobar duality bewtween connected $\infty$-operads and cooperads. We then define divided power, conilpotent coalgebras over a $\infty$-cooperad and define the Koszul duality functor $\indec_{\CO}$.

In \S \ref{Commutative coalgebras in tame spectra}, we define the $\infty$-category of commutative coalgebras in an $\infty$-category. In the case of tame spectra, we prove that it is equivalent to the $\infty$-category of divided power, conilpotent coalgebras.

In \S \ref{The spectral Lie operad}, we define the spectral Lie operad and introduce shifting of an $\infty$-operad and relevant properties.

In \S \ref{Lie Algebras in Tame Spectra}, we define spectral Lie algebras and tame spectral Lie algebras. We then use Koszul duality between the spectral Lie operad and commutative operad to define the Chevalley-Eilenberg functor as a functor from the $\infty$-category of tame spectral Lie algebras to the $\infty$-category of divided power, conilpotent commuative coalgebras in tame spectra.




\section{(co)Operads in Infinity-Categories}
\label{(co)Operads in Infinity-Categories}
In this section, we collect and extend some results from \cite{BrantnerPhD} and \cite{Hadrianphdthesis}.

Let $\Fin^{\simeq}$ denote the $\infty$-category of finite sets with bijections between them, i.e., the core of $\Fin$. 
Note that $\Fin^{\simeq}$ carries a natural symmetric monoidal structure with the tensor product given by disjoint unions. 
Moreover, it's the free symmetric monoidal $\infty$-category generated by the one-object category $\{*\}$.
The $\infty$-category $\CP(\Fin^{\simeq})$ of presheaves admits a symmetric monoidal structure given by Day convolution \cite{HA}[Example 2.2.6.9.], hence we can also consider it as a presentable symmetric monoidal $\infty$-category. This suggests an alternative description of $\CC$:
\[
\Fun_{\CAlg(\Pr^L)}(\CP(\Fin^{\simeq}), \CC) 
\simeq 
\Fun_{\CAlg(\Cat_{\infty})} (\Fin^{\simeq}, \CC)
\simeq
\Fun(\{*\}, \CC)
\simeq 
\CC.
\]


\begin{definition}
	A \emph{symmetric sequence} in $\CC$ is a functor 
	$$
	A:\Fin^{\simeq} \to \CC.
	$$ We denote the $\infty$-category of symmetric sequences in $\CC$ by $\SSeq(\CC)$.
\end{definition}
\begin{remark}
	Informally, one can think of a symmetric sequence $A$ in $\CC$ as a sequence of objects $\{A(n)\}_{n\geq 0}$ in $\CC$ where $A(n):= A((n))$ carries an action of $\Sigma_n$ for each $n$.
	We will refer to the number $n$ as the \emph{arity} of the symmetric sequence $A$. We will sometimes describe an $\infty$-operad informally by giving a sequence $\{A(0), A(1), \dots \}$.
	
\end{remark}
Recall the $\infty$-category of presentable $\infty$-categories admits a symmetric monoidal strucutre. The tensor product $\otimes$ in $\Pr^L$ admits an explicit formula as in \cite[Proposition 4.8.1.17.]{HA}. The following lemma points out that the $\infty$-category $\SSeq(\CC)$ of symmetric sequences in $\CC$ is tensored over the $\infty$-category $\SSeq(\Space)$ of symmetric sequences in the $\infty$-category $\CS$ of spaces.
\begin{lemma}
	 Let $\CC$ be a presentably symmetric moniodal $\infty$-category.
	Then
	$$
	\SSeq(\CC) \simeq \SSeq(\Space)\otimes \CC.
	$$
\end{lemma}
\begin{proof}
First note that 
$
\SSeq(\Space)=\Fun(\Fin^{\simeq}, \Space)\simeq \CP(\Fin^{\simeq}),
$
since $\Fin^{\simeq}$ is isomorphic to its opposite category.
Using \cite{HA}[Proposition 4.8.1.17.], we deduce that there are equivalences
\begin{align*}
	\SSeq(\Space)\otimes \CC 
& \simeq 
\Fun^R(\CP(\Fin^{\simeq})^{op},\CC)\\
&\simeq
(\Fun^{L}(\CP(\Fin^{\simeq}), \CC^{op}))^{op}\\
& \simeq
\Fun(\Fin^{\simeq},\CC^{op}) ^{op}\\
& \simeq \SSeq(\CC).
\end{align*}



\end{proof}

% The category of symmetric sequences $\SSeq(\CD)$ in $\CD$ has a monoidal product $\circ$ called the \emph{composition product}, defined so that the functor $\theta$ is a monoidal functor with respect to the composition monoidal structure in the codomain.   An \emph{ordinary operad} in $\CD$ is then defined as an associative monoid object with respect to this composition product.
%
%For our usage, we need a version of $\infty$-operads enriched in $\CC$. For this we need three ingredients, symmetric sequences, composition products and associative monoid objects. 
%The $\infty$-category $\SSeq(\CC)$ of \emph{symmetric sequences} in $\CC$ is defined as $\Fun(\Fin^{\simeq},\CC)$. 
%Given a symmetric sequence $\CO$ in $\SSeq(\CC)$, it has an associated monad on $\CC$ which is informally on an object $X\in \CC$ given by 
%$$
%\coprod_{n\geq 0} \CO(n)\otimes_{\Sigma_n} X^{\otimes n}.
%$$



We now explain a monoidal structure on $\SSeq(\CC)$ that will be used to define $\infty$-operads. Note that we can view $\CC$ as the full subcategory of $\SSeq(\CC)$ spanned by symmetric sequences that evaluates to the zero object on all non-empty finite sets in $\Fin^{\simeq}$. 
The $\infty$-category $\SSeq(\CC)$ of symmetric sequences is equipped with a monoidal structure that corresponds to the composition of functors in the $\infty$-category $\Fun_{\CAlg(\Pr^L)_{\CC/}}(\SSeq(\CC),\SSeq(\CC))$. More precisely, there are equivalences of $\infty$-categories

\begin{align*}
    	\Fun_{\CAlg(\Pr^L)_{\CC/}}(\SSeq(\CC),\SSeq(\CC))
	& \simeq 
\Fun_{\CAlg(\Pr^L)}(\CP(\Fin^{\simeq}),\SSeq(\CC))\\
& \simeq
\Fun_{\CAlg(\Cat_{\infty})}(\Fin^{\simeq},\SSeq(\CC))\\
& \simeq \Fun(\{*\}, \SSeq(\CC))\\
& \simeq \SSeq(\CC),
\end{align*}
allowing the transfer of the monoidal structure on $\Fun_{\CAlg(\Pr^L)_{\CC/}}(\SSeq(\CC),\SSeq(\CC))$ to $\SSeq(\CC)$.
Given two symmetric sequences $X$ and $Y$ in $\CC$, we will write $X\circ Y$ for this monoidal product on $\SSeq(\CC)$ and refer to it as the \emph{composition product}.


\begin{remark}
\label{formula for composition product}
For some category $\CC$, e.g. $\CC=\operatorname{Set}, \Sp$ etc. with a point-set model, there is a concrete formula for the composition product (cf. \cite[Section 4.1.2.]{BrantnerPhD})
$$
A\circ B (J) \cong \coprod_{n\geq 0} \big(
\coprod_{J = J_1 \coprod \cdots \coprod J_n} A(J) \otimes 
B(J_1)\otimes \cdots \otimes B(J_n)
\big)_{h\Sigma_n}
$$
where the second coproduct runs over all partitions of $J$. 
\end{remark}

\begin{remark}
\label{Schur functor}
Let $X \in \CC$ be an object in $\CC$, which we can identify as a symmetric sequence.
By the equivalence
$$
\Fun_{\CAlg(\Pr^L)_{\CC/}}(\SSeq(\CC),\SSeq(\CC))
\simeq
\SSeq(\CC),
$$
we can identify $X$ as a functor $\SSeq(\CC) \to \SSeq(\CC)$ that factors through $\CC$.
To any symmetric sequence $A$ in $\CC$,  we can then identify $A\circ X$ as an object $A(X)$ in $\CC$.
By Remark \ref{formula for composition product}, we obtain a functor which has an explicit description as follows
\begin{align*}
		F_{(-)}: \SSeq(\CC) & \rightarrow \Fun(\CC,\CC)\\
	A        & \mapsto \big( X\mapsto  A(X):=(\coprod_{n\geq 0} A(n)\otimes X^{\otimes n})_{h\Sigma_n}\big).
\end{align*}
where $\Sigma_n$ acts on the $n$-th tensor power of $X$ via permutation.
By our definition of the composition products, this assignment is lax monoidal with respect to the composition product in $\SSeq(\CC)$ and composition of functors in $\Fun(\CC,\CC)$.
We call the the endofunctor $F_A$ as the \emph{Schur functor associated to the symmetric sequence $A$}.
\end{remark}

We can now define $\infty$-operads (resp. $\infty$-cooperads) as associative algebras (resp. coalgebras) with respect to the composition products in $\SSeq{\CC}$.
\begin{definition}
	\label{inf operads}
	The \emph{$\infty$-category $\Opd(\CC)$ of $\infty$-operads in $\CC$} (resp. $\infty$-category $\Coop(\CC)$ of $\infty$-cooperads in $\CC$) is defined as the $\infty$-category $\Alg(\SSeq(\CC))$ (resp.$\coAlg(\SSeq(\CC))$) of associative algebras (resp. coalgebras) in $\SSeq(\CC)$ with respect to the composition product.
\end{definition}

\begin{remark}
\label{Schur functor and operads}
As a consequence Remark \ref{Schur functor}, the Schur functor $F_\CO$ associated to an operad $\CO$ is a monad on $\CC$.
If additionally we assume $\CC$ is preadditive (Definition \ref{preadditive})
then the Schur functor $F_\CQ$ associated to a cooperad $\CQ$ is a comonad on $\CC$. See \cite[Remark 2.21]{ChingBar} for more details.
\end{remark}

\begin{example}
    	We denote by $\triv_{\CC}$ the symmetric sequence that takes value the tensor unit $1_\CC$ of $\CC$ at $\{*\}$ and the zero object otherwise. One can check that Schur functor $F_{\triv_{\CC}}$ associated to $\triv_{\CC}$ is the identity endofunctor on $\SSeq(\CC)$. Hence $\triv_{\CC}$ serves as a monoidal unit for the composition product in $\SSeq(\CC)$.
	We will refer to $\triv_{\CC}$ as the \emph{trivial $\infty$-operad} in $\CC$.
\end{example}


\begin{example}
\label{Symmetric algebra}
	Consider the \emph{symmetric algebra functor} on $\CC$ defined as 
	\begin{align*}
		\Sym:\CC & \to  \CC\\
		X & \mapsto \amalg_{n\geq 1} (X^{\otimes n})_{h\Sigma_n},
	\end{align*}
	which is the Schur functor of the symmetric sequence $\Com:=(1_\CC,1_\CC,\dots)$, where $1_\CC$ is the tensor unit of $\CC$.
	Since $\Com$ is clearly an associative algebra in $\SSeq(\CC)$ with respect to the 
	composition product, $\Com$ is called \emph{commutative operad} in $\CC$ and hence
	$\Sym$ is a monad on $\CC$.
	If we additionally assume $\CC$ is preadditive, then $\Sym$ associated to the symmetric sequence $\Com$ defines a comonad on $\CC$ and we will abuse notation by writing $\Com$ for the commutative cooperad in $\CC$.
% Not necessairly anymore
% 	Similarly, the \emph{non-unital symmetric algebra functor} $\Sym^{\geq 1}:\CC \to \CC$ given by $X\mapsto \amalg_{n\geq 1} (X^{\otimes n})_{h\Sigma_n}$ defines a non-unital $\infty$-operad in $\CC$ which we will denote by $\Com_{nu}$.
\end{example}

\begin{remark}
Although Lurie's approach to $\infty$-operads \cite{HA}[Chapter 2] is seemingly different from the one we give here,
Haugseng \cite{Haugsengsymseq} and Heine \cite{Hadrianphdthesis} showed that Lurie's model is equivalent to the definition of $\infty$-operads in $\Space$. Hence all the results in $\cite{HA}$ transfer smoothly to the setting of $\infty$-operads in spaces.
\end{remark}

In this paper, we will work exclusively with \emph{non-unital $\infty$-operads}. Roughly speaking, a non-unital $\infty$-operad $\CO$ is an operad without nullary operations.
\begin{definition}
	\label{non-unital operad}
	An $\infty$-operad $\CO$ in $\CC$ is \emph{non-unital} if $\CO(0)$ is equivalent to the zero object of $\CC$. 
\end{definition}

\begin{definition}
	A (non-unital) $\infty$-operad $\CO$ is \emph{connected} \footnote{In \cite{ChingBar}, operads with this property are called reduced. However, the term reduced operads sometimes has another meaning in the literature.} if there is an equivalence $\CO(1)\simeq 1_{\CC}$. 
\end{definition}


% \begin{remark}
% Let $\Fin^{nu}$ be the category of non-empty finite sets.
% Consider the canonical embedding $(\Fin^{nu})^{\simeq}\hookrightarrow \Fin^{\simeq}$. One can then define a \emph{non-unital symmetric sequence in $\CC$}  as a functor 
% \[
% \CO:(\Fin^{nu})^{\simeq}\to \CC.
% \]
% Hence for any symmetric sequence $\CO$ in $\CC$, 
% the underlying non-unital symmetric sequence $\CO_{nu}$ is given by the restriction along $(\Fin^{nu})^{\simeq}\hookrightarrow \Fin^{\simeq}$.
% \end{remark}

% \begin{remark}
% 	Under Lurie's model of $\infty$-operads in $\Space$, a non-unital $\infty$-operad $\CO^{\otimes}$ is defined as an $\infty$-operad whose structure map $\CO^{\otimes} \to \Fin_*$ factors through $\Surj$, where $\Surj$ is the subcategory consisting of surjective maps. 
% 	Note that this is equivalent to our definition in which case $\CO(0)\simeq \emptyset$, the initial object in $\Space$.
% \end{remark}

% No need to make this definition
% \begin{definition}
% \label{non-unitalization of operads}
% 		Let $\CO^{\otimes}$ be an $\infty$-operad in $\Space$, we define (as in \cite{HA}[Definition 5.4.4.1.]) the \emph{non-unitalization of $\CO^{\otimes}$} by $\CO^{\otimes}_{nu}:= \Surj\times_{\Fin_*}\CO^{\otimes}$.
% \end{definition}

% Dually, we define $\infty$-cooperads as coassociative coalgebras in $\SSeq(\CC)$ with respect to the composition product.
% \begin{definition}
% 	\label{infty cooperads}
% 	We define the $\infty$-category $\Coop(\CC):=\Alg(\SSeq(\CC)^{op})^{op}$ of \emph{$\infty$-cooperads in $\CC$} as the opposite $\infty$-category of the $\infty$-category of associtive algebras in of $\SSeq(\CC^{op})$.
% \end{definition}

\begin{definition}
	An $\infty$-operad $\CO$ in $\CC$ is \emph{augmented} if it admits a map of $\infty$-operads $\epsilon:\CO\rightarrow \triv_{\CC}$ such that $\epsilon\circ \eta\simeq id_{\mathds{1}_\CC}$, where $\eta$ is the unit of $\CO$. 
% 	% I don't really need it
% 	We will denote the $\infty$-categories of augmented operads and coaugmented cooperads in $\CC$ by $\operatorname{Op}^{aug}(\CC)$ and $\operatorname{coOp}^{aug}(\CC)$, respectively.

\end{definition}


\begin{remark}
A connected $\infty$-operad is canonically augmented.
\end{remark}  

\begin{remark}
From now on, whenever we say $\infty$-operads (resp. $\infty$-cooperads) we mean non-unital connected $\infty$-operads (resp. $\infty$-cooperads).
\end{remark}


\subsection{Truncations of $\infty$-operads}
% \todo{I'm still expecting Gijs' paper for more details to cite in this subsection}
The goal of this subsection is to set up the prerequisites to state Proposition \ref{inductive construction of coalgebras}  \cite[Theorem 4.12]{Heuts_Koszul} which we will use in the proof of Theorem \ref{2nd Main Theorem}. 
We discuss various notions of truncations of $\infty$-(co)operads, for which we follow the upcoming paper by Heuts \cite{Heuts_Koszul} closely. The upshot of these truncations is to produce natural filtrations on algebras (resp. coalgebras) over operads (resp. cooperads). 


We fix a pointed presentably symmetric monoidal $\infty$-category $\CC$ with tensor product compatible with colimits.
Let $\Fin^{nu}_{\leq n}$ denote the full subcategory of $\Fin^{nu}$ spanned by (non-empty) finite sets with cardinality less or equal to $n$.
We will refer to
$$
\SSeq_{\leq n}(\CC):=\Fun(\Fin^{nu}_{\leq n}, \CC)
$$ 
as the $\infty$-category $\SSeq_{\leq n}(\CC)$ of \emph{$n$-truncated symmetric sequences in $\CC$}.
Restriction along the inclusion $\Fin^{nu}_{\leq n}\to \Fin^{nu}$ induces a
functor 
$$
(-)_{\leq n}: \SSeq(\CC) \to 
\SSeq_{\leq n}(\CC).
$$
which preserves both limits and colimits. Moreover, $(-)_{\leq n}$ induces a
monoidal product on $\SSeq_{\leq n}(\CC)$ by ignoring higher arity operations.
We can now define $n$-truncated operads and cooperads in $\CC$.
\begin{definition}
    We define the $\infty$-category of \emph{$n$-truncated operads in $\CC$} as 
    $$
    \operatorname{Op}_{\leq n}(\CC) :=\Alg(\SSeq_{\leq n}(\CC)).
    $$
    Dually, we define 
    $$
     \operatorname{coOp}_{\leq n}(\CC) := \coAlg(\SSeq_{\leq n}(\CC)) 
    $$
    to be the $\infty$-category of \emph{$n$-truncated cooperads in $\CC$}.
\end{definition}

We now explain the relations between operads and their truncations, following \cite{Heuts_Koszul}.
Consider the functor 
$$
\zeta_n: \SSeq_{\leq n}(\CC) \to \SSeq(\CC)
$$
given by inserting zero objects in the arities above $n$. One can check that it's both left and right adjoint to the restriction functor
$
(-)_{\leq n}.
$

\begin{lemma}
	\cite{Hadrianphdthesis}[Lemma 2.16]
	If $\CC$ is a pointed symmetric monoidal $\infty$-category with colimits.
	Then the restriction functor
$$
(-)_{\leq n}: \SSeq(\CC) \to 
\SSeq_{\leq n}(\CC)
$$
	is monoidal. Hence it lifts to functors between algebras and coalgebras,
	such that the following squares commute
\[
\begin{tikzcd}
%%%% The nodes %%%%%%%%%
	\Opd(\CC)  & 
	\operatorname{Op}_{\leq n}(\CC)\\
	\SSeq(\CC) & \SSeq_{\leq n}(\CC)
%%%% Now the arrows %%%%
	\arrow[from=1-1, to= 1-2, "\rho_n"]
	\arrow[from=1-1, to=2-1, "\oblv" left]
	\arrow[from=1-2, to=2-2, "\oblv'" ]
	\arrow[from=2-1, to= 2-2, "(-)_{\leq n}"]
\end{tikzcd}
\]
and
\[
\begin{tikzcd}
%%%% The nodes %%%%%%%%%
	 \Coop(\CC)  & 
	\operatorname{coOp}_{\leq n}(\CC)\\
	\SSeq(\CC) & \SSeq_{\leq n}(\CC)
%%%% Now the arrows %%%%
	\arrow[from=1-1, to= 1-2, "\rho^n"]
	\arrow[from=1-1, to=2-1, "\oblv" left]
	\arrow[from=1-2, to=2-2, "\oblv'" ]
	\arrow[from=2-1, to= 2-2, "(-)_{\leq n}"].
\end{tikzcd}
\]
where the vertical arrows are forgetful functors.
\end{lemma}

The lemma above has an immediate corollary.
\begin{corollary}
\label{functors between operads adn their truncations}
    The functor 
    $$
    \zeta_n: \SSeq_{\leq n}(\CC) \to \SSeq(\CC)
    $$
    admits both a lax monoidal and an oplax monoidal strucutre. 
\end{corollary}
    The lax monoidal structure on $\zeta_n$ induces a functor on algebras
    $$
    \tau_{n}:\operatorname{Op}_{\leq n}(\CC)
    \to 
    \Opd(\CC)
    $$
    which is right adjoint to $\rho_{n}$. On the other hand, the functor $\rho_n:\Opd(\CC) \to \operatorname{Op}_{\leq n}(\CC)$ preserves limits and filtered colimits (since they are computed in the underlying $\infty$-category of symmetric sequences) hence by the adjoint functor theorem \cite[Corollary 5.5.2.9.]{HTT}, it admits a left adjoint 
    $$
    \varphi_n: \operatorname{Op}_{\leq n}(\CC)
    \to 
    \Opd(\CC).
    $$
    To sum up, we have a diagram consisting of functors described above
\begin{equation}
\label{Op-Op_n adjunction}
	\begin{tikzcd}
	%%%% The nodes %%%%%%%%%
 \Opd(\CC) & \operatorname{Op}_{\leq n}(\CC),
%%%% Now the arrows %%%%
	\arrow[from=1-2, to= 1-1, bend right, shift right = 3, "\varphi_n" above]
	\arrow[from=1-2, to=1-1, bend left, shift left = 3, "\tau_n" below]
	\arrow[from=1-1, to=1-2, "\rho_n"]
\end{tikzcd}
\end{equation}
where the left adjoints are above the right adjoints.
    

    
    Dually, the oplax structure on $\zeta_n$ induces a functor on coalgebras
    $$
    \tau^{n}:\operatorname{coOp}_{\leq n}(\CC)
    \to 
    \Coop(\CC),
    $$
    which is left adjoint to the restriction $\rho^n$. The restriction $\rho^n$ preserves colimits, hence admits a right adjoint
    $$
    \varphi^{n}: \operatorname{coOp}_{\leq n}(\CC)
    \to 
    \Coop(\CC),
    $$
    and we have the following diagram which summarizes the situation for cooperads
    
    \begin{equation}
\label{coOp-coOp_n adjunction}
	\begin{tikzcd}
	%%%% The nodes %%%%%%%%%
\Coop(\CC) & \operatorname{coOp}_{\leq n}(\CC)
%%%% Now the arrows %%%%
	\arrow[from=1-2, to= 1-1, bend right, shift right = 3, "\tau^n" above]
	\arrow[from=1-2, to=1-1, bend left, shift left = 3, "\varphi^n" below]
	\arrow[from=1-1, to=1-2, "\rho^n"].
\end{tikzcd}
\end{equation}


\begin{notation}
	We will abuse notation by writing 
	\begin{itemize}
	    \item 
	    for any operad $\CO$ in $\CC$,
	$\CO \to \tau_{n} \CO$
	for the unit of the bottom adjunction in \eqref{Op-Op_n adjunction} ;
	\item 
	for any operad $\CO$ in $\CC$,
	$
	\varphi_n \CO \to \CO
	$ for the counit of the top adjunction in \eqref{Op-Op_n adjunction} ;
	\item 	for any cooperad $\CQ$ in $\CC$, $\tau^n \CQ \to \CQ$ for the counit of the top adjunction in \eqref{coOp-coOp_n adjunction} ;
	\item 	
	for any cooperad $\CQ$ in $\CC$, 
	$
	\CQ \to \varphi^n \CQ
	$
	for the unit of the bottom adjunction in \eqref{coOp-coOp_n adjunction}.
	\end{itemize}
\end{notation}

\begin{remark}
\begin{itemize}
	\item For any $\infty$-operad $\CO$ in $\CC$, there is a sequence of operad maps
	\[
	\varphi_n\CO \to \CO \to \tau_n \CO.
	\]
	The operad $\tau_n\CO$ is the $n$-truncation of $\CO$, i.e., operations of arity higher than $n$ are set to zero.
	The map $\CO \to \tau_n \CO$ is terminal among those operad maps from $\CO$ that are equivalences in arities up to $n$.
	The operad $\varphi_n\CO$ is equivalent to $\CO$ in arities up to $n$, while higher arity operations are "freely generated" by operations of arity less or equal to $n$.
	Hence, the map $\varphi_n\CO \to \CO$ is initial among those operad maps to $\CO$ that are equivalences in arities up to $n$.

	\item Similarly, for any $\infty$-cooperad $\CQ$ in $\CC$, there is a sequence of  cooperad maps
	\[
	\tau^n \CQ \to \CQ \to \varphi^n \CQ.
	\]
	The cooperad $\tau^n\CQ$ is the $n$-truncation of $\CQ$, i.e., cooperations of arity higher than $n$ are set to zero. The map $\CQ \to \varphi^n\CQ$ is terminal among those cooperad maps from $\CO$ that are equivalences in arities up to $n$.
    The cooperad $\varphi^n \CQ$ is "cofreely generated" by cooperations of arities up to $n$.
    The map $\tau^n\CQ \to \CQ$ is initial among those cooperad maps from $\CQ$ that are equivalences in arities up to $n$.
\end{itemize}
\end{remark}

\begin{remark}
	Using the remark above, we see that there is a direct system for an $\infty$-operad $\CO$, 
\begin{equation}
\varphi_1\CO \to \varphi_2\CO \to \cdots \to \CO.
\end{equation}
Similarly, there is an inverse system for the $n$-truncations of $\infty$-operads.
\begin{equation}
\cdots \to 
\tau_2 \CO
\to 
\tau_1 \CO
\to 
\CO
\end{equation}
\end{remark}
%\label{telescope for O_n}
%	\begin{tikzcd}
%	%%%% The nodes %%%%%%%%%
%	\vdots & \\
%	\varphi_3 \CO & \\
%	\varphi_2 \CO & \\
%	\varphi_1 \CO & \CO
%	%%%% Now the arrows %%%%
%	\arrow[from=2-1, to= 1-1]
%	\arrow[from=2-1, to= 4-2]
%	\arrow[from=3-1, to=2-1]
%	\arrow[from=3-1, to=4-2]
%	\arrow[from=4-1, to=3-1]
%	\arrow[from=4-1, to=4-2]
%	\end{tikzcd}



\begin{proposition}
\label{Operad as a colimits}
	If $\CO$ is an $\infty$-operad in $\CC$, 
	then there is an equivalence
	\[
	\CO \simeq
	\colim_n \varphi_n \CO
	\]
	in $\Opd(\CC)$.
\end{proposition}
\begin{proof}
    It suffices to check the equivalence aritywise.
    For any arity $k$, the direct system of objects 
    $$
    \varphi_1\CO(k) \to \varphi_2\CO(k) \to \cdots    
    $$
    stabilizes for $n \geq k$.
\end{proof}





\section{Algebras over Operads}
\label{Algebras over Operads}
In this section, we will review the definition of algebras over an $\infty$-operad.
Recall that a symmetric sequence $\CO\in\SSeq(\CC)$ acts on $\CC$ via its Schur functor $F_{\CO}$.
% \[
% \SSeq(\CC)
% \simeq
% \Fun_{\CAlg(\Pr^L)_{\CC/}}(\SSeq(\CC)\SSeq(\CC)).
% \]
If $\CO$ is an $\infty$-operad, then its Schur functor $F_\CO$ is a monad, hence we can consider the $\infty$-category $\LMod_{F_\CO}(\CC)$ of left $F_\CO$-modules in $\CC$.
%\begin{align*}
%SSeq(\CC) \times \CC & \to \CC\\
%(\CO, X)   & \mapsto  \coprod_{n\geq 0} (\CO(n)\otimes X^{\otimes n})_{h\Sigma_n}.
%\end{align*}
\begin{definition}
	\label{algebras over an operad}
	Let $\CO$ be an $\infty$-operad in $\CC$. The $\infty$-category of $\CO$-algebras is $\Alg_{\CO}(\CC):= \LMod_{F_\CO}(\CC)$.
\end{definition}

For a symmetric sequence $A$ in $\CC$, we consider the following \emph{extended power functors}
\begin{equation}
\label{extended power}
    D^{A}_n(X): = (A(n)\otimes X^{\otimes n })_{h\Sigma}, 
\quad
D_{A}^n(X): = (A(n)\otimes X^{\otimes n })^{h\Sigma}.
\end{equation}
Informally, an $\CO$-algebra $X$ in $\CC$ is equipped with maps
$$
D^{\CO}_n(X) \to X
$$
for each $n\geq 1$ and homotopy coherent data that keeps track of the associativity.

\begin{definition}
Let $F:\CC\to \CC$ be an endofunctor.
	We define the $\infty$-category $\Alg_{F}(\CC)$ of \emph{$F$-algebras in $\CC$} to be the pullback of the following diagram in $\Cat_{\infty}$
\[
\begin{tikzcd}
%%%% The nodes %%%%%%%%%
	\Alg_{F}(\CC) & 
	\Fun(\Delta^1, \CC)\\
	\CC & 
	\CC \times \CC
%%%% Now the arrows %%%%
	\arrow[from=1-1, to= 1-2]
	\arrow[from=1-1, to=2-1]
	\arrow[from=1-2, to=2-2, "(ev_0\text{,} ev_1)"]
	\arrow[from=2-1, to= 2-2, "(F\text{,} id)"].
\end{tikzcd}
\]
That is, $\Alg_{F}(\CC)$ is the subcategory of objects $X$ equipped with a morphism $F(X)\to X$.
\end{definition}

Proposition \ref{Operad as a colimits} motivates us to ask the following question: Can we write an $\CO$-algebra, as the limit of $\varphi_k\CO$-algebras as in the case of Postnikov decomposition of a simply-connected space? 

Heuts answers this question in the following theorem.
\begin{theorem}
\cite[Theorem 4.1]{Heuts_Koszul}
\label{Thm 4.1 of Heuts Koszul Duality paper}
For each $n \geq 2$, the commutative square of $\infty$-categories
\[
\begin{tikzcd}
%%%% The nodes %%%%%%%%%
	\Alg_{\varphi_n\CO}(\CC) & 
	\Alg_{D^{\CO}_{n}}(\CC)\\
	\Alg_{\varphi_{n-1}\CO} (\CC)  & 
	\Alg_{D_{n}^{\varphi_{n-1}\CO}}(\CC)
%%%% Now the arrows %%%%
	\arrow[from=1-1, to= 1-2]
	\arrow[from=1-1, to=2-1]
	\arrow[from=1-2, to=2-2]
	\arrow[from=2-1, to= 2-2].
\end{tikzcd}
\]
is a pullback square. Furthermore, the natural map
$$
\operatorname{Alg}_{\mathcal{O}}(\mathrm{C}) \rightarrow \lim _{n} \operatorname{Alg}_{\varphi_{n} \mathcal{O}}(\mathcal{C})
$$
is an equivalence of $\infty$-categories.
\end{theorem}

\begin{remark}
Heuts stated Theorem \ref{Thm 4.1 of Heuts Koszul Duality paper} for a stable $\infty$-category, but a careful examination of the proof shows that the theorem holds for a presentbly symmetric monoidal $\infty$-category $\CC$, which is our assumption throughout this chapter.
\end{remark}

\begin{remark}
Theorem \ref{Thm 4.1 of Heuts Koszul Duality paper} has the following informal interpretation: suppose $X$ is a $\varphi_{n-1}\CO$-algebra, then to specify a $\varphi_n\CO$-algebra structure on $X$, it suffices to equip $X$ with a multiplication map $\mu_n: D^{\CO}_n(X)\to X$ that is compatible with the $\varphi_{n-1}\CO$-algebra structure maps. We refer the readers to \cite{Heuts_Koszul} for further details.
\end{remark}





\section{Koszul Duality and Divided Power Conilpotent  Coalgebras }
\label{Coalgebras over cooperads and Koszul Duality}
In this section, we discuss Koszul duality for operads in the sense of \cite{Ginzburg-Kapranov} and define divided power, conilpotent coalgebras over an $\infty$-cooperad. 
We will assume $\CC$ is a presentably, stable, symmetric monoidal $\infty$-category in this section.

We first discuss bar-cobar duality between $\infty$-operads and $\infty$-cooperads. The general form of bar-cobar duality is exhibited in the form of associative algebras and coassociative coalgebras in a nice $\infty$-category in \cite{HA}.
\begin{proposition}
\cite[Remark 5.2.2.19.]{HA}
Let $\CD$ be a pointed monoidal $\infty$-category admitting geometric realizations of simplicial objects and totalizations of cosimplicial objects. Then there is an adjunction
\[
\adj{B}{\Alg^{\operatorname{aug}} (\CD)}{\coAlg^{\operatorname{aug}}(\CD)}{C}.
\]
\end{proposition}
If $\CD= \SSeq(\CC)$, then we obtain an adjunction between (augmented) operads and (coaugmented) cooperads.
	$$
	\adj{B}{\Opd^{aug}(\CC)}{\Coop^{aug}(\CC)}{C}.
	$$
Explicitly, the bar construction $B\CO$ of an $\infty$-operad $\CO$ is computed as the geometric realization of the simplicial bar construction with respect to the composition products \cite[Section 4.4.2]{HA}
	\[
		\Barconstruction(\triv, \CO, \triv)_{\bullet}:=\xymatrix{ \mathds{1}\ar@<0ex>[r]  &  
		\CO \ar@<1ex>[r] \ar@<-1ex>[r] \ar@<1ex>[l] \ar@<-1ex>[l]  &  \CO\circ \CO \ar@<2ex>[l] \ar@<0ex>[l] \ar@<-2ex>[l] \cdots},
\]
and we will write 
$$
B\CO := |\Barconstruction(\triv, \CO, \triv)_{\bullet}|.
$$
	Similarly, the cobar construction $C\CQ$ of a cooperad $\CQ$ is computed as the totalization of the cosimplicial object
	\[
		\Cobar(\triv, \CQ, \triv)^{\bullet}:=\xymatrix{ \mathds{1} \ar@<1ex>[r] \ar@<-1ex>[r] &  
		\CQ \ar@<0ex>[l] \ar@<2ex>[r] \ar@<0ex>[r] \ar@<-2ex>[r] & \CQ\circ \CQ \ar@<1ex>[l] \ar@<-1ex>[l] \cdots}.
\]
and we will write 
$$
C\CQ := \Tot \Cobar(\triv, \CQ, \triv)^{\bullet}.
$$



\begin{definition}
	An $\infty$-operad $\CO\in \Opd^{aug}(\CC)$ is \emph{Koszul} if the unit map
	$$
	\CO\rightarrow CB(\CO)
	$$
	is an equivalence.
\end{definition}

If we restrict to $\infty$-operads in a stable $\infty$-category $\CC$, then every connected $\infty$-operad is Koszul.

\begin{proposition}
\cite[Proposition 3.4]{Heuts_Koszul}
\label{Connected operads are Koszul}
Let $\CC$ be a presentably stable symmtric monoidal $\infty$-category. Then the bar-cobar adjunction
	$$
	\adj{B}{\Opd^{aug}(\CC)}{\Coop^{aug}(\CC)}{C}.
	$$
	restricts to an equivalence on the  $\infty$-categories of connected operads and cooperads.
\end{proposition}


We now introduce functors that will play important roles throughout the rest of the thesis. 
Let $\CO$ be a connected $\infty$-operad with its canonical augmentation $\epsilon:\CO\rightarrow \mathds{1}$.
Restriction along $\epsilon$ induces a functor 
$$
\trivial_{\CO}: \CC\simeq\Alg_{\mathds{1}}(\CC)\rightarrow \Alg_{\CO}(\CC),
$$
which we will call the \emph{trivial $\CO$-algebra functor}.
The trivial $\CO$-algebra functor admits a left adjoint $\cot_{\CO}:\CC\rightarrow \Alg_{\CO}(\CC)$ given by the relative tensor product $\triv\circ_{\CO}(-)$, whose existence can also be deduced from the adjoint functor theorem \cite[Corollary 5.5.2.9.]{HTT}. 
The functor $\cot_{\CO}$ is called the \emph{cotangent fiber functor} in \cite{Heuts_Koszul}. 
\begin{remark}
Informally, a trivial $\CO$-algebra $X$ is an $\CO$-algebra whose structure map 
$X\to F_{\CO}(X)$ factors through $X$ itself.
\end{remark}

\begin{remark}
The functor $\cot_{\CO}$ is often called \emph{Topological Andr\'e-Quillen homology} in the literature. The choice of the terminology from \cite{Heuts_Koszul}
comes from the following fact.
Let $\CC$ be the category of chain complexes over a commutative ring $k$ and $\CO$ the non-unital commutative operad. If $A$ is a commutative $k$-algebra, then $\cot_{\CO}(A)$ computes the fiber of the cotangent complex $L_{A/k}$ at the $k$-point of $\operatorname{Spec}(A)$ determined by the augmentation of $A$.
\end{remark}


Similarly, restriction along the unit map $\eta:\mathds{1}\rightarrow \CO$ induces a functor
$$
\oblv_{\CO}: \Alg_{\CO}(\CC)
\rightarrow 
\CC\simeq\Alg_{\mathds{1}}(\CC),
$$
which we will refer as the \emph{$\CO$-forgetful functor}.
The $\CO$-forgetful functor admits a left adjoint $\Free_{\CO}:\CC\rightarrow \Alg_{\CO}(\CC)$ given by $\CO\circ (-)$ which we will call the \emph{free $\CO$-algebra functor}.
    
The composite
$$
\mathds{1}\xrightarrow{\eta} \CO \xrightarrow{\epsilon} \mathds{1}
$$
is equivalent to the identity, hence we have equivalences
$$
\cot_{\CO} \circ \Free_{\CO} \simeq \id
$$
and 
$$
\oblv_{\CO}\circ \trivial_{\CO} \simeq \id.
$$
The situation can be summarized in the following diagram of adjunctions
\[
\begin{tikzcd}
	%%%% The nodes %%%%%%%%%
\CC & \Alg_{\CO}(\CC) & \CC
%%%% Now the arrows %%%%
	\arrow[from=1-1, to= 1-2, shift left = 1, "\Free_{\CO}" above]
	\arrow[from=1-2, to=1-1, shift left = 1, "\oblv{\CO}" below]
	\arrow[from=1-2, to= 1-3, shift left = 1, "\cot_{\CO}" above]
	\arrow[from=1-3, to=1-2, shift left = 1, "\trivial_{\CO}" below]
\end{tikzcd}
\]
in which the left adjoints are above their right adjoints and both horizontal composites are equivalent to the identity functor.

The following corollary of the Barr-Beck-Lurie theorem \cite[Theorem 4.7.3.5.]{HA} says that any $\CO$-algebra $X$ is canonically equivalent to the geomectric realization of a simplicial object of $\CO$-algebras.
\begin{proposition}
\label{free resolutino of O-algebras}
Let $X$ be an $\CO$-algebra, then it can be resolved as the geometric realization of the simplicial objects $(\Free_{\CO}\circ \oblv_{\CO})_{\bullet + 1}X$, i.e., 
\begin{equation}
\label{(3.5)}
    X \simeq
|(\Free_{\CO}\circ \oblv_{\CO})_{\bullet + 1}X|
\simeq |\Barconstruction(\CO, \CO, X)_{\bullet}|.
\end{equation}
\end{proposition}
\begin{proof}
    This follows from the fact that the adjunction $(\Free_{\CO}, \oblv_{\CO})$ is monadic and Corollary \ref{Cor of Barr-Beck-Lurie theorem}.
    
\end{proof}
Applying $\cot_{\CO}$ to both sides of \eqref{(3.5)} yields a formula for computing the cotangent fiber of $X$. 
\begin{corollary}
\cite[Proposition 4.4]{Heuts_Koszul}
The cotangent fiber of an $\CO$-algebra $X$ can be computed as 
$$
\cot_{\CO}X \simeq 
|(\oblv_{\CO}\circ \Free_{\CO})_{\bullet}\oblv_{\CO}X|,
$$
where the geometric realization is computed in the underlying $\infty$-category $\CC$.
\end{corollary}

As a consequence, we can identify the comonad $\cot_{\CO}\circ \trivial_{\CO}$ in terms of the bar construction of $\CO$.
\begin{proposition}
\label{cot triv is F_BO}
\cite[Proposition 4.5]{Heuts_Koszul}
The comonad $\cot_{\CO}\circ \trivial_{\CO}$ is naturally equivalent to the comonad $F_{B\CO}$ associated with the cooperad $B\CO$.
\end{proposition}

We are now ready to define the $\infty$-category of divided power, conilpotent coalgebras over an $\infty$-cooperad.
\begin{definition}
\label{coalgebras over a cooperad}
	Let $\CQ$ be an $\infty$-cooperad in $\CC$. We define the $\infty$-category of divided power, conilpotent coalgebras over a cooperad $\CQ$ as the $\infty$-category of left comodules over $F_{\CQ}$,
	$$
	\coAlg^{\divpow, \nil}_{\CQ}(\CC):= \operatorname{LcoMod}_{F_{\CQ}}(\CC).
	$$
\end{definition}

Analogous to the case of algebras over an operad, we have two pairs of adjunctions 
\[
\begin{tikzcd}
%%%% The nodes %%%%%%%%%
\CC & \coAlg^{\divpow, \nil}_{\CQ}(\CC) & \CC
%%%% Now the arrows %%%%
	\arrow[from=1-1, to= 1-2, shift left = 1, "\cofree^{\nil}_{\CQ}" above]
	\arrow[from=1-2, to=1-1, shift left = 1, "\oblv^{\nil}_{\CQ}" below]
	\arrow[from=1-2, to= 1-3, shift left = 1, "\operatorname{Prim}^{\nil}_{\CQ}" above]
	\arrow[from=1-3, to=1-2, shift left = 1, "\trivial^{\nil}_{\CQ}" below]
\end{tikzcd}
\]
where the right adjoints are above the left adjoints,
and compositions of the horizontal functors are equivalent to the identity functor on $\CC$.

Moreover, by an argument dual to the proof of Proposition \ref{free resolutino of O-algebras}, we can resolve any divided power, conilpotent coalgebra by a totalization of cosimplicial cofree coalgebras.
\begin{proposition}
Let $X$ be a divided power, conilpotent $\CQ$-coalgebra, then it can be resolved as the totalization of the cosimplicial object $(\cofree_{\CQ}\circ \oblv_{\CQ})^{\bullet + 1}X$, i.e. 
\begin{equation}
\label{(3.6)}
    X \simeq
\Tot (\cofree_{\CQ}\circ \oblv_{\CQ})^{\bullet + 1}X
\simeq \Tot \Cobar(\CQ, \CQ, X)^{\bullet}.
\end{equation}
\end{proposition}



The functor $\cot_{\CO}$ canonically factors through the $\infty$-category $\LcoMod_{\cot_{\CO}\circ \trivial_{\CO}}(\CC)$ of left comodules over the comonad $(\cot_{\CO}\circ \trivial_{\CO})$, which can be identified with the $\infty$-category of left comodules over the monad $F_{B\CO}$ by Proposition \ref{cot triv is F_BO}, hence we can write the resulting factorization as
\[
\begin{tikzcd}
     \Alg_{\CO}(\CC) &   & 
     \CC\\
                     & \coAlg^{\divpow, \nil}_{B\CO}(\CC) &
    \arrow[from = 1-1, to = 1-3, "\cot_{\CO}"]
    \arrow[from = 1-1, to = 2-2, "\indec_{\CO}" below, outer sep=3pt]
    \arrow[from = 2-2, to = 1-3, "\oblv_{B\CO}^{\nil}" below, outer sep=3pt].
\end{tikzcd}
\]
% The functor $\indec_{\CO}$ can be computed explicitly as 
% \begin{align*}
%     \indec_{\CO}(X) &  \simeq \indec_\CO  
%     |(\Free_{\CO}\circ \oblv_{\CO})_{\bullet + 1}X|\\
%     & \simeq | \indec_{\CO}\circ \Free_{\CO} (F_{\CO})_{\bullet} \oblv_{\CO}X |\\
%     & \simeq |\trivial_{B\CO} (F_{\CO})_{\bullet} \oblv_{\CO}X  |.
% \end{align*}
The functor $\indec_{\CO}$ admits a right adjoint $\Prim_{B\CO}$ by the adjoint functor theorem \cite[Corollary 5.5.2.9.]{HTT}, which we will call the \emph{primitives functor};
for $Y \in \coAlg^{\divpow, \nil}_{B\CO}(\CC)$, the \emph{primitives} $\Prim_{B\CO}^{\nil}(Y)$ of $Y$ can be computed explicitly as \cite[Lemma 4.7]  {Heuts_Koszul}
$$
\Prim_{B\CO}^{\nil}(Y) \simeq \Tot \trivial_{\CO} (F_{B\CO})^{\bullet} \oblv^{nil}_{B\CO} (Y).
$$

% I don't need this anymoree
% We now introduce other types of coalgebra over a cooperad $\CQ$. 
% As the dual notion of algebras over operad, one would expect a colagebra $X$ over a cooperad $\CQ$ to be an object $X$ in $\CC$ equipped with maps
% $$
% \epsilon_n: X \to  (\CQ(n) \otimes X^{\otimes n})^{h\Sigma_n}
% $$
% that are homotopy-coherently coassociative. Recall we have defined a cooperad $\CQ$ to be a comonoid in $\SSeq(\CC)$ with respect to the composition product, hence it's a monoid in $\Fun(\CC^{op}, \CC^{op})$. 
% \begin{definition}
%     \label{coalgebras over cooperads}
%     We define the $\infty$-category of \emph{coalgebras over the cooperad $\CQ$} to be 
%     $$
%     \coAlg_{\CQ}(\CC) := \Alg_{\CQ}(\CC^{op})^{op}.
%     $$
% \end{definition}
For future application, we record the following lemma.
\begin{lemma}
\cite[(3.4) and (3.5)]{Francis-Gaitsgory}
\label{CE of Trivial is cofree}
There are formal equivalences
$$
\indec_{\CO}\circ\trivial_{\CO} \simeq \cofree^{\nil}_{B\CO}  \text{ and } 
\Prim_{B\CO}^{\nil} \circ \cofree^{\nil}_{B\CO} \simeq \trivial_{\CO}.
$$
\end{lemma}
\begin{proof}
    The first equivalence is a direct consequence of Proposition \ref{cot triv is F_BO}.
    The composite $\Prim_{B\CO}^{\nil} \circ \cofree^{\nil}_{B\CO}$ is right adjoint to
    $\oblv{_{\CO}\circ \indec_{\CO}}\simeq \cot_{\CO}$, hence $\Prim_{B\CO}^{\nil} \circ \cofree^{\nil}_{B\CO} \simeq \trivial_{\CO}$.
\end{proof}



We also have the following decomposition result for divided power conilpotent $\CQ$-coalgebras, which will be used in the proof of the essential surjectivity of the functor $C_{\operatorname{tame}}$. Let $\CQ$ be a connected $\infty$-cooperad.
\begin{proposition}
\cite[Theorem 4.12]{Heuts_Koszul}
\label{inductive construction of coalgebras}
    For $n\geq 2$, the following commutative square of $\infty$-categories 
\[
\begin{tikzcd}
%%%% The nodes %%%%%%%%%
	\coAlg_{\varphi^n \CQ}(\CC) & 
	\coAlg_{D^n_{\CQ}}(\CC)\\
	\coAlg_{\varphi^{n-1} \CQ}(\CC) & 
	\coAlg_{D^n_{\varphi^{n-1}\CQ}}(\CC)
%%%% Now the arrows %%%%
	\arrow[from=1-1, to= 1-2]
	\arrow[from=1-1, to=2-1]
	\arrow[from=1-2, to=2-2]
	\arrow[from=2-1, to= 2-2]
\end{tikzcd}
\]
is a pullback square. Moreover, the natural map
$$
\coAlg_{\CQ}(\CC) \to \lim_n \coAlg_{\varphi^n\CQ}(\CC)
$$
is an equivalence.
\end{proposition}

\section{Commutative Coalgebras in Tame Spectra}
\label{Commutative coalgebras in tame spectra}

In this section, we define and study the $\infty$-category of commutative coalgebras in the category of $r$-tame spectra. 
The main aim is to show the $\infty$-category of divided power, conilpotent commutative coalgebras is equivalent to the $\infty$-category of commutative coalgebras in $\Sp^{\geq r}_{\tame}$.
We start by collecting some results from \cite{LurieEllipticI}.

\begin{definition}
    Let $\mathcal{C}$ be a symmetric monoidal $\infty$-category. The $\infty$-category $\coCAlg(\CC)$ of commutative coalgebras in $\CC$ is defined to be
    $
    \coCAlg(\CC):= \CAlg(\CC^{op})^{op}.
    $
\end{definition}



\begin{proposition}
\label{Cor 3.1.5. Ellip}
\cite[Corollary 3.1.5]{LurieEllipticI}
	Let $\mathcal{C}$ be a symmetric monoidal $\infty$-category. Suppose that the $\infty$-category $\mathcal{C}$ is presentable and that the tensor product functor $\otimes: \mathcal{C} \times \mathcal{C} \rightarrow \mathcal{C}$ preserves colimits separately in each variable. Then the forgetful functor $\coCAlg(\mathcal{C}) \rightarrow \mathcal{C}$ admits a right adjoint $\cofree: \mathcal{C} \rightarrow \coCAlg(\mathcal{C})$.
\end{proposition}


\begin{corollary}
The forgetful functor $\oblv_{\coCAlg}:\coCAlg(\Sp_{\tame}^{\geq r})\to \Sp^{\geq r}_{\tame}$ admits a right adjoint functor, which we will denote by $\cofree_{\tame}$.
\end{corollary}
\begin{proof}
	 Note that $\Sp^{\geq r}_{\tame}$ is presentable as it's a localization of the presentable $\infty$-category $\Sp^{\geq r}$. 
	 Since the tensor product $\hat{\otimes}$ in $\Sp_{\tame}^{\geq r}$ preserves colimits separately in each variable by Remark \ref{symmetric monoidal structure on tame spectra}, the existence of the cofree commutative coalgebra functor is ensured by Proposition \ref{Cor 3.1.5. Ellip}.
	 
\end{proof}


For a general symmetric monoidal $\infty$-category $\CC$, we don't know any explicit identification of the cofree comonad $Q:=\oblv_{\coCAlg}\circ \cofree$. 
However, we do know an explicit formula for the comonad $Q_{\tame}:=\oblv_{\coCAlg}\circ \cofree_{\tame}$ in the $\infty$-category of tame spectra. To explain this, we need the following lemma which can be deduced from \cite[Proposition 3.1.3.3]{HA} and \cite[Example 3.1.1.17]{HA}.
\begin{lemma}
\label{Cofree commutative coalgebra}
Let $\CC$ be a presentable symmetric monoidal $\infty$-category.
Define $F$ to be the endofunctor on $\CC$
$$
F(X):= \prod_{n\geq 1} (X^{\otimes n})^{h\Sigma_n}.
$$
If the canonical map 
$$
\gamma: \prod_{n\geq 1} (X^{\otimes n})^{h\Sigma_n} \otimes Y \to 
\prod_{n\geq 1} (X^{\otimes n} \otimes Y)^{h\Sigma_n}
$$
is an equivalence for any $Y\in \CC$ with trivial $\Sigma_n$-action, for all $n$,
then the cofree comonad $Q$ is given by $F$.
\end{lemma}
\begin{proof}
    By \cite[Definition 3.1.3.1.]{HA} the cofree coalgebra of an obejct $X$ in $\CC$ is defined as a operadic limit diagram (cf. \cite[Definition 3.1.1.2.]{HA}) $p:\Fin_{*} \to \CC$.
    By a dual statement of \cite[Proposition 3.1.3.3]{HA} in the special case where $\CA$ is the trivial $\infty$-operad, $\CB$ and $\CO$ are the commutative $\infty$-operad $\Fin_{*}$, such a operadic limit exists.
    The lemma then follows from the dual version of \cite[Example 3.1.1.17]{HA}, which says that a diagram $p:\Fin_{*} \to \CC$ is an operadic limit diagram if it remains as a limit diagram in $\CC$ after tensoring with any object $Y\in\CC$. 
    In our case, this means that $\prod_{n\geq 1} (X^{\otimes n})^{h\Sigma_n}$ is an operadic limit diagram if 
    $\prod_{n\geq 1} (X^{\otimes n})^{h\Sigma_n} \otimes Y$ is a limit diagram for any $Y\in \CC$, which is the condition of $\gamma$ being an equivalence.
    
\end{proof}

We now prove that the $\infty$-category of tame spectra satisfies the conditions in Lemma
\ref{Cofree commutative coalgebra}, therefore we obtain a simple description of the comonad $Q_{\tame}$.
We write $\Sym_{\tame}$ for the symmetric algebra functor on $\Sp^{\geq r}_{\tame}$, which is given explicitly by
$$
\Sym_{\tame}(X) :=\bigoplus_{n\geq 1} L_{\tame}(X^{\otimes n})_{h\Sigma_n} .
$$
\begin{proposition}
\label{Cofreee commutative comonad}
	The cofree commutative comonad on the $\infty$-category $\Sp^{\geq r}_{\tame}$ of tame spectra is given by
	 $$
	 Q_{\tame}(X) \simeq \bigoplus_{n\geq 1} L_{\tame}(X^{\otimes n})_{h\Sigma_n}
	 \simeq
	 \Sym_{\tame}(X)
	 $$
	 for any $X \in \Sp^{\geq r}_{\tame}$.
\end{proposition}
\begin{proof}
We first claim that the canonical map 
    $$
	\gamma: \bigoplus_{n\geq 1} (L_{\tame}X^{\otimes n})_{h\Sigma_n}
	\to 
	\prod_{n\geq 1} (L_{\tame}X^{\otimes n})^{h\Sigma_n}
	$$
	is an equivalence for any $X \in \Sp^{\geq r}_{\tame}$.
% 	Since both the $\infty$-categories $\coCAlg^{\divpow,\nil}(\Sp^{\geq r}_{\tame})$ and $\coCAlg(\Sp^{\geq r}_{\tame})$ are comonadic over $\Sp^{\geq r}_{\tame}$, it suffices to check their correspoding comonads are equivalent.
% 	$$
% 	\adj{U}{\coCAlg^{\nil}(\Sp^{\geq r}_{\tame})}{\Sp^{\geq r}_{\tame}}{\cofree^{\nil}}
% 	$$
% 	agrees with the comonad $F_{\Com}$. Indeed, we have a natural map
    Note that the target is equivalent to 
    $$
    \prod_{n\geq 1} (L_{\tame} X^{\otimes n})_{h\Sigma_n},
    $$
    by Proposition \ref{Tate vanishing for tame spectra}.
    
	Fix an integer $k$ and consider the $k$-truncation of $\gamma$. There exists a maximum integer $l(k)$ such that $X^{\otimes (l+1)}$ is $(k+1)$-connective.
	Hence the $k$-truncation of $\gamma$ can be identified as a map
	$$
	\tau_{\leq k} \gamma: \bigoplus_{n\geq 1}^{l(k)}
	\tau_{\leq k} (L_{\tame}X^{\otimes n})_{h\Sigma_n}
	\to 
	\bigoplus_{n\geq 1}^{l(k)}
	\tau_{\leq k} (L_{\tame}X^{\otimes n})^{h\Sigma_n}
	$$
	 which is an equivalence for every $k$. Hence, $\gamma$ is an equivalence.
	 Since tensor product commutes with colimits in $\Sp^{\geq r}_{\tame}$, $F(X):=\prod_{n\geq 1} (L_{\tame} X^{\otimes n})^{h\Sigma_n}$ satisfies the condition of Lemma
    \ref{Cofree commutative coalgebra}, and thus the comonad $Q_{\tame}$ is given by 
    $$
    Q_{\tame}(X) \simeq  \Sym_{\tame}(X).
    $$
	\end{proof}


We now define the $\infty$-category of divided power, conilpotent coalgebras in $\Sp_{\tame}^{\geq r}$. Note that $\Sp_{\tame}^{\geq r}$ is a non-unital symmetric monoidal $\infty$-category, so there is no commutative cooperad in $\Sp_{\tame}^{\geq r}$. However, we can define a comonad which comes from the commutative cooperad in $\Sp$ via tame localization.

By example \ref{Symmetric algebra}, the Schur functor $F_{\Com}$ associated to the commutative cooperad in $\Sp$ is a comonad and is given by
$$
F_{\Com}(X) := \coprod_{n\geq 1} (X^{\otimes n})_{h\Sigma_n}.
$$
Note that $F_{\Com}$ restricts to a comonad on the $\infty$-category $\Sp^{\geq r}$ of $r$-connective spectra, since colimits in $\Sp^{\geq r}$ are computed in $\Sp$.
Applying the localization functor $L_{\tame}$, we obtain a comonad $L_{\tame}F_{\Com}$ on $\Sp_{\tame}^{\geq r}$, given by 
$$
L_{\tame} F_{\Com}(X) := \coprod_{n\geq 1} L_{\tame}(X^{\otimes n})_{h\Sigma_n}.
$$
as $L_{\tame}$ is colimit-preserving and symmetric monoidal.
\begin{definition}
\label{Def of tame dp, nil coalgebras}
    The $\infty$-category of \emph{divided power, conilpotent coalgebras in tame spectra} is defined to be the $\infty$-category of left comodules over the comonad $L_{\tame}F_{\Com}$
    \[
    \coCAlg^{\divpow, \nil}(\Sp_{\tame}^{\geq r}):= \operatorname{LcoMod}_{L_{\tame}F_{\Com}}(\Sp_{\tame}^{\geq r}).
    \]
\end{definition}
We now want to show these two $\infty$-categories of coalgebras in tame spectra are actually equivalent. 
Since the underlying endofunctors of $Q_{\tame}$ and $L_{\tame}F_{\Com}$ are equivalent, it suffices to show there is a natural transformation of comonads
$$
\Gamma: L_{\tame}F_{\Com}\to 
Q_{\tame},
$$
which restricts to the equivalence $L_{\tame}F_{\Com}\simeq Q_{\tame}$.
By the discussion in Appendix \ref{Construction of the Comparison Functor}, there is a map of comonads on $\Sp^{\geq r}_{\tame}$
$$
\Gamma: (L_{\tame})_{*} j^{*} F_{\Com} \to Q_{\tame}.
$$
which is the unique map that makes the diagram in Remark \ref{F_Com to Q_C} commute.
The map of comonads $\Gamma$ then induces a comparison functor 
$$
\zeta:\coCAlg^{\divpow,\nil}(\Sp^{\geq r}_{\tame}) \to
\coCAlg(\Sp^{\geq r}_{\tame}).
$$
The following proposition now follows immediately from Proposition \ref{Cofreee commutative comonad}.
\begin{proposition}
\label{all coalgebras are equivalent}
	The comparison functor $\zeta$ is an equivalence of $\infty$-categories.
\end{proposition}
\begin{proof}
Since the forgetful functor $\Comonad(\Sp^{\geq r}_{\tame}) \to \End(\Sp^{\geq r}_{\tame})$ is conservative,
it suffices to show the $\Gamma$ induces an equivalence between the underlying endofunctors of $L_{\tame}F_{\Com}$ and $Q_{\tame}$. For $X\in \Sp^{\geq r}_{\tame}$, the map $\Gamma$ is obtained from the projection $L_{\tame}F_{\Com}X \to X$ to the first factor, which agrees with the canonical projection $Q_{\tame}X\to X$ by \ref{Cofreee commutative comonad}. Hence $\Gamma$ is an equivalence.

\end{proof}




%Combining the two adjunctions, we obtain a diagram:
%\[
%\begin{tikzcd}
%\CC \arrow[r,shift left=1,"\free_{\CO}"]  & \Alg_{\CO}(\CC) \arrow[r,shift left=1,"\TAQ_{\CO}"] \arrow[l,shift left=1,"\oblv_{\CO}"]  & \CC \arrow[l,shift left=1,"\trivial_{\CO}"] 
%\end{tikzcd}
%\]
%where left adjoints are on top and both composites are equivalent to $id_{\CC}$.
%% Denote $F_{\CO}:\CC\rightarrow \CC$ its associated monad.
%\subsection{Conilpotent coalgebras over a cooperad}
%Let $\CP$ be a connected cooperad.
%Let $F_{\CP}$ be the comonad associated to $\CP$. We denote the category of coalgebras over $F_{\CP}$ by $\coAlg^{nil,dp}_{\CP}(\CC)$. An object $X\in \coAlg^{nil,dp}_{\CP}(\CC)$ is equipped with a structure map:
%$$
%X\rightarrow \bigoplus_{n\geq 1}\big( \CP(n)\otimes X^{\otimes n}\big)_{h\Sigma_n}.
%$$
%We will refer to $\coAlg^{nil,dp}_{\CP}(\CC)$ \textit{the category of conilpotent divided power $\CP$-coalgebras}. 
%Dual to the algebra case, we have a diagram of two pair of adjoint functors:
%\[
%\begin{tikzcd}
%\CC \arrow[r,shift left=1,"\cofree^{nil,dp}_{\CP}"]  & \coAlg^{nil,dp}_{\CP}(\CC) \arrow[r,shift left=1,"\Prim^{nil,dp}_{\CP}"] \arrow[l,shift left=1,"\oblv^{nil,dp}_{\CP}"]  & \CC \arrow[l,shift left=1,"\trivial^{nil,dp}_{\CP}"] 
%\end{tikzcd}
%\]
%where functors at the bottom are left adjoints and both composites are equivalent to $id_\CC$.
%The adjective conilpotent stands for the direct sum and the adjective divided power corresponds to coinvariants in $F_{\CP}$.
%
%
%
%\begin{theorem}[\cite{Francis-Gaitsgory}, \cite{HeutsSurvey}]
%The functor $\TAQ_{\CO}$ factors as 
%$$
%\Alg_{\CO}(\CC) \xrightarrow{B_\CO} \coAlg^{nil,dp}_{\Barconstruction(\CO)}(\CC)\xrightarrow{\oblv_{\Barconstruction (\CO)}} \CC.
%$$	
%
%\end{theorem}
\section{The spectral Lie operad}
\label{The spectral Lie operad}
In this section, we introduce an operad in $\Sp$ which will be central to this thesis. 
We first need to provide some background on partition posets. 
Consider the set $\mathbf{P}(n)$ of partitions (i.e., equivalence relations) on the finite set $\{1, \cdots, n \}$. Note that $\mathbf{P}(n)$ is a poset with minimal element the trivial partition and maximal element the discrete partition.
Write $\mathbf{P}^{+}(n)$ (resp. $\mathbf{P}^{-}(n)$ ) for the subposet of $\mathbf{P}(n)$ obtained by deleting the minimal (resp. maximal) partition.

	The Goodwillie derivative $\partial_* \id$ of the identity functor $\id: \pSpace \to \pSpace$ forms a symmetric sequence in $\Sp$, i.e. each $\partial_n \id$ is a $\Sigma_n$-spectrum in the naive sense. In \cite{JohnsonDerivative} and \cite{Arone-Mahowald}, it was shown that
	$$
	\partial_n \id \simeq \BD(\Sigma^{\infty} K_n),
	$$
	where 
	$
	K_{n}:=|\mathbf{P}(n)| /\left(\left|\mathbf{P}^{+}(n)\right| \cup\left|\mathbf{P}^{-}(n)\right|\right)
	$
	and $\BD$ denotes the Spanier-Whitehead dual. 
	Let $\Com$ denote the commutative cooperad in $\Sp$.
	Ching \cite{ChingBar} identified the derivatives of the identity functor as the cobar construction of the commutative cooperad.
	\begin{proposition}
	\cite[Remark 8.9]{ChingBar}
	The cobar construction of the commutative cooperad is equivalent to the derivatives of the identity functor
	$$
	\Cobar(\Com)\simeq \partial_* \id.
	$$
	\end{proposition}
	
	\begin{remark}
    The classical Koszul duality between the Lie operad and commutative cooperad in (graded) abelian groups is studied in \cite{Ginzburg-Kapranov} and \cite{Loday-Vallette};
    there is an equivalence of operads in (graded) abelian groups
    \[
    \Lie[-1] \simeq \Com^{\vee}
    \]
    where $\Com^{\vee}$ denotes the linear dual and $\Lie[-1](n)_p:=\Lie(n)_{p+1}$.
    \end{remark}
    
	\begin{definition}
	\label{Shifted Spectral Lie Operad}
	    	We define the \emph{shifted spectral Lie operad} to be
	    	$$
	    	\BL:= \partial_* \id.
	    	$$
	\end{definition}

	The name comes from the following computation. By the computation in \cite[Example 9.50]{ChingBar},
	$$
	H_{*}\BL(n) \cong 	
	\begin{cases}
	\Lie(n) \otimes \operatorname{sgn}& \text{ if $*=1-n$},\\
	0 & \text{otherwise}
	\end{cases}
	$$	
	where $\operatorname{sgn}$ denotes the sign representation.


The mismatch of degrees between the spectral Lie operad $\BL$ and the Lie operad $\Lie$ in abelian groups is rather inconvenient for applications. To remedy this, we introduce a shift operation for $\infty$-operads in a stable $\infty$-category $\CC$. 
\begin{definition}
\cite[Section 3]{Camarena_mod2_free_spectral_Lie_algebra} 
    If $T$ is a monad on $\Sp$, the \emph{desuspended monad} $\mathbf{\Sigma}^{-1}  T$ in $\CC$ is defined as 
	$
	\Omega T\Sigma.
	$
	If the monad $T$ is equivalent to the Schur functor $F_{\CO}$ of an $\infty$-operad $\CO$, then there is a \emph{desuspended operad} $\Sigma^{-1}\CO$ corresponding to the monad $\mathbf{\Sigma}^{-1} F_{\CO}$, which has underlying symmetric sequence 
	$$
	(\mathbf{\Sigma^{-1}} \CO ) (n)= \BS^{n}\otimes\Sigma^{-1} \CO(n)
	$$
	where $\Sigma_n$ acts on $\BS^{n}$ by permuting the $n$ factors of $\BS^{1}$ and $\Sigma_n$ acts on $\mathbf{\Sigma^{-1}} \CO(n)$ via the its action on $\CO(n)$.
\end{definition}

We now state some results concerning shifts of monads and $\infty$-operads without proof. We refer the reader to \cite[Section 2.2.4.]{Hadrianphdthesis} and \cite{Camarena_mod2_free_spectral_Lie_algebra} for more details.
The following lemma indicates that shifting an $\infty$-operad is indeed harmless to the category of algebras we want to study.
\begin{lemma}
\cite[Section 2.2.4.]{Hadrianphdthesis}
\label{Shift has no harm}
Let $\CC$ be a stable $\infty$-category.
There is a pullback diagram
\[
    \begin{tikzcd}
	\Alg_{\bf{\Sigma}^{-1} \CO }(\CC)    & \Alg_{\CO}(\CC) \\
	 \CC & \CC
	\arrow[from=1-1, to = 1-2, "\Sigma'", "\simeq" below]
	\arrow[from=1-1 , to =2-1,  left]
	\arrow[from=1-2 , to =2-2,  right]
	\arrow[from=2-1, to=2-2, "\Sigma", "\simeq" below].
\end{tikzcd}
\]
of $\infty$-categories.
\end{lemma}
As a consequence, we write  
$$
\Omega':
\Alg_{\CO}(\CC) \to \Alg_{\bf{\Sigma}^{-1} \CO }(\CC)
$$
for the inverse of $\Sigma'$
which fits into a commutative diagram
\[
    \begin{tikzcd}
	\Alg_{\CO}(\CC)    & \Alg_{\bf{\Sigma}^{-1} \CO }(\CC) \\
	 \CC & \CC
	\arrow[from=1-1, to = 1-2, "\Omega'", "\simeq" below]
	\arrow[from=1-1 , to =2-1,  left]
	\arrow[from=1-2 , to =2-2,  right]
	\arrow[from=2-1, to=2-2, "\Omega", "\simeq" below].
\end{tikzcd}
\]

Moreover, 
we have two commutative diagrams of right adjoints
\begin{equation}
    \label{trivial commutes with shifts}
    \begin{tikzcd}
	\Alg_{\CO}(\CC)    & \Alg_{\bf{\Sigma}^{-1} \CO }(\CC) \\
	 \CC & \CC
	\arrow[from=1-1, to = 1-2, "\Omega'", "\simeq" below]
	\arrow[from=2-1 , to =1-1, "\trivial_{\CO}" left]
	\arrow[from=2-2 , to =1-2,  "\trivial_{\bf{\Sigma}^{-1} \CO }" right]
	\arrow[from=2-1, to=2-2, "\Omega", "\simeq" below],
\end{tikzcd}
\end{equation}
and 
\begin{equation}
    \label{diagram of equivalences in shifting}
    \begin{tikzcd}
	\Alg_{\CO}(\CC)    & \Alg_{\bf{\Sigma}^{-1} \CO }(\CC) \\
	 \Alg_{\CO}(\CC) & \Alg_{\bf{\Sigma}^{-1} \CO }(\CC)
	\arrow[from=1-1, to = 1-2, "\Omega'", "\simeq" below]
	\arrow[from=1-1 , to =2-1, "\Omega_{\CO}" left]
	\arrow[from=1-2 , to =2-2, "\Omega_{\mathbf{\Sigma}^{-1} \CO}" right]
	\arrow[from=2-1, to=2-2, "\Omega'", "\simeq" below],
\end{tikzcd}
\end{equation}
where $\Omega_{\CO}$ and $\Omega_{\mathbf{\Sigma}^{-1} \CO}$ denote the loop functor in $\Alg_{\CO}(\CC)$ and $\Alg_{\bf{\Sigma}^{-1} \CO }(\CC) $, respectively.
The commutativity of (\ref{trivial commutes with shifts}) follows from the fact that $\Omega' \circ \trivial_{\CO}$ and $\trivial_{\bf{\Sigma}^{-1} \CO }\circ \Omega$ are right adjoint to $ \cot_{\CO}\circ \Sigma'$ and $\Sigma \circ \cot_{\bf{\Sigma}^{-1} \CO}$, respectively; and there are equivalences
\begin{align*}
    \cot_{\CO}\circ (\Sigma' X) & \simeq |(F_{\CO})_{\bullet} (\Sigma X)|\\
    & \simeq |\Sigma (\Omega F_{\CO}\Sigma)_{\bullet} (X) |\\
    & \simeq \Sigma \cot_{\bf{\Sigma}^{-1} \CO}.
\end{align*}

The commutativity of (\ref{trivial commutes with shifts}) says that a trivial $\CO$-algebra under $\Omega'$ is a equivalent to a trivial $\bf{\Sigma}^{-1} \CO$-algebra, and the commutativity of (\ref{diagram of equivalences in shifting}) says that there is an equivalence $\Omega_{\bf{\Sigma}^{-1} \CO}X \simeq \Omega'\circ \Omega_{\CO}(\Sigma' X)$ for $X\in \Alg_{\bf{\Sigma}^{-1} \CO}(\CC)$.
Therefore, we obtain the following lemma.
\begin{lemma}
\label{transition of triviality}
Let $Y\in \Alg_{\bf{\Sigma}^{-1} \CO}(\CC)$.
If $\Omega_{\CO}(\Sigma' Y)$ is a trivial $\CO$-algebra, then
$\Omega_{\bf{\Sigma}^{-1} \CO}Y$ is a trivial $\bf{\Sigma}^{-1} \CO$-algebra.
\end{lemma}
\begin{proof}
    By the assumption, $\Omega_{\CO}(\Sigma' Y)$ is a trivial $\CO$-algebra.
    Moreover, by the commutativity of (\ref{trivial commutes with shifts}) and (\ref{diagram of equivalences in shifting}), there are equivalences 
    $$
    \Omega_{\bf{\Sigma}^{-1} \CO}Y \simeq \Omega' \Omega_{\CO}(\Sigma' Y)
    \simeq 
    \Omega'\trivial_{\CO}(Y)
    \simeq \trivial_{\bf{\Sigma}^{-1} \CO}(\Omega Y).
    $$
\end{proof}

% By Proposition \ref{Prop 2.8 GMN} and Corollary \ref{Alg(C) equivalent to groups in Alg(C)}, the composite
% $$
% \Alg_{\CO}(\CC) \xrightarrow{\Omega_{\CO}} 
% \Grp(\Alg_{\CO}(\CC))
% \xrightarrow{\oblv_{\CO}}
% \Alg_{\CO}(\CC)
% $$
% is an equivalence of $\infty$-categories.
% We abuse notation by still writing $\Omega_{\CO}$ as the composition above.
% Therefore, we have a commutative diagram of equivalences


Note also that there is a map of monads 
\begin{equation}
\label{suspension morphism}
    \sigma: F_{\CO}\to \Omega F_{\CO}\Sigma, 
\end{equation}
which will be called the \emph{suspension morphism}. \todo{still need to ask Gijs how to construct this in general}
The suspension morphism $\sigma$ induces a functor
$$
\sigma^*: \Alg_{\bf{\Sigma}^{-1} \CO }(\CC) \to \Alg_{\CO }(\CC)
$$
by restriction. 
One then observes that there is a factorization 
\[
\begin{tikzcd}
     \Alg_{\CO}(\CC) &   & \Alg_{\CO}(\CC)\\
                     & \Alg_{\bf{\Sigma}^{-1} \CO }(\CC)  &
    \arrow[from = 1-1, to = 1-3, "\Omega_{\CO}"]
    \arrow[from = 1-1, to = 2-2, "\Omega'"]
    \arrow[from = 2-2, to = 1-3, "\sigma^*"]
\end{tikzcd}
\]
where $\oblv_{\bf{\Sigma}^{-1}\CO}\circ \Omega' X \simeq \Omega X$ and $\oblv_{\CO}( \sigma^* X)\simeq X$.



% \begin{lemma}
% \label{Shift has no harm}
%     The functor $\Sigma': \Alg_{\bf{\Sigma}^{-1} \CO }(\CC) \to \Alg_{\CO }(\CC)$ is an equivalence of $\infty$-categories.
% \end{lemma}
% \begin{proof}
% 	Consider the commuting diagram
% \[
% \begin{tikzcd}
% 	\Alg_{\bf{\Sigma}^{-1} \CO }(\CC) &   & \Alg_{\CO}(\CC) \\
% 	& \CC &
% 	\arrow[from=1-1, to = 1-3, "\Sigma'"]
% 	\arrow[from=1-1 , to =2-2, "\Sigma \circ  \oblv_{\bf{\Sigma}^{-1}\CO}" left]
% 	\arrow[from=1-3, to=2-2, "\oblv_{\CO}"].
% \end{tikzcd}
% \]
% We want to check the conditions of \cite{HA}[Corollary 4.7.3.16.].
% Conditions (1), (2), (3) are immediate since both diagonal arrows in the diagram above are conservative, hence we are left to check (5).
% Consider the map 
% \[
% \varphi: \Free_{\CO}(X) \to \Sigma' 
% \circ \Free_{\bf{\Sigma}^{-1}\CO} (\Sigma^{-1} X)
% \]
% that is left adjoint to
% \[
% X \to 
% \oblv_{\CO} 
% \circ
% \Sigma' 
% \circ\Free_{\bf{\Sigma}^{-1}\CO} (\Sigma^{-1} X),
% \]
% and we want to check $\varphi$ is an equivalence for any $X\in \CC$.
% Note that $\varphi$ is induced from the map of monads 
% % By Theorem \ref{Monads-Alg correspondence}, 
% % it suffices to check the induced map on monads
% $$
% F_{\CO} \to \Sigma\Omega F_{\CO} \Sigma \Omega,
% $$
% which is an equivalence because we assume $\CC$ is stable.

% % \[
% % \oblv_{\mathbf{\Sigma}\CO}(\varphi):
% % \mathbf{\Sigma}\CO(X) \to 
% % \oblv_{\mathbf{\Sigma}\CO} \circ \Omega'\circ \Free_{\CO} (\Sigma X)
% % \]
% % is an equivalence, where the latter is equivalent to 
% % $$
% % \Omega \circ \oblv_{\CO}\circ \Free_{\CO}(\Sigma X)\simeq \mathbf{\Sigma}\CO(X),
% % $$
% % and the proof is now complete.
% \end{proof}

\begin{definition}
	\label{spectral Lie operad}
	We define the \emph{spectral Lie operad} as $$
	\spLie := \mathbf{\Sigma}^{-1} \BL.
	$$
\end{definition}

\begin{remark}
\label{Remark on shifted Lie operads}
The underlying symmetric sequence of the spectral Lie operad $\spLie:= \mathbf{\Sigma}^{-1}\BL$
	has the form 	
	$$
	\spLie(n)= \BS^{n} \otimes \Sigma^{-1} \BL(n),
	$$
	whose homology is exactly $\Lie(n)$ concentrated in degree $0$.
	Moreover, the homology of the spectral Lie operad $H_{*}\spLie$ is the Lie operad $\Lie$ in graded abelian groups.
	On the other hand, $\spLie$ is the Goodwillie derivative of the functor $\Omega\Sigma$ on $\Space_*$, see \cite[Section 8]{GoodwillieIII}.
	Moreover, one can check the monad associated to $F_{\spLie}$ is indeed given by $\Omega\BL \Sigma$; this can be seen by inspecting their associated monads
	\begin{align*}
		F_{\spLie}(X) &\simeq \coprod_{n\geq 1}
		\big(
		\BS^{n} \otimes \Sigma^{-1}\BL(n) 
		\otimes X^{\otimes n}
		\big)_{h\Sigma_n}\\
		& \simeq 
		\Sigma^{-1}
		\coprod_{n\geq 1}
		\big(
		\BL(n) 
		\otimes (\Sigma X)^{\otimes n}
		\big)_{h\Sigma_n}\\
		& \simeq 
		\Omega F_{\BL}\Sigma(X).
	\end{align*}	
\end{remark}






\section{Lie Algebras in Tame Spectra}
\label{Lie Algebras in Tame Spectra}
In this section, we discuss spectral Lie algebras in $\Sp$ and $\Sp^{\geq r}_{\operatorname{tame}}$. We will prove that the $\infty$-category of tame spectral Lie algebras can be identified as the full subcategory of spectral Lie algebras whose underlying spectra are tame.
We then define the Chevalley-Eilenberg functor that goes from the $\infty$-category of spectral Lie algebras to the $\infty$-category of divided power, conilpotent commutative coalgebras.


% Using the theory of Koszul duality introduced in the previous section, we will state the main result of \cite{Ching-Harper} and apply to the tame spectra case. The main result is that the $\infty$-category $\Alg_{\Lie}(\Sp_{tame})$ of tame spectral Lie algebras is equivalent to the $\infty$-category $\coAlg^{nil, dp}_{\Com}(\Sp_{tame})$ of conilpotent divided power coalgebras.
\begin{definition}
\label{Spectral Lie operad}
We define the $\infty$-category $\Alg_{\spLie}(\Sp)$ of \emph{spectral Lie algebras} to be the $\infty$-category of algebras over the spectral Lie operad $\spLie$.
\end{definition}

\begin{remark}
Note that our definition of spectral Lie algebras might be different from the one given in the literature, since we define $\spLie$ to be the spectral Lie operad. 
The advantage of this definition is that the homology $H_{*}\spLie$ is precisely the Lie operad in abelian groups, hence the homology of a spectral Lie algebra is a graded Lie algebra. 
Furthermore, it is shown in \cite{Camarena_mod2_free_spectral_Lie_algebra} that using this definition of spectral Lie algebras makes computation easier.
\end{remark}


To define the $\infty$-category of tame spectral Lie algebras, we need to introduce an action of the spectral Lie operad $\spLie$ on $\Sp^{\geq r}_{\operatorname{tame}}$.
Note that the Schur functor $F_{\spLie}$ associated to $\spLie$ restricts to a monad on the $\infty$-category $\Sp^{\geq r}$ of $r$-connective spectra, since the colimits in $\Sp^{\geq r}$ are computed in $\Sp$.

The following lemma from Heine can be seen as a special case of \cite[Remark 4.6.2.9.]{HA}.
\begin{lemma}
\label{induced maps on SSeq}
\cite[Remark 5.81]{Hadrianphdthesis}
Let $\CC$ and $\CD$ be preadditive symmetric monoidal $\infty$-categories whose tensor products compatible with colimits. Let
$$
f: \CC \to \CD
$$
be a colimit-preserving, symmetric monoidal functor.
Then $f$ induces a functor
$$
f_{!}: \Monad(\CC) \to \Monad(\CD)
$$
on the $\infty$-category of monads and a functor
$$
f_{!}:
\Comonad(\CC) \to \Comonad(\CD)
$$
on the $\infty$-category of comonads.
\end{lemma}	

Since the Schur functor $F_{\CO}$ associated to an $\infty$-operad $\CO$ (resp. $\infty$-cooperad $\CQ$) in $\CC$ is a monad (resp. comonad) on $\CC$, under the assumption of Lemma \ref{induced maps on SSeq}, we obtain a monad $f_{!}(F_\CO)$ (resp. comonad $f_{!}(F_\CQ)$) on $\CD$. Moreover, $f$ induces a functor:
\[
f' : \Alg_{F_\CO}(\CC) \to \Alg_{f_{!}(F_\CO)}(\CD).
\]
resp. 
$$
f'' : \coAlg^{\divpow,\nil}_{F_\CQ}(\CC) \to \coAlg^{\divpow,\nil}_{f_{!}(F_\CQ)}(\CD).
$$

Take $\CO=\spLie$ and $f=L_{\tame}:\Sp^{\geq r} \to \Sp^{\geq r}_{\tame}$ the localization functor from $r$-connective spectra to $r$-tame spectra. The monad $f_{!}(F_\CO)$ on $\Sp^{\geq r}_{\tame}$ is simply given by the tame localization of the free spectral Lie algebra monad on $\Sp^{\geq r}$:
$$
X \mapsto  
L_{\tame}
\coprod_{n\geq 1}
		\big(
		\spLie(n) \otimes X^{\otimes n}
		\big)_{h\Sigma_n}
		\simeq 
		L_{\tame}F_{\spLie}(X).
$$


We are now ready to define the $\infty$-category of tame spectral Lie algebras.
\begin{definition}
\label{Def of tame spectral Lie algebras}
    We define the $\infty$-category $\Alg_{\spLie}(\Sp^{\geq r}_{\tame})$ of \emph{tame spectral Lie algebras} to be 
    $$
    \Alg_{\spLie}(\Sp^{\geq r}_{\tame}) := \LMod_{	(L_{\tame})_{!}F_{\spLie}}(\Sp^{\geq r}_{\tame}).
    $$
    Similarly, we define 
    $$
    \Alg_{\BL}(\Sp^{\geq r}_{\tame}) := \LMod_{	(L_{\tame})_{!}F_{\BL}}(\Sp^{\geq r}_{\tame}).
    $$
\end{definition}

On the level of underlying spectra,
the induced functor 
$$
f': \Alg_{\spLie}(\Sp)^{\geq r} 
\to 
\Alg_{\spLie}(\Sp^{\geq r}_{\tame}).
$$
is simply given by sending a $r$-connective spectrum $X$ to its tame localization $L_{\tame}X$.

\begin{remark}
Although the monad $L_{\tame}F_{\Lie}$ does not come from an operad in $\Sp^{\geq r}_{\tame}$, it does have all the associated functors we discussed in Section \ref{Coalgebras over cooperads and Koszul Duality}.
Indeed, the spectral Lie monad on $\Sp^{\geq r}$ has unit $\eta$ and augmentation $\epsilon$ natural transformations so that the composition 
\[
\id_{\Sp^{\geq r}}
\xrightarrow{\eta}
F_{\Lie}\xrightarrow{\epsilon}
\id_{\Sp^{\geq r}}
\]
is the identity natural transformation.
Applying tame localization, there is again a composite of monads on $\Sp^{\geq r}_{\tame}$
\[
\id_{\Sp^{\geq r}_{\tame}}
\xrightarrow{\eta}
L_{\tame}F_{\Lie}\xrightarrow{\epsilon}
\id_{\Sp^{\geq r}_{\tame}}
\]
which is equivalent to the identity natural transformation on $\Sp^{\geq r}_{\tame}$.
Therefore, we have the adjoint pairs between tame spectral Lie algebras and tame spectra
\[
\begin{tikzcd}
	%%%% The nodes %%%%%%%%%
\Sp^{\geq r}_{\tame} & \Alg_{\spLie}(\Sp^{\geq r}_{\tame}) & \Sp^{\geq r}_{\tame}
%%%% Now the arrows %%%%
	\arrow[from=1-1, to= 1-2, shift left = 1, "\Free_{\spLie}" above]
	\arrow[from=1-2, to=1-1, shift left = 1, "\oblv{\spLie}" below]
	\arrow[from=1-2, to= 1-3, shift left = 1, "\cot_{\spLie}" above]
	\arrow[from=1-3, to=1-2, shift left = 1, "\trivial_{\spLie}" below].
\end{tikzcd}
\]
\end{remark}

\begin{remark}
Dually, recall we have defined the $\infty$-category of divided power, conilpotent tame commutative coalgebras in Definition \ref{Def of tame dp, nil coalgebras}. 
As in the case of spectral Lie algebras, we have a pair of adjunctions
\[
\begin{tikzcd}
%%%% The nodes %%%%%%%%%
\Sp^{\geq r}_{\tame} & \coAlg^{\divpow, \nil}_{\Com}(\Sp^{\geq r}_{\tame}) & \Sp^{\geq r}_{\tame}.
%%%% Now the arrows %%%%
	\arrow[from=1-1, to= 1-2, shift left = 1, "\cofree^{\nil}_{\Com}" above]
	\arrow[from=1-2, to=1-1, shift left = 1, "\oblv^{\nil}_{\Com}" below]
	\arrow[from=1-2, to= 1-3, shift left = 1, "\operatorname{Prim}^{\nil}_{\Com}" above]
	\arrow[from=1-3, to=1-2, shift left = 1, "\trivial^{\nil}_{\Com}" below]
\end{tikzcd}
\]
\end{remark}





We claim the $\infty$-category of tame spectral Lie algebras is a full subcategory of the $\infty$-category of $r$-connective spectral Lie algebras whose underlying spectra are tame. It turns out that this is a consequence of the following more general result.
\begin{proposition}
\label{Lifting localization to algebra category}
Let $\CC$ and $\CD$ be symmetric monoidal $\infty$-categories whose tensor products are compatible with colimits.
Suppose $f:\CC \to \CD$ is a symmetric monoidal localization functor.
If $\CO$ is an $\infty$-operad in $\CC$, then the induced functor
$$
f' : \Alg_{F_\CO}(\CC) \to \Alg_{f_{!}(F_\CO)}(\CD).
$$
is a localization.
\end{proposition}
\begin{proof}
	The existence of the functor $f'$ follows from Lemma \ref{induced maps on SSeq}. It remains to show $f'$ is a localization functor. Let $g$ be the fully faithful right adjoint of $f$. We first claim that $g$ lifts to a functor 
	$$
	g': \Alg_{f_{!}(F_\CO)}(\CD)
	\to 
	\Alg_{F_\CO}(\CC),
	$$
	which is right adjoint to $f'$. 
	Indeed, since $f$ is symmetric monoidal, it follows that $g$ is lax symmetric monoidal; hence $g$ induces a functor $g': \Alg_{f_{!}(F_\CO)}(\CD)
	\to 
	\Alg_{F_\CO}(\CC)$.
	To see that $g'$ is right adjoint to $f'$, let $\eta: id_{\CC} \to gf$ be the unit natural transformation.
	Note that $\eta$ is a $\End(\CC)$-linear natural transformation in the sense of \cite[Definition 4.6.2.7.]{HA}, hence it induces a natural transformation $\eta': \id_{\Alg_{F_{\CO}}(\CC)} \to g'\circ f'$.
	We claim the map
	$$
	\rho:
	\map_{\Alg_{f_{!}(F_\CO)}(\CD)}(f'M, N)
	\to 
	\map_{\Alg_{F_\CO}(\CC)}(M, g'N),
	$$
	induced from $\eta'$,
	is an equivalence for any $M\in \Alg_{F_\CO}(\CC)$ and $N\in \Alg_{f_{!}(F_\CO)}(\CD)$. First observe that $f'$ sends free $F_\CO$-algebras to free $f_{!}(F_\CO)$-algebras, i.e.
	$$
	f'(\Free_{F_{\CO}} X)\simeq \Free_{f_{!}(F_\CO)} f(X)
	$$
	for $X\in \CC$,
	as $f$ is symmetric monoidal and colimit-preserving. 
	Hence, 
    \begin{align*}
        \map_{\Alg_{f_{!}(F_\CO)}(\CD)}(f'\Free_{F_{\CO}} X, N) & \simeq \map_{\Alg_{f_{!}(F_\CO)}(\CD)}(\Free_{f_{!}(F_\CO)} f(X), N)\\
        & \simeq \map_{\CD}(f(X), \oblv_{f_{!}F_{\CO}}N)\\
        & \simeq \map_{\CC}(X, g \circ \oblv_{f_{!}F_{\CO}}N)\\
        & \simeq \map_{\CC}(X, \oblv_{F_{\CO}}\circ g' N)\\
        & \simeq \map_{\CC}(\Free_{F_{\CO}}  X, g' N)
    \end{align*}	
    and we conclude that
	$\rho$ is an equivalence whenever $M$ is a free $F_{\CO}$-algebra. For a general $F_{\CO}$-algebra, we use Proposition \ref{free resolutino of O-algebras} to write $M$ as a geometric realization of a $\oblv_{F_{\CO}}$-split simplicial free $F_{\CO}$-algebras. Then the claim follows from the Barr-Beck-Lurie theorem \cite[Theorem 4.7.3.5.]{HA} and the fact that $f'$ preserves geometric realizations of $\oblv_{F_{\CO}}$-split simplicial objects.
	
    For the fully-faithfulness of $g'$, it suffices to show the counit map 
    $f'g'N \to N$ is an equivalence for any $N\in \Alg_{f_{!}(F_\CO)}(\CD)$. This follows from the fact that the forgetful functor $\oblv_{f_{!}(F_\CO)}$ is conservative and $g$ is fully faitfhul.
\end{proof}

\begin{corollary}
\label{Identification of tame Lie algebras}
	The $\infty$-category $\Alg_{\spLie}(\Sp^{\geq r}_{\tame})$ can be identified with the full subcategory of $\Alg_{\spLie}(\Sp)^{\geq r}$ spanned by spectral Lie algebras whose underlying spectra are tame.
\end{corollary}
\begin{proof}
    Proposition \ref{Lifting localization to algebra category} provides a localization functor
	$$
	f: \Alg_{\spLie}(\Sp)^{\geq r} \to \Alg_{\spLie}(\Sp^{\geq r}_{\tame}).
	$$
	The corollary then follows from the commuting diagram
\[
\begin{tikzcd}
	\Alg_{\spLie}(\Sp)^{\geq r} & \Alg_{\spLie}(\Sp^{\geq r}_{\tame})\\
	\Sp^{\geq r}  & 
	\Sp^{\geq r}_{\tame}
	\arrow[from=1-2, to= 1-1]
	\arrow[from=1-1, to=2-1, "\oblv" left]
	\arrow[from=1-2, to=2-2, "\oblv'" ]
	\arrow[from=2-2, to= 2-1]
\end{tikzcd}
\]
where the horizontal functors are fully faithful. 

\end{proof}

By an argument dual to the proof of Corollary \ref{Identification of tame Lie algebras}, we obtain the analogous result for tame commutative coalgebras in $\Sp^{\geq r}_{\tame}$. 
\begin{corollary}
\label{Identification of tame dp,nil coalgebras}
    The $\infty$-category $\coCAlg(\Sp^{\geq r}_{\tame})$ can be identified with the full subcategory of $\coCAlg^{\divpow, \nil}(\Sp)^{\geq r}$ spanned by commutative divided power, conilpotent coalgebras whose underlying spectra are tame.
\end{corollary}

We now give an algebraic characterization of tame spectral Lie algebras and tame commutative coalgebras.
Recall from Theorem \ref{algebraic description of tame spectra} that we have the identification
$$
\Sp^{\geq r}_{\tame} \simeq
\Mod_{H\BZ,\tame}^{\geq r},
$$
and the localization $L_{\tame}:\Mod_{H\BZ}^{\geq r} \to \Mod_{H\BZ,\tame}^{\geq r}$. By the same proof as in Corollary \ref{Identification of tame Lie algebras}, we obtain the following corollaries.
\begin{corollary}
\label{Identification of tame Lie algebras in Mod_HZ}
    The $\infty$-category $\Alg_{\spLie}(\Sp^{\geq r}_{\tame})$ of tame spectral Lie algebras can be identified with the full subcategory of $\Alg_{\spLie}(\Mod_{H\BZ})^{\geq r}$ spanned by Lie algebras whose underlying $H\BZ$-module spectra are tame.
\end{corollary}

\begin{corollary}
\label{Identification of tame commutative coalgebras in Mod_HZ}
    The $\infty$-category $\coCAlg(\Sp^{\geq r}_{\tame})$ of tame commutative coalgebras can be identified with the full subcategory of $\coCAlg^{\divpow, \nil}(\Mod_{H\BZ})^{\geq r}$ spanned by commutative divided power, conilpotent coalgebras whose underlying $H\BZ$-module spectra are tame.
\end{corollary}


\begin{remark}
\label{Identify tame Lie algebra with Dwyer's Lie algebra}
Since $\Mod_{H\BZ}^{\geq r}$ models the derived $\infty$-category $D(\BZ)^{\geq r}$ of $r$-connective chain complexes, Corollary \ref{Identification of tame Lie algebras in Mod_HZ} implies that the $\infty$-category $\Alg_{\spLie}(\Sp^{\geq r}_{\tame})$ of tame Lie algebras models Dwyer's tame Lie algebras in \cite{Dwyer}.
\end{remark}











To conclude this section,
we discuss Koszul duality between Lie algebras and divided power, conilpotent commutative coalgebras.
That is, we consider the functor
$$\indec_{\CO}: \Alg_{\CO}(\CC)\to \coAlg_{\Barconstruction(\CO)}(\CC)$$
for $\CO$ the Lie operad. 
Classically, if $\CC$ is the $\infty$-category $\Sp_{\BQ}$ of rational spectra, or equivalently, the $\infty$-category of rational chain complexes, then the functor 
$$
\cot_{\BL}: \Alg_{\BL}(\Sp_{\BQ}) 
\to 
\Sp_{\BQ}
$$
computes the Chevalley-Eilenberg cohomology of a dg Lie algebra over $\BQ$. 

\begin{definition}
\label{CChevalley-Eilenberg functor}
  We define two versions of the \emph{Chevalley-Eilenberg} functor on $\Sp$:
    \begin{itemize}
        \item  
        $
        \operatorname{CE}: \Alg_{\spLie}(\Sp)\xrightarrow{\Sigma'} \Alg_{\BL}(\Sp)
        \xrightarrow{\cot_{\BL}}
        \Sp
        $
        where $\Sigma'$ is the equivalence in Lemma \ref{Shift has no harm};
        \item $
        \widetilde{\operatorname{CE}}: \Alg_{\spLie}(\Sp)\xrightarrow{\Sigma'} \Alg_{\BL}(\Sp)
        \xrightarrow{\indec_{\BL}}
        \coAlg^{\divpow, \nil}_{\Com}(\Sp).
        $
    \end{itemize}
\end{definition}

For later application, we record the following theorem from \cite{Ching-Harper}.
\begin{theorem}
\cite{Ching-Harper}
\label{Ching-Harper's Koszul duality}
	Let $R$ be an $\E_{\infty}$-ring spectrum and $\CO$ an operad in the $\infty$-category $\Mod_R$ of $R$-module spectra with $\CO(1)=R$. Suppose $R$ and all $\CO(n)$ are connective. Then the functor 
	\[
	\indec_{\CO}: \Alg_{\CO}(\Mod_R)^{\geq 1} \to \coAlg^{nil, dp}_{\Barconstruction(\CO)}(\Mod_R)^{\geq 1}
	\]
	 is an equivalence.
\end{theorem}

Taking $R=H\BZ$, we obtain the following corollary.
\begin{corollary}
	For $r\geq 1$, the Chevalley-Eilenberg functor 
	\[
	\widetilde{\operatorname{CE}}: \Alg_{\spLie}(\Mod_{H\BZ})^{\geq r} \to \coAlg^{nil, dp}_{\Com}(\Mod_{H\BZ})^{\geq r}
	\]
	is an equivalence of $\infty$-categories.
\end{corollary}


We now define the Chevalley-Eilenberg functor for tame spectral Lie algebras. First we need the following lemma regarding the bar construction of the localized monad $L_{\tame}F_{\CO}$.
\begin{lemma}
    Let $\CO$ be a connected $\infty$-operad in $ \Sp$ and let $F_{\CO}$ be the induced monad on $\Sp^{\geq r}$,
    then 
    $$
    \Barconstruction(L_{\tame}F_\CO ) \simeq L_{\tame} \Barconstruction(F_\CO).
    $$
\end{lemma}
\begin{proof}
    Unravelling the definition of bar construction, we need to show there is an equivalence
    $$
     |\Barconstruction(\id, L_{\tame}\CO , \id)_{\bullet}|
     \simeq
     L_{\tame}|\Barconstruction(\id, \CO , \id)_{\bullet}|.
    $$
    Since $L_{\tame}$ preserves colimits, we are reduced to checking that
    $$
    L_{\tame} (\underbrace{\CO \circ \cdots \circ \CO}_{\text{$n$-fold}} )
    \simeq 
    \underbrace{ L_{\tame}\CO \circ \cdots \circ  L_{\tame}\CO}_{\text{$n$-fold}} 
    $$
    for each $n\geq 1$.
    It suffices to check this for two-fold composition, which can be done directly on the composition of the associated Schur functors
    $$
    F_{\CO}(X):= \coprod_{n\geq 1} (\CO(n) \otimes X^{\otimes n})_{h\Sigma_n}.
    $$ 
    The desired equivalence follows from the fact that $L_{\tame}$ is symmetric monoidal and preserves colimits.
\end{proof}

As a consequence, taking $\CO=\spLie$ allows us to identify the bar construction of the monad $L_{\tame}F_{\spLie}$ as the comonad $L_{\tame}F_{\Com}$. Explicitly, the comonad $L_{\tame}F_{\Com}$ on a tame spectrum $X$ is given by
$$
X \mapsto \coprod_{n\geq 1} L_{\tame} (X^{\otimes n})_{h\Sigma_n}.
$$
Therefore, we obtain a functor 
$$
L_{\tame} \indec_{\BL}:
\Alg_{\BL}(\Sp^{\geq r+1}_{\tame})
\to
\coAlg^{\divpow, \nil}_{\Com}(\Sp^{\geq r+1}_{\tame}).
$$

Note that there is an equivalence of $\infty$-categories between $r$-tame spectra and $(r+1)$-tame spectra by Lemma \ref{Relations between different tameness}, hence we obtain a pullback diagram 
\[
    \begin{tikzcd}
	\Alg_{\spLie}(\Sp^{\geq r}_{\tame})    & \Alg_{\BL}(\Sp^{\geq r+1}_{\tame})\\
	 \Sp^{\geq r}_{\tame} & \Sp^{\geq r+1}_{\tame}
	\arrow[from=1-1, to = 1-2, "\Sigma'", "\simeq" below]
	\arrow[from=1-1 , to =2-1,  left]
	\arrow[from=1-2 , to =2-2,  right]
	\arrow[from=2-1, to=2-2, "\Sigma", "\simeq" below]
\end{tikzcd}
\]
as in Lemma \ref{Shift has no harm}.

\begin{definition}
\label{Chevalley-Eilenberg functor for tame spectra}
We define the Chevalley-Eilenberg functor for tame spectra
to be the following composite
$$
\tildeCE_{\tame}: \Alg_{\spLie}(\Sp^{\geq r}_{\tame})
\xrightarrow[\simeq]{\Sigma'}
\Alg_{\BL}(\Sp^{\geq r+1}_{\tame})
\xrightarrow{L_{\tame} \indec_{\BL}}
\coAlg^{\divpow, \nil}_{\Com}(\Sp^{\geq r+1}_{\tame}).
$$
\end{definition}

\begin{remark}
The functor $\tildeCE_{\tame}$ admits a right adjoint by the adjoint functor theorem, which we will denote by $\widetilde{\Prim}$.
\end{remark}

% We end this section with a result concerning the Chevalley-Eilenberg functor on tame spectra of different degrees.
% \begin{proposition}
% $$

% $$
% \end{proposition}

\begin{remark}
\label{Formula for computing CE_tame(L)}
The underlying spectrum of $\tildeCE_{\tame}(L)$ can be computed as the geometric realization of the bar construction 
$$
|B(\id_{\Sp^{\geq r+1}_{\tame}}, L_{\tame}F_{\BL}\circ \Sigma', \id_{\Sp^{\geq r+1}_{\tame}})|
$$
which is equivalent to $L_{\tame}\tildeCE(L)$ as $L_{\tame}$ is colimit-preserving and symmetric monoidal.
\end{remark}

\begin{lemma}
\label{Lemma 3.6.20}
There are formal equivalences
$$
\tildeCE_{\tame}\circ \trivial_{\spLie}\simeq \Sym_{\tame}\circ \Sigma
$$
and
$$
\widetilde{\Prim}\circ \Sym_{\tame} \simeq \trivial_{\spLie}\circ \Omega.
$$
\end{lemma}
\begin{proof}
    The proof is analogous to that of Lemma \ref{CE of Trivial is cofree}, plus the equivalences 
    $
    \trivial_{\BL}\circ \Sigma \simeq \Sigma'\circ \trivial_{\spLie} 
    $
    and $\Omega' \circ \trivial_{\BL} \simeq \trivial_{\spLie}\circ \Omega $.
    
\end{proof}



Consider the functor $\trivial_{\spLie}:\Sp^{\geq r}_{\tame}\to \Alg_{\spLie}(\Sp^{\geq r}_{\tame})$ and $$
\indec_{\BL}:\Alg_{\BL}(\Sp^{\geq r+1}_{\tame})
\to
\coAlg^{\divpow, \nil}_{\Com}(\Sp^{\geq r+1}_{\tame}).
$$



Proposition \ref{all coalgebras are equivalent} allows us to identify the codomain of the Chevalley-Eilenberg functor $\tildeCE_{\tame}$ on tame spectra as $\coCAlg(\Sp^{\geq r}_{\tame})$, whose categorical product is given by the tensor product $\hat{\otimes}$ in $\Sp^{\geq}_{\tame}$.
We now show that the Chevalley-Eilenberg functor $\widetilde{\operatorname{CE}}$ preserves finite products, hence it induces a functor on the categories of group objects.
\begin{lemma}
\label{CE preserves products}
	The functor 
	$\widetilde{\operatorname{CE}}_{\tame}$
	preserves finite products.
\end{lemma}
\begin{proof}
    Since $\Sigma'$ is an equivalence, it suffices to show
	the natural morphism 
	\[
	L_{\tame}\indec_{\BL}(L \times L') 
	\to 
	L_{\tame}\indec_{\BL}(L)\hat{\otimes} 
	L_{\tame}\indec_{\BL}(L') 
	\]
	is an equivalence for $L,L' \in \Alg_{\spLie}(\Sp^{\geq r}_{\tame}$). 
	By Remark \ref{Formula for computing CE_tame(L)}, the proof reduces to showing that the natural map 
	\[
    c:	\indec_{\BL}(L \times L') 
	\to 
	\indec_{\BL}(L)\otimes \indec_{\BL}(L') 
	\]
	is a tame equivalence when we regard $L,L'$ as objects in $\Alg_{\spLie}(\Sp)$.

	For a spectral Lie algebra $X$, we let $X_*$ denote its canonical filtration from Theorem \ref{Canonical grading on an O-algebra}. At degree $n$,
	the associated graded of the canonical filtration admits an explicit description 
	$$
	\Free_{\BL}((B\BL(n)\otimes X^{\otimes n})_{h\Sigma_n}.
	$$
	If $f: X \to Y $ is a map between filtered spectral Lie algebras, then
	postcomposing with $\indec_{\BL}$ induces a map on their canonical filtrations
	$$
	\indec_{\BL}: X_* \to Y_*. 
	$$
	
	Since $X\simeq \colim X_*$ and $\indec_{\BL}$ preserves colimits, it suffices to show $c$ induces an equivalence on the canonical filtrations, i.e.
	$$
	\indec_{\BL}(L_* \times L'_*)
	\to 
	\indec_{\BL}(L_*) \otimes \indec_{\BL}(L'_*).
	$$
	Note also that the associated graded functor $\Graded$ is conservative by Lemma \ref{Ass-gr is conservative}, hence the proof further reduces to proving that the induced map on associated gradeds
    \[
    \nu:
	\indec_{\BL}( \Graded(L \times L')_{*} )
	\to 
	\indec_{\BL}(\Graded(L_*))\otimes \indec_{\BL}(\Graded(L'_*))
	\]
	is an equivalence.
	Since
	$$
	\indec_{\BL}\big(
	\Free_{\spLie}((B\spLie(n)\otimes X^{\otimes n})_{h\Sigma_n} )
	\big)
	\simeq 
	(X^{\otimes n})_{h\Sigma_n}
	\simeq
	\Sym^n(X)
	$$
	for any $X\in \Sp$.
	Hence the map $\nu$ is equivalent to the natural map
	$$
	\Sym (L\times L') \to \Sym(L) \otimes \Sym(L'),
	$$
	which is a tame equivalence since the symmetric coalgebra functor $\Sym$ is the cofree coalgebra functor which preserves products and the tensor product in $\Sp^{\geq r}_{\tame}$ is the product in $\coCAlg^{\divpow,\nil}(\Sp^{\geq r}_{\tame})\simeq \coCAlg(\Sp^{\geq r}_{\tame})$.

\end{proof}













