\chapter{A Hopf Algebra Model for Tame spaces}

In this last chapter, we prove Theorem \ref{Theorem B} and Theorem \ref{Theorem C} stated in the introduction,
%The first theorem is a direct consequence of our discussion in the \textbf{chapter of Koszul duality}. 
%\begin{theorem}
%	The universal enveloping algebra functor
%	$$
%	U: \Alg_{\Lie}(\Sp^{\geq r-1}_{(r-1)\text{-}\tame})
%	\to
%	\Hopfalgebra(\Sp^{\geq r-1}_{(r-1)\text{-}\tame})
%	$$
%	is an equivalence of $\infty$-categories.
%\end{theorem}
%\begin{remark}
%\begin{enumerate}
%	\item Here $\Hopfalgebra(\Sp^{\geq r-1}_{(r-1)\text{-}\tame})$ denotes the $\infty$-category of Hopf algebra objects in $(r-1)$-tame spectra as defined in \todo{previous section}. 
%	\item By Proposition \ref{Identification of tame Lie algebras}, the $\infty$-category $\Alg_{\Lie}(\Sp^{\geq r-1}_{(r-1)\text{-}\tame})$ of tame Lie algebras can be identified with the full subcategory of $\Alg_{\Lie}(\Mod_{H\BZ})^{\geq r-1}$ spanned by Lie algebras whose underlying $H\BZ$-modules are tame.
%\end{enumerate}
%\end{remark}
concerning a new Hopf algebra model for $r$-tame spaces. Note that every pointed space $X$ is a commutative coalgebra in $\Space_*$ via the diagonal $X\to X\times X \to X\wedge X$. Since the functors $\Sigma^{\infty}$ and $L_{\tame}$ are symmetric monoidal, the functor
$$
\Sigma^{\infty}_{\tame}: \Space^{\geq r}_{\tame}  \to \Sp^{\geq r}_{\tame}
$$
factors through the $\infty$-category $\coCAlg(\Sp^{\geq r}_{\tame})$ of commutative coalgebras in $\Sp^{\geq r}_{\tame}$, and we denote the resulting functor as
$$
C_{\tame}:  \Space^{\geq r}_{\tame}
\to 
\coCAlg(\Sp^{\geq r}_{\tame}).
$$
The funcor $C_{\tame}$ is colimit-preserving, hence it admits a right adjoint $R$ by the adjoint functor theorem \cite[Corollary 5.5.2.9.]{HTT}. 
We summarize the situation in the following diagram
\[
\begin{tikzcd}
	\Space^{\geq r}_{\tame} &   & \Sp^{\geq r}_{\tame} \\
	&  \coCAlg(\Sp^{\geq r}_{\tame}) &
	\arrow[from=1-1, to = 1-3, "\Sigma^{\infty}_{\tame}", shift left]
	\arrow[from=1-3, to = 1-1, "\Omega^{\infty}", shift left]
	\arrow[from=1-1 , to =2-2, "C_{\tame}", shift left]
	\arrow[from=2-2 , to =1-1, "R", shift left]
	\arrow[from=2-2, to=1-3, "\oblv", shift left]
	\arrow[from=1-3, to=2-2, "\Sym_{\tame}", shift left],
\end{tikzcd}
\]
where left adjoints sit above right adjoint.

Classically, cocommutative Hopf algebras can be identified as group objects in the Cartesian category of cocommutative counital coalgebras \cite{Milnor-Moore}. This motivates us to give the following definition of the  $\infty$-category of Hopf algebras in tame spectra. 
\begin{definition}
\label{tame Hopf algebras}
    We define the $\infty$-category of \emph{Hopf algebras in $r$-tame spectra} to be the category of group objects in $\coCAlg(\Sp^{\geq r}_{\tame})$, i.e., 
    $$
    \Hopfalgebra(\Sp^{\geq r}_{\tame}):= \Grp(\coCAlg(\Sp^{\geq r}_{\tame})).
    $$
\end{definition}	

Our first goal is to establish a Hopf algebra model for tame spaces. The following is the precise statement of Theorem \ref{Theorem B}.
\begin{theorem}
\label{2nd Main Theorem}
The composite
$$
\Space^{\geq r}_{\tame} \xrightarrow{\Omega}
\Grp(\CS^{\geq r-1}_{\tame})
\xrightarrow{\Grp(C_{\tame})}
\Hopfalgebra(\Sp^{\geq r-1}_{\tame})
$$
is an equivalence of $\infty$-categories.
\end{theorem}

We now sketch our strategy for the proof of Theorem \ref{2nd Main Theorem}.
We originally thought that there should be a coalgebra model for tame spaces, i.e., 
we suspected that the functor
	$$
	C_{\operatorname{tame}}:\Space^{\geq r}_{tame}\to \coCAlg(\Sp_{tame})
	$$
	was an equivalence, as this is indeed the case in Quillen's rational homotopy theory and Mandell's $p$-adic homotopy thoery. However, this functor fails to be fully faithful; indeed, if $C_{\tame}$ is fully faithful, then the composition 
	$$
	\oblv_{\Com}\circ C_{\tame}\simeq \Sigma^{\infty}_{\tame}
	$$
	would be conservative. 
	Since the homotopy groups of $\Sigma^{\infty}_{\tame}X$ are computed as the homology groups of $X$ with coefficients in the tame ring system by Corollary \ref{htpy groups of tame spectra can be computed by homology}, the statement that
	$\Sigma^{\infty}_{\tame}$ is conservative is equivalent to a Whitehead theorem for tame spaces.
	That is, a map $f:X \to Y$ of tame spaces is an equivalence if and only if the induced map on homology groups with coefficients in the tame ring system is an isomorphism
	$$
	H_{r+j}f\otimes R_j:
	H_{r+j}X \otimes R_j
	\to 
	H_{r+j}Y \otimes R_j
	$$
	for all $j\geq 0$.
	
	However, we do not have a Whitehead theorem for tame spaces, as we need more conditions on the cokernel $\coker(h_{r+j, X})$ of the Hurewicz map
	$$
	h_{r+j,X}: \pi_{r+j}X\otimes R_j \to 
	H_{r+j}X\otimes R_j.
	$$
	\begin{proposition}
	\cite[Proposition 1.1]{Felix-LemaireII}
	If $f:X \to Y$ is a map between $r$-connective spaces, then the following are equivalent:
\begin{enumerate}
    \item $f$ is a tame equivalence.
    \item For all $j\geq 0$, 
    $$
    H_{r+j}f \otimes R_j: 
    H_{r+j}X \otimes R_j
	\to 
	H_{r+j}Y \otimes R_j
    $$
    is an isomorphism and 
    $$
    \coker(h_{r+j,X}) \to \coker(h_{r+j,Y})
    $$
    is surjective.
\end{enumerate}
	\end{proposition}
	
	Taking a step back, we prove that $C_{\operatorname{tame}}$ is fully faithful when restricted to the full subcategory of tame Eilenberg-Maclane spaces.
We then show that the functors $\Omega$ and $\Grp(C_{\operatorname{tame}})$ are both equivalences of $\infty$-categories. 
The crux of the latter is that the loop of a $r$-tame space splits into a product of tame Eilenberg-Maclane spaces, and therefore we can conclude that $\Grp(C_{\operatorname{tame}})$ is fully faithful.

The second goal of this final chapter is to connect the Hopf algebra model for tame spaces with tame spectral Lie algebras defined in \S \ref{Lie Algebras in Tame Spectra}.

Since Lemma \ref{CE preserves products} ensures the functor $\tildeCE_{\tame}$ preserves group objects, there is a functor on the categories of groups
\[
	\Grp(\tildeCE_{\tame}):
	\Grp(\Alg_{\spLie}(\Sp^{\geq r-1}_{\tame}))
	\to 
	\Grp(\coCAlg(\Sp^{\geq r-1}_{\tame})).
\]

Let $\Omega_{\spLie}$ denote the loop functor in $\Alg_{\spLie}(\Sp^{\geq r}_{\tame}) $. We prove in Proposition \ref{B and Omega are mutally inverses} that $\Omega_{\spLie}$ factors through the $\infty$-category of group objects $\Grp(\Alg_{\spLie}(\Sp^{\geq r-1}_{\tame}))$ in tame spectral Lie algebras. Moreover, we prove that the factorization
$$
\Omega_{\spLie}: \Alg_{\spLie}(\Sp^{\geq r}_{\tame})
\to 
\Grp(\Alg_{\spLie}(\Sp^{\geq r-1}_{\tame}))
$$
is an equivalence of $\infty$-categories.

In \cite{KnudsenHEA}, it is shown that the universal enveloping algebra $U(L)$ of a spectral Lie algebra defined there satisfies a Poincare-Birkhoff-Witt theorem
\cite[Theorem B]{KnudsenHEA}.
Motivated by this, we define the universal enveloping algebra functor as follows.

\begin{definition}
	The \emph{universal enveloping algebra functor} is defined as the following composite
	$$
	U: \Alg_{\spLie}(\Sp^{\geq r}_{\tame}) 
	\xrightarrow[\simeq]{\Omega_{\spLie}}
	\Grp(\Alg_{\spLie}(\Sp^{\geq r-1}_{\tame}))
	\xrightarrow{\Grp(\tildeCE_{\tame})}
	\Hopfalgebra(\Sp^{\geq r}_{\tame}).
	$$
\end{definition}


The following is the precise statement of Theorem \ref{Theorem C}.
\begin{theorem}
\label{FIRST_MAIN_THEOREM}
	The universal enveloping algebra functor
	$$
	U: \Alg_{\spLie}(\Sp^{\geq r}_{\tame})
	\to
	\Hopfalgebra(\Sp^{\geq r}_{\tame})
	$$
	is an equivalence of $\infty$-categories.
\end{theorem}

Combining Theorem \ref{FIRST_MAIN_THEOREM} and Theorem \ref{2nd Main Theorem}, we establish an $\infty$-categorical version of Dwyer's Lie algebra model for tame spaces in \cite{Dwyer}.
\begin{theorem}
\cite{Dwyer}
	There is an equivalence of $\infty$-categories
	\[
	\CS^{\geq r}_{\tame} 
	\to
	\Alg_{\spLie}(\Sp^{\geq r-1}_{\tame}).
	\]
\end{theorem}

In \S \ref{Section: Fully faithfulness on Eilenberg-Maclane spaces}, we prove the functor $C_{\operatorname{tame}}$ is fully faithful on the full subcategory of Eilenberg-Maclane spaces by computing the cohomology of Eilenberg-Maclane spaces with coefficients in the tame ring system.
In \S \ref{Hopf algebra model for tame spaces}, we finalize our proof of Theorem \ref{2nd Main Theorem}.
In \S \ref{Proof of the first main Theorem}, we show that there is an equivalence from the $\infty$-category of tame Lie algebras to the $\infty$-category of tame Hopf algebras. 
%\section{Proof of Theorem \ref{first main theorem}}


\section{Fully faithfulness on Eilenberg-MacLane spaces}\label{Section: Fully faithfulness on Eilenberg-Maclane spaces}

Our goal in this section is to prove the following proposition.
\begin{proposition}
\label{Fully faithfulness of C_tame on EM spaces}
	The functor
	$$
	C_{\operatorname{\tame}}:\Space^{\geq r}_{\tame}\to \coCAlg(\Sp^{\geq r}_{\tame})
	$$
is fully faithful on the full subcategory spanned by Eilenberg-MacLane spaces.
\end{proposition}

%Note that $C_{\operatorname{tame}}$ admits a right adjoint $R$, where $X$ should be the monoidal unit for the symmetric monoidal structure on $\Sp_{tame}$ and it is equipped with the trivial coalgebra structure. 

To simplify notation, we let $n=r+k$ in this section, where $k$ is a non-negative integer. 
Fix an  $R_{k}$-module $V$.
We first show that $C_{\tame}K(V, n)$ is equivalent to the cofree commutative coalgebra generated by $\Sigma^{n}HV$.
By Proposition \ref{all coalgebras are equivalent}, the cofree commutative coalgebra generated by $\Sigma^{n}HV$ is given by the symmetric coalgebra 
$\Sym_{\tame} \Sigma^{n}HV$ in $\Sp^{\geq r}_{\tame}$.
There is a canonical map 
\begin{equation}
\label{(4.1.2.)}
    \Sigma_{\tame}^{\infty}K(V,n)\to  \Sigma^{n}HV
\end{equation}
given by $n$-truncation. By the forgetful-cofree adjunction, the map (\ref{(4.1.2.)}) corresponds to a map
\begin{equation}
\label{C_tame to Sigma^n HV}
    C_{\tame}K(V,n)\to \Sym_{\tame} \Sigma^{n}HV.
\end{equation}

Since the forgetful functor 
$$
\oblv: \coCAlg(\Sp^{\geq r}_{\tame}) \to 
\Sp^{\geq r}_{\tame}.
$$
is conservative, it suffices to check that the map obtained by applying forgetful functor to (\ref{C_tame to Sigma^n HV})
\begin{equation}
\label{(4.1.1.)}
\gamma:  \Sigma^{\infty}_{\tame}K(V,n)\to \Sym_{\tame} \Sigma^{n}HV
\end{equation}
is an equivalence. Here we abuse notation by writing $\Sym_{\tame} \Sigma^{n}HV$ also for the underlying spectrum of the cofree coalgebra generated by $\Sigma^{n}HV$.
% A few further reductions is in order.
Since both $\Sigma^{\infty}_{\tame}\Omega^{\infty}$ and $\Sym_{\tame}$ preserves filtered colimits, we can reduce to the case of $V$ being a finitely generated $R_k$-module. 

Let $M(V,n)$ be the Moore space of type $(V,n)$.
By Remark \ref{suspension tame sends Moore space to EM-space}, $L_{\tame}\Sigma^{\infty}$ sends $M(V,n)$ to a tame Eilenberg-Maclane spectrum, i.e., 
$$
L_{\tame}\Sigma^{\infty} M(V,n) \simeq  \Sigma^n H V.
$$
Let $X$ be a pointed connected space. The underlying space of the
free $\E_{\infty}$-space generated by $X$ is given by the symmetric algebra (see Example \ref{Symmetric algebra})
\[
\Sym(X) = \bigvee_{m\geq 1} (X^{\wedge m})_{h\Sigma_m};
\]
If $X$ is tame, then the free tame $\E_{\infty}$-space generated by $X$ is 
\[
\Sym_{\tame}(X) = \bigvee_{m\geq 1} L_{\tame}(X^{\wedge m})_{h\Sigma_m};
\]
We claim the functor $\Sym_{\tame}$ on tame spaces is analogous to the infinite symmetric product construction, i.e., it sends the tame localization of Moore spaces to Eilenberg-MacLane spaces.
\begin{lemma}
\label{Lemma 4.1.2.}
	Let $V$ be a finitely generated $R_k$-module.
	Then there is an equivalence of tame spaces
	$$
	 \Sym_{\tame} L_{\tame}M(V, n)\simeq K(V, n).
	$$
\end{lemma}
\begin{proof}
	By May's theorem \cite[6.3]{May_Iterated_Loop_Spaces}, there is an equivalence 
	$$
	\Sym X \simeq \Omega^{\infty}\Sigma^{\infty} X
	$$
	for any connected space $X$.
	Hence, the space $\Sym_{\tame} L_{\tame} M(V,n)$ is equivalent to the tame localization of $\Omega^{\infty}\Sigma^{\infty} L_{\tame}M(V, n)$.
	Since both $\Sigma^{\infty}$ and $\Omega^{\infty}$ preserve tame equivalences,
	$$
	L_{\tame}\Omega^{\infty}\Sigma^{\infty} L_{\tame}M(V, n) 
	\simeq 
	L_{\tame}\Omega^{\infty}\Sigma^{\infty}M(V, n) 
	$$
	the homotopy groups of which are given by 
	\begin{align*}
		H_* M(V, n)\otimes R_{*-r}= \begin{cases}
			V & \text{for } * = n;\\
			0 & \text{otherwise. }
		\end{cases}
	\end{align*}
\end{proof}

Since both the functors $L_{\tame}:\Sp^{\geq r}\to \Sp^{\geq r}_{\tame}$ and $\Sigma^{\infty}_{\tame}:\Space^{\geq r}_{\tame}\to \Sp^{\geq r}_{\tame}$ are colimit-preserving and symmetric monoidal, we have the following lemma.
\begin{lemma}
\label{Sym commutes with Sigma infinity}
There is a commutative diagram
\[
\begin{tikzcd}
    	\Space^{\geq r}_{\tame} & \Space^{\geq r}_{\tame}\\
	\Sp_{\tame}^{\geq r} & \Sp_{\tame}^{\geq r}.
	\arrow[from=1-1, to= 1-2, "\Sym_{\tame}"]
	\arrow[from=1-1, to=2-1, "\Sigma_{\tame}^{\infty}" left]
	\arrow[from=1-2, to=2-2, "\Sigma_{\tame}^{\infty}"]
	\arrow[from=2-1, to= 2-2, "\Sym_{\tame}" below]
\end{tikzcd}
\]
\end{lemma}

\begin{proposition}
\label{Prop 4.1.4.}
The map 
$$
\gamma:  \Sigma^{\infty}_{\tame}K(V,n)\to \Sym_{\tame} \Sigma^{n}HV
$$
of (\ref{(4.1.1.)}) is an equivalence.
\end{proposition}

\begin{proof}
Combining Lemma \ref{Lemma 4.1.2.} and Lemma \ref{Sym commutes with Sigma infinity}, we have equivalences
\begin{align*}
	\Sigma_{\tame}^{\infty} K(V,n) 
& \simeq \Sigma_{\tame}^{\infty} \Sym_{\tame}L_{\tame} M(V,n)\\
& \simeq   \Sym_{\tame} \Sigma_{\tame}^{\infty}L_{\tame}M(V,n)\\
& \simeq   \Sym_{\tame} L_{\tame} \Sigma^{\infty}M(V,n)\\
& \simeq \Sym_{\tame} \Sigma^n HV,
\end{align*}
where the last equivalence follows from Remark \ref{suspension tame sends Moore space to EM-space}.
Hence $\gamma$ can be identified as the unique map that lifts $\Sigma^{\infty}_{\tame}K(V,n) \to \Sigma^n HV$ through the projection
$$
\Sym_{\tame} \Sigma^n HV \to \Sigma^n HV,
$$
which is an equivalence.
\end{proof}


Now we can prove Proposition \ref{Fully faithfulness of C_tame on EM spaces}. We denote the right adjoint of $C_{\tame}$ by $R$ as above.
\begin{proof}
[Proof of Proposition \ref{Fully faithfulness of C_tame on EM spaces}:]
Using Proposition \ref{Prop 4.1.4.}, we see that
\begin{align*}
    R C_{\tame} K(V,n) & \simeq R\Sym_{\tame}\Sigma^n HV \\
    & \simeq \Omega^\infty \Sigma^n HV\\
    & \simeq K(V,n),
\end{align*}
    so $C_{\tame}$ is fully faithful on the full subcategory of tame Eilenberg-Maclane spaces.
    
\end{proof}





%We will argue that for any $X \in \mathcal{S}_{\mathbb{Q}}^{\geq 2}$ the unit map of this adjunction
%$$
%X \rightarrow \operatorname{Map}_{\text {coCAlg }}\left(H \mathbb{Q}, \widetilde{C}_{\mathbb{Q}} X\right)
%$$
%is an equivalence. Here and above, the mapping spaces refer to those in the $\infty-$ category $\operatorname{coCAlg}^{\mathrm{nu}}\left(\mathrm{Sp}_{\mathbb{Q}}\right)$












%Let $R$ be a ring containing $\BQ$. 
%\begin{corollary}
%The map 
%\[
%SR\to HR
%\]
%induced from the unit map $\BS\to H\BZ$ is an equivalence.
%\end{corollary}
%\begin{proof}
%	Consider the following commutative diagram induced from the short exact sequence in universal coefficient theorem:
%	\[
%	\begin{tikzcd}
%		\pi_{*}\BS\otimes R & \pi_*(SR)\\
%		\pi_*{H\BZ}\otimes R & \pi_* (HR)
%		\ar[from=1-1, to=1-2]
%		\ar[from=1-1, to=2-1]
%		\ar[from=2-1, to=2-2]
%		\ar[from=1-2, to=2-2]
%	\end{tikzcd}
%	\]
%	where the two horizontal maps are isomorphisms by the universal coefficient theorem.
%	Since $\pi_{*}\BS$ only contains torsions for $*\geq 1$, hence $\pi_*\BS\otimes R=0$ for $*\geq 1$. The left vertical map is an isomorphism at degree $0$, therefore the corollary follows.
%	
%\end{proof}
%
%\begin{lemma}
%	The map 
%	$HR \to HR\otimes HR$ induced from the unit map $\BS \to HR$
%	is an equivalence.
%\end{lemma}
%\begin{proof}
%	Indeed, we have an isomorphism 
%	\[
%	\pi_{*} HR\otimes R \cong
%	\pi_{*} (HR\otimes SR)
%	\cong 
%	\pi_*(HR\otimes HR),
%	\]
%	hence $\pi_*(HR\otimes HR)$ is non-zero only when $*=0$, in which case we have 
%	$$
%	R\otimes R\cong R
%	$$ since $R$ is a solid ring, i.e., the multiplication map $R\otimes R\to R$ is an isomorphism.
%	
%\end{proof}
%
%\begin{corollary}
%	The $R$-localization functor $L_R:\Sp\to \Sp$ is a smashing localization.
%\end{corollary}
%\begin{proof}
%	Since smashing localization is in one-to-one correspondence with idempotent $\E_{\infty}$-rings.
%\end{proof}



\section{A Hopf algebra model for tame spaces}
\label{Hopf algebra model for tame spaces}

First we establish some preliminary results.
\begin{proposition}
\label{tame recognition theorem}
For $r\geq 4$,
	the loop functor $\Omega: \CS^{\geq r}_{\tame}\to \Grp(\CS^{\geq r-1}_{\tame})$ is an equivalence. 
\end{proposition}

\begin{proof}
	We have a commutative diagram 
	\[
	\begin{tikzcd}
		\CS^{\geq r}_{\tame} & \Grp(\CS^{\geq r-1}_{\tame})\\
		\CS^{\geq r}_*   &  \Grp(\CS^{\geq r-1}_{*})  	
	\arrow[from=1-1, to=1-2,"\Omega"]
	\arrow[from=2-1, to=2-2, "\Omega","\simeq" below]
	\arrow[hook, from=1-1, to=2-1]
	\arrow[hook, from=1-2, to=2-2]
	\end{tikzcd}
	\]
	where the left vertical functor is fully faithful as is right vertical functor by Proposition \ref{induced fully faithfulness on group objects}. The bottom arrow is an equivalence by May's recognition theorem (cf. see \cite[Theorem 5.2.6.10.]{HA}). Hence the top loop functor $\CS^{\geq r}_{\tame} \to \Grp(\CS^{\geq r-1}_{\tame})$ is fully faithful.
	For the essential surjectivity, we know that any $X\in \Grp(\CS^{\geq r-1}_{\tame})$ is equivalent to $\Omega Y$ for some $Y\in \CS^{\geq r}_*$, while $Y$ has to be $r$-tame since $X$ is $(r-1)$-tame.
	
\end{proof}

\begin{lemma}
\label{conservativity of forgetful functor (general)}
	Let $\CC$ be a Cartesian symmetric monoidal $\infty$-category and let $L:\CC\to \CC'$ be a localization functor that preserves finite products.
	Then both forgetful functors 
	\[
	\Grp(\CC)\to \CC
	\]
	and 
	\[
	\Grp(\CC')\to \CC'
	\]
	are conservative.
% 	\begin{enumerate}
% 		\item the forgetful functor
% 	\[
% 	\Grp(\CC)\to \CC
% 	\]
% 	is conservative;
% 	\item the forgetful functor
% 	\[
% 	\Grp(\CC')\to \CC'
% 	\]
% 	is also conservative.
% 	\end{enumerate}
\end{lemma}
\begin{proof}
	Since the category of groups is a full subcategory of monoids, it suffices to prove the forgetful functor $\Mon(\CC)\to \CC$ is conservative.
	The statements then follows from \cite[Proposition 2.4.2.5]{HA} and \cite[Lemma 3.2.2.6.]{HA}.
	
% 	For the second statement, consider the commutative diagram
% 	\[
% \begin{tikzcd}
% 	\Mon(\CC') & \CC'\\
% 	\Mon(\CC) & 
% 	\CC	
% 	\arrow[from=1-1, to= 1-2, "\oblv_{\Mon}"]
% %	\arrow[from=1-2, to= 1-1, shift left, "R"]
% 	\arrow[from=1-1, to=2-1]
% 	\arrow[from=1-2, to=2-2]
% 	\arrow[from=2-1, to= 2-2, "\oblv_{\Mon}"]
% %	\arrow[from=2-2, to= 2-1, shift left, "R'"]
% \end{tikzcd}
% \]
% where both vertical arrows are fully faithful (hence conservative) and the bottom arrow is conservative by the first statement. Hence the top arrow is conservative as well.
\end{proof}

Applying Lemma \ref{conservativity of forgetful functor (general)} to the case of tame spaces, we have the following corollary.

\begin{corollary}
\label{conservativity of forgetful functor}
	The forgetful functor 
	\[
	\Grp(\CS^{\geq r-1}_{\tame}) \xrightarrow{\oblv_{\Grp}} 
	\CS^{\geq r-1}_{\tame}
	\]
	is conservative.
\end{corollary}

Since the categorical product of $\coCAlg(\Sp^{\geq r-1}_{\tame})$ is given by the smash product in $\Sp^{\geq r-1}_{\tame}$, 
the smash product equips $\coCAlg(\Sp^{\geq r-1}_{\tame})$ with the structure of a Cartesian symmetric monoidal $\infty$-category. Moreover, the functor 
$$C_{tame}: \CS^{\geq r-1}_{\tame} \to
\coCAlg(\Sp^{\geq r-1}_{\tame})$$
sends products to smash products, and therefore induces a functor
\begin{equation}
\label{G_tame}
    G_{tame}:\Grp(\CS^{\geq r-1}_{\tame})
\to
\Grp(\coCAlg(\Sp^{\geq r-1}_{\tame}))
\end{equation}
where the latter $\infty$-category is the $\infty$-category of tame Hopf algebras (Definition \ref{tame Hopf algebras}).

Applying Lemma \ref{conservativity of forgetful functor (general)} again, we have the following result.
\begin{corollary}
\label{conservativity of the forgetful functor for Hopf algebras}
		The forgetful functor 
	\[
	\Hopfalgebra(\Sp^{\geq r-1}_{\tame}) \xrightarrow{\oblv_{\Grp}} 
	\coCAlg(\Sp^{\geq r-1}_{\tame})
	\]
	is conservative.
\end{corollary}
\begin{proof}
	This follows from the fact that $\coCAlg(\Sp^{\geq r-1}_{(r-1)\text{-}\tame})$ is a Cartesian symmetric monoidal $\infty$-category and Lemma \ref{conservativity of forgetful functor (general)}.

\end{proof}

Recall that in rational homotopy theory, any rational H-space splits as a product of Eilenberg-Maclane spaces.
Our proof of the Hopf algebra model of tame spaces will be based on an analogous splitting result for tame H-spaces. 
The following proposition was already proved in \cite[Proposition 1.7]{Scheerer-Tanre} using Dwyer's Lie algebra model for tame spaces. 
\begin{proposition}
\label{Decomp of tame H-spaces}
	Let $X$ be an $r$-tame $\E_1$-space. Then $X$ is equivalent to a product of Eilenberg-Maclane spaces, that is, 
	\[
	X \simeq \prod_i K(\pi_i X, i).
	\]
\end{proposition}
\pagebreak
\begin{proof}
We first assume $X$ is of finite type, i.e., all its homotopy groups are finitely generated abelian groups. If $X$ is rational, then the result follows from rational homotopy theory. Otherwise, we can assume $X$ is $p$-local for every some prime $p$.

Observe that if $X$ is both tame and $p$-local, then $X$ is $(r+2p-4)$-truncated; since $X$ is $p$-local and $\pi_{n}X$ is uniquely $p$-divisible for $n \geq r+2p-3$. Note also that if $p=2$, then $X$ is an Eilenberg-MacLane space $K(\pi_r X, r)$ ($2$ is inverted at degree $r+1$), hence we can assume $p$ is an odd prime. 

We proceed by induction on the Postnikov tower of $X$.
The base case is obvious. 	
For the inductive step, consider the principal fiber sequence
$$
K(\pi_{n+1}X,n+1)\to 
\tau_{\leq n+1}X
\to
\tau_{\leq n}X
\xrightarrow{k_n}
K(\pi_{n+1}X,n+2).
$$
We want to show that the $k$-invariant $[k_n]\in H^{n+2}(\tau_{\leq n}X; \pi_{n+1}X)$ vanishes. 
Note that $k_n$ represents a primitive element in the $p$-local Hopf algebra $H^*(\tau_{\leq n}X; \pi_{n+1}X)$ (\cite[Theorem 3.2]{D.Kahn}).
We claim $[k_n]$ is also indecomposable.
A primitive element in a $p$-local commutative Hopf algebra is decomposable if and only if it's a $p$-th power \cite[Proposition 4.21]{Milnor-Moore}.
However, $[k_n]$ cannot be a $p$-th power for degree reasons. As
$n<r+2p-3$, whence
\[
n-rp<r+2p-3-rp< (2-r)(p-1)<0.
\]
	
By the inductive hypothesis, the cohomology of $\tau_{\leq n} X$ is a product of Eilenberg-Maclane spaces. By the K\"{u}nneth formula, 
it suffices to show 
$$
H^{n+2}(K(\pi_i X, i); \pi_{n+1}X) = 0
$$
for $r\leq i \leq n$.
Cartan \cite{Cartan_I} shows that if $R$ is a $p$-local ring, all cohomology classes  $[\alpha]\in H^{*}(K(A,k);R)$ are decomposable in degree $k<*<k+2(p-1)$. 
Since $n+2 < i + 2(p-1)$ for any $r\leq i \leq n$,
$[k_n]=0$ for degree reasons.

Now a general tame $\E_1$-space $X$ can be written as a filtered colimits of tame $\E_1$-spaces of finite type
$$
X\simeq \colim_{\alpha} X_{\alpha}.
$$
Since the $n$-truncation commutes with colimits, there are equivalences:
\begin{align*}
 \tau_{\leq n} X & \simeq \colim_{\alpha} \tau_{\leq n} X_{\alpha}\\
 & \simeq \colim_{\alpha} \prod^{n}_{i=r}K(\pi_iX_{\alpha} , i)\\
 & \simeq \prod^{n}_{i=r} \colim_{\alpha}K(\pi_{i}X_{\alpha},i)\\
 & \simeq \prod^{n}_{i=r} K(\colim_{\alpha}\pi_{i}X_{\alpha},i)\\
 & \simeq \prod^{n}_{i=r} K(\pi_{i}X,i)\\
\end{align*}
where we also used that filtered colimits commute with finite limits and filtered colimits commutes with $\pi_*$.
Therefore, the proposition is proved.
\end{proof}
\begin{remark}
		We learned part of the proof of Proposition \ref{Decomp of tame H-spaces} from Soul\'e \cite{soule}[Proposition 3], where the author further attributed the idea to L.Smith.	 
\end{remark}

We need the following technical lemma for the proof of Theorem \ref{2nd Main Theorem}.
\begin{lemma}
\label{RC_tame preserves connectivity}
The monad $RC_{\tame}: \CS^{\geq r}_{\tame} \to 
\CS^{\geq r}_{\tame}$ preserves connectivity.
\end{lemma}
\begin{proof}
Suppose $X$ is a $k$-connective tame space.
By Corollary \ref{Cor of Barr-Beck-Lurie theorem}, 
$RC_{\tame} X$ is given by the totalization 
$$
RC_{\tame} X \simeq \Tot (\Omega^{\infty} \Sigma^{\infty}_{\tame})^{\bullet +1}  X.
$$
Let $Q:= \Omega^{\infty} \Sigma^{\infty}_{\tame}$. Recall that $\Tot (\Omega^{\infty} \Sigma^{\infty}_{\tame})^{\bullet +1}  X$ can be written as the inverse limit 
$$
\lim_n \Tot^n Q^{\bullet + 1} X,
$$
where $\Tot^n Q^{\bullet + 1} X$ denotes the limit of the diagram over $Q^{\bullet + 1} X|_{\Delta_{\leq n}}$.
Let $T$ be a set with $n$ elements and $\CP(T)$ be the poset of subsets of $T$. 
We define a diagram $\CF$ over $\CP(T)$ by
$$
\CF(S):= Q^{|T-S| + 1} X.
$$
By the appendix of \cite{Arone-Kankaanrinta98},
$$
\fib(\Tot^{n} Q^{\bullet + 1}X \to 
\Tot^{n-1} Q^{\bullet + 1}X) 
\simeq 
\Omega^{n-1}F_n(X)
$$
is $nk$-connective. Hence, the lemma follows from the Milnor exact sequence
$$
0 \to {\lim}^{1}_n \pi_{m+1}\Tot^{n} Q^{\bullet + 1}X \to \pi_m \Tot Q^{\bullet + 1}X 
\to 
\lim_n \pi_m \Tot^n Q^{\bullet + 1}X
\to 
0.
$$ 
% Then for any $r \leq l<k$, we have
% \begin{align*}
%     \pi_l \Tot (\Omega^{\infty} \Sigma^{\infty}_{\tame})^{\bullet +1}  X 
%     & \cong  
%     \pi_0
%     \map_{\Space_{*}}(S^l, \Tot (\Omega^{\infty} \Sigma^{\infty}_{\tame})^{\bullet +1}  X)\\
%     & \simeq 
%     \
% \end{align*}
  
% \Tot \map_{\CS^{\geq r}_{\tame}}(S^l,  (\Omega^{\infty} \Sigma^{\infty}_{\tame})^{\bullet +1}  X) \simeq *

% since $\Omega^{\infty}\Sigma^{\infty}_{\tame}$ preserves connectivity.

\end{proof}

\begin{lemma}
\label{RV commutes infinite products of EM-spaces}
If $F:\CS^{\geq r}_* \to \CS^{\geq r}_*$ preserves connectivity and finite products, then 
$$
F(\prod_n K(A_n, n)) \simeq \prod_n F(K(A_n, n))
$$
where $\{A_n\}_{n\leq r}$ is a sequence of abelian groups.
\end{lemma}
\begin{proof}
It suffices to check the canonical map 
$$
F(\prod_n K(A_n, n)) \to \prod_n F(K(A_n, n))
$$ 
is an equivalence after $k$-truncation $\tau_{\leq k}$ for every $k\geq r$. Since $F$ preserves connectivity and finite products, there is an equivalence after $k$-truncation
$$
F(\prod_{n=r}^{k} K(A_n, n)) \xrightarrow{\simeq} \prod_{n=r}^{k}F( K(A_n,n)).
$$
\end{proof}

% We are now ready to state and prove the second main theorem in this thesis.
\begin{theorem}
	The functor $G_{\tame}: \Grp(\CS^{\geq r-1}_{\tame})\to \Hopfalgebra(\Sp^{\geq r-1}_{\tame})$ (cf. (\ref{G_tame})) is an equivalence of $\infty$-categories.
\end{theorem}
\begin{proof}
First, we prove the functor $G_{\tame}$ is fully faithful.
Since both $C_{tame}$ and $R$ preserve finite products, they lift to a pair of adjunction
\begin{equation}
\label{adj on grps}
	\adj{G_{tame}}{\Grp(\CS^{\geq r-1}_{\tame})}{\Hopfalgebra(\Sp^{\geq r-1}_{\tame})}{R'}.
\end{equation}
We observe that there are commutative diagrams of $\infty$-categories
\[
\begin{tikzcd}
	\Grp(\CS^{\geq r-1}_{\tame}) & \Hopfalgebra(\Sp^{\geq r-1}_{\tame})\\
	\CS^{\geq r-1}_{\tame}  & 
	\coCAlg(\Sp^{\geq r-1}_{\tame})
	\arrow[from=1-1, to= 1-2, "G_{tame}"]
%	\arrow[from=1-2, to= 1-1, shift left, "R"]
	\arrow[from=1-1, to=2-1, "\oblv" left]
	\arrow[from=1-2, to=2-2, "\oblv'" ]
	\arrow[from=2-1, to= 2-2, "C_{tame}"]
%	\arrow[from=2-2, to= 2-1, shift left, "R'"]
\end{tikzcd}
\]
and 
\[
\begin{tikzcd}
	\Grp(\CS^{\geq r-1}_{\tame}) & \Hopfalgebra(\Sp^{\geq r-1}_{\tame})\\
	\CS^{\geq r-1}_{\tame}  & 
	\coCAlg(\Sp^{\geq r-1}_{\tame})
%	\arrow[from=1-1, to= 1-2, "G_{tame}"]
	\arrow[from=1-2, to= 1-1, "R'"]
	\arrow[from=1-1, to=2-1, "\oblv" left]
	\arrow[from=1-2, to=2-2, "\oblv'" ]
%	\arrow[from=2-1, to= 2-2, "G_{tame}"]
	\arrow[from=2-2, to= 2-1, "R"]
\end{tikzcd}
\]
where $\oblv$ and $\oblv'$ denote the forgetful functors from the category of groups in $\CS^{\geq r-1}_{\tame}$ and from $\Hopfalgebra(\Sp^{\geq r-1}_{\tame})$ to the underlying categories of coalgebras.

We want to show the unit map $X \to R' G_{\tame}X$ is an equivalence for any $X\in \Grp(\CS^{\geq r-1}_{\tame})$.
Since the functor $\oblv$ is conservative by Lemma \ref{conservativity of forgetful functor}, it suffices to show the map $\oblv(X) \to \oblv(R'G_{\tame}X)$ is an equivalence.
By the commutativity of the diagrams above, 
$$
\oblv(R'G_{\tame}X) \simeq R \circ \oblv'(G_{\tame}X)\simeq RC_{\tame}(\oblv(X)).
$$
	Note that Proposition \ref{tame recognition theorem} implies that any  $(r-1)$-tame group $X$ is equivalent to the loop space of a $r$-tame space $Y$. Since $X\simeq \Omega Y$ is in particular an $\E_1$-space, it splits into a product of Eilenberg-Maclane spaces by Proposition \ref{Decomp of tame H-spaces}. Hence, 
		\begin{align*}
		RC_{\tame}(\oblv(X)) & \simeq RC_{\tame}(\prod_{i} K(\pi_i X, i)))\\
		& \simeq \prod_{i} RC_{\tame}(K(\pi_i X, i))\\
		& \simeq \prod K(\pi_i X, i)\\
		& \simeq X
	\end{align*}
	where the second equivalence follows from Lemma \ref{RV commutes infinite products of EM-spaces}.
	Hence, we conclude that the functor $G_{\tame}$ is fully faithful.
	
	To finish the proof, it suffices to show the right adjoint $R'$ is conservative. 
	Since $\oblv'$ is conservative by Corollary \ref{conservativity of the forgetful functor for Hopf algebras}, we are reduced to showing that the functor 
	\[
	R: \coCAlg(\Sp^{\geq r-1}_{\tame}) \to 
	\CS^{\geq r-1}_{\tame}
	\]
	is conservative.
	
	Let $f:U\to V$ be a morphism in $\coCAlg(\Sp^{\geq r-1}_{\tame})$ and suppose that $R(f): RU\to RV$ is an equivalence.
	For any coalgebra $X\in \coCAlg(\Sp^{\geq r-1}_{\tame})$, we claim that the induced map on the mapping space
	$$
	\map_{\coCAlg(\Sp^{\geq r-1}_{\tame})}(X, U)\to 
	\map_{\coCAlg(\Sp^{\geq r-1}_{\tame})}(X, V)
	$$
	is an equivalence, which implies $f$ is an equivalence.
	Note that if $X$ lies in the essential image of $C_{tame}$, then this certainly holds. Hence, it suffices to show that the functor
	$$
	C_{tame}: \CS^{\geq r-1}_{\tame}  
	\to 
	\coCAlg(\Sp^{\geq r-1}_{\tame})
	$$
	is essentially surjective, which we will prove in Lemma \ref{Essential Surjectivity of C_tame}.
	
\end{proof}

% Note that the composite $L_{\operatorname{tame}}\circ \Sigma^{\infty}: \Space_{*}^{\geq r-1}\to \Sp^{\geq r-1}_{\tame}$ factorizes as $C'_{tame}: \Space_{*}^{\geq r-1}\to \coCAlg(\Sp^{\geq r-1}_{\tame})$ followed by a forgetful functor. 
The crux of the essential surjectivity argument for $C_{\tame}$ is the following lemma.
\begin{lemma}
	\label{Sigma S^n is a trivial coalgebra}
	$C_{\tame} L_{\tame}S^{r-1}$ is equivalent to the trivial tame coalgebra on $\Sigma^{r-1}H\BZ$.
\end{lemma}
\begin{proof}
    Since the coalgebra $C_{\tame} L_{\tame}S^{r-1}$ is obtained by applying tame localization to the coalgebra
	$Y:=\Sigma^{\infty}L_{\tame}S^{r-1}$ in $\coCAlg(\Sp)$, 
	it suffices to prove $Y$ is a trivial coalgebra after tame localization.
	
	By Proposition \ref{inductive construction of coalgebras}, we can build a commutative coalgebra $X$ in $\Sp$ by assembling compatible coalgebra structures of $X$ in $\coAlg_{\varphi^n \Com}(\Sp)$ for each $n$. Moreover, by Proposition \ref{Operad as a colimits} there is an equivalence 
	$$
	\colim_{n} L_{\tame} F_{\varphi^{n}\Com} \simeq L_{\tame}F_{\Com}.
	$$ 
	It suffices to prove by induction that, $C_{\tame}L_{\tame}S^{r-1}$ is a trivial $L_{\tame}  F_{\varphi^n\Com}$-coalgebra for each $n\geq 1$. 
	The case for $n=1$ is obvious since $\varphi^1\Com$ is the trivial operad.
	
	Assume $L_{\tame} Y$ is a trivial $(L_{\tame}F_{\varphi^{n-1} \Com})$-coalgebra, i.e., the structure map of $Y$ as an $(\varphi^{n-1} \Com)$-coalgebra becomes trivial after apply $L_{\tame}$.
	By Proposition \ref{inductive construction of coalgebras} and the vanishing of the Tate construction in tame spectra, specifying a $L_{\tame}F_{\varphi^{n} \Com}$-coalgebra structure on $C_{\tame}L_{\tame}S^{r-1}$ is equivalent to a lift in the following diagram
		\[
	\begin{tikzcd}
		& L_{\tame}(Y \otimes \cdots \otimes Y)_{h\Sigma_n}\\
		 L_{\tame}Y & L_{\tame}(\varphi^{n-1}\Com(n)\otimes Y \otimes \cdots \otimes Y)_{h\Sigma_n}
		\arrow[from = 2-1, to = 1-2, dashed]
		\arrow[from = 2-1, to = 2-2, "0" below]
		\arrow[from = 1-2, to = 2-2]
	\end{tikzcd}
	\]
	where the vertical map is induced from the map of operads $\Com\to \varphi^{n-1}\Com$.
% 	\[
% 	\begin{tikzcd}
% 		& L_{\tame}(\Sigma^{\infty}_{\tame}L_{\tame}S^{r-1} \otimes \cdots \otimes \Sigma^{\infty}_{\tame}L_{\tame}S^{r-1})_{h\Sigma_n}\\
% 		\Sigma^{\infty}_{\tame}L_{\tame}S^{r-1} & L_{\tame}(\varphi^{n-1}\Com(n)\otimes \Sigma^{\infty}_{\tame}L_{\tame}S^{r-1} \otimes \cdots \otimes\Sigma^{\infty}_{\tame}L_{\tame}S^{r-1})_{h\Sigma_n}
% 		\arrow[from = 2-1, to = 1-2, dashed]
% 		\arrow[from = 2-1, to = 2-2, "0" below]
% 		\arrow[from = 1-2, to = 2-2].
% 	\end{tikzcd}
% 	\]
	Let $F$ denote the fiber of the vertical map. We claim that the connectivity of $F$ is at least $r$, hence any lift is null-homotopic and has to be the trivial map.
	The connectivity of $L_{\tame}(Y \otimes \cdots \otimes Y)_{h\Sigma_n}$ is at least $n(r-1)$, which is larger than $r+1$ (recall $r\geq 4$). 
% 	Since the possible largest dimension of a non-degenerate simplex in $\varphi^{n-1}\Com(n)$ is $n-3$ , 
    The connectivity of 
	$$
    L_{\tame}(\varphi^{n-1}\Com(n)\otimes Y \otimes \cdots \otimes Y)_{h\Sigma_n}
	$$
	is at least $n(r-1)-(n-3) > r+1$ (cf. \cite[Proposition 4.10 and Example 4.7]{Heuts_Goodwillie}). Hence the connectivity of $F$ is larger than $r$, and the lemma is proved.
	
\end{proof}

\begin{remark}
	The proof of Lemma \ref{Sigma S^n is a trivial coalgebra} is almost identical to the proof of \cite[Lemma 6.17]{Heuts_Goodwillie}. 
	There Heuts shows that $\Sigma^{\infty}S^{r-1}$ is a trivial coalgebra in the $\infty$-category $\coAlg^{\nil, \divpow}_{\Com}(\tau_{p-1}\Sp^{\geq r-1})$.
	Informally, $\coAlg^{\nil, \divpow}_{\Com}(\tau_{p-1}\Sp^{\geq r-1})$ consists of conilpotent, divided power coalgebras $X$ in $\Sp$ with $(p-1)!$ inverted in $\Sp$ and with coherent structure maps $X\to (X^{\otimes k})_{h\Sigma_k}$ for $1\leq k \leq p-1$. 
	The important ingredients of both proofs are the inductive construction of coalgebras and the vanishing of Tate construction.
\end{remark}

Using Lemma \ref{Sigma S^n is a trivial coalgebra}, we can now prove the essential surjectiveness of the functor $C_{\tame}$.
\begin{lemma}
\label{Essential Surjectivity of C_tame}
	The functor 
	$$
	C_{tame}: \CS^{\geq r-1}_{\tame}  
	\to 
	\coCAlg(\Sp^{\geq r-1}_{\tame})
	$$
	is essentially surjective.
\end{lemma}
\begin{proof}
Recall that Theorem \ref{Ching-Harper's Koszul duality}
gives an equivalence
$$
\tildeCE: \Alg_{\spLie}(\Mod_{H\BZ})^{\geq r-1} \to \coCAlg^{\divpow, \nil}(\Mod_{H\BZ})^{\geq r-1}
$$
of $\infty$-categories. The $\infty$-category $\Alg_{\spLie}(\Mod_{H\BZ})^{\geq r-1}$ is generated by the free Lie algebra $\Free_{\spLie}(\Sigma^{r-1} H\BZ)$ (in $\Mod_{H\BZ}$) under colimits.
 It follows that the $\infty$-category $\coCAlg^{\divpow, \nil}(\Mod_{H\BZ})^{\geq r-1}$ is generated under colimits by the trivial commutative coalgebra $\trivial_{\Com}(\Sigma^{r-1}H \BZ)$, since we have an equivalence of functors
$$
\tildeCE \circ \Free_{\spLie} \simeq \trivial_{\Com}.
$$

Since $\coCAlg(\Sp^{\geq r}_{\tame})$ is
a localization of $\coCAlg^{\divpow, \nil}(\Mod_{H\BZ}^{\geq r-1})$ by Remark \ref{Identification of tame commutative coalgebras in Mod_HZ},
it is generated under colimits by the trivial tame coalgebra 
$$
\trivial_{\Com,\tame}(\Sigma^{r-1}H \BZ),
$$ 
where $\trivial_{\Com,\tame}$ denotes the composite $L_{\tame}\circ \trivial_{\Com}$.
Hence it suffices to show 
$$
C_{\tame} L_{\tame}S^{r-1}\simeq
\trivial_{\Com,tame}(\Sigma^{r-1}H \BZ),
$$ which is the content of Lemma \ref{Sigma S^n is a trivial coalgebra}.

\end{proof}



% Consider the identity map $$
% \id: \oblv_{\tame}\circ\trivial_{\Com,\tame}(\Sigma^{r-1}H\BZ) \to \Sigma^{r-1}H\BZ,
% $$
% which is adjoint to a map of coalgebras
% $$
% \eta:
% \trivial_{\Com,\tame} (\Sigma^{r-1}H\BZ) 
% \to 
% \Sym_{\tame}(\Sigma^{r-1}H\BZ).
% $$
% By Proposition \ref{Prop 4.1.4.}, the tame cofree coalgebra $\Sym_{\tame}(\Sigma^{r-1}H\BZ)$ is equivalent to $C_{\tame}(K(\BZ, r-1))$ but its underlying spectrum is $\Sigma^{r-1}H\BZ$ by Example \ref{C_tame of suspension of Eilenberg-Maclane space}.
% We conclude that $\eta$ is an equivalence by conservativity of the forgetful functor, hence $C_{\tame}(K(\BZ, r-1))\simeq \trivial_{\Com,tame}(\Sigma^{r-1}H \BZ)$.




% We claim that $\Free_{\spLie,\tame}(\Sigma^{r-1} H\BZ)$ is also trivial;
% indeed, note that the monad $L_{\tame}F_{\spLie}$ preserves tame equivalences, hence there is an equivalence 
% $$
% \Free_{\spLie,\tame} (\Sigma^{r-1}H\BZ) \simeq
% \Free_{\spLie, \tame}(\BS^{r-1})
% $$
% but $\Free_{\spLie,\tame}(\BS^{r-1})$ is a trivial tame Lie algebra by Theorem \ref{odd spheres}.
% To conclude the proof, note that there are equivalences
% \begin{align*}
% \trivial_{\Com,tame}(\Sigma^{r-1}H \BZ) & \simeq \tildeCE_{\tame} \circ \Free_{\spLie, \tame} (\Sigma^{r-1}H\BZ) \\
% & \simeq 
% \tildeCE_{\tame}\circ \trivial_{\spLie, \tame} (\Sigma^{r-1}H\BZ) \\
% & \simeq 
% \Sym_{\tame}(\Sigma^{r-1}H\BZ)
% \end{align*}
% and the latter is equivalent to $C_{\tame}(K(\BZ, r-1))$ by Proposition \ref{Prop 4.1.4.}.

% It suffices to show  $\Free_{\BL}(\Sigma^{r-1}H\BZ)$ is trivial. 
% Since the monad $\Free_{\BL}$ 

% Since $S^{r-1}\to K(\BZ,r-1)$ is a tame equivalence, we have that
% $$
% C_{tame}K(\BZ,r-1) \simeq C_{tame} S^{r-1}.
% $$
% The statement now follows from Lemma \ref{Sigma S^n is a trivial coalgebra}.


\section{The equivalence of tame Lie algebras and tame Hopf algebras}
\label{Proof of the first main Theorem}
In the last section of this chapter, we prove Theorem \ref{FIRST_MAIN_THEOREM}.
We first prove an important proposition about the loop of a tame spectral Lie algebra.
Over $\BQ$, if $L$ is a Lie algebra in $\Ch_{\BQ}$, then $\Omega_{\Lie}L$ is a trivial Lie algebra for degree reasons.
We claim the same phenomenon happens in the case of tame spectral Lie algebras.

\begin{proposition}
\label{Triviality of loop of an O-algebra}
	Let $L\in \Alg_{\spLie}(\Sp^{\geq r}_{\tame}) $ be a tame spectral Lie algebra.
	Then there is an equivalence
	$$
	\Omega_{\spLie} L\simeq \trivial_{\spLie}(\Omega L).
	$$ 
\end{proposition}

Assuming Proposition \ref{Triviality of loop of an O-algebra}, we can now prove Theorem \ref{FIRST_MAIN_THEOREM}.
\begin{proof}
[Proof of Theorem \ref{FIRST_MAIN_THEOREM}:]
	First we claim the universal enveloping algebra functor
	\[
	U:
	\Alg_{\spLie}(\Sp^{\geq r}_{\tame}) 
	\xrightarrow{\Omega_{\spLie}}
	\Grp(\Alg_{\spLie}(\Sp^{\geq r-1}_{\tame}))
	\xrightarrow{\Grp(\tildeCE_{\tame})}
	\Hopfalgebra(\Sp^{\geq r-1}_{\tame})
	\]
	is fully faithful. 
% 	Since $\Omega_{\spLie}X$ is also an $(r-2)$-tame spectrum, we can factor $U$ has follows,
% 	\[
% 	\begin{tikzcd}
% 	\Alg_{\Lie}(\Sp^{\geq r-1}_{\text{($r-1$)-tame}})	&    \Grp(\Alg_{\Lie}(\Sp^{\geq r-2}_{\text{($r-2$)-tame}}))
%  &
%   \Hopfalgebra(\Sp^{\geq r-1}_{\text{($r-2$)-tame}})
%     \\
% 		& \Grp(\Alg_{\Lie}(\Sp^{\geq r-2}_{\text{($r-1$)-tame}}))
%   & \Hopfalgebra(\Sp^{\geq r-1}_{\text{($r-1$)-tame}})
%   	\arrow[from=1-1, to=1-2, "\Omega_{\Lie}'"]
%   	\arrow[from=1-2, to=1-3, "\widetilde{\operatorname{CE}}'"]
%   	\arrow[from=1-1, to=2-2, "\Omega_{\Lie}" left]
%   	\arrow[from=1-2, to=2-2]
%   	\arrow[from=1-3, to=2-3]
%   	\arrow[from=2-2, to=2-3, "\widetilde{\operatorname{CE}}"]
% 	\end{tikzcd}
% 	\]
% 	where the two vertical arrows are fully faithful.
% 	We claim the upper horizontal arrow $\widetilde{\operatorname{CE}}'\circ \Omega_{\Lie}'$ is fully faithful.
    Let $\Grp(\widetilde{\Prim})$ denote the right adjoint of
    $\Grp(\tildeCE_{\tame})$. Note that $\Omega_{\spLie}$ is an equivalence, hence it suffices to show the unit map
    $$
    \eta:\Omega_{\spLie} X \to \Grp(\widetilde{\Prim})\circ \Grp(\tildeCE_{\tame}) (\Omega_{\spLie}X)
    $$
    is an equivalence.
    By Proposition \ref{Triviality of loop of an O-algebra}, $\Omega_{\spLie}X\simeq \trivial_{\spLie} \Omega X $.
    Hence $\eta$ can be written as
    $$
    \eta: \trivial_{\spLie} \Omega X \to 
    \Grp(\widetilde{\Prim})\circ \Grp(\tildeCE_{\tame}) (\trivial_{\spLie} \Omega X).
    $$
    The claim then follows from the fact that the unit from the Koszul duality adjunction $\tildeCE_{\tame}$-$\widetilde{\Prim}$ is an equivalence on trivial algebras, which is due to the following formal equivalences
    $$
    \tildeCE_{\tame} \circ \trivial_{\spLie} \simeq \Sym_{\tame}\Sigma
    \quad
    \widetilde{\Prim} \circ \Sym_{\tame} \simeq \trivial_{\spLie}\Omega
    $$
    by Lemma \ref{Lemma 3.6.20}.
    This completes the proof of the fully faithfulness of $U$.
%     and let $B_{\spLie}$ be an inverse to $\Omega_{\spLie}$ by Proposition \ref{B and Omega are mutally inverses}. Consider the unit map
% 	$$
% 	X \to B_{\spLie}\circ \widetilde{\Prim} \circ \Grp(\widetilde{\operatorname{CE}_{\tame}})\circ \Omega_{\spLie}(X)
% 	\simeq
% 	B_{\spLie}\circ
% 	\widetilde{\Prim} \circ\Grp(\widetilde{\operatorname{CE}} ) \circ \trivial_{\spLie}(\Omega X)
% 	$$
% where the latter equivalence follows from 
% Apply $\Omega_{\spLie}$ and use the fact that the forgetful functor $\oblv_{\spLie}$ is conservative, we are reduced to check the natural map
% \[
% \eta:  \Omega X
% \to 
% \oblv_{\spLie}\circ 
% \widetilde{\Prim} \circ \widetilde{\operatorname{CE}}\circ \trivial_{\spLie}(\Omega X)
% \]
% is an equivalence.
% %, this is equivalent to show
% %	the unit map	
% %	
% %	is an equivalence for any $(r-1)$-tame Lie lagebra $X$.
% 	By \cite{Francis-Gaitsgory}[Lemma 3.3.4.], we have a natural equivalence 
% 	$$\widetilde{\operatorname{CE}}\circ \trivial_{\spLie}\circ \Omega (X) 
% 	\simeq 
% 	\cofree(\Omega X);
% 	$$ 
% 	and the map $\eta$ now reads as 
% 	$$
% 	\eta: \Omega X \to \widetilde{\Prim} \circ \cofree (\Omega X),
% 	$$
% 	which is an equivalence is the natural transformation $\id \to \widetilde{\Prim} \circ \cofree$ is an equivalence. 

	
	
% 	apply the functor $\oblv_{\spLie} \circ \widetilde{\Prim}$ on both sides, we have
% 	$$
% 	\oblv_{\spLie} \circ \widetilde{\Prim}\circ \widetilde{\operatorname{CE}} \circ 
% 	\trivial_{\spLie}(\Omega X) \simeq   	\oblv_{\spLie} \circ \widetilde{\Prim}\circ \cofree(\Omega X)
% 	\simeq  \Omega X
% 	$$
    For essential surjectivity, note first that  $U$ preserves colimits, as it is a composition of an equivalence $\Omega_{\spLie}$ and a left adjoint $\Grp(\tildeCE_{\tame})$. 
    Since we have an equivalence of $\infty$-categories 
    $$\Hopfalgebra(\Sp^{\geq r-1}_{\tame})
    \simeq
    \CS^{\geq r}_{tame}
    $$  by Theorem \ref{2nd Main Theorem},
    it suffices to show that the Hopf algebra $H$ corresponding to the generator $K(\BZ,r)$ of $\CS^{\geq r}_{tame}$, lies in the essential image of $U$. 
    Observe that the underlying $(r-1)$-tame spectrum of $H$ is 
    $$
    \Sigma^{\infty}_{\operatorname{tame}}\Omega K(\BZ,r)
    \simeq 
    \Sigma^{\infty}_{\operatorname{tame}} K(\BZ, r-1)
    \simeq \Sym_{\tame}(\Sigma^{r-1}H\BZ),
    $$ 
    where the second equivalence follows from Proposition \ref{Prop 4.1.4.}.
    Moreover, for the trivial $r$-tame Lie algebra $\trivial_{\spLie}(\Sigma^{r}H\BZ)$,
    \begin{align*}
        \tildeCE_{\tame} \circ  \Omega_{\spLie}(\trivial_{\spLie}(\Sigma^{r}H\BZ))
        & \simeq 
        \tildeCE_{\tame}\circ \trivial_{\spLie} (\Sigma^{r-1}H\BZ)\\
        & \simeq \Sym_{\tame}(\Sigma^{r-1}H\BZ),
    \end{align*}
    hence we see that $H$ is the image of $\trivial_{\spLie}(\Sigma^{r}H\BZ)$ under $U$.
    
\end{proof}

We conclude this chapter with the proof of Proposition \ref{Triviality of loop of an O-algebra}.
We learned this proof from Heuts. 
Recall that there is a pair of adjoint equivalences of $\infty$-categories by Lemma \ref{Shift has no harm} 
$$
\adj{\Sigma'}{\Alg_{\Omega \BL \Sigma}(\Sp) }{\Alg_{\BL}(\Sp)}{\Omega'}.
$$
By Lemma \ref{transition of triviality},
it suffices to show $\Omega_{\BL} X$ is a trivial $\BL$-algebra for any $X\in \Alg_{\BL}(\Sp^{\geq r})$ after tame localization.

Let $\sigma$ denote the \emph{suspension morphism} in (\ref{suspension morphism})
\[
F_{\BL} \xrightarrow{\sigma} F_{\Omega \BL \Sigma},
\]
or equivalently, this can be obtained via the induced map on Goodwillie derivatives
$\partial_{*}\id \to \partial_* \Omega \Sigma$.
The restriction along $\sigma$ fits in a factorization
\[
\begin{tikzcd}
	\Alg_{\BL}(\Sp) &   & \Alg_{\BL}(\Sp) \\
	&  \Alg_{\Omega \BL \Sigma}(\Sp) &
	\arrow[from=1-1, to = 1-3, "\Omega_{\BL}"]
	\arrow[from=1-1 , to =2-2, "\Omega'" below]
	\arrow[from=2-2, to=1-3, "\sigma^*" below].
\end{tikzcd}
\]
We claim $\sigma:F_{\BL}\to F_{\Omega \BL \Sigma}$ factors through the identity monad as the augmentation of $F_{\BL}$ followed by the unit of $F_{\Omega \BL \Sigma}$, after tame localization, which would complete the proof of Proposition \ref{Triviality of loop of an O-algebra},
since the restriction along $F_\BL \to \id$ is indeed the trivial Lie algebra functor. 
% For convenience, we abuse notation by writing $\BS^k$ for $L_{\tame}\BS^k$ for $k\geq r$.

The $\infty$-category $\Sp$ is generated under sifted colimits by wedge sum of shifted spheres $\BS^k$.
Therefore, it suffices to prove the claim when evaluating $\sigma$ at a wedge of spheres
$\BS^{k_1}\oplus \cdots \oplus \BS^{k_n}$.
That is, we want to show 
$$
\sigma_{\BS^{k_1}\oplus \cdots \oplus \BS^{k_n}}: 
\BL(\BS^{k_1}\oplus \cdots \oplus \BS^{k_n})
\to 
\Omega \BL \Sigma (\BS^{k_1}\oplus \cdots \oplus \BS^{k_n}).
$$
factors as the coaugmentation of $\BL$ follows by the augmentation of $\Omega \BL \Sigma$ on $\BS^{k_1}\oplus \cdots \oplus \BS^{k_n}$ after tame localization.
\begin{theorem}
    [Hilton-Milnor Theorem, \cite{Brantner-Heuts}, \cite{Arone-Kankaanrinta98}]
    For any collection of spheres $\BS^{k_1}, \dots, \BS^{k_n}$, there is an equivalence
    $$
    \Omega\BL\Sigma(\BS^{k_1}\oplus \cdots \oplus \BS^{k_n})
    \to 
    \sideset{}{'}\prod_{\omega \in \Lie_n} \Omega \BL (\Sigma \omega(\BS^{k_1 }, \cdots, \BS^{k_n })),
    $$
    where $\prod'$ denotes the weak product (filtered colimits of finite products) and $\Lie_n$ denotes the ordered set of Lie words with $n$ generators, i.e., every $\omega\in \Lie_n$ is a basis element of the free Lie algebra on $n$ generators.
\end{theorem}

Using the Hilton-Milnor theorem, we can describe the suspension morphism $\sigma$ on the underlying spectra as
$$
\sigma: 
\sideset{}{'}\prod_{w \in \Lie_n} \BL(\Sigma \omega (\BS^{k_1 -1}, \cdots, \BS^{k_n -1}))
\to 
\sideset{}{'}\prod_{w \in \Lie_n} \Omega \BL(\Sigma \omega (\BS^{k_1 }, \cdots, \BS^{k_n })).
$$
Since $\sigma$ comes from a map of spectral Lie algebras, it should preserve components indexed by the same Lie word.
We claim that $\sigma$ restricts to a null map on components corresponding to those Lie words $\omega$ with length $n\geq 2$. 
Indeed, write $F:\Sp^n \to \Sp$ for the multivariable functor
$
\BL (\Sigma \omega(-, \cdots, -)).
$
Then we can identify the suspension morphism $\sigma$ as 
$$
F(\BS^{k_1 -1}, \cdots, \BS^{k_n -1})
\to 
\Omega F(\Sigma \BS^{k_1 -1}, \cdots, \Sigma\BS^{k_n -1}).
$$
Observe that this map can be further factored by first applying the suspension morphism component wise, then applying the ``diagonal embedding'' $\BS^{-n}\to \BS^{-1}$. This can be summarized as the diagram below
\[
\begin{tikzcd}
     F(\BS^{k_1 -1}, \cdots, \BS^{k_n -1}) & \Omega F(\Sigma \BS^{k_1 -1}, \cdots, \Sigma\BS^{k_n -1})\\
     &     \Omega^{n} F(\Sigma \BS^{k_1 -1}, \cdots, \Sigma\BS^{k_n -1}).
     \arrow[from = 1-1, to = 1-2, "\sigma"]
     \arrow[from = 1-1, to = 2-2]
     \arrow[from = 2-2, to = 1-2, "0"]
\end{tikzcd}
\]
We remind the reader that the map $\BS^{-n}\to \BS^{-1}$ is obtained from the Spainer-Whitehead dual of the diagonal embedding $\BS^1\to \BS^{n}$, which is null if $n \geq 2$.

Therefore $\sigma$ factors as a map
$$
\bigoplus_{i=1}^{n} \BL(\BS^{k_i}) \to 
\bigoplus_{i=1}^{n} \Omega\BL\Sigma(\BS^{k_i}),
$$
which sends $\BL(S^{k_i})$ to $\Omega\BL \Sigma(\BS^{k_i})$ for every $i$.
It now suffices to show that for each $k$, the suspension map 
$$
\sigma':\BL(\BS^{k}) \to 
\Omega\BL\Sigma(\BS^{k})
$$
factors through $\BS^{k}$ after tame localization, for which we need the following theorem by Arone-Mahowald.

% Let $\Lie_n$ denote the ordered set of Lie words with $n$ generators, i.e. every $w \in \Lie_k$ is a basis element of the free Lie algebra on $n$ generators.
% For $X_1, \dots, X_n \in \Sp$, we define $w(X_1, \cdots, X_2):= X_1\otimes \cdots \otimes X_n$; that is, we let the Lie brackets in $w$ act as the smash products in $\Sp$. 
% The following is a variant of the Hilton-Milnor theorem.
% \todo{Do we need this ?}
% \begin{theorem}
% 	[Hilton-Milnor,  \cite{Arone-Kankaanrinta98}, \cite{Brantner-Heuts}]
% 	For any collection of spheres $\BS^{k_1}, \cdots, \BS^{k_n}$, there is an equivalence
% 	$$
% 	\Omega \BL \Sigma(\BS^{k_1}\oplus \cdots \oplus \BS^{k_n})
% 	\simeq 
% 	\bigoplus_{w\in \Lie_n} \Omega \BL( \Sigma w(\BS^{k_1}, \cdots, \BS^{k_n})).
% 	$$
% \end{theorem}

% Now we can rewrite the suspension morphism $\sigma$ evaluating at the wedge of spheres $\BS^{k_1}\oplus \cdots \oplus \BS^{k_n}$ as 
% $$
% \sigma: 
% \BL(\BS^{k_1}\oplus \cdots \oplus \BS^{k_n})
% \to 
% \bigoplus_{w\in \Lie_n} \Omega \BL (\Sigma w(\BS^{k_1}, \cdots, \BS^{k_n})).
% $$
% Note that any map 
% $$
% \BS^{l} \to 
% \BS^{k}
% $$
% is null-homotopic if $l < k$,
% hence $\sigma$ has to factor through the wedge sum of Lie words of weight one, i.e. we have factorization
% \[
% \begin{tikzcd}
% 	 \bigoplus_{i=1}^{n} \BL(\BS^{k_i}) &   &  \bigoplus_{w\in \Lie_n} \Omega \BL (\Sigma w(\BS^{k_1}, \cdots, \BS^{k_n}))\\
% 	& \bigoplus_{i=1}^{n}\Omega \BL (\Sigma \BS^{k_i})  &
% 	\arrow[from=1-1, to = 1-3, "\sigma"]
% 	\arrow[from=1-1 , to =2-2]
% 	\arrow[from=2-2, to=1-3]
% \end{tikzcd}
% \]
%  factors over $\bigoplus_{i=1}^{n}\Omega \BL (\Sigma \BS^{k_i})$.
%  It now suffices to check the following.
 
 \begin{theorem}
 	\cite[Theorem 3.13, Theorem 4.4]{Arone-Mahowald}
 	\label{odd spheres}
 	Let $X$ denote the $p$-localization of the sphere $\BS^k$ at a prime $p$, then
 	\begin{enumerate}
 		\item if $k$ is odd, then 
 			$$
 			\big(\BL(n)\otimes X^{\otimes n}
 			\big)_{h\Sigma_n}
 			\simeq *.
 			$$
 			if $n\neq p^l$ for some $l$.
 		\item if $k$ is even, then 
 			$$
 			\big(\BL(n)\otimes X^{\otimes n}
 			\big)_{h\Sigma_n}
 			\simeq *.
 			$$
 			if $n$ is not equal to $p^l$ or $2p^j$ for $l, j>0$.
 	\end{enumerate} 	
 \end{theorem}
 \begin{lemma}
 
 	The suspension morphism evaluated on a sphere admits a factorization
\[
\begin{tikzcd}
	 \BL(\BS^{k}) &   &  \Omega \BL (\Sigma \BS^{k})\\
	& \BS^{k}  &
	\arrow[from=1-1, to = 1-3, "\sigma"]
	\arrow[from=1-1 , to =2-2]
	\arrow[from=2-2, to=1-3]
\end{tikzcd}
\]
after tame localization.
\end{lemma}
\begin{proof}
	The lemma follows from Theorem \ref{odd spheres}; indeed, for $n=p^l$ or $n=2p^l$ and $l>0$, the connectivity of
	\[
	L_{\tame}\big(
	\BL(n)\otimes (\BS^{k})^{\otimes n}
	\big)_{h\Sigma_n}
	\]
	is at least $kn-n+1$, which is larger than $r+2p-3$ (note that $k\leq r$), hence 
	$L_{\tame}\big(
	\BL(n)\otimes (\BS^{k})^{\otimes n}
	\big)_{h\Sigma_n}$ is contractible for $n>1$.
	
	\end{proof}

 
 The proof of Theorem \ref{FIRST_MAIN_THEOREM} is now complete.

%\begin{corollary}
%The suspension morphism $\sigma$ factors as 
%	$$
%\BL \to \triv 
%\to 
%\Omega \BL \Sigma
%$$
%after tame localiztion. 
%\end{corollary}
%
%\begin{proof}
%	Since all these three functors are sifted-colimit preserving functors, so it will suffice to check this on objects of the form of wedge of spheres.
%\end{proof}





%\begin{proof}
%	Still working on this proof, and here are some ideas on how to attack this:
%	\begin{itemize}
%		\item Need to define the notion of tensoring an $\CO$-algebra $X$ with a commutative algebra $A$ (this should work in any symmtric monoidal $\infty$-category). That is, a functor
%		\[
%		A \otimes -: \Alg_\CO\to \Alg_\CO
%		\]
%		\item In the rational case, this is indeed true; because $C^*(S^1;\BQ)$ is quasi-isomorphic to the cdga $H^*(S^1;\BQ)$.
%		\item Show $X\otimes A$ is trivial if $A$ is a trivial commutative algebra.
%		\item Show $\Omega L$ is equivalent to $L_{tame}H\BZ^{S^1}\otimes L$ and hence trivial. 
%	\end{itemize}
%\end{proof}
%The spectrum $H\BQ^{S^1}$ models the rational cochain of $S^1$. Moreover, by Sullivan's formality result, we have an equivalence of $\E_\infty$-algebras 
%\[
%H\BQ^{S^1} \simeq H^{*}(S^1; \BQ)
%\]
%in $\Mod_{H\BQ}$. Hence, the rational cochain $H\BQ^{S^1}$ is a trivial $\E_{\infty}$-algebra.
%
%Now to show $L_{tame}H\BZ^{S^1}$ is a trivial $\E_{\infty}$-algebra, we need a formality result for $L_{tame}H\BZ^{S^1}$.
%\begin{question}
%	How is the formality argument proved in the rational case?
%\end{question}
%
%\begin{proposition}
%	There is an equivalence 
%	\[
%	L_{tame}C^{*}(S^1;\BZ) \simeq \tilde{H}^*(S^1; \BZ)
%	\]
%of $\E_\infty$-algebras if we invert primes quick enough to kill all the Steenrod operations.
%\end{proposition}
%
%
%
%
%Maybe one can proceeds as follows:
%One first shows $L_{tame}H\BZ^{S^1}$ is a trivial $\E_{\infty}$-algebra in $\Sp$. For this, we need to show some kind of formality argument:
%the cochain $L_{tame}H\BZ^{S^1}$ is tame equivalent to the cdga $H^*(S^1; \BZ)$ in $\Mod_{\BZ}$.








