\chapter{Hopf Algebra Model}

In this last chapter, we will give the proof second main main theorem stated in the introduction.
%The first theorem is a direct consequence of our discussion in the \textbf{chapter of Koszul duality}. 
%\begin{theorem}
%	The universal enveloping algebra functor
%	$$
%	U: \Alg_{\Lie}(\Sp^{\geq r-1}_{\text{($r-1$)-tame}})
%	\to
%	\Hopfalgebra(\Sp^{\geq r-1}_{\text{($r-1$)-tame}})
%	$$
%	is an equivalence of $\infty$-categories.
%\end{theorem}
%\begin{remark}
%\begin{enumerate}
%	\item Here $\Hopfalgebra(\Sp^{\geq r-1}_{\text{($r-1$)-tame}})$ denotes the $\infty$-category of Hopf algebra objects in $(r-1)$-tame spectra as defined in \todo{previous section}. 
%	\item By Proposition \ref{Identification of tame Lie algebras}, the $\infty$-category $\Alg_{\Lie}(\Sp^{\geq r-1}_{\text{($r-1$)-tame}})$ of tame Lie algebras can be identified with the full subcategory of $\Alg_{\Lie}(\Mod_{H\BZ})^{\geq r-1}$ spanned by Lie algebras whose underlying $H\BZ$-modules are tame.
%\end{enumerate}
%\end{remark}
Our second result concerns a new Hopf algebra model for $r$-tame spaces. 
\begin{theorem}
\label{2nd Main Theorem}
The following composition 
$$
\CS^{\geq r}_{tame} \xrightarrow{\Omega}
\Grp(\CS^{\geq r-1}_{\text{($r-1$)-tame}})
\xrightarrow{\Grp(C_{tame})}
\Hopfalgebra(\Sp^{\geq r-1}_{\text{($r-1$)-tame}})
$$
is an equivalence of $\infty$-categories.
\end{theorem}
Combinging Theorem \ref{first main theorem} and Theorem \ref{2nd Main Theorem}, we reproduce Dwyer's Lie algebra model for tame spaces in \cite{Dwyer}.
\begin{theorem}
\cite{Dwyer}
	There is an equivalence of $\infty$-categories
	\[
	\CS^{\geq r}_{tame} 
	\to
	\Alg_{\Lie}(\Sp^{\geq r-1}_{\text{($r-1$)-tame}}).
	\]
\end{theorem}

We now sketch our strategy of the proof of Theorem \ref{2nd Main Theorem}.
We originally thought that there is a coalgebra model for tame spaces, i.e. we suspected the functor
	$$
	C_{\operatorname{tame}}:\Space^{\geq r}_{tame}\to \coCAlg(\Sp_{tame})
	$$
	is an equiavlence. However, this functor fails to be fully faithful; we will explain this failure later in this chapter.
	Instead, we prove that $C_{\operatorname{tame}}$ is fully faithful when restricted to the full subcategory of tame Eilenberg-Maclane spaces.
We then show both the functors $\Omega$ and $\Grp(C_{\operatorname{tame}})$ are equivalences of $\infty$-categories. 
The crux of the latter being an equivalence is that the loop of an $r$-tame space splits into a product of Eilenberg-Maclane spaces, and therefore we can conclude $\Grp(C_{\operatorname{tame}})$ is fully faithful.


In section \ref{Section: Fully faithfulness on Eilenberg-Maclane spaces}, we prove the functor $C_{\operatorname{tame}}$ is fully faithful on the full subcategory of Eilenberg-Maclane spaces by computing the cohomology of Eilenberg-Maclane spaces with coefficients in the tame ring system.
In section \ref{proof of the 2nd main theorem}, we finalize our proof of Theorem \ref{2nd Main Theorem}.

%\section{Proof of Theorem \ref{first main theorem}}


\section{Fully faithfulness on Eilenberg-Maclane spaces}\label{Section: Fully faithfulness on Eilenberg-Maclane spaces}

Our goal in this section is to prove the following.
\begin{proposition}
\label{Fully faithfulness of C_tame on EM spaces}
	The functor
	$$
	C_{\operatorname{tame}}:\Space^{\geq r}_{tame}\to \coCAlg(\Sp_{tame})
	$$
restricts to an equivalence on the full subcategory spanned by Eilenberg-Maclane spaces.
\end{proposition}

%Note that $C_{\operatorname{tame}}$ admits a right adjoint $R$, where $X$ should be the monoidal unit for the symmetric monoidal structure on $\Sp_{tame}$ and it is equipped with the trivial coalgebra structure. 

Let $V$ be a $R_{k}$-module. We will show that $C_{tame}(K(V,r+k))$ is equivalent to the cofree commutative coalgebra generated by $\Sigma^{r+k}HV$.
Hence, it suffices to check the following map:
$$
\Sigma_{tame}^{\infty}K(V,r+k)\to \cofree \Sigma^{r+k}HV
$$
is an equivalence. Since both $\Sigma^{\infty}_{tame}\Omega^{\infty}$ and $\cofree_{tame}$ preserves filtered colimits, we are reduced to the case of $V$ being a finitely generated $R_k$-module. Since $R_k$ is a PID, by the structure theorem of finitely generated modules over PIDs, we are further reduced to check the cases $V=R_k$ and $V=\BZ/p^m$ where $p$ is a prime not inverted in $R_k$. 



\begin{lemma}
	For $V=R_k$, we have
\begin{enumerate}
	\item If $r+k$ is odd,
	 $$
	 \pi_* \Sigma_{tame}^{\infty}K(V,r+k) =
	 \begin{cases}
		R_k & \text{if $*=r+k$,}\\
		0 & \text{otherwise}.
	 \end{cases}
	 $$
	\item If $r+k$ is even,
	$$
	 \pi_* \Sigma_{tame}^{\infty}K(V,r+k) =
	 \begin{cases}
		R_{n(r+k)-r} & \text{if $*=n(r+k)$,}\\
		0 & \text{otherwise}.
	 \end{cases}
	 $$
\end{enumerate}
\end{lemma}
\begin{proof}
Observe that $K(V,r+k)$ is the $R_k$-localization of $K(\BZ,r+k)$ and combine with the remark above, we see that 
	\begin{align*}
		\pi_* \Sigma_{tame}^{\infty}K(V,r+k) & = H_* K(V,r+k)\otimes R_{*-r}\\
		& = H_* K(\BZ,r+k)\otimes R_{*-r}.
	\end{align*}
	Recall that the first $p$-torsion in the homology groups of $K(\BZ,r+k)$ is at degree $2p-2+r+k$, but $R_{k+2p-2}$ already contains $\frac{1}{p}$. Hence 
	$$
	H_* K(\BZ,r+k)\otimes R_{*-r} = \big(H_* K(\BZ,r+k)/\text{torsions}\big) \otimes R_{*-r}
	$$ and the result follows from the classical calculation of $H_* K(\BZ,r+k)/\text{torsions}$ (e.g. see Hatcher's book on spectral sequence Proposition 1.20). 
\end{proof}

We now consider the case for $V=\BZ/p$ where $p$ is a prime not inverted in $R_k$.
We want to compute the integral cohomology of $K(\BZ/p,n)$.
The $\BF_p$-cohomology of $K(\BZ/p,n)$ is known classically, as the free $\BF_p$-algebra generated by the set 
$$
\{
St^I(\iota_n)=\beta^{\epsilon_0}P^{s_1}\cdots \beta^{\epsilon_m}P^{\epsilon_m}(\iota_n) |
e(I)<n 
\}
$$
where $\iota_n$ is the fundamental class of $K(\BZ/p,n)$ and $e(I)$ is the \emph{excess} computed as
\[
e(I):=2s_1p+2\epsilon_0-\abs{I}.
\]
Note that the Steenrod operation $P^i$ has degree $2i(p-1)$ and the prime $p$ is inverted in $R_{2p-3}$, hence the generators of the form
$$
P^i(\iota_n), \text{ for } i\geq 1
$$  
live in degrees where $p$ has been inverted.
\begin{corollary}
\label{F_p tame cohomology of K(Z/p,n)}
	The cohomology $H^*(K(\BZ/p,n);\BF_p)\otimes R_{*-r}$ of the Eilenberg-Maclane space with coefficient in the tame ring system is the truncated polynomial algebra generated by 
	$\iota_n$ and $\beta(\iota_n)$, with range in degree $[n, r+2p-4]$.
\end{corollary}

\begin{proposition}
[\cite{Hatcher} Theorem 3E.3]
	Let $H_{n}(X;\BZ)$ be finitely generated for all $n$, then each $\BZ/p^{k}$ summand of $H^n(X;\BZ)$ with $k>1$ contributes $\BZ/p$ summands to both $BH^{n-1}(X;\BF_p)$ and $BH^{n}(X;\BF_p)$, where $BH^n(X;\BF_p)$ denotes the $n$-th Bockstein cohomology group.
\end{proposition}

\begin{theorem}
	Let $p$ be a prime.
	The integral cohomology group 
	$H^*K(\BZ/p,n)$ doesn't have summand $\BZ/p^k$ for $k>1$ in degree less than $np$.
\end{theorem}

We claim the integral cohomology $H^*(K(\BZ/p,n);\BZ)\otimes R_{*-r}$ admits no summand of $\BZ/p^k$ for $k>1$; we show the Bockstein cohomology groups $BH^*(K(\BZ/p,n);\BF_p)$ are trivial in degree $[n, r+2p-4]$.
By Corollary \ref{F_p tame cohomology of K(Z/p,n)}, we only need to consider the differential graded algebra generated by the classes $\iota_n$ and $\beta(\iota_n)$. 

When $n$ is odd, $BH^*(K(\BZ/p,n);\BZ)\otimes R_{*-r}$ is a truncated $\BF_{p}$-algebra of the DGA
$$
\Lambda_{\BF_p}(\iota_n)\otimes \BF_p[\beta(\iota_n)]
$$
with differential $d(\iota_n)=\beta(\iota_n)$ and $d(\beta(\iota_n))=\beta^2(\iota_n)=0$. Hence $\Lambda_{\BF_p}(\iota_n)\otimes \BF_p[\beta(\iota_n)]$ is acyclic.

When $n$ is even, $H^*(K(\BZ/p,n);\BZ)\otimes R_{*-r}$ is a truncated $F_{p}$-algebra of the DGA
$$
 \BF_p[\iota_n]\otimes \Lambda_{\BF_p}(\beta(\iota_n))
$$
with the same differential as in the odd case. We claim that $BH^j(K(\BZ/p,n);\BF_p)$ is trivial for $j<pn$; this follows from the fact that $d\iota_n^k=k\iota_n^{k-1}$, and so it is an isomorphism for $k<p$ and is zero for $k=p$. Recall that we are only interested in spaces which are at least $3$-connective, hence the Bockstein cohomology groups $BH^j(K(\BZ/p,n);\BF_p)$ are trivial within the tame range.

Now, we can slice the long exact sequence 
\[
\cdots \to 
H^{n}(X;\BZ)\xrightarrow{\cdot p} H^{n}(X;\BZ)
\xrightarrow{\rho}
H^{n}(X;\BZ/p)\xrightarrow{\tilde{\beta}}
H^{n+1}(X;\BZ)
\to
\cdots
\]
into short exact sequences 
\[
0 \to H^{m}(X;\BZ)
\xrightarrow{\rho}
H^{m}(X;\BZ/p)
\xrightarrow{\tilde{\beta}}
H^{m+1}(X;\BZ)
\to 
0
\]
for $m<np$.
Observe that $\rho$ is injective, and we have 
$$
\operatorname{Im}\rho = \operatorname{Im}(\beta:H^{m-1}(X;\BZ/p)\to H^{m}(X;\BZ/p)).$$ 
Hence we conclude the following.
\begin{corollary}
	\[
	H^{*}(K(\BZ/p,n);\BZ)\otimes R_{*-r}\cong
	\begin{cases}
		$R_{k-r}\otimes \BZ/p$,  & \text{ for $k$ a multiple of $n+1$ in $[r, r+2p-3]$;} \\
		$0$,        &\text{ otherwise.}
	\end{cases}
	\]
\end{corollary}

\begin{corollary}
	There is an isomorphism of graded $\BF_p$-vector spaces
	\[
	\tau_{\leq r+2p-3}H^{*}( K(\BZ/p,n);\BF_p) 
	\to
	\tau_{\leq r+2p-3} (H\BF_p)^*\Sigma^nH\BF_p.
	\]
	
\end{corollary}


For $p$ an odd prime and $I=(\epsilon_1, s_1, \dots, \epsilon_k, s_k)$ such that $\epsilon_j\in \{0,1\}$.
We say $I$ is \emph{admissible} if $s_i\geq ps_{i+1}\epsilon_i$ for $1\leq i <k$. We write 
$$
P^I:=\beta^{\epsilon_0}P^{s_1}\beta^{\epsilon_1}\cdots P^{s_k}\beta^{\epsilon_k}.
$$
\begin{theorem}
	[??]\todo{ref?}
	The admissible monomials $P^I\in \CA^{*}$ form basis for the mod-$p$ Steenrod algebra $\CA^*$.
\end{theorem}


Now observe that both $H^{*}(K(\BZ/p,n);\BF_p)$ and $H_{*}(\Sym \Sigma^nH\BF_p;\BF_p)$ are isomorphic as DGAs.
We can then run the Bockstein argument, and see that their integral cohomology doesn't contain $\BZ/p^k$ for $k>1$ below degree $np$.
Hence we conclude
\begin{theorem}
\label{base case}
	The map $\Sigma^{\infty}K(\BZ/p,n) \to 
	\Sym \Sigma^nH\BF_p$ is a tame equivalence.
\end{theorem}

For the inductive step, recall that there is a principal fiber sequence
\[
K(\BZ/p^k,n)\to 
K(\BZ/p^k,n+1)
\xrightarrow{\beta} K(\BZ/p^{k+1},n+1)
\]
and hence we can write $K(\BZ/p^{k+1},n+1)\simeq \abs{\Barconstruction(K(\BZ/p^k,n+1),K(\BZ/p^k,n),*)}$.
Note that 
\begin{equation}
\label{principal fiber seq for EM-spaces}
	\Sigma^{\infty}_{+}K(\BZ/p^k,n)\to 
\Sigma^{\infty}_{+}K(\BZ/p^k,n+1)
\to 
\Sigma^{\infty}_{+}
K(\BZ/p^{k+1},n+1)
\end{equation}
is a principal fiber sequence of $\E_{\infty}$-rings,
since $\Sigma^{\infty}_{+}$ is symmetric monoidal and preserves small colimits.

\begin{theorem}
	The map $\Sigma^{\infty}K(\BZ/p^k,n) \to 
	\Sym \Sigma^nH\BZ/{p^k}$ is a tame equivalence for any $k\geq 1$.
\end{theorem}
\begin{proof}
	The case for $k=1$ is theorem \ref{base case}.
	For the inductive step, observe that 
	\[
		\Sigma^{\infty}_{+}K(\BZ/p^k,n)\to 
\Sigma^{\infty}_{+}K(\BZ/p^k,n+1)
\to 
\Sigma^{\infty}_{+}
K(\BZ/p^{k+1},n+1)
\]
is a cofiber sequence in $\Sp$ by the discussion above.
Consider now the cofiber sequence in $\Sp$
\[
\Sigma^nH\BZ/{p^k}\to \Sigma^{n+1}H\BZ/{p^k}
\to 
\Sigma^{n+1}H\BZ/{p^{k+1}},
\]
apply the functor $\Sym$ and we obtain a cofiber sequence in $\CAlg(\Sp)$
\[
\Sym\Sigma^nH\BZ/{p^k}\to 
\Sym\Sigma^{n+1}H\BZ/{p^k}
\to 
\Sym
\Sigma^{n+1}H\BZ/{p^{k+1}}.
\]
The theorem now follows from the fact that the collection of tame equivalences is closed under colimits.

\end{proof}

To conclude, we record the following corollary, which completes the proof of Proposition \ref{Fully faithfulness of C_tame on EM spaces}.
\begin{corollary}
	For any $R_k$-module $V$, the map 
	$$\Sigma^{\infty}K(V,r+k) \to 
	\Sym \Sigma^{r+k}HV$$ is a tame equivalence for any $k\geq 1$.
\end{corollary}

%We will argue that for any $X \in \mathcal{S}_{\mathbb{Q}}^{\geq 2}$ the unit map of this adjunction
%$$
%X \rightarrow \operatorname{Map}_{\text {coCAlg }}\left(H \mathbb{Q}, \widetilde{C}_{\mathbb{Q}} X\right)
%$$
%is an equivalence. Here and above, the mapping spaces refer to those in the $\infty-$ category $\operatorname{coCAlg}^{\mathrm{nu}}\left(\mathrm{Sp}_{\mathbb{Q}}\right)$






%Todo:
%\begin{enumerate}
%	\item Argue that the cofree conilpotent coalgebra is given by the symmetric algebra in $\Sp_{tame}$. Check Higher Algebra 3.1.1. and 3.1.3.
%	\item Argue that the map $
%\Sigma_{tame}^{\infty}K(V,r+k)\to \cofree_{tame} \Sigma^{r+k}HV
%$ is indeed an equivalence. This can be done by reducing to the case of finitely generated abelian groups.
%    \item An Eilenberg-Moore type of argument to finish ?
%\end{enumerate}
%
%
%Since the underlying spectrum of $C_{tame}(M(G_n,n+1))$ is an Eilenberg-Maclane spectrum $\Sigma^{n+1}HG_n$. Let's consider the unit map for $M(G_n,n+1)$:
%\begin{align*}
%	\map_{\coCAlg(\Sp_{tame})}(H\BZ, C_{tame}(M(G_n,n+1))) & \simeq  \map_{\Sp_{tame}}(H\BZ, \Prim \circ C_{tame}(M(G_n,n+1)))
%	\\ & \simeq 
%\end{align*}










%Let $R$ be a ring containing $\BQ$. 
%\begin{corollary}
%The map 
%\[
%SR\to HR
%\]
%induced from the unit map $\BS\to H\BZ$ is an equivalence.
%\end{corollary}
%\begin{proof}
%	Consider the following commutative diagram induced from the short exact sequence in universal coefficient theorem:
%	\[
%	\begin{tikzcd}
%		\pi_{*}\BS\otimes R & \pi_*(SR)\\
%		\pi_*{H\BZ}\otimes R & \pi_* (HR)
%		\ar[from=1-1, to=1-2]
%		\ar[from=1-1, to=2-1]
%		\ar[from=2-1, to=2-2]
%		\ar[from=1-2, to=2-2]
%	\end{tikzcd}
%	\]
%	where the two horizontal maps are isomorphisms by the universal coefficient theorem.
%	Since $\pi_{*}\BS$ only contains torsions for $*\geq 1$, hence $\pi_*\BS\otimes R=0$ for $*\geq 1$. The left vertical map is an isomorphism at degree $0$, therefore the corollary follows.
%	
%\end{proof}
%
%\begin{lemma}
%	The map 
%	$HR \to HR\otimes HR$ induced from the unit map $\BS \to HR$
%	is an equivalence.
%\end{lemma}
%\begin{proof}
%	Indeed, we have an isomorphism 
%	\[
%	\pi_{*} HR\otimes R \cong
%	\pi_{*} (HR\otimes SR)
%	\cong 
%	\pi_*(HR\otimes HR),
%	\]
%	hence $\pi_*(HR\otimes HR)$ is non-zero only when $*=0$, in which case we have 
%	$$
%	R\otimes R\cong R
%	$$ since $R$ is a solid ring, i.e., the multiplication map $R\otimes R\to R$ is an isomorphism.
%	
%\end{proof}
%
%\begin{corollary}
%	The $R$-localization functor $L_R:\Sp\to \Sp$ is a smashing localization.
%\end{corollary}
%\begin{proof}
%	Since smashing localization is in one-to-one correspondence with idempotent $\E_{\infty}$-rings.
%\end{proof}



\section{Proof of Theorem \ref{2nd Main Theorem}}
\label{proof of the 2nd main theorem}
\begin{proposition}
\label{tame recognition theorem}
For $r\geq 4$,
	the loop functor $\Omega: \CS^{\geq r}_{\text{$r$-tame}}\to \Grp(\CS^{\geq r-1}_{\text{($r-1$)-tame}})$ is an equivalence. 
\end{proposition}

\begin{proof}
	We have a commutative diagram 
	\[
	\begin{tikzcd}
		\CS^{\geq r}_{\text{$r$-tame}} & \Grp(\CS^{\geq r-1}_{\text{($r-1$)-tame}})\\
		\CS^{\geq r}_*   &  \Grp(\CS^{\geq r-1}_{*})  	
	\arrow[from=1-1, to=1-2,"\Omega"]
	\arrow[from=2-1, to=2-2, "\Omega","\simeq" below]
	\arrow[from=1-1, to=2-1]
	\arrow[from=1-2, to=2-2]
	\end{tikzcd}
	\]
	where the left vertical functor is fully faithful and the right vertical functor is also fully faithful by Proposition \ref{induced fully faithfulness on group objects}. The bottom arrow is an equivalence by May's recognition theorem (e.g. see \cite{HA}[Theorem 5.2.6.10.]). Hence the top loop functor $\CS^{\geq r}_{\text{$r$-tame}} \to \Grp(\CS^{\geq r-1})$ is fully faithful.
	For the essential surjectivity, we know that any $X\in \Grp(\CS^{\geq r-1}_{\text{($r-1$)-tame}})$ is equivalent to $\Omega Y$ for some $Y\in \CS^{\geq r}_*$, while $Y$ has to be $r$-tame since $X$ is $(r-1)$-tame.
	
\end{proof}

\begin{lemma}
\label{conservativity of forgetful functor (general)}
	Let $\CC$ be a Cartesian symmetric monoidal $\infty$-category and let $L:\CC\to \CC'$ be a localization functor that preserves finite products.
	Then 
	\begin{enumerate}
		\item the forgetful functor
	\[
	\Grp(\CC)\to \CC
	\]
	is conservative;
	\item the forgetful functor
	\[
	\Grp(\CC')\to \CC'
	\]
	is also conservative.
	\end{enumerate}
\end{lemma}
\begin{proof}
	Since the category of groups is a full subcategory of monoids, it suffices to prove the forgetful functor $\Mon(\CC)\to \CC$ is conservative.
	The first statement then follows from \cite{HA}[Proposition 2.4.2.5] and \cite{HA}[Lemma 3.2.2.6.].
	
	For the second statement, consider the commutative diagram:
	\[
\begin{tikzcd}
	\Mon(\CC') & \CC'\\
	\Mon(\CC) & 
	\CC	
	\arrow[from=1-1, to= 1-2, "\oblv_{\Mon}"]
%	\arrow[from=1-2, to= 1-1, shift left, "R"]
	\arrow[from=1-1, to=2-1]
	\arrow[from=1-2, to=2-2]
	\arrow[from=2-1, to= 2-2, "\oblv_{\Mon}"]
%	\arrow[from=2-2, to= 2-1, shift left, "R'"]
\end{tikzcd}
\]
where both horizontal arrows are fully faithful and the bottom arrow is conservative by the first statement. Hence the top arrow is conservative as well.
\end{proof}

Apply Lemma \ref{conservativity of forgetful functor (general)} to the case of tame spaces, we have the following corollary.

\begin{corollary}
\label{conservativity of forgetful functor}
	The forgetful functor 
	\[
	\Grp(\CS^{\geq r-1}_{\text{($r-1$)-tame}}) \xrightarrow{\oblv_{\Grp}} 
	\CS^{\geq r-1}_{\text{($r-1$)-tame}}
	\]
	is a conservative functor.
\end{corollary}

Since the underlying product of $\coCAlg(\Sp^{\geq r-1}_{\text{($r-1$)-tame}})$ is given by the smash product in $\Sp^{\geq r-1}_{\text{($r-1$)-tame}}$, we can identify $\coCAlg(\Sp^{\geq r-1}_{\text{($r-1$)-tame}})$ with a Cartesian symmetric monoidal $\infty$-category. Moreover, the functor 
$$C_{tame}: \CS^{\geq r-1}_{\text{($r-1$)-tame}} \to
\coCAlg(\Sp^{\geq r-1}_{\text{($r-1$)-tame}})$$
sends products to smash products, therefore it induces a functor
$$
G_{tame}:\Grp(\CS^{\geq r-1}_{\text{($r-1$)-tame}})
\to
\Grp(\coCAlg(\Sp^{\geq r-1}_{\text{($r-1$)-tame}})),
$$
where we identify the latter as the $\infty$-category of cocommutative Hopf algebras in $(r-1)$-tame spectra.

\begin{definition}
	Let $\CC$ be a symmetric monoidal $\infty$-category. We let 
	$$
	\Hopfalgebra(\CC):=\Grp(\coCAlg(\CC))
	$$
	denote the $\infty$-category of (cocommutative) Hopf algebra objects in $\CC$.
\end{definition}


Apply Lemma \ref{conservativity of forgetful functor (general)} again, we have the following result.
\begin{corollary}
\label{conservativity of the forgetful functor for Hopf algebras}
		The forgetful functor 
	\[
	\Hopfalgebra(\Sp^{\geq r-1}_{\text{($r-1$)-tame}}) \xrightarrow{\oblv_{\Grp}} 
	\coCAlg(\Sp^{\geq r-1}_{\text{($r-1$)-tame}})
	\]
	is a conservative functor.
\end{corollary}
\begin{proof}
	This follows from the fact that $\coCAlg(\Sp^{\geq r-1}_{\text{($r-1$)-tame}})$ is a Cartesian symmetric monoidal $\infty$-category and Lemma \ref{conservativity of forgetful functor (general)}.
\end{proof}

Recall that in rational homotopy theory, any rational H-space splits as a product of Eilenberg-Maclane spaces.
Our proof of the Hopf algebra model of tame spaces will be based on an analogous splitting result for tame H-spaces.


\begin{proposition}
\label{Decomp of tame H-spaces}
	Let $X$ be a $r$-tame H-space of finite type. Then $X$ is equivalent to a product of Eilenberg-Maclane spaces, that is, we have an equivalence
	\[
	X \simeq \prod_i K(\pi_i X, i).
	\]
\end{proposition}
\begin{proof}
Since $X$ is of finite type, we are reduced to prove the case for $X$ being $p$-local for every prime $p$. Asumme $X$ is both tame and $p$-local, then $X$ is $(r+2p-4)$-truncagted, since $\pi_{n+1}X$ is uniquely $p$-divisible  for $n \geq r+2p-3$.

	We proceed by induction on the Postnikov tower of $X$.
	The base case is obvious. 	
	For the inductive step, consider the principal fiber sequence
	$$
	K(\pi_{n+1}X,n+1)\to 
	\tau_{\leq n+1}X
	\to
	\tau_{\leq n}X
	\xrightarrow{k_n}
	K(\pi_{n+1}X,n+2),
	$$
	and we want to show the $k$-invariant $[k_n]\in H^{n+2}(\tau_{\leq n}X; \pi_{n+1}X)$ vanishes. 
	Note that $k_n$ is a map of $H$-spaces, hence it represents a primitive element in the $p$-local Hopf algebra $H^*(\tau_{\leq n}X; \pi_{n+1}X)$. Let $H$ be a commutative $p$-local Hopf algebra, the kernel of the natural map $\Prim(H)\to \Indec(H)$ consists of elements of $p^k$-th power by a variant of \cite{Milnor-Moore}[Proposition 4.21]. However, this cannot happen because of degree reason, as
	$n<r+2p-3$ but
	\[
	n-rp<r+2p-3-rp= (2-r)(p-1)<0.
	\]
 	By inductive hypothesis, the cohomology of $\tau_n X$ is a direct sum of cohomology of Eilenberg-Maclane spaces;
	\todo{Ref} Cartan shows that if $R$ is a $p$-local ring, all cohomology classes  $[\alpha]\in H^{*}(K(A,k);R)$ are decomposable in degree $k<*<k+2(p-1)$, which concludes that $[k_n]=0$.
\end{proof}
\begin{remark}
		We learned part of the proof of Proposition \ref{Decomp of tame H-spaces} from \cite{soule}[Proposition 3], where the author further atributed the idea to L.Smith.	
\end{remark}

Since we have shown in \todo{ref: previous notes}, the functor $C_{tame}$ is fully faithful when restricted to the full subcategory spanned by Eilenberg-Maclane spaces. 
Theorem \ref{2nd Main Theorem} would follow if we can show every group-like $(r-1)$-tame space splits as a product of Eilenberg-Maclane spaces.



We are now ready to state and prove the second main theorem in this paper.
\begin{theorem}
	The functor $G_{tame}: \Grp(\CS^{\geq r-1}_{\text{($r-1$)-tame}})\to \Hopfalgebra(\Sp^{\geq r-1}_{\text{($r-1$)-tame}})$ is an equivalence of $\infty$-categories.
\end{theorem}
\begin{proof}
Since both $C_{tame}$ and $R$ preserve finite products, there is a pair of adjunctions
\begin{equation}
\label{adj on grps}
	\adj{G_{tame}}{\Grp(\CS^{\geq r-1}_{\text{($r-1$)-tame}})}{\Hopfalgebra(\Sp^{\geq r-1}_{\text{($r-1$)-tame}})}{R'}.
\end{equation}
We observe that there are commutative diagrams of $\infty$-categories
\[
\begin{tikzcd}
	\Grp(\CS^{\geq r-1}_{\text{($r-1$)-tame}}) & \Hopfalgebra(\Sp^{\geq r-1}_{\text{($r-1$)-tame}})\\
	\CS^{\geq r-1}_{\text{($r-1$)-tame}}  & 
	\coCAlg(\Sp^{\geq r-1}_{\text{($r-1$)-tame}})
	\arrow[from=1-1, to= 1-2, "G_{tame}"]
%	\arrow[from=1-2, to= 1-1, shift left, "R"]
	\arrow[from=1-1, to=2-1, "\oblv" left]
	\arrow[from=1-2, to=2-2, "\oblv'" ]
	\arrow[from=2-1, to= 2-2, "C_{tame}"]
%	\arrow[from=2-2, to= 2-1, shift left, "R'"]
\end{tikzcd}
\]
and 
\[
\begin{tikzcd}
	\Grp(\CS^{\geq r-1}_{\text{($r-1$)-tame}}) & \Hopfalgebra(\Sp^{\geq r-1}_{\text{($r-1$)-tame}})\\
	\CS^{\geq r-1}_{\text{($r-1$)-tame}}  & 
	\coCAlg(\Sp^{\geq r-1}_{\text{($r-1$)-tame}})
%	\arrow[from=1-1, to= 1-2, "G_{tame}"]
	\arrow[from=1-2, to= 1-1, "R'"]
	\arrow[from=1-1, to=2-1, "\oblv" left]
	\arrow[from=1-2, to=2-2, "\oblv'" ]
%	\arrow[from=2-1, to= 2-2, "G_{tame}"]
	\arrow[from=2-2, to= 2-1, "R"]
\end{tikzcd}
\]
where $\oblv$ and $\oblv'$ denote the forgetful functor from the category of groups in $\CS^{\geq r-1}_{\text{($r-1$)-tame}}$ (resp. $\Hopfalgebra(\Sp^{\geq r-1}_{\text{($r-1$)-tame}})$ ) to the underlying categories.

We want to show the unit map $X \to R' G_{tame}X$ is an equivalence for any $X\in \Grp(\CS^{\geq r-1}_{\text{($r-1$)-tame}})$.
Since the functor $\oblv$ is conservative by Lemma \ref{conservativity of forgetful functor}, it suffices to show the map $\oblv(X) \to \oblv(R'G_{tame}X)$ is an equivalence.
By the commutativity of the diagrams above, we have
$$
\oblv(R'G_{tame}X) \simeq R \circ \oblv'(G_{tame}X)\simeq RC_{tame}(\oblv(X)).
$$
	Note that Proposition \ref{tame recognition theorem} implies that any  $(r-1)$-tame group $X$ is equivalent to the loop space of a $r$-tame space $Y$. Since $X\simeq \Omega Y$ is in particular an $H$-space, it splits into a product of Eilenberg-Maclane spaces by Proposition \ref{Decomp of tame H-spaces}. Hence, 
		\begin{align*}
		RC_{tame}(\oblv(X)) & \simeq RC_{tame}(\prod_{i} K(\pi_i X, i)))\\
		& \simeq \prod_{i} RC_{tame}(K(\pi_i X, i))\\
		& \simeq \prod K(\pi_i X, i)\\
		& \simeq X
	\end{align*}
	where the second equivalence follows from Proposition \todo{$RC_{tame}$ preserves finite products and connectivity} and Lemma \ref{products commute with products of Eilenberg-Maclane spaces}.
	Hence, we conclude that the functor $G_{tame}$ is fully faithful.
	
	To finish the proof, it suffices to show the right adjoint $R'$ is conservative. 
	Since both $\oblv$ and $\oblv'$ are conservative by Corollary \ref{conservativity of forgetful functor} and Corollary \ref{conservativity of the forgetful functor for Hopf algebras}, we are reduced to show the functor 
	\[
	R: \coCAlg(\Sp^{\geq r-1}_{\text{($r-1$)-tame}}) \to 
	\CS^{\geq r-1}_{\text{($r-1$)-tame}}
	\]
	is conservative. Let $f:U\to V$ be a morphism in $\coCAlg(\Sp^{\geq r-1}_{\text{($r-1$)-tame}})$ and suppose that $R(f): RU\to RV$ is an equivalence.
	For any coalgebra $X\in \coCAlg(\Sp^{\geq r-1}_{\text{($r-1$)-tame}})$, we claim that the induced map on the mapping space
	$$
	\map_{\coCAlg(\Sp^{\geq r-1}_{\text{($r-1$)-tame}})}(X, U)\to 
	\map_{\coCAlg(\Sp^{\geq r-1}_{\text{($r-1$)-tame}})}(X, V)
	$$
	is an equivalence.
	Note that if $X$ lies in the essential image of $C_{tame}$, then the proof is complete. Hence, it is reduced show the functor
	$$
	C_{tame}: \CS^{\geq r-1}_{\text{($r-1$)-tame}}  
	\to 
	\coCAlg(\Sp^{\geq r-1}_{\text{($r-1$)-tame}})
	$$
	is essentially surjective, which we will prove in Lemma \ref{Essential Surjectivity of C_tame}.
	
\end{proof}






\clearpage
Note that the composite $L_{tame}\circ \Sigma^{\infty}: \Space_{*}^{\geq r-1}\to \Sp^{\geq r-1}_{\text{($r-1$)-tame}}$ factorizes as $C'_{tame}: \Space_{*}^{\geq r-1}\to \coCAlg(\Sp^{\geq r-1}_{\text{($r-1$)-tame}})$ followed by a forgetful functor. The lemma below would be crucial to establish the essential surjectivity of $C_{tame}$.

\begin{lemma}

	\label{Sigma S^n is a trivial coalgebra}
	$C'_{tame} S^{r-1}$ is a trivial coalgebra in $\coCAlg(\Sp^{\geq r-1}_{\text{($r-1$)-tame}})$.
\end{lemma}
\begin{proof}
%	We need to show all the coalgebra structure maps 
%	$$
%	\Sigma^{\infty}_{tame} S^{r-1} 
%	\xrightarrow{\delta_n} 
%	\big( (\Sigma^{\infty}_{tame} S^{r-1})^{\hat{\otimes}n} \big)_{h\Sigma_n}
%	$$
%	are trivial. 
	By Proposition \todo{ref?}, we can build a commutative tame coalgebra $X$ by assembling compatible coalgebra structures of $X$ in $\coAlg_{\phi^n \Com}(\Sp_{tame}^{r-1})$ for each $n$.
	
	We prove by induction on $n$.
	Assume $\Sigma^{\infty}_{tame}S^{r-1}$ is a trivial $(\phi^n \Com)$-coalgebra, by Proposition \ref{inductive construction of coalgebras} and the vanishing of Tate diagonal in tame spectra, specifying a $(\phi^{n+1} \Com)$-coalgebra structure  on $\Sigma^{\infty}_{tame}S^{r-1}$ is equivalent to a lift in the following diagram
	\[
	\begin{tikzcd}
		& (\Sigma^{\infty}_{tame}S^{r-1} \hat{\otimes} \cdots \hat{\otimes} \Sigma^{\infty}_{tame}S^{r-1})_{h\Sigma_n}\\
		\Sigma^{\infty}_{tame}S^{r-1} & (\phi^{n-1}\Com(n)\hat{\otimes} \Sigma^{\infty}_{tame}S^{r-1} \hat{\otimes} \cdots \hat{\otimes}\Sigma^{\infty}_{tame}S^{r-1})_{h\Sigma_n}
		\arrow[from = 2-1, to = 1-2, dashed]
		\arrow[from = 2-1, to = 2-2, "0" below]
		\arrow[from = 1-2, to = 2-2].
	\end{tikzcd}
	\]
	Let $F$ denote the fiber of the vertical map. We claim that the connectivity of $F$ is at least $r$, hence any lift is null-homotopic.
	The connectivity of $(\Sigma^{\infty}_{tame}S^{r-1} \hat{\otimes} \cdots \hat{\otimes} \Sigma^{\infty}_{tame}S^{r-1})_{h\Sigma_n}$ is $n(r-1)$, which is larger than $r+1$ (recall $r\leq 4$). Since the possible largest dimension of a non-degenerate simplex of in $\Part_{n-1}(n)$ is $n-3$, the connectivity of 
	$$
	(\phi^{n-1}\Com(n)\hat{\otimes} \Sigma^{\infty}_{tame}S^{r-1} \hat{\otimes} \cdots \hat{\otimes}\Sigma^{\infty}_{tame}S^{r-1})_{h\Sigma_n}
	$$
	is at least $n(r-1)-(n-3) > r+1$. Hence the connectivity of $F$ is larger than $r$, and the lemma is proved.
\end{proof}

\begin{remark}
	The proof above is almost identical to the proof in \cite{Heuts_Goodwillie}[Lemma 6.17]. 
	There he shows that $\Sigma^{\infty}S^{r-1}$ is a trivial coalgebra in $\coAlg^{nil, dp}_{\Com}(\tau_{p-1}\Sp^{\geq r-1})$,
	where $\coAlg^{nil, dp}_{\Com}(\tau_{p-1}\Sp^{\geq r-1})$ denotes the $\infty$-category of conilpotent, divided power coalgebras $X$ in $\Sp$ with $(p-1)!$ inverted in $\Sp$ and with coherent structure maps $X\to (X^{\otimes k})_{h\Sigma_k}$ for $1\leq k \leq p-1$. 
	The important ingredients of both proof are the inductive construction of coalgebras and the vanishing of Tate construction.
\end{remark}

\begin{lemma}
\label{Essential Surjectivity of C_tame}
	The functor 
	$$
	C_{tame}: \CS^{\geq r-1}_{\text{($r-1$)-tame}}  
	\to 
	\coAlg^{nil, dp}_{\Com}(\Sp^{\geq r-1}_{\text{($r-1$)-tame}})
	$$
	is essentially surjective.
\end{lemma}
\begin{proof}
Recall that Theorem \ref{Ching-Harper's Koszul duality}
gives an equivalence
$$
B_{\Lie}: \Alg_{\Lie}(\Mod_{H\BZ})^{\geq r-1} \to \coAlg^{nil, dp}_{\Com}(\Mod_{H\BZ})^{\geq r-1}
$$
of $\infty$-categories. The $\infty$-category $\Alg_{\Lie}(\Mod_{H\BZ})^{\geq r-1}$ is generated by the free Lie algebra $\Free_{\Lie}(\Sigma^{r-1} H\BZ)$ under colimits,
 it follows that the $\infty$-category $\coAlg^{nil, dp}_{\Com}(\Mod_{H\BZ})^{\geq r-1}$ is generated under colimits by the trtivial coalgebtra $\trivial_{\Com}(\Sigma^{r-1}H \BZ)$ as we have an equivalence of functors
$$
B_{\Lie}\circ \Free_{\Lie} \simeq \trivial_{\Com}.
$$
Since $\coAlg^{nil, dp}_{\Com}(\Sp^{\geq r-1}_{\text{($r-1$)-tame}})$ is a full subcategory of $\coAlg^{nil, dp}_{\Com}(\Mod_{H\BZ}^{\geq r-1})$,
it suffices to show the essential image of $K(\BZ, r-1)$ is equivalent to the trivial coalgebra $\trivial_{\Com,tame}(\Sigma^{r-1}H \BZ)$.

Since $S^{r-1}\to K(\BZ,r-1)$ is a tame equivalence, we have that
$$
C_{tame}K(\BZ,r-1) \simeq C'_{tame} S^{r-1}.
$$
The statement now follows from Lemma \ref{Sigma S^n is a trivial coalgebra}.
\end{proof}






