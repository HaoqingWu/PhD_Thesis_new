\chapter{Tame Spaces and Tame Spectra}
In this chapter, we give a modern treatment of Dwyer's tame spaces in \S \ref{Section Tame spaces} using the theory of localizations.
In \S \ref{Section: Characterization of Tame Equivalences}, we give a concrete characterization of tame equivalences.
We then extend the definition of tameness to spectra in \S \ref{Section: Tame spectra}.
We prove that the $\infty$-category of $r$-tame spaces (resp. $r$-tame spectra) is a localization of the $\infty$-category of $r$-connective spaces (resp. $r$-connective spectra).
We then discuss properties of these two categories and establish an algebraic characterization of $r$-tame spectra.

\section{Tame Spaces}
\label{Section Tame spaces}
In this section, we first define the notion of a tame space.
Then we explain how to obtain the $\infty$-category of $r$-tame spaces as a localization of the $\infty$-category $\Space_{*}^{\geq r}$ of pointed $r$-connective spaces.
Recall that we have defined the tame ring system $\{R_{j}\}_{j \leq 0}$ in Definition \ref{tame ring system}.

\begin{definition}
	\label{Tame spaces}
	Let $X\in \Space_{*}^{\geq r}$ be a pointed $r$-connective space.
	We say $X$ is \emph{$r$-tame} if for all $j\geq 0$, the $(r+j)$-th homotopy group $\pi_{r+j}(X)$ is uniquely $p$-divisible for all $p\leq \frac{j+3}{2}$. 
	This is equivalent to saying that $\pi_{r+j}(X)$ is a $R_j$-module for each $j$, that is,
	\[
	\pi_{r+j}(X)\cong \pi_{r+j}(X)\otimes R_{j}.
	\]
\end{definition}

    \begin{remark}
        Note that the definition of tame spaces depends on an integer $r$.
        It is clear from the definiton that,
        for two integers $s\geq r \geq 3$, if a $s$-connective space $X$ is $r$-tame, then it is also $s$-tame. 
    \end{remark}
    
    \begin{notation}
        We let $\Space^{\geq r}_{\tame}$ denote the full subcategory of $\Space^{\geq r}_{*}$ spanned by $r$-tame spaces.
       We decide to omit the $r$ from the subscript since the level of tameness will be clear from the superscript.
       Moreover, we will simply call a space \emph{tame} if the level of tameness is clear from the context.
    \end{notation}
    

    
Given a space $X\in \Space^{\geq r}_{*}$, we want to produce a space $X_{\tame}$ that is the universal tame space receiving a map from $X$. Moreover, we want this assignment to be functorial. The theory of localization, well-studied in Bousfield \cite{BousfieldSpaces}, \cite{BousfieldSpectra} and Farjoun \cite{Farjoun}, provides powerful machinery for doing so in the context of model categories.
For localization of $\infty$-categories, we will follow \cite{HTT}.
\begin{definition}
    \cite[Definition 5.2.7.2.]{HTT}
    A functor $F:\CC\to \CD$ between $\infty$-categories is a \emph{localization} if $F$ has a fully faithful right adjoint.
\end{definition}
In our case, we will construct a \emph{tame localization functor}
$$
L_{\tame}: \Space^{\geq r}_* \to \Space^{\geq r}_{\tame}
$$
so that the inclusion functor is its right adjoint.
Our strategy for proving the existence of such a localization functor is to show the effect of inverting primes in homotopy groups is equivalent to inverting some maps in $\CS^{\geq r}_{*}$.

To explain this idea, we now recall some terminology from Definition \ref{localization}. 
Let $\CC$ be an $\infty$-category and $S$ a collection of morphisms in $\CC$.
An object $Z$ is said to be \emph{$S$-local} if for every morphism $f:X\to Y$ in $S$, restricting along $f$ 
$$
\map_{\CC}(Y,Z) \to \map_{\CC}(X,Z)
$$
is a weak equivalence.
A morphism $f:X\to Y$ in $\CC$ is an \emph{$S$-equivalence} if, for every $S$-local object $Z$, composition with $f$ induces a weak equivalence 
$$
\map_{\CC}(Y,Z) \to \map_{\CC}(X,Z).
$$
\begin{remark}
    Let $L:\CC \to L\CC \hookrightarrow \CC$ be a localization of an $\infty$-category $\CC$, and let $S$ be the class of morphisms $f$ in $\CC$ such that $Lf$ is an equivalence.
    By \cite[Proposition 5.5.4.2.]{HTT}, the full subcategory spanned by $S$-local objects is precisely the essential image $L\CC$ under $L$, and every $S$-equivalence in $\CC$ belongs to $S$. Moreover,  $L\CC$ is accessible if $\CC$ is accessible.
\end{remark}

% \begin{proposition}
%     \cite[Proposition 5.5.4.2.]{HTT}
%     Let $L:\CC \to \CC$ be a localization functor and $S$ be the class of morphisms $f$ in $\CC$ such that $Lf$ is an equivalence. Then
%     \begin{enumerate}
%         \item An object $C\in\CC$ is $S$-local if and only $C\in L\CC$.
%         \item Every $S$-equivalence in $\CC$ belongs to $S$.
%         \item Suppose that $\CC$ is accessible, then the following are equivalent:
%         \begin{enumerate}[(i)]
%             \item The $\infty$-category $L\CC$ is accessible.
%             \item The functor $L\CC \to \CC$ is accessible.
%             \item There exists a (small) subset $S_0\subset S$ such that every $S_0$-local object is $S$-local.
%         \end{enumerate}
        
%     \end{enumerate}
% \end{proposition}



To understand the effect of inverting primes in homotopy groups, let's first consider the following simple case.
Let $m_p:S^n \xrightarrow{p}S^n$ be a map of degree $p$.
Observe that a space $X$ is $m_p$-local if and only if the multiplication-by-$p$ map on the homotopy groups
\[
\pi_{*}(X) \xrightarrow{\cdot p} \pi_{*}(X)
\]
is an isomorphism in degrees $*\geq n$. In other words, a space $X$ is $m_p$-local if and only if its homotopy groups $\pi_*(X)$ are uniquely $p$-divisible in degrees large or equal to $n$.

Let $P$ denote the set of all primes. For each prime $q$, we consider the following composition 
\[
 m_q: S^{r+2q-3} 
\xrightarrow{\cdot q} S^{r+2q-3} \xrightarrow{\text{inclusion}}  \bigvee_{p \in P}S^{r+2p-3},
\]
which induces a canonical map  
\begin{equation}
\label{f map}
	\vee_{p\in P} m_p: \bigvee_{p\in P}S^{r+2p-3} \to \bigvee_{p\in P}S^{r+2p-3}.
\end{equation}
To simplify notation, we will denote the map $ \vee_{p\in P} m_p$ by $f$.


\begin{lemma}
\label{tame is equivalent to f-local}
	A space $X\in \Space_{*}^{\geq r}$ is tame if and only if it is $f$-local.
\end{lemma}
\begin{proof}
	A space $X$ is $f$-local if and only if the induced map on mapping spaces
	\[
	\map_{*}(\bigvee_{p\in P}S^{r+2p-3},X) \xrightarrow{f^*} \map_{*}(\bigvee_{p\in P}S^{r+2p-3},X)
	\]
	is a weak equivalence. This is equivalent to asking that the homotopy groups $\pi_{r+j}(X)$ be uniquely $p$-divisible for all primes $p\leq \frac{j+3}{2}$ for every fixed $j\geq 0$. 	The lemma now follows from the following commuting diagram
	\[	
	\begin{tikzcd}
		\pi_{r+j}(X) & \pi_{r+j}(X)\\
		\pi_{j-(2p-3)}(\map_*(S^{r+2p-3},X)) & \pi_{j-(2p-3)}(\map_*(S^{r+2p-3},X)),
		\arrow[from=1-1, to=1-2,"\cdot p"]
		\arrow[from=2-1, to=2-2,"\cdot p"]
		\arrow[from=1-1, to=2-1]
		\arrow[from=1-2, to=2-2]
	\end{tikzcd}
	\]
	where the vertical arrows are isomorphisms. Hence the top horizontal map is an isomorphism if and only if the bottom horizontal map is an isomorphism.
	
\end{proof}

The following proposition guarantees the existence of the localization of presentable $\infty$-categories.
\begin{proposition}
    \cite[Proposition 5.5.4.15.]{HTT}
    \label{Prop 5.5.4.15. HTT}
    Let $\mathcal{C}$ be a presentable $\infty$-category, and let $S$ be a (small) set of morphisms of $\CC$. Let $\overline{S}$ denote the strongly saturated (cf. \cite[Definition 5.5.4.5.]{HTT}) class of morphisms generated by $S$. If $\mathcal{C}^{\prime} \subseteq \CC$ denotes the full subcategory of $\CC$ spanned by $S$-local objects, then
    \begin{enumerate}
        \item For each object $C \in \CC$, there exists a morphism $s: C \rightarrow C^{\prime}$ such that $C^{\prime}$ is $S$-local and $s$ belongs to $\overline{S}$.
        \item The $\infty$-category $\CC'$ is presentable.
        \item The inclusion $\CC' \subseteq \mathcal{C}$ has a left adjoint $L$.
        \item For every morphism $f$ of $\mathcal{C}$, the following are equivalent:
        \begin{enumerate}[(i)]
            \item  The morphism $f$ is an $S$-equivalence.
            \item The morphism $f$ belongs to $\overline{S}$.
            \item The induced morphism $L f$ is an equivalence.
        \end{enumerate}
    \end{enumerate}
\end{proposition}
If $\CC = \Space^{\geq r}_{*}$ and $S= \{f\}$ in (\ref{f map}), we will also call an $S$-local space (resp. $S$-equivalence) $L_f$-local (resp. $L_f$-equivalence).
The proposition above guarantees the existence of the tame localization functor. 
\begin{corollary}
\label{existence of tame localization}
There exists a localization functor
$$
L_{\tame}: \Space^{\geq r}_* \to \Space^{\geq r}_{\tame}.
$$
Moreover, the $\infty$-category $\Space^{\geq r}_{\tame}$ of tame spaces is a presentable $\infty$-category.
\end{corollary}
\begin{definition}
    A morphism $g:X \to Y$ in $\Space^{\geq r}_*$ is said to be a \emph{tame equivalence} if it is an $L_f$-equivalence in the sense of Definition \ref{localization}.
\end{definition}

\section{Characterization of Tame Equivalences}
\label{Section: Characterization of Tame Equivalences}
In this section, we give an explicit characterization of tame equivalences by showing that the functor $L_{tame}:\Space^{\geq r}_* \to \Space^{\geq r}_{tame}$ is an infinite composite of localization functors.

We first consider the effect of localizing with respect to the multiplication-by-$p$ map $$
m_p:S^{r+2p-3}\to S^{r+2p-3}$$ for a fixed prime $p$. 
Note that a space $X\in \Space^{\geq r}_*$ is $L_{m_p}$-local if and only if $\pi_{*}X$ is uniquely $p$-divisible for $* \geq r+2p-3$.
Let $\CS^{\geq r}_{m_p}$ denote the full subcategory of $\CS^{\geq r}_*$ spanned by spaces whose homotopy groups are uniquely $p$-divisible in degree larger or equal to $r+2p-3$.
Proposition \ref{Prop 5.5.4.15. HTT} guarantees the existence of a localization functor
\[
L_{m_p}:\Space^{\geq r}_{*} \to \CS^{\geq r}_{m_p}.
\]
A map $g:X\to Y$ is a $L_{m_p}$-equivalence if and only if 
$L_{m_p}g: L_{m_p}X \to L_{m_p}Y$ is an equivalence.
We first show that $L_{m_p}$-localization does not change the homotopy groups below degree $r+2p-3$. 
\begin{proposition}
\label{connectivity of local equivalences of spaces}
	If $g: X\to Y$ is a $n$-connective map in $\Space_*$, then every $L_g$-equivalence is $n$-connective.
\end{proposition}
\begin{proof}
	Take $S=\{g\}$ and denote by $\overline{S}$ the strongly saturated class of morphisms generated by $S$. Proposition \ref{Prop 5.5.4.15. HTT} implies that  $\overline{S}= \{L_g\text{-equivalences}\}$, hence it suffices to show that $\overline{S}$ is contained in the class of $n$-connective maps. Note that the class of $n$-connective maps is saturated in the sense of \cite[Definition 5.5.5.1]{HTT}. Since every strongly saturated class of morphisms in a presentable $\infty$-category is also saturated \cite[Example 5.5.5.5.]{HTT}, we have 
	$$
	\{L_g\text{-equivalences}\}=\overline{S} \subseteq \tilde{S} \subseteq \{n\text{-connective maps}\}
	$$
	where $\tilde{S}$ denotes the saturated class of morphisms generated by $S$.
\end{proof}

\begin{corollary}
\label{L_g doesn't change lower hpty groups if g is n-conn}
        If $g:A\to B$ is $n$-connective, then  the localization $L_g:\CS_*\to \CS_*$ doesn't change the homotopy groups of a space $X$ below degree $n$, i.e. 
	\[
	\pi_{i}(L_gX) \cong \pi_i(X)
	\]
	for $i< n$.
\end{corollary}

We now compare $L_{m_p}$-localization with
the Bousfield localization with respect to the homology theory $H_{*}(-;\BZ[\frac{1}{p}])$.

In this case, we let $S'$ be the collection of morphisms that induce isomorphisms on homology with coefficients in $\BZ[\frac{1}{p}]$, or equivalently on $\pi_*(-)\otimes \BZ[\frac{1}{p}]$ by \cite[Proposition 4.3]{Bousfield96}, as we are working in $\CS^{\geq r}_*$.
We will call an $S'$-local space \emph{$\BZ[\frac{1}{p}]$-local}.
By \cite[Theorem 5.5]{Bousfield96}, 
a space whose homotopy groups are uniquely $p$-divisible is $\BZ[\frac{1}{p}]$-local.
Hence, any $\BZ[\frac{1}{p}]$-local space is also $L_{m_p}$-local, and any $L_{m_p}$-equivalence also induces isomorphisms on $\pi_*(-)\otimes \BZ[\frac{1}{p}]$.

\begin{lemma}
\label{L-equivalence implies Z[1/p]-iso in spaces}
	If $f:X\to Y$ is a $L_{m_{p}}$-equivalence of spaces, then it induces isomorphisms
	$$
	\textstyle{\pi_{*}(X)\otimes \BZ[\frac{1}{p}]
	\to
	\pi_{*}(Y)\otimes \BZ[\frac{1}{p}]}.
	$$
	Moreover, if $f:X\to Y$ is a map between $(r+2p-3)$-connective spaces, then $f$ is a $L_{m_p}$-equivalence if and only if it is a $\BZ[\frac{1}{p}]$-equivalence.
\end{lemma}
\begin{proof}
    The first part is clear.
    The second part follows from the fact that a $(r+2p-3)$-connective space is $L_{m_p}$-local if and only if it is $\BZ[\frac{1}{p}]$-local.
\end{proof}

Let $\tau_{\geq r+2p-3}X$ denote the $(r+2p-3)$-connective cover and $\tau_{\leq r+2p-4}X$ the $(r+2p-4)$-truncation of $X$.
We claim that $L_{m_p}$-localization commutes with taking $(r+2p-3)$-connective cover.

\begin{lemma}
\label{m_p localization commutes with k-th connective cover}
	The localization functor $L_{m_p}$ commutes with taking $(r+2p-3)$-th connective cover, i.e., there is an equivalence
	$$
	L_{m_p}(\tau_{\geq r+2p-3}X)\simeq \tau_{\geq r+2p-3}(L_{m_p} X)
	$$
	for any $X\in \CS^{\geq r}_*$.
\end{lemma}
\begin{proof}
	It suffices to show the map $\tau_{\geq r+2p-3}X\to \tau_{\geq r+2p-3}L_{m_{p}}X$ is a $L_{m_p}$-equivalence, since $\tau_{\geq r+2p-3}L_{m_{p}}X$ is $L_{m_p}$-local. Consider the following commutative diagram of fiber sequences
	\[
	\begin{tikzcd}
		\tau_{\geq r+2p-3}X & X & \tau_{\leq r+2p-4}X    \\
		\tau_{\geq r+2p-3}L_{m_p}X & L_{m_p}X & \tau_{\leq r+2p-4} L_{m_p}X\simeq \tau_{\leq r+2p-4}X,
		\arrow[from=1-1, to=1-2]
		\arrow[from=2-1, to=2-2]
		\arrow[from=1-3, to=2-3]
		\arrow[from=1-2, to=1-3]
		\arrow[from=2-2, to=2-3]
		\arrow[from=1-1, to=2-1]
		\arrow[from=1-2, to=2-2]
	\end{tikzcd}
	\]
	where the equivalence at the lower right corner follows from Corollary \ref{L_g doesn't change lower hpty groups if g is n-conn}.
	Note that the diagram above induces a commutative diagram of two long exact sequences of homotopy groups. For each $n\geq r+2p-3$, there is a commutative diagram
	\[
	\begin{tikzcd}
		\pi_{n}(\tau_{\geq r+2p-3}X)\otimes \BZ[\frac{1}{p}] & \pi_{n}(X) \otimes \BZ[\frac{1}{p}] \\
		\pi_{n}(\tau_{\geq r+2p-3}L_{m_p}X)\otimes \BZ[\frac{1}{p}]  & \pi_{n}(L_{m_p}X)\otimes \BZ[\frac{1}{p}] 
		\arrow[from=1-1, to=1-2]
		\arrow[from=2-1, to=2-2]
		\arrow[from=1-1, to=2-1]
		\arrow[from=1-2, to=2-2],
	\end{tikzcd}
	\]
	where the horizontal maps are isomorphisms and the right vertical map is also an isomorphism by Lemma \ref{L-equivalence implies Z[1/p]-iso in spaces}, hence the left map is also an isomorphism.
	The lemma then follows from the second part of Lemma \ref{L-equivalence implies Z[1/p]-iso in spaces}.
	
\end{proof}
	
We can now compute the homotopy groups of the $L_{m_p}$-localization of a space.
\begin{corollary}
\label{computation of the htpy groups of L_m_p X}
	The homotopy groups of $L_{m_p}X$ are given by
	\[
	\pi_* L_{m_p}X = \begin{cases}
		\pi_*X & \text{if $*< r+2p-3$}\\
		\pi_*X\otimes \BZ[\frac{1}{p}] & \text{if $*\geq r+2p-3$}.
	\end{cases}
	\]
\end{corollary}
\begin{proof}
	Since the map $m_p$ is $(r+2p-3)$-connective, the map $X \to L_{m_p}X$ is also $(r+2p-3)$-connective. By Proposition \ref{connectivity of local equivalences of spaces}. Hence there are isomorphisms in homotopy groups
	$$
	\pi_* L_{m_p}X \cong \pi_* X
	$$
	for $*<r+2p-3$. 
	Since $L_{m_p}$ commutes with taking $(r+2p-3)$-connective cover by Lemma \ref{m_p localization commutes with k-th connective cover}, we can assume $X$ is $(r+2p-3)$-connective. 
	We claim that $L_{m_p}X\simeq L_{\BZ[\frac{1}{p}]}X$, that is, $L_{m_p}X$ is the localization of $X$ with respect to the homology theory $H_{*}(-;\BZ[\frac{1}{p}])$.
	
	Indeed, the homotopy groups of $L_{m_{p}}X$ are uniquely $p$-divisible, so $L_{m_{p}}X$ is $L_{\BZ[\frac{1}{p}]}$-local and we have a unique factorization
	\[
	\begin{tikzcd}
		X &   & L_{m_p}X\\
		  & L_{\BZ[\frac{1}{p}]}X &
	\arrow[from=1-1, to=1-3]
	\arrow[from=1-1, to=2-2]
	\arrow[from=2-2, to=1-3, dashed].
	\end{tikzcd}
	\]
By Lemma \ref{L-equivalence implies Z[1/p]-iso in spaces}, the horizontal map is an $L_{\BZ[\frac{1}{p}]}X$-equivalence.
	Hence the dashed diagonal map is also an $L_{\BZ[\frac{1}{p}]}$-equivalence, which implies that $L_{m_p}X\simeq L_{\BZ[\frac{1}{p}]}X$. Therefore, we conclude that $\pi_*L_{m_p}X=\pi_*X\otimes \BZ[\frac{1}{p}]$ for $*\geq r+2p-3$.
	
\end{proof}

\begin{corollary}
\label{Char of tame equivalence in spaces}
	A map $g:X \to Y$ is an $L_{m_p}$-equivalence if and only if the induced maps on homotopy groups satisfy:
	\begin{enumerate}
		\item  $\pi_i(X) \to \pi_i (Y)$ is an isomorphism for $i <r+2p-3$;
		\item $\pi_i(X)\otimes \BZ[\frac{1}{p}] \to \pi_i(Y)\otimes \BZ[\frac{1}{p}]$ is an isomorphism for $i \geq r+2p-3$.
	\end{enumerate}
\end{corollary}
\begin{proof}
Note that $f$ is a $L_{m_p}$-equivalence if and only if  
$$
L_{m_p}X \to L_{m_p}Y
$$
is an equivalence. By our computation of the homotopy groups in Corollary \ref{computation of the htpy groups of L_m_p X}, this is equivalent to requiring that $f$ is $(r+2p-3)$-connective and induces isomorphisms on 
	$
	\pi_{n+r+2p-3}X\otimes \BZ[\frac{1}{p}]\to \pi_{n+r+2p-3}Y\otimes \BZ[\frac{1}{p}]
	$
	for any $n\geq 0$.

\end{proof}

We conclude this section by showing that localization with respect to the map $f$ is equivalent to localization with respect to an infinite composite of localizations.
\begin{proposition}
	\label{Characterization of tame localization}
	If $f$ is the map in (\ref{f map}), then there is an equivalence
	\[
	L_f X \simeq \colim(X\to L_{m_2}X\to L_{m_3}L_{m_2}X\to \cdots)
	\]	
	in $\CS^{\geq r}_*$. 
\end{proposition}
\begin{proof}
	Let $X_{\infty}:=\colim(X\to L_{m_2}X\to L_{m_3}L_{m_2}X\to \cdots)$. 
	The space $X_\infty$ is tame; indeed, for fixed $j$, we have
	\[
	\pi_{r+j}(X_{\infty})\cong \pi_{r+j}(L_{m_q}\cdots L_{m_2}X)\cong \pi_{r+j}(X) \otimes R_{j},
	\]
	where $q$ is the largest prime number less than or equal to $\frac{j+3}{2}$. 
	Since there is also an isomorphism $\pi_{r+j}(L_f X)\cong \pi_{r+j}X\otimes R_j$, we conclude that the canonical map
	$$
	L_f X \to X_{\infty}
	$$
	is an equivalence.

\end{proof}
As a consequence, a map $g: X\to Y$ in $\CS^{\geq r}_{*}$ is a tame equivalence if and only if it is a $L_{m_p}$-equivalence for every prime $p$, which implies the following corollary.
\begin{corollary}
\label{char of tame equivalences for spaces}
        A map $g: X\to Y$ between $r$-connective spaces is a tame equivalence if and only if the induced maps
        $$
        \pi_{r+j}X \otimes R_{j} 
        \to
        \pi_{r+j}Y \otimes R_{j}
        $$
        are isomorphisms for all $j\geq 0$.
\end{corollary}

\begin{remark}
	Let $s\leq r$ be non-negative integers.
	If $X\to Y$ is an $r$-tame equivalence, then it's also an $s$-tame equivalence.
\end{remark}

We end this section with a basic example of tame equivalences.
\begin{example}
\label{odd sphere is tame equivalent to EM-space}
For $r$ an odd number,
the canonical map 
$$
S^r \to K(\BZ, r)
$$
is a tame equivalence since it is a rational equivalence.
\end{example}

\section{Tame Spectra}
\label{Section: Tame spectra}

In this section, we introduce the notion of tame spectra.
These spectra are defined analogously to tame spaces, that is, we impose divisibility conditions on their homotopy groups.
Then we establish functors which connect tame spaces and tame spectra.
Finally, we establish an algebraic characterization of the $\infty$-category of tame spectra.

As opposed to the case of spaces, we allow $r$ to be any non-negative integer, and we denote the $\infty$-category of $r$-connective spectra by $\Sp^{\geq r}$.

\begin{definition}
	A $r$-connective spectrum $X$ is \emph{$r$-tame} if the $(r+j)$-th homotopy group $\pi_{r+j}(X)$ is an $R_j$-module for $j\geq 0$, or equivalently, there are isomorphisms
	\[
	\pi_{r+j}(X)\cong \pi_{r+j}(X)\otimes R_{j}
	\]
	for all $j\leq 0$.
\end{definition}

\begin{notation}
We let $\Sp^{\geq r}_{tame}$ denote the full subcategory of $\Sp^{\geq r}$ spanned by $r$-tame spectra.
\end{notation}


Let $f$ be the map 
\begin{equation}
\label{f for spectra}
	f: \bigvee_{p\in P}\BS^{r+2p-3} \to \bigvee_{p\in P}\BS^{r+2p-3}
\end{equation}
defined by assembling the multiplication by different primes as we did for spaces in (\ref{f map}).
\begin{definition}
    A map $g:X \to Y$ between $r$-connective spectra is a \emph{tame equivalence} if it is an $L_f$-equivalence.
\end{definition}

We
summarize the results regarding tame localization of spectra and tame equivalences below. 

\begin{proposition}
        \begin{enumerate}
            \item A spectrum $X\in \Sp^{\geq r}$ is tame if and only if it is $L_f$-local.
            \item The tame localization
$$
L_{\tame}: \Sp^{\geq r} \to \Sp^{\geq r}_{\tame}
$$
exists. Moreover, the $\infty$-category $\Sp^{\geq r}_{\tame}$ of tame spectra is presentable.
\item  A map $g: X\to Y$ between $r$-connective spectra is a tame equivalence if and only if the induced maps
        $$
        \pi_{r+j}X \otimes R_{j} 
        \to
        \pi_{r+j}Y \otimes R_{j}
        $$
        are isomorphisms for all $j\geq 0$.
        \end{enumerate}
\end{proposition}
\begin{proof}
    The proofs of these statements are completely analogous to those of Lemma \ref{tame is equivalent to f-local}, Corollary \ref{existence of tame localization} and Proposition \ref{char of tame equivalences for spaces}.
    
\end{proof}

Let $\CC$ be a presentable $\infty$-category and let $S$ be a collection of morphisms in $\CC$.
We now state a closure property $S$-equivalences in $\CC$. In particular, the lemma below implies that the collection of tame equivalences is closed under colimits.
\begin{lemma}
\label{closure property of tame equivalences}
		Let $K$ be a simplicial set and
		let $f:X\to Y$ be a pointwise $S$-equivalence in $\Fun(K,\CC)$.
	Then the induced map on colimits 
		$$
		\colim_{k\in K}X_k \to \colim_{k\in K}Y_k
		$$ is an $S$-equivalence.
		
%		\item Conjecture: The collection of tame equivalences is closed under finite products and inverse limits. 
\end{lemma}
\begin{proof}
    This follows immediately from Proposition \ref{Prop 5.5.4.15. HTT}, since the collection of $S$-equivalences is strongly saturated, and hence is closed under colimits in $\Fun(\Delta^1, \CC)$.
% 	It suffices to show the induced map on colimits
% 	$\colim_{K}X_k \to \colim_{K}Y_k$ is an equivalence after applying the localization functor $L$.
% 	Since $L:\CC \to L\CC$ preserves all colimits,
% 	the map $L \colim_{K}X_k \to L \colim_{K}Y_k$ is equivalent to the map 
% 	$$
% 	\colim_{K}L X_k \to \colim_{K}L Y_k
% 	$$
% 	in $L
% 	\CC$, which is an equivalence since $L X_k \to L Y_k$ is an equivalence for each vertex $k$.
\end{proof}
% \begin{lemma}
% \label{closure property of tame equivalences}

% 		Let $K$ be a simplicial set, and let $f:X\to Y$ be a pointwise tame equivalence in $\Fun(K,\Sp^{\geq r})$.
% 	Then the induced map on colimits 
% 		$$
% 		\colim_{K}X_k \to \colim_{K}Y_k
% 		$$ is a tame equivalence.
		
% %		\item Conjecture: The collection of tame equivalences is closed under finite products and inverse limits. 

% \end{lemma}

% \begin{proof}
% 	It suffices to show the induced map on colimits
% 	$\colim_{K}X_k \to \colim_{K}Y_k$ is an equivalence after applying $L_{tame}$.
% 	Since $L_{tame}:\Sp^{\geq r}\to \Sp_{tame}^{\geq r}$ preserves all colimits,
% 	the map $L_{tame}\colim_{K}X_k \to L_{tame}\colim_{K}Y_k$ is equivalent to the map 
% 	$$
% 	\colim_{K}L_{tame}X_k \to \colim_{K}L_{tame}Y_k
% 	$$
% 	in $\Sp_{tame}^{\geq r}$, which is an equivalence since $L_{tame}X_k \to L_{tame}Y_k$ is an equivalence for each vertex $k$.
% \end{proof}
%	let $Z$ be a tame spectrum.
%Consider the following commutative diagram:	
%	\[
%	\begin{tikzcd}
%		\map_{\Sp}(\colim_{K}Y_k,Z)& \map_{\Sp}(\colim_{K}X_k,Z)\\
%		\lim_K\map_{\Sp}(Y_k,Z)
%  & \lim_K\map_{\Sp}(X_k,Z)
%		\arrow[from=1-1, to=1-2]
%		\arrow[from=2-1, to=2-2,"\simeq"]
%		\arrow[from=1-1, to=2-1, "\simeq"]
%		\arrow[from=1-2, to=2-2, "\simeq"],
%	\end{tikzcd}
%	\]
%	in which the bottom and the vertical maps are equivalences.
%Hence the top map is an equivalence and the map 
%$\colim_{K}X_k \to \colim_{K}Y_k$ is a tame equivalence.
	
%	For $(b)$, observe that for fixed $j$, we have an isomorphism
%	$$
%	\pi_{r+j}X_k\otimes R_j \xrightarrow{\simeq} 
%	\pi_{r+j}Y_k\otimes R_j 
%	$$
%	for every $k\in K'_0$. Therefore, we have
%	$$
%	\lim_{K'}(\pi_{r+j}X_k\otimes R_j) \xrightarrow{\simeq}
%	\lim_{K'}(\pi_{r+j}Y_k\otimes R_j).
%	$$
%	Since $R_j$ is torsion free, $(-)\otimes R_j$ is left exact, then we have
%	$$
%%	\pi_{r+j}(\lim_{K'} X_k)\otimes R_j\cong 
%	(\lim_{K'}\pi_{r+j}X_k)\otimes R_j
%	\cong
%	\lim_{K'}(\pi_{r+j}X_k\otimes R_j).
%	$$
%which implies $(\lim_{K'}\pi_{r+j}X_k)\otimes R_j \to (\lim_{K'}\pi_{r+j}Y_k)\otimes R_j$ is an isomorphism for every $j\geq 0$, 



We now discuss the relation between the $\infty$-category $\CS^{\geq r}_{\tame}$ of tame spaces and the $\infty$-category $\Sp^{\geq r}_{\tame}$ of tame spectra.
Since there are isomorphisms
$$
\pi_*(\Omega^{\infty}X) \cong \pi_* (X)
$$
for any spectrum $X$,
		the infinite loop space of a tame spectrum is a tame space. 
Therefore, the functor $\Omega^{\infty}:\Sp^{\geq r}\to 
\CS^{\geq r}$ restricts to a functor
$$
\Omega^{\infty}: \Sp^{\geq r}_{\tame} 
\to 
\CS^{\geq r}_{\tame}.
$$



\begin{lemma}
\label{Suspension preserves tame equivalence}
	The suspension functor $\Sigma^{\infty}:\Space^{\geq r}_{*} \to \Sp^{\geq r}$ sends tame equivalences of spaces to tame equivalences of spectra.
\end{lemma}
\begin{proof}
	Let $f:X\to Y$ be a tame equivalence in $\Space^{\geq r}_{*}$. By definition, $\Sigma^{\infty}f: \Sigma^{\infty}X \to \Sigma^{\infty}Y$ is a tame equivalence if, for every tame spectrum $Z$, the induced map
	$$
	\map_{\Sp^{\geq r}}(\Sigma^{\infty} Y, Z) \to 	\map_{\Sp^{\geq r}}(\Sigma^{\infty} X, Z)
	$$
	is a weak equivalence. Note that the map above is equivalent to the map
	$$
	\map_{\Space^{\geq r}_{*}}(Y, \Omega^{\infty} Z) \to 	\map_{\Space^{\geq r}_{*}}(X, \Omega^{\infty} Z)
	$$
	which is a weak equivalence as $\Omega^{\infty}Z$ is a tame space and $f$ is a tame equivalence by assumption.
\end{proof}




%I can't prove the following proposition, neither that I think it is correct.
%We now show that tame equivalences of spectra can also be detected on the homology level.
%\begin{proposition}
%	\label{Tame equivalence=tame homology equivalence}
%	A map of spectra $g:X\to Y$ in $\Sp^{\geq s}$ is a tame equivalence if and only if its induced map on homology $H_{s+j}(X)\otimes R_j \to H_{s+j}(Y)\otimes R_j$ is an isomorphism for $j\geq 0$. 
%\end{proposition}

	

%Is this even true?
%	\begin{proposition}
%	\label{Tame localization is symmetric monoidal}
%	The Tame localization functor 
%\[
%L_{tame}: \Sp^{\geq s}\to \Sp^{\geq s}_{tame}
%\]
%is symmetric monoidal.
%\end{proposition}
We now discuss some basic examples of tame equivalences in $\Sp^{\geq r}$.
\begin{example}
\label{C_tame of suspension of Eilenberg-Maclane space}
\begin{enumerate}
    \item For any integer $r\geq 0$, the $r$-truncation map
$$
\BS^{r} \to \Sigma^r H\BZ
$$
is an $r$-tame equivalence.
	Indeed, the first $p$-torsion in the homotopy groups of the shifted sphere spectrum $\BS^{r}$ appears in degree $r+2p-3$, hence its $r$-tame localization is the shifted Eilenberg-MacLane spectrum $\Sigma^r H\BZ$. 
	This example indicates that $\Sp^{\geq r}_{\tame}$ is generated by $\Sigma^{r}H\BZ$ under colimits, since $\Sp^{\geq r}$ is generated by $\BS^r$ under colimits.
	
	\item Let $r\geq 3$ be an odd integer, then the canonical map
	$$
	S^{r} \to K(\BZ, r) 
	$$
	is an $r$-tame equivalence by Example \ref{odd sphere is tame equivalent to EM-space}. Then Lemma \ref{Suspension preserves tame equivalence} implies that 
	$$
	\BS^{r} \to 
	\Sigma^{\infty}K(\BZ, r) 
	$$
	is a tame equivalence.
	Combining with the previous example, we conclude that the $r$-truncation map
	$$
	\Sigma^{\infty}K(\BZ, r) \to
	\Sigma^{r} H\BZ
	$$
	is also a tame equivalence. Since $\Sigma^{r} H\BZ$ is $r$-tame, hence we have 
	$$
	L_{\tame}\Sigma^{\infty}K(\BZ, r)
	\simeq
	\Sigma^{r} H\BZ.
	$$
\end{enumerate}
\end{example}

We now show that the class of $r$-tame equivalences is closed under smashing with a connective spectrum. 
\begin{lemma}
\label{Shift of tame equivalences}
Let $r\geq 0$ and
let $E$ be a connective spectrum. If $f:X\to Y$ is an $r$-tame equivalence in $\Sp^{\geq r}$, then $E\otimes X \to E\otimes Y$ is an $r$-tame equivalence in $\Sp^{\geq r}$.
\end{lemma}
\begin{proof}
Let $\CE$ be the full subcategory of $\Sp^{\geq 0}$ spanned by spectra $F$ satisfying the condition that
$$
F\otimes X \to F\otimes Y
$$
is an $r$-tame equivalence whenever $X\to Y$ is an $r$-tame equivalence. Clearly, the sphere spectrum $\BS$ is in $\CE$. 
Since $\Sp^{\geq 0}$ is generated by $\BS$ under colimits, the claim will follow if we can show $\CE$ is closed under colimits. Let $Z:K\to \CE$ be a diagram in $\CE$.
Note that by Lemma \ref{closure property of tame equivalences} and the fact that smash product commutes with colimits in $\Sp^{\geq 0}$, we conclude that
$$
(\colim_K Z_k )\otimes X \to (\colim_K Z_k )\otimes Y
$$
is an $r$-tame equivalence.

%Fix some $j\geq 0$, we claim the map $\pi_{r+j}(E\otimes X)\otimes R_j \to \pi_{r+j}(E \otimes Y)\otimes R_j$ is an isomorphism. We let $F$ denote the fiber of the induced map $\tau_{\leq r+j}X \to \tau_{\leq r+j}Y$. Observe that $R_j$ is a flat, we have a long exact sequence
%$$
%0 \to \pi_{r+j}F \otimes R_j \to \pi_{r+j}X \otimes R_j  \to \pi_{r+j}Y \otimes R_j  \to \pi_{r+j-1}F \otimes R_j \to \cdots.
%$$
%Since $f:X \to Y$ is a tame equivalence, we conclude that $F$ is $(r+j)$-truncated and $\pi_{r+k}F\otimes R_j=0$ for $0\leq k \leq j$. Under the terminology of Bousfield localization, we say the spectrum $F$ is $\BS R_j$-acyclic, where $\BS R_j$ denotes the Moore spectrum associated to $R_j$.
%Tensoring with $E$ preserves fiber sequence, the lemma would follow if we can show $E\otimes F$ is also $\BS R_j$-acyclic, which follows from the fact that $\BS R_j$-localization is symmetric monoidal.
\end{proof}


\begin{remark}
\label{symmetric monoidal structure on tame spectra}
\begin{enumerate}
 	\item As a direct consequence of Lemma \ref{Shift of tame equivalences}, if $X\to Y$ is an $r$-tame equivalence, then $X\otimes H\BZ \to Y\otimes H\BZ$ is also an $r$-tame equivalence. Hence, for every $k\geq 0$, the induced map on homology
 	\[
 	H_{r+k}(X)\otimes R_{k}
 	\to
 	H_{r+k}(Y)\otimes R_{k}
 	\]
 	is an isomorphism.
	\item Lemma \ref{Shift of tame equivalences} above implies that $r$-tame localization is compatible with the smash product in $\Sp^{\geq r}$ in the sense of \cite[Definition 2.2.1.6.]{HA}. By \cite[Proposition 2.2.1.9.]{HA}, the $\infty$-category $\Sp^{\geq r}_{\tame}$ of tame spectra admits a (nonunital) symmetric monoidal structure given by 
$$
X\hat{\otimes} Y := L_{\tame}(X\otimes Y).
$$
Moreover, the tame localization 
$$
L_{\tame}: \Sp^{\geq r} \to 
\Sp^{\geq r}_{\tame}
$$
is symmetric monoidal, hence the tensor product $\hat{\otimes}$ preserves colimits in each variable.
\end{enumerate}

\end{remark}

\begin{corollary}
\label{HZ commutes with tame localization}
	Let $E\in \Sp^{\geq r}$. Then the map
	$$
	E \simeq E\otimes \BS  \to E\otimes H\BZ
	$$ 
	is an $r$-tame equivalence.
\end{corollary}
\begin{proof}
	This is immediate from Lemma \ref{Shift of tame equivalences}, since $\BS \to H\BZ$ is a $0$-tame equivalence, and $E$ is $r$-connective.
\end{proof}


\begin{remark}
One might ask whether $H\BZ$-module spectra are $r$-tame for all $r$, but this is false. For instance, $\Sigma H\BZ$ is a $1$-connective $H\BZ$-module but it's not $0$-tame, as $\pi_{1}\Sigma H\BZ\cong \BZ$ is not uniquely $2$-divisible.
\end{remark}

We now give an algebraic characterization of tame spectra. 
Let $\Mod_{H\BZ}$ denote the $\infty$-category of $H\BZ$-modules. Note that $\Mod_{H\BZ}$ can be identified with the derived $\infty$-category $D(\BZ)$ by \cite[Remark 7.1.1.16.]{HA}.
We let $(\Mod^{\geq r}_{H\BZ})_{\tame}$ denote the full subcategory of $\Mod^{\geq r}_{H\BZ}$
% full subcategory of $\Mod_{H\BZ}$ 
spanned by $r$-connective $H\BZ$-modules whose underlying spectra are tame.

%%% Added a construction of tame localization of 
\begin{construction}
We now explain how to realize $(\Mod^{\geq r}_{H\BZ})_{\tame}$ as a localization of 
$\Mod^{\geq r}_{H\BZ}$. 
We first remark that this procedure is completely analogous to the case of spaces and spectra.
Similar to (\ref{f for spectra}), we assemble the multiplication-by-$p$ maps
$
m_p: \Sigma^{r+2p-3}H\BZ
\to 
\Sigma^{r+2p-3}H\BZ
$
to a map
\begin{equation}
	f: \bigvee_{p\in P}\Sigma^{r+2p-3} H\BZ \to \bigvee_{p\in P}\Sigma^{r+2p-3} H\BZ.
\end{equation}
By definition, $X\in \Mod^{\geq r}_{H\BZ}$ is $f$-local if $$
\map_{\Mod_{H\BZ}}(\bigvee_{p\in P}\Sigma^{r+2p-3} H\BZ, X) 
\to 
\map_{\Mod_{H\BZ}}(\bigvee_{p\in P}\Sigma^{r+2p-3} H\BZ, X) 
$$
is a weak equivalence. 
Since we have an equivalence
$$
\map_{\Mod_{H\BZ}}(\Sigma^{r+2p-3} H\BZ, X)
\simeq 
\map_{\Sp}(\BS^{r+2p-3}, X),
$$
$X$ is $f$-local if and only if its underlying spectrum is tame, i.e., $X\in (\Mod^{\geq r}_{H\BZ})_{\tame}$. By Proposition \ref{Prop 5.5.4.15. HTT}, there is a localization functor 
$$
\Mod^{\geq r}_{H\BZ} \to
(\Mod^{\geq r}_{H\BZ})_{\tame}
$$
which we will again denote by $L_{\tame}$. We will refer to $L_{\tame}$-equivalences in $\Mod^{\geq r}_{H\BZ}$ as \emph{tame $H\BZ$-equivalences}.

It is clear from the definition of the $\infty$-category $(\Mod^{\geq r}_{H\BZ})_{\tame}$ that a map $f:X \to Y$ in $\Mod^{\geq r}_{H\BZ}$ is a tame $H\BZ$-equivalence if it is a tame equivalence in $\Sp^{\geq r}$.

\begin{remark}
Let $f:M\to M'$ be a tame $H\BZ$-equivalence, then 
$$
f\underset{\BZ}{\otimes} \id:
M\underset{\BZ}{\otimes} P \to M' \underset{\BZ}{\otimes} P
$$
is a tame equivalence for any quasi-free (cf. \cite[Definition 7.2.1.16.]{HA}) $H\BZ$-module $P$, since tame equivalences is compatible with tensor product by Lemma \ref{Shift of tame equivalences}.
Suppose $N$ is an arbitary $H\BZ$-module with a quasi-free resolution $P_{\bullet}$ (cf. \cite[Remark 7.2.1.21.]{HA}), then the map
$$
f\underset{\BZ}{\otimes} N: M\underset{\BZ}{\otimes} N \to 
M'\underset{\BZ}{\otimes} N
$$
induces a tame $H\BZ$-equivalence of simplicial $H\BZ$-modules
$
M \underset{\BZ}{\otimes} P_{\bullet}
\to 
M' \underset{\BZ}{\otimes} P_{\bullet},
$
which implies 
$
f\underset{\BZ}{\otimes} N
$
is a tame $H\BZ$-equivalence.
Therefore, tame $H\BZ$-equivalences are compatible with the symmetric monoidal structure on $\Mod_{H\BZ}^{\geq r}$ and we can equip $(\Mod^{\geq r}_{H\BZ})_{\tame}$ with a symmetric monoidal structure given by
$$
M\hat{\otimes} N \simeq L_{\tame} (M\underset{\BZ}{\otimes} N).
$$
\end{remark}


\end{construction}
Note that the functor 
$$
L_{\tame}(-\otimes H\BZ): \Sp^{\geq r} \to 
(\Mod^{\geq r}_{H\BZ})_{\tame}
$$
is left adjoint to the forgetful functor $U: (\Mod^{\geq r}_{H\BZ})_{\tame} \to \Mod^{\geq r}_{H\BZ}
\to \Sp^{\geq r}
$. Moreover, the forgetful functor has essential images in $\Sp^{\geq r}_{\tame}$, hence by Proposition \ref{Restrict adjoints to full subcategory},
we obtain a pair of adjoint functors
\[
\adj{L_{\tame}(-\otimes H\BZ)}{\Sp^{\geq r}_{\tame}}{(\Mod^{\geq r}_{H\BZ})_{\tame}}{U}.
\]
We can now state and prove Theorem \ref{Theorem A} stated in the introduction.
\begin{theorem}
	\label{algebraic description of tame spectra}
	The functor
	$$
	L_{\tame}(-\otimes H\BZ):
	\Sp^{\geq r}_{\tame} \to (\Mod^{\geq r}_{H\BZ})_{\tame}
	$$
	is a symmetric monoidal equivalence of $\infty$-categories.
\end{theorem}
\begin{proof}
    We first show $L_{\tame}(-\otimes H\BZ)$ is an equivalence of $\infty$-categories.
    Since the forgetful functor $U$ is conservative, it suffices to show $L_{\tame}(-\otimes H\BZ)$ is fully faithful.
	Let $X\in \Sp^{\geq r}_{\tame}$ be a tame spectrum. The map 
	$$
	X\otimes \BS \to X\otimes H\BZ
	$$ is an $r$-tame equivalence by Corollary \ref{HZ commutes with tame localization}
	and 
	$$
	H\BZ \to L_{tame}(X\otimes H\BZ)
	$$
	is a tame equivalence since it's a tame $H\BZ$-equivalence
	. 
	Hence the composite 
	$$
	X \to X\otimes H\BZ \to L_{tame}(X\otimes H\BZ)
	$$ is an equivalence as both $X$ and $L_{\tame}(X\otimes H\BZ)$ are tame.
	
	To show $L_{\tame}(-\otimes H\BZ)$ is symmetric monoidal, by \cite[Proposition 2.2.1.9.]{HA}, it remains to check tame equivalences is compatible with the tensor product on $\Sp^{\geq r}_{\tame}$, which is the content of Lemma \ref{Shift of tame equivalences}.
	
	
% 	for every pair of tame spectra $X,Y$, there is an equivalence
% 	$$
% 	L_{\tame}(L_{\tame}(X\otimes H\BZ) \underset{\BZ}{\otimes} 
% 	L_{\tame} (Y\otimes H\BZ))
% 	\simeq
% 	L_{\tame}(L_{\tame} (X\otimes Y) \otimes H\BZ)
% 	$$
% 	in $\Sp^{\geq r}$; indeed, 
	
% 	Conversely, let $Y\in (\Mod^{\geq r}_{H\BZ})_{\tame}$ be a tame $H\BZ$-module. The map $Y\otimes \BS \to  Y\otimes H\BZ \to L_{\tame}(Y\otimes H\BZ)$ is a map in $\Mod^{\geq r}_{H\BZ}$. Since its underlying map of spectra is a tame equivalence, hence it is a tame $H\BZ$-equivalence of $H\BZ$-modules.
% 	% need to show the composite is an equivalence in the category of tame HZ-modules.
% % 	The map
% % 	$Y\otimes H\BZ \to Y$
% % 	is a map in $\Mod^{\geq r}_{H\BZ}$.
% 	Therefore, we have shown the inclusion 
% 	$
% 	(\Mod^{\geq r}_{H\BZ})_{\tame} 
% 	\to 
% 	\Sp^{\geq r}_{\tame}
% 	$
% 	is an inverse of $L_{\tame}(-\otimes H\BZ)$.
	\end{proof}
	

As a direct corollary, we see that the tame homotopy type of a spectrum is determined by its homology with coefficients in the tame ring system.
\begin{corollary}
\label{htpy groups of tame spectra can be computed by homology}
	If $X\in \Sp^{\geq r}_{\tame}$ is a tame spectrum, then 
	\[
	\pi_{r+k}X \cong H_{r+k}X \otimes R_k
	\]
	for every $k\geq 0$. Therefore, a map $g:X\to Y$ in $\Sp^{\geq r}_{\tame}$ between tame spectra is an equivalence if and only if the induced map on homology with coefficients in the tame ring system
	$$
	H_{r+k}(X)\otimes R_k \to H_{r+k}(Y)\otimes R_k
	$$ 
	is an isomorphism for all $k\geq 0$.    
\end{corollary}
	
\begin{proof}
	Using Theorem \ref{algebraic description of tame spectra}, we compute
	\begin{align*}
		\pi_{r+k}X  &\cong \pi_{r+k}L_{\tame}(X\otimes H\BZ)\\
		& \cong \pi_{r+k}(X\otimes H\BZ)\otimes R_k\\
		& \cong H_{r+k}(X)\otimes R_k
	\end{align*}
\end{proof}

\begin{corollary}
\label{homotopy groups of tame spectra are hmlg with coeff in tame ring system}
		For any spectrum $X\in \Sp^{\geq r}$, 
\begin{align*}
	\pi_{r+k} L_{tame}X & \cong H_{r+k} L_{tame}X \otimes R_{k}\\
		& \cong H_{r+k} X \otimes R_{k}
\end{align*}
for any $k\geq 0$.

\end{corollary}
\begin{proof}
	The first isomorphism follows from corollary \ref{htpy groups of tame spectra can be computed by homology} and the second isomorphism follows from the fact that tensoring with $H\BZ$ preserves $r$-tame equivalences.
	
\end{proof}

\begin{remark}
\label{suspension tame sends Moore space to EM-space}
 As an immediate consequence of Corollary \ref{homotopy groups of tame spectra are hmlg with coeff in tame ring system}, the functor $L_{\tame}\Sigma^{\infty}:\Space^{\geq r}_* \to \Sp^{\geq r}_{tame}$ sends a Moore space $M(V,r+k)$ for $V$ an $R_k$-module to a shifted  Eilenberg-Maclane spectrum $\Sigma^{r+k}HV$, i.e., 
 $$
 L_{\tame} \Sigma^{\infty}M(V, r+k) \simeq \Sigma^{r+k} HV.
 $$
 
\end{remark}
%Recall that an $\infty$-category $\mathcal{C}$ has an \emph{essentially colimit dense subcategory} if there is an essentially small full subcategory $\mathcal{C}_{0} \subset \mathcal{C}$ such that $\CC$ is generated by $\CC_0$ under colimits. Note that the $\infty$-category $\Sp^{\geq s}_{tame}$ of $s$-tame spectra is generated by $\{L_{tame}\BS^{n}\}_{n\geq s}$ under colimits, hence $\Sp^{\geq s}_{tame}$ is essentially colimit dense.
%The existence of the right adjoint of the forgetful functor 
%$$
%\coCAlg(\Sp^{\geq s}_{tame})
%\to \Sp^{\geq s}_{tame}
%$$
%is guaranteed by the following.
%
%\begin{theorem}
%\cite{AFT}
%	 Let $\mathrm{C}$ be a locally small cocomplete $\infty$-category and $\mathcal{D}$ a locally small $\infty$-category. Suppose that $\mathrm{C}$ has an essentially small colimit-dense subcategory. Then a functor $F: \mathrm{C} \rightarrow \mathcal{D}$ is a left adjoint if and only if it preserves small colimits.
%\end{theorem}
%
%\begin{corollary}
%	The forgetful functor $\coCAlg(\Sp^{\geq s}_{tame})
%\to \Sp^{\geq s}_{tame}$ admits a right adjoint which we will denote by $\cofree_{tame}$.
%\end{corollary}
\begin{notation}
Since $\Sigma^{\infty}:\CS^{\geq r}_* \to \Sp^{\geq r}_{\tame}$ preserves tame equivalences by Lemma \ref{Suspension preserves tame equivalence}, we obtain a canonical lift by the universal property of the tame localization
\[
	\begin{tikzcd}
		\CS^{\geq r}_* &   & \Sp^{\geq r}_{\tame}\\
		  & \CS^{\geq r}_{\tame} &
	\arrow[from=1-1, to=1-3, "L_{\tame}\Sigma^{\infty}"]
	\arrow[from=1-1, to=2-2, "L_{\tame}" ]
	\arrow[from=2-2, to=1-3, dashed].
	\end{tikzcd}
	\]
We let $\Sigma^{\infty}_{\tame}$ denote the resulting functor
$$
\Sigma^{\infty}_{\tame}:
\CS^{\geq r}_{\tame}
\to 
\Sp^{\geq r}_{\tame}.
$$
\end{notation}



We now establish a pair of adjoint functors between tame spaces and tame spectra.

\begin{proposition}
	There is an adjoint pair
	\[
	\adj{\Sigma^{\infty}_{\tame}}{\Space^{\geq r}_{\tame}}{\Sp^{\geq r}_{\tame}}{\Omega^{\infty}}
	\]
	% \coAlg_{\Sigma^{\infty}_{tame}\Omega^{\infty}}(\Sp)^{\geq r}
\end{proposition}
\begin{proof}
	We first note that $\adj{L_{\tame}\Sigma^{\infty}}{\Space_{*}^{\geq r}}{\Sp^{\geq r}_{tame}}{\Omega^\infty}$ is an adjoint pair, where we abuse notation by writing $\Omega^{\infty}$ for the composite  $$
	\Sp^{\geq r}_{\tame}\hookrightarrow \Sp^{\geq r} \xrightarrow{\Omega^\infty} \Space^{\geq r}_{*}.
	$$ 
	Indeed, $\Sigma^{\infty}_{\tame}$ is equivalent to the composition of $L_{\tame}$ and $\Sigma^\infty$, which are left adjoint to the inclusion functor $\Sp^{\geq r}_{\tame}\hookrightarrow \Sp^{\geq r}$ and $\Omega^\infty$, respectively. The statement then follows from Proposition \ref{Restrict adjoints to full subcategory} and the fact that $\Space_{\tame}^{\geq r}$ is a full subcategory of $\Space_{*}^{\geq r}$. 
	
\end{proof}

Let $G$ be a finite group.
For any preadditive $\infty$-category $\CC$ (see Definition \ref{preadditive}) with finite limits and colimits, there is a norm natural transformation constructed in \cite[\S 6.1.6]{HA} and \cite[Definition I.1.10]{Nikolaus-Scholze}:
$$
\operatorname{Nm}: (-)_{hG} \to (-)^{hG}
$$
from the homotopy orbits functor $(-)_{hG}:\Fun(BG,\CC)\to \CC$ to the homotopy fixed points functor $(-)^{hG}:\Fun(BG,\CC)\to \CC$.
The \emph{Tate construction} of an object $X\in \Fun(BG,\CC)$ is defined as the cofiber of the norm map:
$$
X^{tG} := \cofib(\operatorname{Nm}_X:(X)_{hG} \to (X)^{hG}).
$$
If $G=\Sigma_n$ and $\CC=\Sp$, then the norm map of a $\Sigma_n$-spectrum $X$
$$
\operatorname{Nm}_X: X_{h\Sigma_n} \to X^{h\Sigma_n}
$$
is an equivalence if $n!$ is invertible in the homotopy groups of $X$.

For any $X\in \Sp^{\geq r}$, we can identify $L_{\tame}X^{\otimes n}$ as an object in $\Fun(B\Sigma_n, \Sp^{\geq r}_{\tame})$.
We end this chapter with the Tate vanishing property of tame spectra.
\begin{lemma}
\label{Tate vanishing for tame spectra}
    The Tate construction of a tame spectrum vanishes, i.e., for all $X\in \Sp^{\geq r}$
	$$
	\big(L_{\tame}(X^{\otimes n})\big)^{t\Sigma_n}\simeq *
	$$
	for all $n\geq 2$. Therefore, 
	$$
	\big(L_{\tame}(X^{\otimes n})\big)_{h\Sigma_n}
	\simeq
	\big(L_{\tame}(X^{\otimes n})\big)^{h\Sigma_n}.
	$$
\end{lemma}
\begin{proof}
    If $X$ is $r$-connective, then its $n$-th tensor power $X^{\otimes n}$ is $rn$-connective. We claim that the homotopy groups of the tame spectrum $L_{\tame}(X^{\otimes n})$ are uniquely $n!$-divisible, from which the conclusion follows. Indeed, since   
    $$
    nr-2n-r+3 = (r-2)(n-1)+1\geq 0
    $$
    for $r\geq 3$, it follows that $\pi_{*}L_{\tame}(X^{\otimes n})$ is uniquely $k$-divisible for all $k\leq n$ in all degrees.
\end{proof}