\chapter{Tame Spaces and Tame Spectra}

\section{Introduction}
The homotopy theory without torsions is fairly simple, as it is the context of the rational homotopy theory.
However, the homotopy theory of spaces localized at a prime $p$ is much more complicated than the rational case. 

However, it turns out that we do know when the $p$-torsions appears in the homotopy group of spheres. A careful computation of the homology of Eilenberg-Maclane spaces and Serre spectral sequence would lead to the following famous theorem by Serre.
\begin{theorem}[Serre]
	For $n\geq 3$, the first $p$-torsion in the homotopy group $\pi_{k}(S^{n})$ appears in degree $ n+2p-3$.
\end{theorem}

The goal of this note is to answer the following two questions:
\begin{enumerate}
	\item Can we focus on studying those spaces which are equivalent to the spheres modulo \say{asymptotic behavior}?
    \item Is this kind of spaces has an easy algebraic description?
\end{enumerate}

Dwyer \cite{Dwyer} call the spaces in the first question \emph{tame spaces}.
He also showed the homotopy theory of tame spaces is equivalent to the homotopy theory of certain integral dg Lie algebras.

Our goal is to use modern technology to upgrade Dwyer's algebraic model for tame spaces.
We will show the \emph{tame homotoy type} of a space is completely characterize by its essential image under the free spectral Lie algebra functor. 
 In addition, we also provide a coalgebra model for tame spaces.
 \todo[inline]{Essentially, the tame homotopy type is completely determined by its essential image under these functors. Maybe we can do some computations with them?}



Through out this note, We fix an integer $r\geq 3$ and a sequence of rings $R_{j}:=\BZ[\frac{1}{k}|k\leq \frac{j+3}{2}]$ for $j\geq 0$. The integer $r$ will serve as the lower bound for the connectivity of a space. 

\section{Tame Spaces}

\subsection{Tame localization}

\begin{definition}
	\label{Tame spaces}
	Let $X\in \pSpace^{\geq r}$ be a $r$-connective space.
	We say $X$ is \emph{tame} if 
	\[
	\pi_{r+j}(X)\cong \pi_{r+j}(X)\otimes R_{j}.
	\]
	That is to say, the $(r+j)$-th homotopy group $\pi_{r+j}(X)$ of $X$ is uniquely $p$-divisible for $p\leq \frac{j+3}{2}$. We let $\Space^{\geq r}_{tame}$ denote the full subcategory spanned by $r$-connective tame spaces.
\end{definition}

\begin{remark}
Note that the condition of being tame depends on the connectivity bound $r$ we choose, but we will mostly work in the context when $r$ is fixed.
	For the rest of this section, we will avoid mentioning the lower bound of the connectivity of the tame spectrum as it will be clear from our notation.	
	\end{remark}
Given a space $X\in \pSpace^{\geq r}$, we hope to produce a space $X_{tame}$ that is in some sense the universal tame space associated to $X$.
The theory of localization of Bousfield and Farjoun gives a modern way to streamline this procedure.
\todo{We should give a brief introduction of Bousfield localization in the preliminaries}

To illurstrate this idea, we first give an equivalent characterization of a space being tame.
Let's first recall a simple fact of localization with respect to the degree $p$ map.
Let $g:S^n \xrightarrow{p}S^n$ be a map of degree $p$ for $n\geq r$. A space $X$ is said to be \emph{$g$-local} if $f$ induces a weak homotopy equivalence 
\[
\map_*(S^n,X) \to \map_*(S^n,X)
\]
on the mapping space. Observe that this is equivalent to requiring the multiplication by $p$ map
\[
\pi_{*}(X) \xrightarrow{\cdot p} \pi_{*}(X)
\]
to be an isomorphism for $*\geq n$. In other words, a space $X$ is $g$-local if and only if its homotopy groups $\pi_*(X)$ is uniquely $p$-divisible for degree large and equal to $n$.

Let $P$ denote the set of primes and fix a prime $q$, we consider the following composition 
\[
\bigvee_{p \in P}S^{r+2p-3} \xrightarrow{\text{collapsing}} S^{r+2q-3} 
\xrightarrow{\cdot q} S^{r+2q-3}
\]

Assembling these maps together, we obtain a map
\begin{equation}
\label{f}
	f: \bigvee_{p\in P}S^{r+2p-3} \to \bigvee_{p\in P}S^{r+2p-3}.
\end{equation}


\begin{remark}
	The map $f$ can be seen as a diagonal matrix with primes on the diagonal. \todo{how to elobrate this?}
\end{remark}

\begin{lemma}
	A space $X\in \pSpace^{\geq r}$ is tame if and only if it is $f$-local.
\end{lemma}
\begin{proof}
	A space $X$ is $g$-local if and only if the map
	\[
	\map_{*}(\bigvee_{p\in P}S^{r+2p-3},X) \xrightarrow{g^*} \map_{*}(\bigvee_{p\in P}S^{r+2p-3},X)
	\]
	is a weak homotopy equivalence. This is equivalent to asking the homotopy groups $\pi_{r+j}(X)$ are uniquely $p$-divisible for those primes $p\leq \frac{j+3}{2}$ for every fixed $j\geq 0$. 	The lemma now follows from the following commuting diagram
	\[	
	\begin{tikzcd}
		\pi_{r+j}(X) & \pi_{r+j}(X)\\
		\pi_{j-(2p-2)}(\map_*(S^{r+2p-2},X)) & \pi_{j-(2p-2)}(\map_*(S^{r+2p-2},X))
		\arrow[from=1-1, to=1-2,"\cdot p"]
		\arrow[from=2-1, to=2-2,"\cdot p"]
		\arrow[from=1-1, to=2-1]
		\arrow[from=1-2, to=2-2].
	\end{tikzcd}
	\]
	where the vertical arrows are isomorphisms.
\end{proof}


\begin{remark}
By the theory of Bousfield localzation, we conclude that the tame localization functor $L_{tame}$ agrees with $L_f$.	
\end{remark}

\begin{definition}
	\label{Tame equivalences}
	A map $g:X\to Y$ in $\Space^{\geq r}_{*}$ is a \emph{tame equivalence} if for any tame space $Z$, the induced map
	\[
	\map_*(B,Z) \xrightarrow{g^*} \map_*(A,Z)
	\]
	is an equivalence.
\end{definition}

In the following, we give an explicit description of tame equivalences and show the functor $L_{tame}:\Space^{\geq r}_* \to \Space^{\geq r}_{tame}$ is an infinite composite of localization functors.

For every prime $p$, we let $f_p:S^{r+2p-3}\to S^{r+2p-3}$ denote the degree $p$ map. A space is $f_p$-local if and only if its homotopy groups are uniquely $p$-divisible above degree $r+2p-2$.

By the theory of localization developed by Bousfield and Farjoun, we have a localization functor
\[
L_{f_p}:\Space^{\geq r}_{*} \to \Space^{\geq r}_{*}.
\]
A map is a \emph{$L_{f_p}$-equivalence} if the induced map
\[
	\map_*(Y,Z) \xrightarrow{f_{p}^*} \map_*(X,Z)
\]
is a homotopy equivalence for every $f_p$-local space $Z$. We now want to give an explicit identification of the space $L_{f_p}X$ under the localization functor $L_{f_p}$. For this, we need a few lemma.

\begin{lemma}
	If $g:A\to B$ is $n$-connective, then
	\[
	\pi_{i}(L_gX) \cong \pi_i(X)
	\]
	for $i< n$.
\end{lemma}
\begin{proof}
	
	
	
\end{proof}

\begin{lemma}
	The localization functor $L_{f_p}$ commutes with connective cover.
\end{lemma}
\begin{proof}
	
	
	
\end{proof}

\begin{proposition}
	Let $X\in \pSpace$, then we have 
	\[
	\pi_{i}(L_{f_p}X)= \begin{cases}
		\pi_{i}(X) \otimes \BZ[\frac{1}{p}] &\text{for $i\geq n$},\\
		\pi_{i}(X) &\text{for $i< n$.}
	\end{cases}
	\]
\end{proposition}
\begin{proof}
	
\end{proof}

\begin{corollary}
\label{Char of tame equivalence}
	A map $g:X \to Y$ is a $L_{f_p}$-equivalence if and only if the induced maps on homotopy groups satisfies:
	\begin{enumerate}
		\item  $\pi_i(X) \to \pi_i (Y)$ is an isomorphism for $i <n$,
		\item $\pi_i(X)\otimes \BZ[\frac{1}{p}] \to \pi_i(Y)\otimes \BZ[\frac{1}{p}]$ is an isomorphism for $i \geq n$.
	\end{enumerate}
	Moreover, $g:X\to Y$ is a tame equivalence if and only if it is a $L_{f_p}$-equivalence for every prime $p$.
\end{corollary}

We conclude this section by showing that the localization with respect to the map $f$ is equivalent to the localization with respect to the infinite composites of localizations.
\begin{corollary}
	\label{Characterization of tame localization}
	Let $f$ be as in (\ref{f}), then we have
	\[
	L_f X \simeq \colim(X\to L_{f_2}X\to L_{f_3}L_{f_2}X\to \cdots)
	\]	
\end{corollary}
\begin{proof}
	We define $X_{\infty}:=\colim(X\to L_{f_2}X\to L_{f_3}L_{f_2}X\to \cdots)$. 
	The space $X_\infty$ is tame. Indeed, for fixed $j$, we have
	\[
	\pi_{r+j}(X_{\infty})\cong \pi_{r+j}(L_{f_q}\cdots L_{f_2}X)\cong \pi_{r+j}(X) \otimes R_{j}
	\]
	where $q$ is the largest prime number less or equal to $\frac{j+3}{2}$. Moreover, the natural map $X\to X_{\infty}$ is a tame equivalence by Corollary \ref{Char of tame equivalence}. Therefore, the map $X\to X_{\infty}$ is a tame localization.
\end{proof}
\subsection{Tame spectra}

In this section, we discuss the tame homotopy theory for spectra. Again, we fix an integer $s\in \BZ_{\geq 0}$ and we will denote the $\infty$-category of $s$-connective spectra by $\Sp^{\geq s}$.
\begin{definition}
	An $s$-connective spectrum $X$ is \emph{$s$-tame} if 
	\[
	\pi_{s+j}(X)\cong \pi_{s+j}(X)\otimes R_{j}.
	\]
	We let $\Sp^{\geq s}_{tame}$ denote the full subcategory spanned by $s$-tame spectra.
\end{definition}

\begin{remark}
	Note that unlike in the case of spaces, we could consider $0$-connective tame spectra in the stable case.
	The key observation is that the first $p$-torsion in the homotopy group of the sphere spectrum $\BS$ appears at degree $2p-3$.
\end{remark}

\subsection{Tame equivalences in $\Sp$}
We identify the existence of tame localization of spectra in this section. Moreover, we characterize tame equivalences as those maps inducing isomorphisms after tensoring with tame ring system.
Let $m_p:\BS^n\to \BS^n$ denote the multiplication-by-$p$ map on $\BS^n$. A connective spectrum $X$ is $m_p$-local if 
\[
\map_{\Sp}(\BS^n,X) \xrightarrow{m_p^*} \map_{\Sp}(\BS^n,X)
\]
is a weak homotopy equivalence.
\begin{lemma}
\label{local w.r.t. multiplication-by-p map}
	A spectrum $X$ is $m_p$-local if and only if $\pi_{n+k}(X)$ is uniquely $p$-divisible for $k\geq 0$.
\end{lemma}
\begin{proof}
	We compute the induced map on the homotopy groups of $\map_{\Sp}(\BS^n,X)$. Note that for $k\geq 0$, one has
	\begin{align*}
		\pi_k\map_{\Sp}(\BS^n,X) & \cong \pi_0 \Omega^k\map_{\Sp}(\BS^n,X)\\
		     					 & \cong \pi_0 \map_{\Sp}(\BS^{n+k},X)\\
		     					 & \cong \pi_{n+k} (X),
	\end{align*}
	and $m_p^*$ induces a multiplication-by-$p$ map $ \pi_{n+k} (X) \xrightarrow{\cdot p} \pi_{n+k} (X)$. Therefore, $X$ is $m_p$-local if and only if $\pi_{n+k}X$ is uniquely $p$-divisible for all $k\geq 0$.
\end{proof}
Let $\Sp_{m_p}$ denote the full subcategory of spectra whose homotopy groups are uniquely $p$-divisible above degree $n$.
Note that the localization functor $L_{m_p}:\Sp \to \Sp_{m_p}$ exists by Proposition 5.5.4.15. of \cite{HTT}.
Let's now compare $L_{m_p}$ with the Bousfield localization with respect to the homology theory $H_{*}(-,\BZ[\frac{1}{p}])$. A spectrum is $\BZ[\frac{1}{p}]$-local if and only if its homotopy groups are uniquely $p$-divisible \cite[Proposition 2.4]{BousfieldSpectra}. Therefore, a $\BZ[\frac{1}{p}]$-local spectrum is also $L_{m_p}$-local and we record the following useful lemma.
\begin{lemma}
\label{L-equivalence implies Z[1/p]-iso}
	If $f:X\to Y$ is a $L_{m_{p}}$-equivalence of spectra, then it induces an isomorphism on 
	$$
	\pi_{*}(X)\otimes \BZ[\frac{1}{p}]
	\to
	\pi_{*}(Y)\otimes \BZ[\frac{1}{p}].
	$$
\end{lemma}

The inverse is true if we restricts to connective spectra.
\begin{lemma}
\label{L-equivalence is equivalent to pi_* iso on n-conn spectra}
	Let $f:X\to Y$ be a map between $n$-connective spectra, then $f$ is a $L_{m_{p}}$-equivalence if and only if it induces an isomorphism on 
	$
	\pi_{*}(X)\otimes \BZ[\frac{1}{p}]
	\to
	\pi_{*}(Y)\otimes \BZ[\frac{1}{p}].
	$
\end{lemma}
\begin{proof}
	Let $Z$ be a $L_{m_p}$-local spectrum and let $f$ be a $\BZ[\frac{1}{p}]$-equivalence. Consider the commutative diagram
	\[
	\begin{tikzcd}
		\map_{\Sp}(Y,Z) & \map_{\Sp}(X,Z)\\
		\map_{\Sp}(Y,\tau_{\geq n}Z) & \map_{\Sp}(X,\tau_{\geq n}Z)
		\arrow[from=1-1, to=1-2]
		\arrow[from=2-1, to=2-2]
		\arrow[from=1-1, to=2-1,"\simeq"]
		\arrow[from=1-2, to=2-2,"\simeq"]
	\end{tikzcd}
	\]
	where the vertical maps are equivalences. Since the homotopy groups of $\tau_{\geq n}Z$ are uniquely $p$-divisible, it is $\BZ[\frac{1}{p}]$-local and hence the bottom map is an equivalence as well.
\end{proof}



\begin{proposition}
\label{connectivity of local equivalences}
	Let $g: X\to Y$ be a $n$-connective map in $\Sp$. Then every $L_g$-equivalence is $n$-connective.
\end{proposition}
\begin{proof}
	Let $S=\{g\}$ and denote $\overline{S}$ the strongly saturated class of morphisms generated by $S$. Proposition 5.5.4.15. of \cite{HTT} then implies that  $\overline{S}= \{L_g\text{-equivalences}\}$, hence it suffices to show that $\overline{S}$ is contained in the class of $n$-connective maps in $\Sp$. Note that the class of $n$-connective maps is saturated in the sense of Definition 5.5.5.1 of \cite{HTT}. Since every strongly saturated class of morphisms in $\Sp$ is also saturated by Example 5.5.5.5. of \cite{HTT}, we have
	$$
	\{L_g\text{-equivalences}\}=\overline{S} \subseteq \tilde{S} \subseteq \{n\text{-connective maps}\}
	$$
	where $\tilde{S}$ is the collection of saturated morphisms generated by $S$.
\end{proof}
\begin{remark}
	So in particular, every $L_{m_p}$-equivalence is $n$-connective.
\end{remark}

As a corollary, we deduce that $L_{m_p}$-localization commutes with taking $n$-connective cover.
\begin{corollary}
\label{L-localization commutes with connective cover}
	The $L_{m_p}$-localization commutes with $n$-connective cover, that is, 
	$$
	L_{m_p}(\tau_{\geq n}X)\simeq \tau_{\geq n}(L_{m_p} X).
	$$
	for any $X\in \Sp$.
\end{corollary}
\begin{proof}
	It suffices to show $\tau_{\geq n}X\to \tau_{\geq n}L_{m_{p}}X$ is a $L_{m_p}$-equivalence, as $\tau_{\geq n}L_{m_{p}}X$ is $L_{m_p}$-local. Consider the following ladder of fiber sequences
	\[
	\begin{tikzcd}
		\tau_{\geq n}X & X & \tau_{\leq n-1}X    \\
		\tau_{\geq n}L_{m_p}X & L_{m_p}X & \tau_{\leq n-1} L_{m_p}X\simeq \tau_{\leq n-1}X
		\arrow[from=1-1, to=1-2]
		\arrow[from=2-1, to=2-2]
		\arrow[from=1-3, to=2-3]
		\arrow[from=1-2, to=1-3]
		\arrow[from=2-2, to=2-3]
		\arrow[from=1-1, to=2-1]
		\arrow[from=1-2, to=2-2]
	\end{tikzcd}
	\]
	which induces a ladder of long exact sequences of homotopy groups. For each $k\geq n$, we have a commutative diagram
	\[
	\begin{tikzcd}
		\pi_{k}(\tau_{\geq n}X)\otimes \BZ[\frac{1}{p}] & \pi_{k}(X) \otimes \BZ[\frac{1}{p}] \\
		\pi_{k}(\tau_{\geq n}L_{m_p}X)\otimes \BZ[\frac{1}{p}]  & \pi_{k}(L_{m_p}X)\otimes \BZ[\frac{1}{p}] 
		\arrow[from=1-1, to=1-2]
		\arrow[from=2-1, to=2-2]
		\arrow[from=1-1, to=2-1]
		\arrow[from=1-2, to=2-2],
	\end{tikzcd}
	\]
	where the horizontal maps are isomorphisms and the right vertical map is also an isomorphism by Lemma \ref{L-equivalence implies Z[1/p]-iso}, hence the left map is also an isomorphism.
	The corollary then follows from Lemma \ref{L-equivalence is equivalent to pi_* iso on n-conn spectra}.
\end{proof}


We can now compute the homotopy groups of the $L_{m_p}$-localization on a spectrum.


\begin{corollary}
	The homotopy groups of $L_{m_p}X$ is given by
	\[
	\pi_* L_{m_p}X = \begin{cases}
		\pi_*X & \text{if $*< n$}\\
		\pi_*X\otimes \BZ[\frac{1}{p}] & \text{if $*\geq n$}.
	\end{cases}
	\]
\end{corollary}
\begin{proof}
	Since $m_p$ is $n$-connective, the map $X \to L_{m_p}X$ is also $n$-connective by Proposition \ref{connectivity of local equivalences}. Since $L_{m_p}$ commutes with taking $n$-connective cover by Corollary \ref{L-localization commutes with connective cover}.
	 Let's now assume $X$ is $n$-connective. We claim that $L_{m_p}X\simeq L_{\BZ[\frac{1}{p}]}X$, that is, $L_{m_p}X$ is the localization of $X$ with respect to the homology theory $H_{*}(-,\BZ[\frac{1}{p}])$.
	
	Indeed, the homotopy groups of $L_{m_{p}}X$ is uniquely $p$-divisible, so it's $L_{\BZ[\frac{1}{p}]}$-local and we have a unique factorization
	\[
	\begin{tikzcd}
		X &   & L_{m_p}X\\
		  & L_{\BZ[\frac{1}{p}]}X &
	\arrow[from=1-1, to=1-3]
	\arrow[from=1-1, to=2-2]
	\arrow[from=2-2, to=1-3, dashed].
	\end{tikzcd}
	\]
By Lemma \ref{L-equivalence is equivalent to pi_* iso on n-conn spectra}, the horizontal map is a  $L_{\BZ[\frac{1}{p}]}X$-equivalence
	Hence the dashed diagonal map is also a $L_{\BZ[\frac{1}{p}]}$-equivalence and we have $L_{m_p}X\simeq L_{\BZ[\frac{1}{p}]}X$, which implies $\pi_*L_{m_p}X=\pi_*X\otimes \BZ[\frac{1}{p}]$ for $*\geq n$.
\end{proof}

\begin{lemma}
	A map $f:X \to Y$ of spectra is a $L_{m_p}$-equivalence if and only if 
	\begin{enumerate}
	 	\item $f$ is $n$-connective.
		\item $f$ induces isomorphism on 
	$$
	\pi_{n+k}X\otimes \BZ[\frac{1}{p}]\to \pi_{n+k}Y\otimes \BZ[\frac{1}{p}]
	$$
	for $k\geq 0$.

	\end{enumerate}
	\end{lemma}
\begin{proof}
Note that $f$ is a $L_{m_p}$-equivalence if and only if  
$$
L_{m_p}X \to L_{m_p}Y
$$
is an equivalence. By our computation of the homotopy groups, this is equivalent to requiring that $f$ is $n$-connective and induces isomorphism on 
	$
	\pi_{n+k}X\otimes \BZ[\frac{1}{p}]\to \pi_{n+k}Y\otimes \BZ[\frac{1}{p}]
	$
	for $k\geq 0$.



\end{proof}

To obtain tame localization, we only need to assemble the multiplication maps for different primes.
We define a map between wedges of sphere spectra
\begin{equation}
\label{f}
	f: \bigvee_{p\in P}\BS^{s+2p-3} \to \bigvee_{p\in P}\BS^{s+2p-3}
\end{equation}
which can be seen as a diagonal matrix with primes on the diagonal. Combining with Lemma \ref{local w.r.t. multiplication-by-p map}, this allows us to give the following equivalent characterization of tame spectra.
\begin{corollary}
	A spectrum $X\in \Sp^{\geq s}$ is $s$-tame if and only if it is $f$-local.
\end{corollary}

%\begin{proof}
%	A spectrum $X$ is $s$-tame if and only if, for every $j\geq 0$, $\pi_{s+j}X$ is uniquely $p$-divisible for $p\leq \frac{j+3}{2}$. By Lemma \ref{local w.r.t. multiplication-by-p map}, this is equivalent to requiring that $X$ 
%	$$
%	\bigvee_{p\in P_j}\BS^{s+2p-3} \to \bigvee_{p\in P_j}\BS^{s+2p-3}
%	$$
%	where $P_{j}=\{p\in \BZ_{\geq 0} | p \leq \frac{j+3}{2}\}$.
%\end{proof}





\begin{definition}
	A map of spectra $g:X\to Y$ in $\Sp^{\geq s}$ is a \emph{$s$-tame equivalence} if for any $s$-tame spectrum $Z$, the induced map
	\[
	\map_*(Y,Z) \xrightarrow{g^*} \map_*(X,Z)
	\]
	is an equivalence.
\end{definition}



\begin{proposition}
	A map of spectra $g:X\to Y$ in $\Sp^{\geq s}$ is a $s$-tame equivalence if and only if it induces isomorphisms
	\[
	\pi_{s+j}(X)\otimes R_j \to \pi_{s+j}(Y) \otimes R_j
	\]
	for $j\geq 0$.
\end{proposition}





\begin{remark}
	Let $s\leq s'$ be non-negative integers.
	If $X\to Y$ is a $s'$-tame equivalence, then it's also a $s$-tame equivalence.
\end{remark}

By work of Bousfield and Farjoun, there is a reflective localization $L_{tame}: \Sp^{\geq s} \to \Sp^{\geq s}_{tame}$


\begin{remark}
		The infinite loop space of a tame spectrum is a tame space. That is, $\Omega^{\infty}X$ is a tame space if $X$ is a tame spectrum.
\end{remark}




\begin{lemma}
\label{Suspension preserves tame equivalence}
	The functor $\Sigma^{\infty}:\Space^{\geq r}_{*} \to \Sp^{\geq r}$ sends $r$-tame equivalences of spaces to $r$-tame equivalences of spectra.
\end{lemma}
\begin{proof}
	Let $f:X\to Y$ be a tame equivalence in $\Space^{\geq r}_{*}$. Note that $\Sigma^{\infty}f: \Sigma^{\infty}X \to \Sigma^{\infty}Y$ is a tame equivalence if, for every tame spectrum $Z$, the induced map
	$$
	\map_{\Sp^{\geq r}}(\Sigma^{\infty} Y, Z) \to 	\map_{\Sp^{\geq r}}(\Sigma^{\infty} X, Z)
	$$
	is a weak equivalence. Indeed, the map above is equivalent to 
	$$
	\map_{\Space^{\geq r}_{*}}(Y, \Omega^{\infty} Z) \to 	\map_{\Space^{\geq r}_{*}}(X, \Omega^{\infty} Z)
	$$
	which is a weak equivalence as $\Omega^{\infty}Z$ is a tame space and $f$ is a tame equivalence by assumption.
\end{proof}

We give some closure properties for tame equivalences.
\begin{lemma}
\label{closure property of tame equivalences}
	\begin{enumerate}[(a)]
		\item Let $K$ be a simplicial set and let $f:X\to Y$ be a point-wise $r$-tame equivalence in $\Fun(K,\Sp)$.
	Then the induced map on colimits 
		$$
		\colim_{K}X_k \to \colim_{K}Y_k
		$$ is a $r$-tame equivalence.
%		\item Conjecture: The collection of tame equivalences is closed under finite products and inverse limits. 
	\end{enumerate}
\end{lemma}

\begin{proof}
	For $(a)$, it suffices to show the induced map on colimits
	$\colim_{K}X_k \to \colim_{K}Y_k$ is an equivalence after applying $L_{tame}$.
	Observe that if we view $L_{tame}$ as a functor from $\Sp^{\geq r}\to \Sp_{tame}^{\geq r}$, it preserves all colimits.
	Hence the map $L_{tame}\colim_{K}X_k \to L_{tame}\colim_{K}Y_k$ is equivalent to the map 
	$$
	\colim_{K}L_{tame}X_k \to \colim_{K}L_{tame}Y_k
	$$
	in $\Sp_{tame}^{\geq r}$, which is an equivalence since $L_{tame}X_k \to L_{tame}Y_k$ is an equivalence for each vertex $k$.
	
%	let $Z$ be a tame spectrum.
%Consider the following commutative diagram:	
%	\[
%	\begin{tikzcd}
%		\map_{\Sp}(\colim_{K}Y_k,Z)& \map_{\Sp}(\colim_{K}X_k,Z)\\
%		\lim_K\map_{\Sp}(Y_k,Z)
%  & \lim_K\map_{\Sp}(X_k,Z)
%		\arrow[from=1-1, to=1-2]
%		\arrow[from=2-1, to=2-2,"\simeq"]
%		\arrow[from=1-1, to=2-1, "\simeq"]
%		\arrow[from=1-2, to=2-2, "\simeq"],
%	\end{tikzcd}
%	\]
%	in which the bottom and the vertical maps are equivalences.
%Hence the top map is an equivalence and the map 
%$\colim_{K}X_k \to \colim_{K}Y_k$ is a tame equivalence.
	
%	For $(b)$, observe that for fixed $j$, we have an isomorphism
%	$$
%	\pi_{r+j}X_k\otimes R_j \xrightarrow{\simeq} 
%	\pi_{r+j}Y_k\otimes R_j 
%	$$
%	for every $k\in K'_0$. Therefore, we have
%	$$
%	\lim_{K'}(\pi_{r+j}X_k\otimes R_j) \xrightarrow{\simeq}
%	\lim_{K'}(\pi_{r+j}Y_k\otimes R_j).
%	$$
%	Since $R_j$ is torsion free, $(-)\otimes R_j$ is left exact, then we have
%	$$
%%	\pi_{r+j}(\lim_{K'} X_k)\otimes R_j\cong 
%	(\lim_{K'}\pi_{r+j}X_k)\otimes R_j
%	\cong
%	\lim_{K'}(\pi_{r+j}X_k\otimes R_j).
%	$$
%which implies $(\lim_{K'}\pi_{r+j}X_k)\otimes R_j \to (\lim_{K'}\pi_{r+j}Y_k)\otimes R_j$ is an isomorphism for every $j\geq 0$, 
\end{proof}

%I can't prove the following proposition, neither that I think it is correct.
%We now show that tame equivalences of spectra can also be detected on the homology level.
%\begin{proposition}
%	\label{Tame equivalence=tame homology equivalence}
%	A map of spectra $g:X\to Y$ in $\Sp^{\geq s}$ is a tame equivalence if and only if its induced map on homology $H_{s+j}(X)\otimes R_j \to H_{s+j}(Y)\otimes R_j$ is an isomorphism for $j\geq 0$. 
%\end{proposition}

	

%Is this even true?
%	\begin{proposition}
%	\label{Tame localization is symmetric monoidal}
%	The Tame localization functor 
%\[
%L_{tame}: \Sp^{\geq s}\to \Sp^{\geq s}_{tame}
%\]
%is symmetric monoidal.
%\end{proposition}



	

\subsection{Algebraic models for tame spaces}
\begin{remark}
	Since the first $p$-torsion of the sphere spectrum $\BS$ appears in degree $2p-3$, its tame localization is the Eilenberg-Maclane spectrum $H\BZ$.
\end{remark}

We now show that the class of $r$-tame equivalences is compatible with smashing with connective spectrum.
\begin{lemma}
\label{Shift of tame equivalences}
Let $r,s\geq 0$ and
let $E$ be a $r$-connective spectrum and let $f:X\to Y$ be a $s$-tame equivalence in $\Sp^{\geq s}$. Then $E\otimes X \to E\otimes Y$ is a $(r+s)$-tame equivalence.
\end{lemma}
\begin{proof}
Let $\CE$ be the full subcategory of $\Sp^{\geq r}$ spanned by spectra $F$ satisfying the condition that
$$
F\otimes X \to F\otimes Y
$$
is a $(r+s)$-tame equivalence whenever $X\to Y$ is a $s$-tame equivalence. Clearly, the shifted sphere spectrum $\Sigma^{r}\BS$ is in $\CE$. 
The claim would follow if we can show $\CE$ is closed under colimits. Let $Z:K\to \CE$ be a diagram in $\CE$ so that its colimit $\colim_K Z_k$ in $\Sp^{\geq r}$ exists.
Note that by Lemma \ref{closure property of tame equivalences} and the fact that smash product commutes with colimits, we have
$$
(\colim_K Z_k )\otimes X \to (\colim_K Z_k )\otimes Y
$$
is a $(r+s)$-tame equivalence.

%Fix some $j\geq 0$, we claim the map $\pi_{r+j}(E\otimes X)\otimes R_j \to \pi_{r+j}(E \otimes Y)\otimes R_j$ is an isomorphism. We let $F$ denote the fiber of the induced map $\tau_{\leq r+j}X \to \tau_{\leq r+j}Y$. Observe that $R_j$ is a flat, we have a long exact sequence
%$$
%0 \to \pi_{r+j}F \otimes R_j \to \pi_{r+j}X \otimes R_j  \to \pi_{r+j}Y \otimes R_j  \to \pi_{r+j-1}F \otimes R_j \to \cdots.
%$$
%Since $f:X \to Y$ is a tame equivalence, we conclude that $F$ is $(r+j)$-truncated and $\pi_{r+k}F\otimes R_j=0$ for $0\leq k \leq j$. Under the terminology of Bousfield localization, we say the spectrum $F$ is $\BS R_j$-acyclic, where $\BS R_j$ denotes the Moore spectrum associated to $R_j$.
%Tensoring with $E$ preserves fiber sequence, the lemma would follow if we can show $E\otimes F$ is also $\BS R_j$-acyclic, which follows from the fact that $\BS R_j$-localization is symmetric monoidal.
\end{proof}

\begin{remark}
\label{symmetric monoidal structure on tame spectra}
\begin{enumerate}

 	\item If $X\to Y$ is a $s$-tame equivalence, then $X\otimes H\BZ \to Y\otimes H\BZ$ is also a $s$-tame equivalence. Hence, for every $k\geq 0$, the induced map on homology
 	\[
 	H_{s+k}(X)\otimes R_{k}
 	\to
 	H_{s+k}(Y)\otimes R_{k}
 	\]
 	is an isomorphism.
	\item The lemma above implies that $s$-tame localization is compatible with the smash product in $\Sp$ in the sense of \cite[Definition 2.2.1.6.]{HA}. By \cite[Proposition 2.2.1.9.]{HA}, the $\infty$-category $\Sp^{\geq s}_{tame}$ of tame spectra admits a symmetric monoidal structure given by 
$$
X\hat{\otimes} Y = L_{tame}(X\otimes Y).
$$
	\item The $\infty$-category $\Sp^{\geq s}$ of $s$-connective spectra is presentable, hence by \cite[Proposition 5.5.4.15.]{HTT}, the $\infty$-category $\Sp^{\geq s}_{tame}$ of tame spectra is also presentable.
\end{enumerate}


\end{remark}

\begin{corollary}
\label{HZ commutes with tame localization}
	Let $E\in \Sp^{\geq r}$. Then the map
	$$
	E \simeq E\otimes \BS  \to E\otimes H\BZ
	$$ 
	is a $r$-tame equivalence.
\end{corollary}
	


\begin{proof}
	This is immediate from Lemma \ref{Shift of tame equivalences}, since $\BS \to H\BZ$ is a $0$-tame equivalence and $E$ is $r$-connective.
\end{proof}

\begin{comment}
I think this lemma is wrong...
\begin{lemma}
	The tame localization functor $L_{tame}:\Sp^{\geq r} \to \Sp^{\geq r}$ is symmetric monoidal.
\end{lemma}
\end{comment}

One might ask whether $H\BZ$-module spectra are tame for all $r$. The answer to that question is false. For instance, $\Sigma H\BZ$ is a $1$-connective $H\BZ$-module but it's not $0$-tame as $\pi_{1}\Sigma H\BZ\cong \BZ$ is not uniquely $2$-divisible.


We now give an algebraic characterization of tame spectra. We let $(\Mod^{\geq r}_{H\BZ})_{tame}$ denote the full subcategory of $\Sp^{\geq r}_{tame}$ spanned by $r$-connective tame $H\BZ$-modules.

\begin{theorem}
	\label{algebraic description of tame spectra}
	There is an equivalence of $\infty$-categories
	$$
	\Sp^{\geq r}_{tame} \xrightarrow{L_{tame}(-\otimes H\BZ)} (\Mod^{\geq r}_{H\BZ})_{tame}
	$$
	where the inverse is given by inclusion.
\end{theorem}
\begin{proof}
	Let $X\in \Sp^{\geq r}_{tame}$ be a tame spectrum. The map $X\otimes \BS \to X\otimes H\BZ$ is a $r$-tame equivalence by Corollary \ref{HZ commutes with tame localization}. Hence the composite $X \to X\otimes H\BZ \to L_{tame}(X\otimes H\BZ)$ is an equivalence as both $X$ and $L_{tame}(X\otimes H\BZ)$ are tame.
	
	Similarly, let $Y\in (\Mod^{\geq r}_{H\BZ})_{tame}$ be a tame $H\BZ$-module. The composite $Y\otimes \BS \to  Y\otimes H\BZ \to L_{tame}(Y\otimes H\BZ)$ is a composite of two tame equivalences, and hence is an equivalence as both source and target are tame.
	\end{proof}
	

As a direct corollary, we see that the tame homotopy type is determined by homology with coefficients in the tame ring system.
\begin{corollary}
\label{htpy groups of tame spectra can be computed by homology}
	Let $X\in \Sp^{\geq r}_{tame}$ be a tame spectrum, then we have 
	\[
	\pi_{r+k}X = H_{r+k}X \otimes R_k
	\]
	for every $k\geq 0$.
\end{corollary}
	
\begin{proof}
	Using theorem \ref{algebraic description of tame spectra}, we calculate
	\begin{align*}
		\pi_{r+k}X  &= \pi_{r+k}L_{tame}(X\otimes H\BZ)\\
		& =\pi_{r+k}(X\otimes H\BZ)\otimes R_k\\
		& = H_{r+k}(X)\otimes R_k
	\end{align*}
\end{proof}

\begin{corollary}
		For any spectrum $X\in \Sp^{\geq r}$, we have the following observation:
\begin{align*}
	\pi_* L_{tame}X & = H_* L_{tame}X \otimes R_{*-r}\\
		& = H_* X \otimes R_{*-r}.
\end{align*}
Hence, we have for any spectrum $X$,
\[
\pi_*X \otimes R_{*-r}\cong H_*X \otimes R_{*-r}.
\]
\end{corollary}
\begin{proof}
	The first equality follows from corollary \ref{htpy groups of tame spectra can be computed by homology} and the second equality follows from the fact that tensoring with $H\BZ$ preserves $r$-tame equivalences.
\end{proof}


\begin{remark}
 As an immediate consequence, the functor $\Sigma^{\infty}_{tame}:\Space^{\geq r}_* \to \Sp^{\geq r}_{tame}$ sends any Moore space $M(V,r+k)$ for $V$ an $R_k$-module to a shift of  Eilenberg-Maclane spectrum $\Sigma^{r+k}HV$.
\end{remark}
%Recall that an $\infty$-category $\mathcal{C}$ has an \emph{essentially colimit dense subcategory} if there is an essentially small full subcategory $\mathcal{C}_{0} \subset \mathcal{C}$ such that $\CC$ is generated by $\CC_0$ under colimits. Note that the $\infty$-category $\Sp^{\geq s}_{tame}$ of $s$-tame spectra is generated by $\{L_{tame}\BS^{n}\}_{n\geq s}$ under colimits, hence $\Sp^{\geq s}_{tame}$ is essentially colimit dense.
%The existence of the right adjoint of the forgetful functor 
%$$
%\coCAlg(\Sp^{\geq s}_{tame})
%\to \Sp^{\geq s}_{tame}
%$$
%is guaranteed by the following.
%
%\begin{theorem}
%\cite{AFT}
%	 Let $\mathrm{C}$ be a locally small cocomplete $\infty$-category and $\mathcal{D}$ a locally small $\infty$-category. Suppose that $\mathrm{C}$ has an essentially small colimit-dense subcategory. Then a functor $F: \mathrm{C} \rightarrow \mathcal{D}$ is a left adjoint if and only if it preserves small colimits.
%\end{theorem}
%
%\begin{corollary}
%	The forgetful functor $\coCAlg(\Sp^{\geq s}_{tame})
%\to \Sp^{\geq s}_{tame}$ admits a right adjoint which we will denote by $\cofree_{tame}$.
%\end{corollary}

	The following proposition is a folklore to many seasoned homotopy theoriests.
\begin{proposition}
	\label{Coalgebra model for simply-connected spaces}
	For $r\geq 2$, the functor $\Sigma^{\infty}:\Space^{\geq r}_{*} \to \Sp^{\geq r}$ is comonadic. In other words, there is an equivalence of $\infty$-categories:
	\[
	\phi: \Space^{\geq r}_{*} \to \coAlg_{\Sigma^{\infty}\Omega^{\infty}}(\Sp^{\geq r}).
	\]
 between the $\infty$-category of $r$-connective spaces and the $\infty$-category of $r$-connective $\Sigma^{\infty}\Omega^{\infty}$-coalgebra.
\end{proposition}
\begin{proof}
By Corollary \ref{Cor of Barr-Beck-Lurie theorem}, it suffices to show there is an equivalence 
$$
X\to \Tot (\Omega^{\infty}\Sigma^{\infty})^{\bullet+1}X
$$
for every $X\in \Space_{*}^{\geq r}$. We prove this by induction on the Postnikov tower. For $X$ being an Eilenberg-Maclane space $K(A,n)\simeq \Omega^{\infty-n}HA$, its associated augmented cosimplicial object $(\Omega^{\infty}\Sigma^{\infty})^{\bullet+1}X$ splits, with the contracting homotopy induced by the counit $\Sigma^{\infty}\Omega^{\infty} \Omega^n HA\to \Omega^n HA$. For the inductive step, we have a principal fibration sequence
\[
K(\pi_n X,n) \to \tau_{\leq n}X \to \tau_{\leq n-1}X 
\to 
K(\pi_n X,n+1).
\]
By the principal fibration lemma, the functor $\Tot (\Omega^{\infty}\Sigma^{\infty})^{\bullet+1}$ preserves principal fibrations. Hence the vertical sequences in the following diagram are fiber sequences.
\[
\begin{tikzcd}
	\tau_{\leq n}X & \Tot (\Omega^{\infty}\Sigma^{\infty})^{\bullet+1} (\tau_{\leq n}X)\\
	\tau_{\leq n-1}X  & \Tot (\Omega^{\infty}\Sigma^{\infty})^{\bullet+1} (\tau_{\leq n-1}X)\\
	K(\pi_n X,n+1)   & \Tot (\Omega^{\infty}\Sigma^{\infty})^{\bullet+1} K(\pi_n X,n+1)
	\arrow[from=1-1, to=1-2]
	\arrow[from=1-1, to=2-1]
	\arrow[from=2-1, to=2-2]
	\arrow[from=1-2, to=2-2]
	\arrow[from=2-1, to=3-1]
	\arrow[from=3-1, to=3-2]
	\arrow[from=2-1, to=3-1]
	\arrow[from=2-2, to=3-2]
\end{tikzcd}
\]
Observe that the bottom two horizontal arrows are equivalences by the inductive hypothesis, hence the induced map on the fibers is an equivalence. This completes the inductive step.
\end{proof}

We now establish a pair of adjoint functors between tame spaces and tame spectra.
Observe that the infinite loop space of a $r$-tame spectrum is an $r$-tame space. For convenience, we write $\Sigma^{\infty}_{tame}$ as a short hand for the composite $L_{tame}\circ \Sigma^{\infty}$.
\begin{proposition}
	We have an adjoint pair
	\[
	\adj{\Sigma^{\infty}_{tame}}{\Space^{\geq r}_{tame}}{\Sp^{\geq r}_{tame}}{\Omega^{\infty}}
	\]
	% \coAlg_{\Sigma^{\infty}_{tame}\Omega^{\infty}}(\Sp)^{\geq r}
\end{proposition}
\begin{proof}
	We first note that $\adj{\Sigma^{\infty}_{tame}}{\Space_{*}^{\geq r}}{\Sp^{\geq r}_{tame}}{\Omega^\infty}$ is an adjoint pair, where we abuse notation by writing $\Omega^{\infty}$ as the composite of $\Sp^{\geq r}_{tame}\hookrightarrow \Sp^{\geq r} \xrightarrow{\Omega^\infty} \Space^{\geq r}_{*}$; indeed, $\Sigma^{\infty}_{tame}$ is the composition of $L_{tame}$ and $\Sigma^\infty$, which are left adjoint to the inclusion functor $\Sp^{\geq r}_{tame}\hookrightarrow \Sp^{\geq r}$ and $\Omega^\infty$, respectively. The statement then follows from Proposition \ref{Restrict adjoints to full subcategory} and the fact that $\Space_{tame}^{\geq r}$ is a full subcategory of $\Space^{\geq r}$. 
\end{proof}


The main theorem of this note is an analogue of Proposition \ref{Coalgebra model for simply-connected spaces} in the setting of tame spaces.
Before we give the proof of the main theorem, we make an important observation.
We claim that the following composite:
\[
\coAlg_{\Sigma^{\infty}\Omega^{\infty}}(\Sp^{\geq r})
\xrightarrow{\oblv}
\Sp^{\geq r}
\xrightarrow{L_{tame}}
\Sp^{\geq r}_{tame}
\]
factors through the $\infty$-category $\coAlg_{\Sigma^{\infty}_{tame}\Omega^{\infty}}(\Sp^{\geq r}_{tame})$ of $\Sigma^{\infty}_{tame}\Omega^{\infty}$-coalgebras in $\Sp^{\geq r}_{tame}$;
that is to say, the tame localization of a $\Sigma^{\infty}\Omega^\infty$-coalgebra admits a $\Sigma^{\infty}_{tame}\Omega^{\infty}$-coalgebra structure.
Indeed, 
for $X\in \Space_*^{\geq r}$, we have an equivalence $\Sigma^{\infty}_{tame}X \simeq \Sigma^{\infty}_{tame}L_{tame}X$, since $\Sigma^{\infty}_{tame}$ sends tame equivalences to equivalences of tame spectra. Then we obtain a commuting diagram
\[
\begin{tikzcd}
\Space^{\geq r}_{*} & & \Sp^{\geq r}\\
& \coAlg_{\Sigma^{\infty}\Omega^{\infty}}(\Sp^{\geq r})&\\
\Space^{\geq r}_{tame}&  & \Sp^{\geq r}_{tame}
\arrow[from=1-1,to=1-3, "\Sigma^{\infty}"]
\arrow[from=1-1,to=2-2, "\simeq"]
\arrow[from=2-2,to=1-3, "\oblv" below]
\arrow[from=1-1, to=3-1, "L_{tame}"]
\arrow[from=1-3, to=3-3, "L_{tame}"]
\arrow[from=3-1, to=3-3, "\Sigma_{tame}^{\infty}"]
\end{tikzcd}
\]
which implies $\Sigma^{\infty}_{tame}(L_{tame}X)\simeq  L_{tame}(\oblv\circ\phi(X))$. 
Hence we obtain a factorization
\[
\begin{tikzcd}
	\coAlg_{\Sigma^{\infty}\Omega^{\infty}}(\Sp^{\geq r}) &  & \Sp^{\geq r}_{tame}\\
	 & \coAlg_{\Sigma^{\infty}_{tame}\Omega^{\infty}}(\Sp_{tame}^{\geq r}) &
	 \arrow[from=1-1, to=1-3, "L_{tame}\circ \oblv"]
	 \arrow[from=1-1, to=2-2, "L'_{tame}"]
	 \arrow[from=2-2, to=1-3, "\oblv_{tame}"]	 
\end{tikzcd}
\]
where $\oblv_{tame}$ denotes the forgetful functor sending a $\Sigma^{\infty}_{tame}\Omega^{\infty}$-coalgebra to its underlying tame spectrum. 

\begin{lemma}
	\label{Lemma for the main theorem}
	Let $X\in \coAlg_{\Sigma^{\infty}\Omega^{\infty}}(\Sp^{\geq r})$. Then the induced map 	\[
	\Tot (\Omega^{\infty}\Sigma^{\infty})^{\bullet} \Omega^{\infty}X
	\to 
	\Tot (\Omega^{\infty}\Sigma^{\infty}_{tame})^{\bullet} \Omega^{\infty}L_{tame}X
	\]
	on the totalization of cosimplicial spaces $(\Omega^{\infty}\Sigma^{\infty})^{\bullet} \Omega^{\infty}X$ and $(\Omega^{\infty}\Sigma^{\infty}_{tame})^{\bullet} \Omega^{\infty}L_{tame}X$
	is a tame equivalence of spaces.
\end{lemma}
\begin{proof}
	Since tame equivalences is closed under limits\todo{not true!!!}, it suffices to show 
	\[
	(\Omega^{\infty}\Sigma^{\infty})^{n} \Omega^{\infty}X
	\to 
    (\Omega^{\infty}\Sigma^{\infty}_{tame})^{n} \Omega^{\infty}L_{tame}X
    \]
    is a tame equivalence for each $n\geq 0$.
    For $n=0$, this is obvious since $\Omega^{\infty}$ preserves tame equivalence. Suppose now 
    $(\Omega^{\infty}\Sigma^{\infty})^{n-1} \Omega^{\infty}X
	\to 
    (\Omega^{\infty}\Sigma^{\infty}_{tame})^{n-1} \Omega^{\infty}L_{tame}X$ is a tame equivalence;
    observe that $(\Omega^{\infty}\Sigma^{\infty})^{n} \Omega^{\infty}X
	\to 
    (\Omega^{\infty}\Sigma^{\infty}_{tame})^{n} \Omega^{\infty}L_{tame}X$ is also a tame equivalence as it is the composition of two tame equivalences
    $$ \Omega^{\infty}\Sigma^{\infty}(\Omega^{\infty}\Sigma^{\infty})^{n-1} \Omega^{\infty}X
	\to 
    \Omega^{\infty}\Sigma^{\infty}(\Omega^{\infty}\Sigma^{\infty}_{tame})^{n-1} \Omega^{\infty}L_{tame}X
    $$  
    and 
    $$\Omega^{\infty}\Sigma^{\infty}(\Omega^{\infty}\Sigma^{\infty}_{tame})^{n-1} \Omega^{\infty}L_{tame}X
    \to \Omega^{\infty}\Sigma^{\infty}_{tame}(\Omega^{\infty}\Sigma^{\infty}_{tame})^{n-1} \Omega^{\infty}L_{tame}X.$$
	\end{proof}

\begin{theorem}
	The adjoint pair $(\Sigma^{\infty}_{tame},\Omega^{\infty})$ is comonadic. That is, we have an equivalence of $\infty$-categories
	\[
	\Sigma^{\infty}_{tame}: \Space^{\geq r}_{tame} \to \coAlg_{\Sigma^{\infty}_{tame}\Omega^{\infty}}(\Sp^{\geq r}_{tame}).
	\]
\end{theorem}
\begin{proof}



%The functor $L'_{tame}$ preserves all colimits 
%$$
%L'_{tame}: \coAlg_{\Sigma^{\infty}\Omega^{\infty}}(\Sp^{\geq r}) \to \coAlg_{\Sigma^{\infty}_{tame}\Omega^{\infty}}(\Sp_{tame}^{\geq r})
%$$ 
%by $L'_{tame}$.


We let $\CC'$ denote the full subcategory of $\coAlg_{\Sigma^{\infty}\Omega^{\infty}}(\Sp^{\geq r})$ spanned by coalgebras whose underlying spectra satisfying:
\[
\Tot (\Omega^\infty \Sigma^{\infty})^{\bullet} \Omega^{\infty}X 
\text{ is a tame space.}
\tag{$\ast$}
\]
Note that a $\Sigma^{\infty}\Omega^{\infty}$-coalgebra $Y$ is in $\CC'$ if and only if it is in the esential image of $\Space^{\geq r}_{tame}$ via $\phi$, since $\phi^{-1}(Y)\simeq \Tot (\Omega^\infty \Sigma^{\infty})^{\bullet} \Omega^{\infty}Y $.
Hence, we have $\CC'\simeq \Space^{\geq r}_{tame}$.

We claim the functor $L'_{tame}: \coAlg_{\Sigma^{\infty}\Omega^{\infty}}(\Sp^{\geq r}) \to \coAlg_{\Sigma^{\infty}_{tame}\Omega^{\infty}}(\Sp_{tame}^{\geq r})$ restricts to an equivalence on $\CC'$.
%We first observe that $\CC'$ is equivalent to the $\infty$-category $\Space^{\geq r}_{tame}$ of tame spaces; note that $\phi$ is an equivalence and a $\Sigma^{\infty}\Omega^{\infty}$-coalgebra $\Sigma^{\infty}X$ is tame if and only if its corresponding space $\phi^{-1}(X)$ is tame.

%$$
%\map_{\Space^{\geq r}}(S^n,X)\simeq \map_{\coAlg_{\Sigma^{\infty}\Omega^\infty}}(\Sigma'^{\infty}S^n,\Sigma'^{\infty}X) \neq
%\map_{\Sp}(\BS^n,\Sigma'^{\infty}X)
%$$


%We next claim that $\coAlg_{\Sigma^{\infty}_{tame}\Omega^{\infty}}(\Sp_{tame}^{\geq r})$ is a full subcategory of $\coAlg_{\Sigma^{\infty}\Omega^{\infty}}(\Sp^{\geq r})$ and $L'_{tame}: \coAlg_{\Sigma^{\infty}\Omega^{\infty}}(\Sp^{\geq r}) \to \coAlg_{\Sigma^{\infty}_{tame}\Omega^{\infty}}(\Sp_{tame}^{\geq r})$ is the reflective localization functor at the class of maps whose underlying maps in $\Sp^{\geq r}$ are tame equivalences.

%Let $X\in \coAlg_{\Sigma^{\infty}\Omega^{\infty}}(\Sp^{\geq r})$ and $A\in \Sp^{\geq r}_{tame}$, then we have
%\[
%\begin{align*}
%	\map_{\coAlg_{\Sigma^{\infty}\Omega^{\infty}}(\Sp^{\geq r})}(X,\cofree(A))  & \simeq  \map_{\Sp^{\geq r}}(X,A) \\
%	& \simeq \map_{\Sp^{\geq r}_{tame}}(L_{tame}X,A) &\\
%	& \simeq \map_{\coAlg_{\Sigma^{\infty}_{tame}\Omega^{\infty}}(\Sp_{tame}^{\geq r})}(L'_{tame}X,\cofree_{tame}(A))\\
%	& \simeq \map_{\coAlg_{\Sigma^{\infty}_{tame}\Omega^{\infty}}(\Sp_{tame}^{\geq r})}(L'_{tame}X,L'_{tame}\cofree(A))
%\end{align*}
%\]
%where $\cofree_{tame}:\Sp^{\geq r}_{tame}\to \coAlg_{\Sigma^{\infty}_{tame}\Omega^{\infty}}$ is the right adjoint to the forgetful functor. 

Let $Y,Z\in \CC'$, we compute:
\[
\begin{aligned}
	\map_{\CC'}(Y,Z) & \simeq \map_{\coAlg_{\Sigma^{\infty}\Omega^{\infty}}(\Sp^{\geq r})}\big(Y,Z\big) \\
	& \simeq \map_{\coAlg_{\Sigma^{\infty}\Omega^{\infty}}(\Sp^{\geq r})}\big(Y,\Tot \cofree^{\bullet+1} Z\big)\\
	& \simeq \Tot \map_{\coAlg_{\Sigma^{\infty}\Omega^{\infty}}(\Sp^{\geq r})}\big(Y, \cofree^{\bullet+1} Z\big)\\
	& \simeq \Tot \map_{\Sp^{\geq r}}\big(Y, (\Sigma^{\infty}\Omega^{\infty})^{\bullet} Z\big)\\
	& \simeq  \map_{\Sp^{\geq r}}\big(Y, \Tot (\Sigma^{\infty}\Omega^{\infty})^{\bullet} Z\big)\\
	& \simeqexpl{$(\ast)$ and Lemma \ref{Lemma for the main theorem}}  \map_{\Sp^{\geq r}}\big(Y, \Tot (\Sigma^{\infty}_{tame}\Omega^{\infty})^{\bullet} L_{tame} Z\big)\\
	& \simeq  \map_{\Sp^{\geq r}_{tame}}\big(L_{tame}Y, \Tot (\Sigma^{\infty}_{tame}\Omega^{\infty})^{\bullet} L_{tame} Z\big)\\
	& \simeq \map_{\coAlg_{\Sigma^{\infty}_{tame}\Omega^{\infty}}(\Sp^{\geq r}_{tame})}\big(L'_{tame}Y,\Tot \cofree^{\bullet+1}_{tame} (L'_{tame}Z)\big)\\
	& \simeq \map_{\coAlg_{\Sigma^{\infty}_{tame}\Omega^{\infty}}(\Sp^{\geq r}_{tame})}\big(L'_{tame}Y,L'_{tame}Z \big)
\end{aligned}
\]
which implies that $L'_{tame}|_{\CC'}$ is fully faithful. It remains to show $L'_{tame}$ is essentially surjective. Let $X\in \coAlg_{\Sigma^{\infty}_{tame}\Omega^{\infty}} (\Sp_{tame}^{\geq r})$, we show that $X$ also admits a $\Sigma^{\infty}\Omega^{\infty}$-coalgebra structure. 
This follows from the fully faithfulness of $L'_{tame}|_{\CC'}$; observe that the structure map $X\to \Sigma^{\infty}_{tame}\Omega^{\infty}X$ can be viewed as a map $\gamma_X:X \to \cofree_{tame}(X)$ of $\Sigma^{\infty}_{tame}\Omega^{\infty}$-coalgebras, but this corresponds to a $\Sigma^{\infty}\Omega^{\infty}$-coalgebra map $X \to \cofree (X)$ up to a contractible choice of ambiguity. 
This concludes our proof $\Space^{\geq r}_{tame}\simeq \CC'\simeq \coAlg_{\Sigma^{\infty}_{tame}\Omega^{\infty}}(\Sp_{tame}^{\geq r})$.
\end{proof}

We now turn to the algebraic description of tame spaces. 
Let $D(\BZ)$ denote the derived $\infty$-category of the abelian category $\Ab$ of abelian groups, there is an equivalence of $\infty$-categories $D(\BZ)\simeq \Mod_{H\BZ}$ by Remark 7.1.1.6 of \cite{HA}. Note that the objects of $D(\BZ)$ are projective, bounded below chain complexes of abelian groups.
The $\infty$-category $(\Mod_{H\BZ})_{tame}$ of tame $H\BZ$-modules then can be identified with $D(\BZ)_{tame}^{\geq r}$, the full subcategory spanned by chain complexes $M_{*}$ satisfying the tame condition,
$$
M_{k}\simeq 0 \text{ for $k<r$ and } H_{r+j}(M_{*})\otimes R_{j}\cong H_{r+j}(M_{*}) \text{ for $j\geq 0$ }.
$$

A non-unital cocommutative coalgebra $C$ in $D(\BZ)$ is a chain complex $C$ equipped with maps $\gamma_n: C \to (C^{\otimes n})^{h\Sigma_n}$ for $n\geq 1$, such that it satisfies the usual coassociativity and cocommutativity conditions. 

\begin{remark}
	Note that $\Sp^{\geq r}_{tame}$ admits a symmetric monoidal structure with the tensor product $X\otimes_{tame} Y$ given by the tame localization $L_{tame}(X\otimes Y)$ of the smash product of $X$ and $Y$. Observe that the functor $L_{tame}:\Mod_{H\BZ}^{\geq r}\to (\Mod_{H\BZ}^{\geq r})_{tame}$ is symmetric monoidal; this follows from the fact that the map $M \otimes_{\BZ} N \to L_{tame}M\otimes_{\BZ} N \to L_{tame}M\otimes_{\BZ} L_{tame}N$ is a tame equivalence for any $M,N\in \Mod_{H\BZ}^{\geq r}$.
\end{remark}


%\begin{question}
%    I understand that the goal is to show $\Sigma^{\infty}_{tame}\Omega^{\infty}$ agrees with the cofree commutative comonad on $\Sp^{\geq r}_{tame}$ and we indeed have 
%    $$
%    \Sigma^{\infty}_{tame}\Omega^{\infty}X\simeq L_{tame}\big(\Sym^{\geq 1}(X) \big),
%    $$ since we invert primes fast enough to kill Tate construction; but I don't know how to show the cofree commutative coalgebra comonad is given by $L_{tame}\big(\Sym^{\geq 1}(X)\big)$.
%\end{question}
%
%\begin{lemma}
%	Let $X\in \Sp^{B\Sigma_n}$ be a $\Sigma_n$-spectrum whose underlying spectrum is uniquely $n!$-divisible, then 
%	$$
%	X^{t\Sigma_n}\simeq *.
%	$$
%\end{lemma}

If $X$ is $r$-connective, then its $n$-th tensor power $X^{\otimes n}$ is $(rn)$-connective. Then the tame spectrum $L_{tame}(X^{\otimes n})$ is uniquely $(n!)$-divisible; indeed, the homotopy group $\pi_{nr}L_{tame}(X^{\otimes n})$ is uniquely divisible by any natural number $k\leq \lfloor \frac{nr-r+3}{2}\rfloor$, which would be satisfied since we require $r\geq 3$. (Concretely, we want $n\leq \frac{nr-r+3}{2}$ which is equivalent to $(r-2)(n-1)+1\geq 0$.)
As a consequence, we record the following lemma:
\begin{lemma}
\label{Tate vanishing for tame spectra}
	$$
	\big(L_{tame}(X^{\otimes n})\big)^{t\Sigma_n}\simeq *.
	$$
\end{lemma}




\begin{proposition}
	Let $X\in \Sp^{\geq r}$, then the norm map	
	
	$$
	(X^{\otimes n})_{h\Sigma_n} \xrightarrow{Nm} (X^{\otimes n})^{h\Sigma_n} 
	$$
	admits a retract after tame localization. 
% 	THE FOLLOWING IS NOT KNOWN
%	Moreover, the tame localization of $(X^{\otimes n})^{t\Sigma_n}$ vanishes, i.e. 
%	$$
%	L_{tame}(X^{\otimes n})^{t\Sigma_n}\simeq *.
%	$$
\end{proposition}
\begin{proof}
	Note we have a commutative diagram
	\[
	\begin{tikzcd}
		(X^{\otimes n})_{h\Sigma_n} & (X^{\otimes n})^{h\Sigma_n}\\
		(L_{tame}X^{\otimes n})_{h\Sigma_n} & (L_{tame}X^{\otimes n})^{h\Sigma_n}
		\arrow[from=1-1, to=1-2, "Nm"]
		\arrow[from=1-1, to=2-1]
		\arrow[from=2-1, to=2-2, "\simeq"]
		\arrow[from=1-2, to=2-2]
	\end{tikzcd}
	\]
	where the left map is a tame equivalence and the bottom map is an equivalence by Lemma \ref{Tate vanishing for tame spectra}.
After tame localization we then have 	
\[
	\begin{tikzcd}
		L_{tame}(X^{\otimes n})_{h\Sigma_n} & L_{tame}(X^{\otimes n})^{h\Sigma_n}\\
		L_{tame}(L_{tame}X^{\otimes n})_{h\Sigma_n} & L_{tame}(L_{tame}X^{\otimes n})^{h\Sigma_n}
		\arrow[from=1-1, to=1-2, "Nm"]
		\arrow[from=1-1, to=2-1, "\simeq"]
		\arrow[from=2-1, to=2-2, "\simeq"]
		\arrow[from=1-2, to=2-2]
	\end{tikzcd}
\]
the retract is given by composite of the right map followed by the bottom and left maps.
%	Observe that 
%	$$
%	L_{tame}(X^{\otimes n})_{h\Sigma_n} \xrightarrow{L_{tame}Nm} L_{tame}(X^{\otimes n})^{h\Sigma_n} 
%	\to
%	L_{tame}(X^{\otimes n})^{t\Sigma_n}
%	$$
%	is a cofiber sequence in $\Sp^{\geq r}_{tame}$, hence 
%	$L_{tame}(X^{\otimes n})^{t\Sigma_n}$ is contractible.
\end{proof}

\clearpage
\begin{question}
	Let $G$ be a finite group and let $X\in \Fun(BG,\Sp)$. Do we have 
	$$
	L(X)^{hG}\simeq (LX)^{hG} \text{   ?}
	$$
	Note that we have a fiber sequence in $\Sp$
	$$
	(LX)_{hG} \to (LX)^{hG} \to (LX)^{tG}
	$$
	and a cofiber sequence in $\Sp_{tame}$
	$$
	L(X)_{hG} \to L(X)^{hG} \to L(X)^{tG}.
	$$
	Note that in the case I'm interested in, $(LX)^{tG}\simeq *$ and $L(X)_{hG}\simeq (LX)_{hG}$.
	So we have a cofiber sequence in $\Sp_{tame}$
	$$
	(LX)^{hG} \to L(X)^{hG} \to L(X)^{tG}
	$$
	So it suffices to show $(-)^{tG}$ preserves tame equivalences, but this cannot be ture:
	consider the 0-tame equivalence $\BS\to H\BZ$,
	then $(\BS^{\otimes 2})^{tC_2}\simeq \BS^{\wedge}_{2}\to  (L_{tame}(\BS^{\otimes 2}))^{tC_2}\simeq *$, but $\pi_0 \BS^{\wedge}_2\neq 0$.
\end{question}


%\begin{proposition}
%	The Goodwillie tower of $\Sigma^{\infty}\Omega^{\infty}$ splits after tame localization.
%\end{proposition}
%\begin{proof}
%
%	Note that we have a pullback diagram in $\Sp$
%	\[	
%	\begin{tikzcd}
%		P_n\Sigma^{\infty}\Omega^{\infty} (X) 
%		& (X^{\otimes n})^{h\Sigma_n}\\
%		P_{n-1}\Sigma^{\infty}\Omega^{\infty} (X)  
%		& (X^{\otimes n})^{t\Sigma_n}
%		\arrow[from=1-1, to=1-2]
%		\arrow[from=2-1, to=2-2]
%		\arrow[from=1-1, to=2-1]
%		\arrow[from=1-2, to=2-2].
%	\end{tikzcd}
%	\]
%	For $n=1$, we have $P_1\Sigma^{\infty}\Omega^{\infty} (X)\simeq D_{1}\Sigma^{\infty}\Omega^{\infty} (X)\simeq X$.
%	Suppose we have 
%	$$
%	L_{tame}P_{n-1}\Sigma^{\infty}\Omega^{\infty}(X)\simeq
%	\bigoplus_{i=1}^{n-1} L_{tame} (X^{\otimes n})_{h\Sigma_n}.
%	$$
%	
%\end{proof}





\clearpage
