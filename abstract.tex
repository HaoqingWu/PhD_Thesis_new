%\begingroup
%\let\cleardoublepage\clearpage


% English abstract
\cleardoublepage
\chapter*{Abstract}
~\newline~\newline~
%\markboth{Abstract}{Abstract}
\addcontentsline{toc}{chapter}{Abstract (English/Français)} % adds an entry to the table of contents
% put your text here
In this thesis, we give a modern treatment of Dwyer's tame homotopy theory using the language of $\infty$-categories.
We introduce the notion of tame spectra and show it has a concrete algebraic description.
We then carry out a study of $\infty$-operads and define tame spectral Lie algebras and tame spectral Hopf algebras. 
Finally, we prove that the homotopy theory of tame spectral Hopf algebras is equivalent to that of tame spaces. To recover Dwyer's Lie algebra model for tame spaces, we use Koszul duality to construct a universal enveloping algebra functor, and show it is an equivalence from the $\infty$-category of tame spectral Lie algebras to the $\infty$-category of tame spectral Hopf algebras.



\vskip0.5cm
Key words: $\infty$-category, tame homotopy theory, spectral Lie algebra, Koszul duality
%put your text here





% French abstract
\begin{otherlanguage}{french}
\cleardoublepage
\chapter*{Résumé}
~\newline~\newline~
%\markboth{Résumé}{Résumé}
% put your text here
À travers cette thèse, nous étudions la théorie moderne de l’homotopie modérée de Dwyer en utilisant le langage des $\infty$-catégories.
Nous introduisons la notion de spectres modérée et montrons qu'elle a une description algébrique concrète.
Nous effectuons ensuite une étude des $\infty$-opérades et définissons les algèbres de Lie spectrales modérée et les algèbres de Hopf spectrales modérée. 
Enfin, nous prouvons que la théorie de l'homotopie des algèbres de Hopf spectrales modérée est équivalente à celle des espaces modérée. Pour retrouver le modèle d'algèbre de Lie de Dwyer pour les espaces modérée, nous utilisons la dualité de Koszul pour construire un foncteur universel d'algèbre enveloppante et montrons qu'il s'agit d'une équivalence entre la $\infty$-catégorie des algèbres de Lie spectrales modérée et la $\infty$-catégorie des algèbres de Hopf spectrales modérée.


\vskip0.5cm
Mots clefs: $\infty$-catégorie, théorie de l'homotopie tame, algèbre de Lie spectrale, dualité de Koszul.
%put your text here
\end{otherlanguage}


%\endgroup			
%\vfill
